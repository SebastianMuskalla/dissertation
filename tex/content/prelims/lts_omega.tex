\documentclass[../../diss.tex]{subfiles}
\begin{document}

\section{\texorpdfstring{$\omega$}{Omega}-regular languages}%
\label{LTS:OmegaAutomata}%

We consider variants of finite-state automata that define $\omega$-languages.
Obviously, the acceptance condition of ending in a final state needs to be replaced.
Simply requiring that a final state occurs would limit us to so-called reachability conditions and the associated \emph{safety properties} of programs.
Reachability conditions are not expressive enough to model so-called \emph{liveness properties}~\cite{GradelTW02} like fair scheduling.

\paragraph{Büchi automata}

The easiest way of making an NFA accept an $\omega$-language is to require that final states occur repeatedly.
On the syntactic level, \emph{nondeterministic Büchi automata (NBAs)} are defined just like NFAs.
Their semantics, however, is defined in terms of infinite words.
A run on an infinite \nb{word $w \in \Sigma^\omega$} from state $q \in Q$ is an infinite sequence of states and transitions
\[
    q = q_0 \tow{w_1} q_1 \tow{w_2} \cdots
    \ .
\]
A run of $A$ on $w$ is \emph{accepting} if it starts in the initial state $\qinit$ and in its sequence of states, final states from $\QF$ occur infinitely often.
As $\QF$ is a finite set, this is equivalent to requiring that some state from $\QF$ occurs infinitely often.
The language $\olang{A}$ of an NBA $A$ is the set of all infinite words on which $A$ has an accepting run.
Note that we use the notation $\olang{A}$ to distinguish the language of $A$ seen as Büchi automaton from $\lang{A}$, the language of finite words of $A$ seen as NFA.\@
This should not be confused with the omega-iteration of a language.

An $\omega$-language $\calL \subseteq \Sigma^\omega$ is called \emph{$\omega$-regular} if it is the language of an NBA $A$.
The class of $\omega$-regular languages enjoys many of the effective closure properties that also hold for regular languages of finite words:
They are closed under union, intersection, and complementation.
Furthermore, the following characterization is well-known~\cite{Thomas90}:
A language $\calL \subseteq \Sigma^\omega$ is $\omega$-regular if and only if it can be written as
\[
    \calL = \bigcup_{i \in \oneto{n}} U_i.\paren{V_i}^\omega
\]
where $n \in \N$, and for each $i \in \oneto{n}$, $U_i,V_i \subseteq \Sigma^*$ are regular languages of finite words.

While $\omega$-regular languages are closed under complement, the proof of this fact is much more involved than in the case of NFAs.
Recall that in the case of finite words, one uses that an NFA can be determinized and DFAs are easy to complement.
The same does not work for Büchi automata:
There are languages of NBAs that do not occur as languages of DBAs (deterministic Büchi automata).
The language ${(a \cup b)}^* b^\omega$ of words that contain only finitely many $a$s is one such example.
\cref{Figure:AutomataForFinManyAsNBA} depicts an NBA accepting that language.
It nondeterministically guesses the occurrence of the last $a$.

\begin{figure}[t]
    {\centering\subcaptionbox{An NBA.\label{Figure:AutomataForFinManyAsNBA}}[0.45\textwidth][c]{
        \begin{tikzpicture}[->,>=latex,node distance=7em,semithick]

\node[initial,state,transform shape, initial text={}] (A) {$q_0$};
\node[state, accepting, transform shape] (B) [right of=A] {$q_1$};

\path
    (A) edge [loop above] node [above] {$a,b$} (A)
    (A) edge node [above] {$b$} (B)
    (B) edge [loop above] node [above] {$b$} (B)
;

\end{tikzpicture}
}
    }
    {\centering\subcaptionbox{A DPA.\label{Figure:AutomataForFinManyAsDPA}}[0.45\textwidth][c]{
        \begin{tikzpicture}[->,>=latex,node distance=7em,semithick]

\node[initial,state,transform shape, initial text={}] (A) {$1$};
\node[state, transform shape] (B) [right of=A] {$0$};

\path
    (A) edge [bend left = 15] node [above]  {$b$} (B)
    (B) edge [bend left = 15] node [below] {$a$} (A)
    (A) edge [loop above] node [above] {$a$} (A)
    (B) edge [loop above] node [above] {$b$} (B)
;

\end{tikzpicture}
}
    }
    \caption{Automata accepting the language ${(a \cup b)}^* b^\omega$.}%
    \label{Figure:AutomataForFinManyAs}%
\end{figure}

\paragraph{Parity automata}

To solve the above problem, one possibility is to consider an automata model with a more involved acceptance condition.
There are several choices, including the Rabin~\cite{Rabin68}, Streett~\cite{Streett81}, Muller~\cite{Muller63}, and parity~\cite{Mostowski84} acceptance conditions.
We only consider the parity acceptance condition throughout this thesis as it is widely used in theory and tools.

The idea is to allow the nesting of Büchi and co-Büchi conditions.
The latter allow the specification of states that should not be visited infinitely often.
We partition the states $Q = Q_1 \dotcup Q_2 \dotcup \ldots \dotcup Q_m$ into several sets, where (1)~there is a total order on these sets and (2)~each set is marked either as final or as non-final.
A run is accepting if the maximal set $Q_i$ (\wrt the total order) such that a state from $Q_i$ occurs infinitely often in the run is final.

We follow the usual convention to formalize the partition by a \emph{priority assignment}
\(
    \Omega \colon Q \to \N
\)
on the states.
Each set $Q_i$ is defined as the set of states that have priority $i$, $Q_i = \Set{ q \in Q}{ \Omega(q) = i}$.
The total order is as indicated by the priority, where we fix the convention that higher priorities are dominating.
We consider even priorities as final and odd priorities as non-final.

Altogether, a \emph{(nondeterministic) parity automaton (NPA)} is a tuple
$A = (Q, \Sigma, \delta, \qinit, \Omega)$, where all components but $\Omega$ are as for NFAs and NBAs and $\Omega \colon Q \to \N$ is a priority assignment as introduced above.
A run of an NPA $A$ on an infinite word $w$ is accepting if it starts with the initial \nb{state $\qinit$} and the largest priority occurring infinitely often in the run is even.
The notion of language is again the set of all infinite words that have an accepting run.

As $Q$ is finite, so is the image of $\Omega$, and we may speak of the \emph{largest priority} that is used by a parity automaton.
Similarly, each infinite run has a largest priority that occurs infinitely often.
% It is clear that we
% It is also clear that anything that can be expressed with a parity automaton can already be expressed with a parity automaton using priorities only up to $\card{Q} + 2$.
Büchi automata can be seen as special parity automata that only use two priorities, $1$ for non-final \nb{and $2$ for} final states.

\begin{remark*}
    For finite-state systems, choosing larger priorities to be dominating and even priorities to be final is arbitrary, and in parts of the literature, other conventions are used.
\end{remark*}

In contrast to Büchi automata, parity automata are determinizable.
\emph{Deterministic parity automata (DPAs)} are defined as expected by requiring the transition relation to be deterministic.
For any NPA, there is a language-equivalent DPA.\@
However, there is a potential exponential blowup in the number of states that is even worse than in the case of finite-state automata on finite words.

\begin{theorem}[(\citea{Safra88})]
    For any NPA $A$ with $\card{Q}$ states, there is a language-equivalent DPA with $2^{\bigO{\card{Q} \cdot \log \card{Q}}}$ states.
    The construction is optimal.
\end{theorem}

\cref{Figure:AutomataForFinManyAsDPA} depicts a deterministic parity automaton (where the names of the states indicate the priorities) that accepts the language ${(a \cup b)}^* b^\omega$ for which no deterministic Büchi automaton exists.

Both NPAs and DPAs can be used to obtain an alternative but equivalent definition of the $\omega$-regular languages:
A language is $\omega$-regular iff it occurs as the language of an NBA/NPA/DPA.\@
Converting an NBA into an NPA is trivial, determinizability is Safra's result, and to convert a DPA to an NBA, we let the NBA guess the highest-occurring priority at the beginning of the run and verify its guess later.

\end{document}
