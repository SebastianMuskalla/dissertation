\documentclass[../../diss.tex]{subfiles}
\begin{document}

\section{Higher-ordered recursion schemes}%
\label{Section:HORS}%

We introduce \emph{higher-order recursion schemes (HORSes)}~\cite{Nivat72,CourcelleN78}, a rewriting-based type of language generating mechanism that can be seen as a generalization of context-free grammars.
As briefly mentioned, context-free grammars can be used to model recursive programs;
they can be seen as a HORSes of order one.
Modern programming languages like Haskell~\cite{Haskell03} support higher-order functions, \ie functions whose parameters themselves are functions.
Context-free grammars (and HORSes of order one) are insufficient to model this concept, while HORSes of order greater than one provide a model for such programs.
Our presentation follows~\cite{Haddad12}.

We first introduce a typing discipline.
The typing fixes for each term a type, which in particular determines its order, \ie whether it is a value (a term of order zero), a function of order one that operates on values, or a function of higher order that has functions as parameters.
Representing both functions and values by terms corresponds to the concept of treating functions as \emph{first-class citizens} in modern programming languages.
Instead of having a clear distinction between values and functions, functions are simply seen as values of the corresponding function type.

We will actually not use the name type in the following.
We refer to the concept that was explained above as \emph{kinds}, to avoid confusion with type-based approaches to the verification of HORSes as used \eg in~\cite{Kobayashi09}.

We define $\ground$ to be the unique kind \emph{ground} of data values.
Functions kinds are derived by composition.
Formally, the kinds $\kappa$ are defined by the following grammar:
\begin{align*}
     \kappa\ \bnfdef \ground \bnf ( \kappa \to \kappa)\ .
\end{align*}
Intuitively, a term of kind $\kappa_1 \to \kappa_2$ is a function that takes a value of kind $\kappa_1$ and returns a value of kind $\kappa_2$ (where both the parameter and the return value may be functions).
We define $\kinds$ to be the set of all kinds.

We usually omit unnecessary brackets by assuming right-associativity.
For instance, this means that $\ground \to \ground \to \ground$ denotes $\ground \to (\ground \to \ground)$, \ie the kind of functions that take a ground value and return a function that takes and returns a ground value.

Our definition of kinds does not allow for multi-parameter functions.
Instead, we use the concept of currying.
A function that takes two values and returns one, which intuitively should have kind $\ground \times \ground \to \ground$ is seen as a function of kind $\ground \to (\ground \to \ground)$:
It takes the first parameter and returns a function that takes the second parameter and then returns the final value.
Formally, we consider $f \colon \ground \times \ground \to \ground$ as a function $f' \colon \ground \to (\ground \to \ground)$ by defining $f'(x) \colon \ground \to \ground$ with $(f'(x))(y) = f(x,y)$.
This concept is commonly implemented in functional programming languages like Haskell to simplify the usage of partially evaluated functions.

We define the notion of \emph{arity} that describes the number of parameters that a function takes (in its uncurried form), and its \emph{order} as explained above.
We formally define
\begin{align*}
        \horsarity{\ground} &= 0,
        & \horsorder{\ground} &= 0,\\
        \horsarity{\kappa_1 \to \kappa_2} &= \horsarity{\kappa_2}+1,
        & \horsorder{\kappa_1 \to \kappa_2} &= \max(\horsorder{\kappa_1} +1, \horsorder{\kappa_2})\ .
\end{align*}

For example, we have $\horsarity{\ground \to (\ground \to \ground)} = \horsarity{\ground \to \ground} + 1 = \horsarity{\ground} + 2 = 2$ and $\horsorder{\ground \to (\ground \to \ground)} = 1$.
Indeed, functions of type $\ground \to (\ground \to \ground)$ in their uncurried form take two parameters and are of order one, since their parameters and their return value are ground values.
In contrast, we have $\horsarity{(\ground \to \ground) \to \ground} = 1$ and $\horsorder{(\ground \to \ground) \to \ground} = 2$.
A function of kind $(\ground \to \ground) \to \ground$ takes a function as parameter and returns a ground value.
% , but it  a function as parameter.
% Also note that the kind $(\ground \to \ground) \to \ground$ cannot be simplified by omitting brackets without changing its meaning.


Just as a context-free grammar, a HORS consists of terminals and nonterminals.
In contrast to CFGs, we see the nonterminals of HORSes as functions.
Syntactically, this means that we also need a set of variables.
%
To get a consisting typing, each of the three types of symbols has an associated kind.
More formally, a \emph{kinded symbol} is a symbol $a$ together with a kind $\kappa$, which we write as $a \colon \kappa$.
Let $\Lambda$ be a set of kinded symbols.
When convenient, we see $\Lambda$ as a set of (distinct) symbols and assume that the kind assignment is implicitly given.
By $\Lambda_\kappa$, we denote the restriction of $\Lambda$ to symbols with kind~$\kappa$.

For each kind $\kappa$, $\horsterms{\Lambda}{\kappa}$ is the set of terms of kind $\kappa$.
These sets are defined by simultaneous inductions over all kinds.
They are the smallest sets so that
\begin{enumerate}[(1)]
    \item
        All symbols from $\Lambda$ are terms of the appropriate kind,
        \[
            \forall \kappa \colon
            \Lambda_\kappa \subseteq \horsterms{\Lambda}{\kappa}
        \]
    \item
        If $f  \in \horsterms{\Lambda}{\kappa_1 \to \kappa_2}$ is a (function) term of kind $\kappa_1 \to \kappa_2$, and $v \in \horsterms{\Lambda}{\kappa_1}$ is a term of kind $\kappa_1$, then the \emph{application} $f\ v$ is a term in $\horsterms{\Lambda}{\kappa_2}$ of kind $\kappa_2$,
        \[
            \forall \kappa_1, \kappa_2 \colon
            \Set{f \appl t}{f \in \horsterms{\Lambda}{\kappa_1 \to \kappa_2},  v \in \horsterms{\Lambda}{\kappa_1}} \subseteq \horsterms{\Lambda}{\kappa_2}
            \ .
        \]
    \item
        If $t \in \horsterms{\Lambda}{\kappa_2}$ is a term of kind $\kappa_2$, and $x \in \horsterms{\Lambda}{\kappa_1}$ is a term of kind $\kappa_1$, then the \emph{$\lambda$-abstraction} $\lambda x . t$ is a term of kind $\kappa_1 \to \kappa_2$,
        \[
            \Set{ \lambda x . t}{ t \in \horsterms{\Lambda}{\kappa_2},  x \in \horsterms{\Lambda}{\kappa_1} }
            \subseteq
            \horsterms{\Lambda}{\kappa_1 \to \kappa_2}
            \ .
        \]
\end{enumerate}
The set of all terms $\horsterms{\Lambda}$ is defined as the union of the $\horsterms{\Lambda}{\kappa}$ for all kinds $\kappa$.

As we have seen in the second rule, we usually omit the brackets for function application, \ie we \nb{write $f \appl t$} instead of $f(t)$.
The last rule is interesting as it allows us to construct infinitely many terms of arbitrarily large order assuming there is at least one symbol of kind ground.
Instead of \nb{writing $t \in \horsterms{\Lambda}{\kappa}$}, we simply write $t \colon \kappa$ (where we assume that $\Lambda$ is clear from the context).
%
Terms that do not contain $\lambda$-abstractions, \ie they are build only using the first two cases of the definition of terms, are called $\lambda$-free.

When defining HORSes, we will assume that $\Lambda$ contains a set of variables.
To distinguish these variables form other types of variables that we will consider in \cref{Part:Games}.~of the thesis, we speak of \emph{HORS variables}.
We assume that only the HORS variables are allowed to take the place of $x$ in the $\lambda$-abstraction $\lambda x . t$.
This allows us to speak of the free variables of a term, \ie variables that occur in the term and that are not bound by a preceding $\lambda$-abstraction.
A \emph{variable-closed} term is a term that does not contain any free variables.

We have now gathered the prerequisites to formally defines HORSes.

% \begin{definition}
    A \emph{higher-order recursion scheme (HORS)} is a tuple $G = (V, N, T, P, S)$, where
        $V$ are the HORS variables, $T$ are the terminals, $N$ are the nonterminals.
        All three sets are finite sets of kinded symbols that are pairwise disjoint.
        The symbol $S \in N$ is the initial symbol.
        The set $P$ contains a finite set of \emph{production rules} of the shape $F \to t$, where $t$ is a term over $\Lambda = V \dotcup N \dotcup T$, the union of variables, nonterminals and terminals.
        We require that $P$ is well-typed in that for each \nb{production $X \to t$}, the kinds of $X$ and $t$ coincide.
        We require that each right-hand side $t$ is a variable-closed term $\lambda x_1 \ldots \lambda x_n . e$ where the $x_i$ are variables and $e$ is a $\lambda$-free term of kind $\ground$. (This implies \nb{that $x_1, \ldots, x_n$} is a superset of the variables that occur in $e$.)
% \end{definition}

We generalize the notions of order and arity from kinds to terms of that kind.
The order of a HORS is the maximum order of any of its nonterminals.

The semantics of a HORS $G$ is defined in the form of a term rewriting system, whose terms are the terms (over the union of variables, nonterminals, and terminals), and the rewriting rules of which are induced by the productions of $G$.
To make this definition formal, we define a \nb{\emph{context} $C[\contextplaceholder]$} to be a term over $\Lambda \dotcup \set{ \contextplaceholder \colon \ground}$ in which the placeholder $\contextplaceholder$ of kind ground occurs exactly once.
Given $C[\contextplaceholder]$ and a term $t \colon \ground$ of kind ground, we define $C[t]$ as the term over $\Lambda$ obtained by replacing the unique occurrence of $\contextplaceholder$ by $t$.
Note that the kinds of $C[\contextplaceholder]$ and $C[t]$ coincide.
We introduce a derivation relation on terms over $\Lambda$ by defining $t \derive t'$ to hold iff if there is a \nb{context $C[\contextplaceholder]$}, a production rule $X \to \lambda x_1 \ldots \lambda x_n.e$, and a term $X\ t_1\ \ldots\ t_n \colon \ground$ such that
$t = C[X\ t_1\ \ldots\ t_n]$ and $t' = C[e[x_1 \mapsto t_1, \ldots, x_n \mapsto e_n]]$.
In words, the derivation relation allows us to choose a subterm $X\ t_1\ \ldots\ t_n$ of kind ground (which intuitively means that all parameters of $X$ are present) and replace it by the right-hand side of a production for $X$, with the occurrences of the variables replaced by the appropriate parameters.
We call such a subterm a \nb{\emph{reducible expression (redex)}}.
Note that since we required productions to be well-typed, the derivation relation is kind-preserving: If $t \derive t'$, then $t$ and $t'$ are of the same kind.

To associate a language to a HORS, one usually imposes several restrictions.
The first one is that we require the initial symbol $S$ to be of kind ground.
Since the rewriting relation is kind-preserving, this implies that any term derivable from $S$ is of kind ground.
The second one is that all terminals have order at most one.
Together, this means that any term over the terminals that can be derived from the initial word can be seen as a tree:
The term $a\ t_1\ t_2\ \ldots t_n$, where $a$ is a terminal of arity $n$, is a tree consisting of an $a$-labeled root and $n$ subtrees that correspond to the parameters.
In particular, a terminal of arity $0$ is seen as a tree consisting of a single leaf.
% Note that we have omitted brackets by writing $a\ t_1\ t_2\ \ldots t_n$ for the term $(\ldots((a\ t_1)\ t_2) \ldots ) t_n$, similar to the conventions for kinds.

\paragraph{Word-generating schemes}

We are interested in schemes that define a language of finite words.
To this end, we impose the stronger restriction that the set of terminals is of the shape $T = \Sigma \dotcup \set{ \wordend \colon \ground}$, where $\wordend$ is the \emph{word-end marker} of kind ground, and the other terminals in $\Sigma$ are of kind $\ground \to \ground$, \ie they are first-order and have arity one.

We say that the word $w = a_1 \ldots a_n \in \Sigma^*$ (with the symbols $a_i$ from $\Sigma$ seen as normal letters) can be derived by HORS $G$ if $S \derive^* a_1 (a_2 ( a_3 \ldots a_n(\wordend)))$.
In the following, we will identify terms of this shape with the corresponding words.
The language $\lang{G}$ of $G$ is defined to be the set of all words that can be derived by $G$.

A derivation step is \emph{outermost to innermost (OI)} if there is no redex that contains the one that was replaced as a proper subterm.
\citea{Haddad12} has shown that we do not lose expressiveness by restricting ourselves to OI-derivations:
Any word that can be derived from the initial symbol can be derived by a sequence of OI-derivation steps.
In a sense OI-derivation steps for HORSes correspond to left-derivations in the case of CFGs.

There is the analogue notion of innermost to outermost (IO) derivations.
However, there are words that can be derived by OI derivations, but not by IO derivations.
Also, we will later make use of the fact that each derivable term in a word-generating scheme has a unique outermost redex.
The same is not true for innermost redexes.

\begin{example}%
\label{Example:HORSfromCFG}%
    Let $G = (N,P,S)$ be a context-free grammar over $\Sigma$.
    We define a word-generating \nb{scheme $G' = (V, N', T, P, S')$} of order one with the same language.
    The set of variables consists of the unique variable $x \colon \ground$.
    The set $N'$ consists of the nonterminals $N$ of $G$, each seen $X \in N$ as a symbol of kind $\ground \to \ground$.
    Additionally, we introduce a fresh nonterminal $S' \colon \ground$ that is the initial symbol of the HORS.\@
    The terminals $T$ of the HORS consist of the word end marker $\wordend \colon \ground$ and of the \nb{terminals $\Sigma$} of the grammar, each terminal $a$ seen as a symbol of kind $\ground \to \ground$.

    For the nonterminals in $N$, each rule $X \to \eta$ of the grammar with $\eta = \eta_1 \eta_2 \ldots \eta_m \in {(N \cup \Sigma)}^*$ induces a rule $X \to \lambda x . \eta_1 (\eta_2 ( \ldots \eta_m (x) \ldots )) $.
    Intuitively, a nonterminal takes the suffix of the terminal word that has already been generated as the parameter $x$ of kind $\ground$.
    It then prepends the sentential form $\eta$.
    Technically, the concatenations in $\eta$ are seen as a sequence of function applications.
    % Finally, this allows us to derive $w.x$, where $w$ is a word derived from sentential form $\beta$.
%
    Additionally, there is a rule $S' \to S \appl \wordend$ for the fresh initial symbol $S'$.
    Intuitively, it starts the derivation process with the empty suffix.

    We obtain a one-to-one correspondence between derivation steps in the grammar and derivation steps of the HORS, and between OI-replacement steps and left-derivations.
    To see that this is true, note that any term that can be derived by the HORS from $S$ is of the \nb{shape $\beta_1 (\beta_2 ( \ldots \beta_k (\wordend) \ldots ))$}, where each $\beta_i$ is a terminal or a nonterminal from $N$.
    Each \nb{subterm $\beta_i (\beta_{i+1} ( \ldots \beta_k (\wordend) \ldots ))$} where $\beta_i$ is a nonterminal is a redex.
%
    % Similarly, each left-derivation corresponds to a OI-replacement step:
    % The leftmost occurrence of a nonterminal in a sentential form corresponds to the outermost redex in a HORS term.
%
    Altogether, we obtain $\lang{G} = \lang{G'}$ as desired.

    We instantiate this construction for the grammar from \cref{Example:CFGAnBn} and obtain the \nb{HORS $G' = (\set{x \colon \ground},\set{S' \colon \ground, S \colon \ground \to \ground},\set{a \colon \ground \to \ground, b \colon \ground \to \ground, \wordend \colon \ground}, P, S')$} with the rules $S' \to S \appl \wordend$, $S \to \lambda x . x$, and $S \to \lambda x . a \appl S \appl b \appl x$.
    Note that $a \appl S \appl b \appl x$ stands for $a(S(b(x)))$.
    This HORS generates the language $\Set{a^n b^n}{n \in \N}$ as expected.
\end{example}

\begin{example}
    We give an example showing that HORSes are strictly more expressive than CFGs.
    Consider the \nb{HORS
    $G = (\set{f \colon \ground \to \ground, x \colon \ground}, N, T, P,S)$} where
    $N = \set{S \colon \ground, X, Y \colon (\ground \to \ground) \to \ground \to \ground, Z \colon \ground \to \ground}$,
    $T = \set{a, b, c \colon \ground \to \ground, \wordend \colon \ground}$,
    and the rules are $S \to X \appl Z \appl \wordend$, $X \to \lambda f. \lambda x . f \appl x$, $X \to \lambda f. \lambda x . X \appl (Y \appl f) \appl (c x)$, $Y \to \lambda f.\lambda x . a \appl f \appl b \appl x$, and $Z \to \lambda x . x$.
    It is of order two and generates the non-context-free \nb{language $\Set{a^n b^n c^n}{n \in \N}$}.
\end{example}

\paragraph{Determinism}

A HORS is deterministic if each nonterminal $X$ has a unique rule with $X$ as its left-hand side.
If we restrict ourselves to OI-derivation steps, this makes the transition relation on the semantic level deterministic for word-generating schemes.

Many works in the literature present word-generating schemes in a different way:
They consider deterministic schemes without restricting the terminals to be of arity one.
These schemes have a unique derivation process, the limit of which is a tree.
The language of finite words of the scheme occurs as the set of all labels along finite branches of that tree.
Both views have advantages and drawbacks, and it is easy to convert between them.
We will formally introduce the construction for determinization in \cref{Section:HORSFPSemantics}.
The definition of word-generating schemes here makes it easier to see the correspondence between CFGs and HORSes of order one.

\paragraph{Alternative models}

In the same way that context-free languages can also be defined via pushdown automata, there is a state-based model for languages of word-generating higher-order schemes.
A higher-order pushdown automaton of order $n$ is an automaton that uses an order-$n$ stack as storage. A stack of order one is simply a LIFO stack as defined in \cref{Section:CFG}.
A stack of order $i+1$ for $i \geq 1$ is a sequence of order $i$ stacks.
In addition to the order-one operations $\pushop A$ and $\popop A$ for every symbol of the stack alphabet, the transitions can be labeled by $\pushop_{\!i}$ and $\popop_{\!i}$ for $i \in \fromto{2}{n}$.
An order-$i$ push duplicates the topmost order-$i$ stack (inside the topmost order-$(i+1)$ stack inside the topmost-$(i+2)$ stack and so on, if $i \neq n$).
Similarly, an order-$i$ pop operation removes the topmost order-$i$ stack.

Unfortunately, order-$n$ pushdown automata do not generate all languages of HORSes of order~$n$.
Rather, they generate a subclass, the languages of \emph{safe} HORSes, where safety is a syntactic restriction on the production rules of HORSes~\cite{DammG86,KnapikNU02}.

To get the desired equivalence, one has to enhance the model to obtain collapsible higher-order pushdown systems~\cite{HagueMOS17}.
In this model, each (order-$1$) stack symbol is equipped with a link, a pointer into the stack that represents the context in which the symbol was created.
Additionally, there is a collapse operation that removes a part of the stack so that the target of the link becomes the top-of-stack.
Giving the formal definitions and a more in-depth explanation is not needed for this thesis.

\end{document}
