\documentclass[../../diss.tex]{subfiles}
\begin{document}

\section{Finite-state automata}

We introduce finite-state automata, which are simply LTSes with finitely many configurations.
However, one usually uses modified notation in this case.
We also follow the convention of having a single initial state.
The material is standard and can be found \eg in~\cite{KhoussainovN01,Kozen97}.

Formally, a \emph{(nondeterministic) finite(-state) automaton (NFA)} over alphabet $\Sigma$ is a tuple
\(
    A = (Q, \delta, \qinit, \QF)
    ,
\)
where
    $Q$ is a finite set of \emph{control states},
    $\delta \subseteq Q \times \Sigma \times Q$ is the transition relation,
    $\qinit \in Q$ is the unique initial state, and
    $\QF \subseteq Q$ is a set of final states.

Apart from the different naming of the components, the notation and definitions for LTSes carry over.
For example, we call an NFA a \emph{deterministic finite automaton (DFA)} if it is deterministic.

The size of an automaton $A$ is $\card{A} = \card{Q} + \card{\delta}$.
Since $\card{\delta} \leq \card{Q}^2 \cdot \card{\Sigma}$, to show that an automaton is \eg of size polynomial in $n$, it is sufficient to show that $\card{Q}$ is polynomial in $n$ (assuming that the alphabet is fixed).

A language $\calL \subseteq \Sigma^*$ is called \emph{regular} if there is a finite automaton $A$ over $\Sigma$ with $\lang{A} = \calL$.
The class of all regular languages is denoted by $\REG$.

To decide the word problem of a regular language, we do not need the capabilities of an unrestricted Turing machine, a finite automaton is sufficient.

\begin{example}%
\label{Example:NFAForABStar}%
    The language ${(ab)}^* \subseteq \set{a,b}^*$ is regular: It is the language of the NFA given by \cref{Figure:NFAForABStar}.
\end{example}

\begin{figure}
    \centering%
    \begin{tikzpicture}[->,>=latex,node distance=7.3em,semithick]

\node[initial,state,accepting,transform shape, initial text={}] (A) {$q_0$};
\node[state, transform shape] (B) [right of=A] {$q_1$};

\path
    (A) edge [bend left = 15] node [above]  {$a$} (B)
    (B) edge [bend left = 15] node [below] {$b$} (A)
;

\end{tikzpicture}
%
    \caption{An NFA with language ${(ab)}^*$.}%
    \label{Figure:NFAForABStar}%
\end{figure}

The class of regular languages can also be defined by \emph{regular expressions}, finite expressions that may contain the empty language $\emptyset$, the singleton languages $\set{\eps}$ and $\set{a}$ for each letter $a$ of the underlying alphabet and unions, concatenations, and the Kleene star.
From this, we immediately obtain that every language consisting of finitely many words is regular, and if $\calL, \calL' \subseteq \Sigma^*$ are regular, then so are their union $\calL \cup \calL'$, their concatenation, $\calL.\calL'$, and the Kleene star $\calL^*$.
It can also be shown that the intersection $\calL \cap \calL'$ of two regular languages and that the complement $\overline{\calL}$ of a regular language are regular.
All these closure properties are effective:
Given automata for $\calL$ and $\calL'$, we can construct an automaton for each of the aforementioned languages.
In the case of the intersection, we have explicitly stated the construction in \cref{Section:LTS}.

In addition to the definitions based on automata or closure properties, the regular languages can also be characterized as the solution to closed formulas in the logic $\mathsf{S1S}$, (weak) monadic second-order logic with one successor, see \eg \cite{KhoussainovN01}.
Discussing the details is beyond the scope of this thesis.

\end{document}
