\documentclass[../../diss.tex]{subfiles}
\begin{document}

\section{(Labeled) Transition systems}%
\label{Section:LTS}%

A \emph{transition system} $S = (\configs,T)$ consists of a set of \emph{configurations} $\configs$ and a \emph{transition relation} $T \subseteq \configs \times \configs$.
One can say that the transition systems defined by hardware and software systems are the main object of interest in computer science.

Instead of working directly on the configuration space $\configs$, we want to see transition systems as language-generating devices.
This additional level of indirection allow us to compare transition systems with different sets of configurations as long as they define languages over the \nb{same alphabet}.

Formally, a \emph{labeled transition system (LTS)} over alphabet $\Sigma$ is of the shape $\lts = (\configs,T)$ where $T \subseteq \configs \times \Sigma^* \times \configs$ is a set of transitions labeled by finite words over $\Sigma$.
Most of the time, we are interested in LTSes whose transitions are labeled by words of length exactly one (\ie $T \subseteq \configs \times \Sigma \times \configs$), or of length at most one.
In the latter case, we get $T \subseteq \configs \times \Sigma_\eps \times \configs$ where $\Sigma_\eps = \Sigma \dotcup \set{\eps}$.

In the case of labeled transitions, we usually write $c \tow{a} c'$ instead of $(c,a,c') \in T$.
We extend the notations towards \emph{computations} (also called \emph{runs}) in the expected way and write $c \tow{w} c'$ where $w \in \Sigma^*$ is a word if there is a sequence of states $c_0, \ldots, c_{\card{w}}$ such that $c = c_0 \tow{w_1} c_1 \tow{w_2} \ldots \tow{w_{\card{w}}} c_{\card{w}} = c'$.

To see an LTS as a device that generates finite words, we equip it with sets of \emph{initial} and \emph{final} configurations, respectively.
This means we consider a tuple $\lts = (\configs,T,\configs_\init,\configs_\final)$, where $(\configs,T)$ is as before and $\configs_\init, \configs_\final \subseteq \configs$.
A run of such an LTS is \emph{accepting} if it starts in an initial and ends in a final configuration.
The language $\lang{\lts} \subseteq \Sigma^*$ of an LTS is the set of all words that occur along accepting runs,
\[
    \lang{\lts} = \Set{ w \in \Sigma^*}{c_\init \tow{w} c_\final \text{ for some } c_\init \in \configs_\init, c_\final \in \configs_\final}
    \ .
\]

\begin{remark*}
    Seeing an LTS as a device for generating languages of infinite words is more involved, and we will not give a general definition on the level of LTSes.
    The acceptance condition (ending in a final state) has to be replaced, and there are different ways to do so, each with its own sets of drawbacks and advantages.
    A detailed discussion is given in \cref{LTS:OmegaAutomata}.
\end{remark*}

We define some notation for LTSes with $\Sigma$-labeled transitions.
For a letter $a$ and a \nb{configuration $c$} of some LTS $\lts$, we denote by
%
\begin{align*}
    \post{\lts}{a}{c} &= \Set{ c' \in \configs}{ c \tow{a} c'}
    \ ,
    &
    \pre{\lts}{a}{c} &= \Set{ c' \in \configs}{ c' \tow{a} c}
\end{align*}
%
the set of \emph{(direct) $a$-successors} \resp \emph{predecessors}.
We extend the notation to sets of configurations by taking the union, \eg $\post{\lts}{a}{C} = \bigcup_{c \in C} \post{\lts}{a}{c}$.
Additionally, we extend the notation to words by induction,
%
\begin{align*}
    \post{\lts}{\eps}{C} &= \pre{\lts}{\eps}{C} = C
    \ ,
    \\
    \post{\lts}{w.a}{C} &= \post{\lts}{a}{\post{\lts}{w}{C}}
    \ ,
    &
    \pre{\lts}{w.a}{C} &= \pre{\lts}{w}{\pre{\lts}{a}{C}}
    \ .
\end{align*}
%
We will be mostly interested in the configurations reached from the initial configurations and, similarly, in the configurations from which a final configuration can be reached.
We define
%
\begin{align*}
    \reach{\lts}{w} &= \post{\lts}{w}{\configs_\init}
    \ ,
    &
    \reachrev{\lts}{w} &= \pre{\lts}{w}{\configs_\final}
    \ .
\end{align*}
%
We generalize $\reach, \reachrev, \postT$, and $\preT$ to set of words by taking the union.
A particularly interesting case is the case of all words, which we simply denote by
\begin{align*}
    \reach{\lts} &= \reach{\lts}{\Sigma^*} = \bigcup_{w \in \Sigma^*} \reach{\lts}{w}
    \ ,
    &
    \reachrev{\lts} &= \reachrev{\lts}{\Sigma^*} = \bigcup_{w \in \Sigma^*} \reachrev{\lts}{w}
    \ ,
\end{align*}
%
the set of all configurations reachable from the initial configurations respectively the set of all configurations from which a final configuration is reachable.
It is clear from the definition of the language of an LTS that we have $\lang{\lts} \neq \emptyset$ if and only if one (or all) of the following equivalent conditions is true:
\begin{align*}
    \reach{\lts} \cap \configs_\final &\neq \emptyset
    \ ,
    &
    \reachrev{\lts} \cap \configs_\init &\neq \emptyset
    \ ,
    &
    \reach{\lts} \cap \reachrev{\lts} &\neq \emptyset
    \ .
\end{align*}
%
If the system $\lts$ we are considering is clear from the context, we omit the subscript and simply write \eg $\post{a}{c}$.

The transition relation of an LTS is \emph{finitely branching} if for each configuration $c$ and each \nb{symbol $a \in \Sigma$}, there are at most finitely many successors $c'$ such that $c \tow{a} c'$.
It is called \emph{unique} if there is at most one successor, \emph{complete} if there is at least one successor, and \emph{deterministic} if there is exactly one successor.
If the transition relation $T$ of an LTS is deterministic, we may see it as a function $T \colon \configs \times \Sigma \to \configs$ such that $T(c,a)$ is the unique $c'$ with $c \tow{a} c'$.

An LTS is \emph{finitely branching} if its transition relation is and it additionally has only finitely many initial states.
It is \emph{deterministic} if its transition relation is and it has a unique initial state.

\paragraph{The synchronized product}

We present a standard construction that, given two LTSes, yields an LTS whose language is the intersection of the languages of the LTSes.
Intuitively, the product of the LTSes tracks tuples of configurations, simulating a run of one LTS per component.
Transitions not labeled by $\eps$ need to be applied synchronously to both components.
Formally, the construction is as follows:
Let $\lts = (\configs,T,\configs_\init,\configs_\final)$ and $\lts' = (\configs',T',\configs_\init',\configs_\final')$ be two LTSes with transition labeled by $\Sigma_\eps$.
Their \emph{synchronized product} is $\lts \times \lts' = (\configs \times \configs', T_\times, \configs_\init \times \configs_\init', \configs_\final \times \configs_\final')$ where $T_\times$ is defined by $(c,c') \tow{a} (d,d')$ if
(1) $a = \eps$, and $c \tow{\eps} d$ in $T$ and $c' = d'$,
or
(2) \nb{$a = \eps$, and $c' \tow{\eps} d'$ in $T'$ and $c = d$,}
or
(3) $a \in \Sigma$, and $c \tow{a} d$ in $T$ and $c' \tow{a} d'$ in $T'$.

It is straightforward to show $\lang{\lts \times \lts'} = \lang{\lts} \cap \lang{\lts'}$: For each word $w$, a run of the product LTS gives rise to a run in each LTS.\@
Similarly, two runs on the same word, one for each system, can be merged into a single run of the product system.


\paragraph{Automata}

LTSes are a very general concept since we have not imposed any restriction on the set $\configs$; in particular, it may be infinite.
The drawback is that we cannot treat them algorithmically in this full generality.
%
To solve the problem, we will focus on \emph{automata} in the following.
Automata are LTSes of a special shape:
Their set of configurations can be written as $\configs = Q \times M$ such that $Q$ is a \emph{small} (typically finite) set of \emph{control states}, and $M$ is a potentially large (\eg infinite) set of \emph{memory values}.
The transitions $T$ are allowed to depend arbitrarily on $Q$, but only \emph{locally} on $M$.
The latter means for example that even if $M$ is infinite (which necessarily means that its elements have unbounded size), the transitions should only depend on a bounded part of the current value $m \in M$.
More formally, one can require that there is a finite set of \nb{\emph{transition rules} $\to$} that specify whether a transition $(q,m) \tow{a} (q',m') \in T$ exists.
Hence, an automaton can usually be represented by some \emph{finite syntax} (the set $Q$, a representation of the shape of the elements of $M$, and $\to$), although it gives rise to a potentially infinite \emph{semantics}, the transition system $(Q \times M, T)$.
% In many cases, this allows us to use the finite description of the system to compute properties of its infinite configuration space.

For example, sequential imperative programs can be seen as automata:
The control states are the lines of code while the memory stores the assignment of the variables.
The transitions from one line to another only depend on the memory as specified \eg by a conditional in the source code.
With this view, the source code is the syntax of the automaton, the transition system that describes its runtime behavior is the semantics.
The labeling that we consider for such a system depends on what property we want to analyze.
For example, it may be the atomic commands that are executed by the program.

The above definition of automata is imprecise and fuzzy, but it captures the fundamental idea behind automata theory:
An infinite semantics is represented by a finite description that can serve as input for algorithms.
%
However, requiring a finite representation of the system is not sufficient to get decidability results.
For example, Turing machines also admit a finite description, but by Rice's  theorem~\cite{Rice53} no non-trivial properties of the languages they define are decidable.
Currently, there is no general theory that explains which types of systems admit which decidability results.
The model of valence systems over graph monoids~\cite{Zetzsche15d,Charney07} that we present in \cref{Chapter:ValenceGames} can be seen as a step in that direction, but this model is not powerful enough to capture all types of memories, and it has not (yet) gained wide-spread usage.

This leaves us with having to define various classes of automata for which certain properties are decidable.
In the following, we define various types of \emph{state-based systems} that are automata in the above sense and for which some properties, \eg membership in their language, are decidable.
Finite automata, Petri nets, and well-structured transition systems fall into this category.

We will also consider \emph{rewriting-based systems} like context-free grammars and higher-order recursion schemes.
These systems can also be seen as automata as we will explain in \cref{Section:CFGOmega}.

\end{document}
