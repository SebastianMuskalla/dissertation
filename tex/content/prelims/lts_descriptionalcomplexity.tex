\documentclass[../../diss.tex]{subfiles}
\begin{document}

\section{Descriptional complexity}%
\label{Section:DescriptionalComplexity}%

We introduce some notions regarding the descriptional complexity of regular languages.
Unlike computational complexity (see \cref{Section:Complexity}) which studies the amount of resources needed to decide membership in a language, descriptional complexity studies the size of descriptions of languages.
However, many interconnections between the two areas exist in the form of results that show that languages for which the membership problem has a certain complexity admit descriptions of a certain shape and size and vice versa~\cite{Immerman98}.
Descriptional complexity theory considers several types of objects to represent languages, including logical formulas and circuits.
In this thesis, we are exclusively interested in the descriptional complexity of regular languages, for which we use finite automata as descriptions.

The well-known fact that DFAs and NFAs are equally expressive results in two measures of complexity for regular languages.
We will introduce and use both in the following.
For a regular language $\calL$, its \emph{state complexity} is the minimum number of states that an NFA $A$ with $\lang{A} = \calL$ has.
Its \emph{index} is the minimum number of states that a DFA $A$ with $\lang{A} = \calL$ has.
The latter notion comes from the fact that this number is indeed the index, the number of equivalence classes, of the Nerode right-congruence~\cite{Nerode58}.
It is well-known that every regular language has a unique minimum DFA defining it (up to isomorphism), see \eg~\cite{HopcroftU79}; the same is not true for NFAs~\cite{JiangR93}.

There are languages for which state complexity and index coincide.

\begin{example}%
\label{Example:StateComplexityEqualsIndex}%
    For each $k \in \N$, the language $\set{ a^k } \subseteq a^*$ has state complexity and index $k+1$.
    Obviously, a DFA with states $q_0, q_1, \ldots, q_k$ and transitions $q_{i} \tow{a} q_{i+1}$ for all $i$ accepts the language with the desired number of states.
    Any smaller automaton would necessarily see a state repetition while processing $a^k$.
    This loop could be repeated to accept a word $a^{k+\ell}$ for $\ell > 0$, a contradiction to the assumption that $a^k$ is the only word in the language.

    The same is true for the languages
    \begin{align*}
        \calL_{\leq k} &= \Set{ a^\ell }{\ell \leq k}
        \ ,
        &
        \calL_{\geq k} &= \Set{ a^\ell }{\ell \geq k}
        \ .
    \end{align*}
\end{example}


In general, the index is at most exponentially higher than the state complexity.
The \emph{Rabin-Scott powerset construction}~\cite{RabinS59} turns an NFA with states $Q$ into a language-equivalent DFA with states $Q^{\det} \subseteq \powerset{Q}$, and $\card{\powerset{Q}} = 2^{\card{Q}}$.
This is in fact also used when proving that regular languages are closed under complement: Given an NFA for a language, we transform it into a DFA, which can be complemented by making the final states non-final and vice versa.
In particular, a regular language and its complement always have the same index.
This method of complementation is not sound for NFAs, and we will see in \cref{Section:UCBPP} a language for which the state complexity of its complement is exponentially larger than its state complexity.

In the following, we consider an example for a family of languages in which the worst-case-behavior of an exponential gap between state complexity and index occurs.
% As we have seen above, the gap between state complexity and index does not need to occur.
% The following example presents a c but there are examples in which it happens.

\begin{example}%
\label{Example:StateComplexityVsIndex}%
    For each $k$, the language $\Set{ w \in \B^{\geq k}}{\text{the $k$-last letter of $w$ is $0$}}$ has state complexity $k+1$ but index $2^k$.
    It is not hard to construct automata with the desired number of states:
    An NFA can guess when the $k$-last letter occurs, verify that it is $0$ and then verify that it was indeed the $k$-last letter.
    A DFA can store the $k$-last letters so that when the end of the word occurs, it can check that the $k$-last letter is $0$ as desired.
    Both constructions are optimal.

    The example is mentioned without proof in~\cite{Kozen97}.
    We give a proof in \cref{Section:PNSeparabilityLowerBound} when we use the example to establish a lower bound on the descriptional complexity of separators.
\end{example}

\paragraph{Descriptional complexity for families of languages}

As in the Examples~\ref{Example:StateComplexityEqualsIndex} and~\ref{Example:StateComplexityVsIndex}, we are usually not interested in the state complexity of a single language, but rather in the state complexity of a family of languages.
In the case where we have a family of languages ${(\calL_k)}_{k \in \N}$ consisting of one language per natural number, the meaning of a statement like \enquote{the state complexity is exponential} is obvious.
It could be formalized using the notation that we have introduced in \cref{Section:Complexity}: The state complexity of a family is exponential if there is a constant $\ell$ such that there is a function contained in $2^{\bigO{n^\ell}}$ that maps each number $n$ to an upper bound for the state complexity of $\calL_n$.
When we say that the state complexity is exponential and that this is optimal, we mean that there is no such function that can be bounded by a polynomial.

Sometimes, we will consider families of languages that are not indexed by natural numbers, but by more complex objects.
For example, when we make a statement like \enquote{the languages of NFAs have polynomial state complexity}, we are considering the family of language $\Set{ \lang{A} }{ A \text{ NFA}}$.
In this family of languages, each language $\lang{A}$ is indexed by the NFA $A$ that generates it.
By the above statement, we mean that if we measure the size $\card{A}$ as defined previously, we obtain that the state complexity of each language $\lang{A}$ is polynomial in $\card{A}$.
Luckily, we will only need to consider examples where the index set and the size assignment is clear from the context.

The \emph{index} of a family of languages -- not to be confused with the indexing of languages mentioned above -- can be defined similarly to the state complexity.

\paragraph{$\eps$-Transitions}

We also consider NFAs with internal transitions, \ie NFAs where we allow labels from $\Sigma_\eps = \Sigma \dotcup \set{\eps}$.
Intuitively, transitions labeled by $\eps$ can be taken without generating a letter.
Sometimes it is convenient to allow such transitions to be able to give easier and more intuitive constructions.

It is well-known that given an automaton with $\eps$-transitions, it is possible to compute an NFA without such transitions that has the same set of states.
Hence, we may use NFAs and NFAs with internal transitions interchangeably when talking about the state complexity of languages.

\paragraph{Multiple initial states}

Similar to the case of $\eps$-transitions, constructions may be simpler if we allow multiple initial states.
Such an NFA can be converted to an NFA with a unique initial state as follows: We insert a fresh initial state, and $\eps$-labeled transitions from the fresh state to all former initial states.
As mentioned above, this NFA with $\eps$-transitions can be converted to a normal NFA.\@
Hence, the existence of an NFA with multiple initial states and $k$ control states for a language proves that the language has state complexity at most $k+1$.

In this thesis, we will be exclusively interested in whether the state complexity of a family of languages is polynomial, exponential, and so on.
These classes of functions are closed under the operation of adding $1$, so we may freely use multiple initial states.

Note that both constructions -- $\eps$-transition and multiple initial states -- can be seen as a form of nondeterminism.
For example, the language $\calL_k$ from \cref{Example:StateComplexityVsIndex} can easily be expressed using automata with $k+1$ states that are deterministic but for the existence of $\eps$-transitions or multiple initial states.
Hence, the state complexity is only preserved under these constructions in the case of NFAs.

\end{document}
