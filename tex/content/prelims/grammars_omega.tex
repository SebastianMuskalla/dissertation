\documentclass[../../diss.tex]{subfiles}
\begin{document}

\section{\texorpdfstring{$\omega$}{Omega}-languages of context-free grammars}%
\label{Section:CFGOmega}%

We introduce $\omega$-context-free languages, $\omega$-languages defined by context-free grammars.
In the literature, $\omega$-context-free languages are an established class of languages that can be defined in several equivalent ways, which we will discuss later.
For now, we present a novel definition of this class of languages that we feel is more elegant.
This also means that in contrast to the rest of this chapter, this section contains a contribution from the author of this thesis.
This research is taken from the publication~\cite{MeyerMN17a}, as detailed in \cref{Chapter:Contributions}.

Consider an infinite left-derivation process
\[
    S = \beta_0 \leftderive \beta_1 \leftderive \ldots
\]
for some CFG $G = (N, P, S)$.
As discussed in the previous section, it induces a sequence of terminal words
\(
    \tprefix{\beta_0}, \tprefix{\beta_1}, \ldots
\)
such that for each $i,k \in \N$, $\tprefix{\beta_i}$ is a prefix of $\tprefix{\beta_{i+k}}$.
Intuitively, we want to associate to such a derivation process the infinite word that is the limit of these prefixes.
If the length of the prefixes grows unboundedly, we may define the infinite word
\(
    \lim \tprefix{\beta_i}_{i \in \N} \in \Sigma^\omega
\)
whose $\nth{j}$ letter is the $\nth{j}$ letter of the $\nth{i_j}$ prefix, where $i_j \in \N$ is an arbitrary index so that the corresponding prefix is sufficiently long.
Formally, we define ${(\lim \tprefix{\beta_i}_{i \in \N})}_j = \tprefix{\beta_{i_j}}_j$, where $i_j \in \N$ is chosen so that $\card{ \tprefix{\beta_{i_j}} } \geq j$.
The above observation guarantees that the definition is independent of the choice of each $i_j$.

With this definition at hand, it is possible to associate to a context-free grammar $G$ a \nb{language $\calL' = \Set{ \lim \tprefix{\beta_i}_{i \in \N} \in \Sigma^\omega }{ {(\beta_i)}_{i \in \N} \text{ is an infinite left-derivation process}}$} of all infinite words that occur as the prefix limits along infinite left-derivation processes.

For example, the $\omega$-regular language ${(a \cup b)}^* b^\omega$ of words that contain only finitely many $a$s occurs as the language of the grammar with the rules $S \to XY, X \to Xa \mid Xb \mid \eps, Y \to bY$.
In a left-derivation process, we can first use the rules for $X$ to derive a prefix from $\set{a,b}^*$ of arbitrary, but finite length, and then append $b^\omega$ by using $Y \to bY$ infinitely often.
We cannot obtain infinitely many $a$s, because to do so, we would need to preserve nonterminal $X$ for infinitely many steps.
Such a derivation process consists entirely of sentential forms that start with \nb{nonterminal $X$}, so their prefixes are $\eps$ and the prefix limit is not an infinite word.

However, the definition is too weak to express some languages that certainly should be $\omega$-context-free according to any reasonable definition.
For example, there can be no context-free grammar whose associated language is $\Set{a^n b^n}{n \in \N}^\omega$, which is the $\omega$-iteration of a context-free language.
To sketch the proof of this fact, we observe that any candidate grammar needs to contain a nonterminal $X$ that can be reached from the initial symbol and a rule $X \to aXb$ (or a set of rules that after a finite number of replacement steps lead to such a pattern).
Such a rule is needed to be able to create an unbounded but equal amount of $a$s and $b$s on each side: It is well-known that with rules that are exclusively left- or right-linear (rules that only contain one nonterminal that is the leftmost or rightmost symbol), it is impossible to generate non-regular languages.
By infinitely often applying this rule, we obtain a left-derivation process with prefix limit $a^\omega$, a word which is not in the desired language.

We present a definition that solves this problem.
Let us call an infinite left-derivation process $S = \beta_0 \leftderive \beta_1 \leftderive \ldots$ \emph{right-infinite} if it contains infinitely many sentential forms of the \nb{shape $wX$} where $w \in \Sigma^*$ is a terminal prefix and $X$ is a single nonterminal.
In such a derivation process, the rightmost nonterminal is replaced infinitely often.
Additionally, every other nonterminal is replaced by a terminal word within finitely many derivation steps.

The name right-infinite comes from the shape of the derivation tree associated to this type of derivation process.
A derivation tree (associated to a derivation process) is a tree in which the nodes are labeled by symbol from $N \cup \Sigma$.
It is constructed inductively as follows:
We start with a tree only consisting of a root node labeled by $S$.
If production rule $X \to \eta$ is used in the derivation process, we consider the node of the derivation tree that corresponds to the occurrence of~$X$ that should be replaced and insert children $\eta_1, \ldots, \eta_{\card{\eta}}$.
The limit of this process is a derivation tree in which all inner nodes are labeled by nonterminals, while the leaves are terminals.
The yield of the tree (the leaves read from left to right) form the terminal word that is derived by the derivation process.
The derivation tree associated to a right-infinite left-derivation process is infinite only in its rightmost branch.
Any node in the tree that is not this branch can be reached within finitely many steps.
%  $X$ is (the label of) an inner node and $Y$ is a child of this node that is not the rightmost child, then any descendant, \ie an indirect successor, of $Y$ can be reached from $X$ in finitely many steps.

The proper definition of the omega-language of a context-free grammar associates to a grammar all infinite prefix limits that occur along right-infinite left-derivation processes.

\begin{definition}
    A CFG $G$ defines the $\omega$-language
    \[
        \olang{G} =
        \Set{ \lim \tprefix{\beta_i}_{i \in \N} \in \Sigma^\omega }{ {(\beta_i)}_{i \in \N} \text{ is a right-infinite left-derivation process}}
        \ .
    \]
\end{definition}

In a sense, this definition imposes a Büchi acceptance condition.
We may consider the \nb{LTS $(\sententials,T,S,\Sigma^*N)$} with sentential forms as configurations and transitions induced by left-derivations with their prefix growth as label.
The final configurations are configurations of the \nb{shape $wX \in \Sigma^*N$}, \ie configurations in which the unique nonterminal is the rightmost symbol.
The language of the LTS with a Büchi acceptance condition (\ie requiring that computations contain infinitely many final configurations) is exactly $\olang{G}$ as defined above.

We will later show in \cref{Example:OmegaCFGAnBnOmega} and \cref{Remark:OmegaCFGWeakerDefinition} that this way of defining the $\omega$-languages of CFGs is strictly more expressive than the version without the restriction to right-infinite processes.
On the one hand, $\Set{a^n b^n}{n \in \N}^\omega$ is the $\omega$-language of a suitable CFG.\@
On the other hand, we can recover the weaker definition that does not require right-infinity by adding rules to the grammar.

\paragraph{Characterizing $\omega$-context-free languages}

Let us call a language of infinite words \emph{$\omega$-context-free} if it occurs as $\olang{G}$ of some CFG $G$.
The class of these languages admits the following characterization.

\begin{theorem}%
\label{Theorem:CFGOmegaLangCharacterization}%
    An $\omega$-language $\calL \subseteq \Sigma^\omega$ is $\omega$-context-free iff it can be written as
    \[
        \calL = \bigcup_{i \in \oneto{n}} U_i . V_i^\omega
    \]
    where $n \in \N$ and for each $i \in \oneto{n}$, $U_i, V_i \subseteq \Sigma^*$ are context-free languages.
\end{theorem}

In words, a language is $\omega$-context-free iff it is the finite union of languages that consist of a finite context-free prefix, and a context-free period that is repeated \emph{ad infinitum}.
Note that this is analogous to the characterization of $\omega$-regular languages given in \cref{LTS:OmegaAutomata}.
Once proven, the characterization will in particular show that $\Set{a^n b^n}{n \in \N}^\omega$ is an $\omega$-context-free~language.

Before proving the result, we discuss its implications.
In particular, we use the result to show that our definition of $\omega$-context-free languages coincides with various~definitions from the literature.

\paragraph{Related work}

Other works in the past have considered various definitions of $\omega$-context-free languages.

\citea{CohenG77} have studied how $\omega$-context-free language can be defined by (1) pushdown automata with a Büchi acceptance condition on the states, (2) pushdown automata with the condition that the empty stack is reached infinitely often, and (3) context-free grammars that proceed by left-derivations and are equipped with a so-called repetition set.
The repetition set specifies which productions need to be used infinitely often (or, equivalently, which nonterminals need to be replaced infinitely often) for a derivation process to be accepting.
Hence, it also imposes a Büchi condition.
The authors prove that all three definitions are equivalent, since all three are equal to what they call the Kleene closure of the context-free languages of finite words.
The Kleene closure is exactly the type of languages as in \cref{Theorem:CFGOmegaLangCharacterization}.
Hence, with the theorem at hand, our definition of $\omega$-context-free languages is also equivalent to the three methods.
Arguably, our method is more elegant as the Büchi condition is implicit instead of being explicitly specified in the form of a repetition set.

In the second part of the paper, the authors show that if one allows arbitrary derivations (instead of exclusively considering left-derivations), one obtains a weaker class of languages, even under the presence of a Büchi condition in the form of a repetition set.
For example, the \nb{language $\Set{a^n b^n}{n \in \N}^\omega$} cannot be obtained by such a grammar.

Independently, \citea{Linna76} has given a definition of $\omega$-context-free languages defined by pushdown automata with a Büchi acceptance condition.
He shows that this class is equal to the languages obtained from the $\omega$-regular languages by applying context-free substitutions.
Given the characterization of $\omega$-regular languages, this proves a result analogous to \cref{Theorem:CFGOmegaLangCharacterization} and shows that his definition coincides with our definition.

He also considers the class of languages defined as limits of context-free languages.
The limit of a context-free language is the set of infinite words such that all their finite prefixes are contained in the context-free language.
This class of languages is incomparable to the $\omega$-context-free languages.
On the one hand, it does not contain the $\omega$-regular language ${(a \cup b)}^* b^\omega$.
On the other hand, the limit of the context-free language $\Set{ w.b.a^k}{k \text{ equals the number of $b$s in $w$}}$ is not $\omega$-context-free: It is a language without so-called ultimately periodic words.
An \emph{ultimately periodic word} is an infinite word of the shape $w.v^\omega$ for suitable finite words $w, v \in \Sigma^*$.
From our characterization result, \cref{Theorem:CFGOmegaLangCharacterization}, we immediately obtain that every $\omega$-context-free language is guaranteed to contain at least one such word.

Linna also shows that by closing the class of limits of context-free languages under homomorphisms (or, equivalently, context-free substitutions), one obtains a superclass of both that class and the class of $\omega$-context-free languages.
The algorithms for languages of infinite words that we will consider later, \eg in \cref{Section:EDSOmegaRegIncl}, rely on the presence of ultimately periodic words.
Hence, they would not work for the latter two classes of languages defined by Linna.

\paragraph{Proof of \cref{Theorem:CFGOmegaLangCharacterization}}

The rest of this section is dedicated to the proof of \cref{Theorem:CFGOmegaLangCharacterization}.
One direction is easy, as we can give a straightforward construction.

\begin{lemma}
    If \( \calL = \bigcup_{i \in \oneto{n}} U_i . V_i^\omega \) for $n \in \N$ and context-free $U_i, V_i$, then $\calL$ is $\omega$-context-free.
\end{lemma}

\begin{proof}
    Assume that for each $i$, $U_i = \lang{N_i,\Sigma,P_i,S_i}$ and similarly $V_i = \lang{N_i',\Sigma,P_i',S_i'}$.
    We assume \wolog that all $N_i$ and $N_i'$ (and hence also all $P_i$ and $P_i'$) are pairwise disjoint, and we assume that none of these sets contain $S$ and $X_i$ for $i \in \oneto{n}$.
    We construct a new grammar $G = (N,\Sigma,P,S)$ with
    \(
        N = \bigcup_{i} N_i \cup \bigcup_{i} N_i \dotcup \set{ S } \dotcup \Set{ X_i}{i \in \oneto{n}}
    \)
    and
    \(
        P = \bigcup_{i} P_i \cup \bigcup_{i} P_i' \dotcup \Set{ S \to S_i X_i, X_i \to S_i' X_i}{i \in \oneto{n}}
    \).

    Grammar $G$ intuitively works as follows:
    In the first step, one chooses some part $U_i . V_i^\omega$ of the union by picking the corresponding rule $S \to S_i X_i$.
    From $S_i$, a word in $U_i$ can be derived (recall that $S_i$ is the initial symbol of the corresponding grammar).
    This process has to be finite, since $S_i$ is not the rightmost nonterminal.
    Afterwards, an infinite process starts, in each step of which $X_i$ is replaced using $X_i \to S_i' X_i$.
    After one such step, the occurrence of $S_i'$ can then be used to derive a finite word from $V_i$.
    Hence, the language $\olang{G}$ is indeed \(\bigcup_{i \in \oneto{n}} U_i . V_i^\omega \).
\end{proof}

\begin{example}%
\label{Example:OmegaCFGAnBnOmega}%
    To obtain a grammar for $\Set{a^n b^n}{n \in \N}^\omega$, we may apply a simplified version of the construction from the above proof, obtaining $G = (\set{S,X},\set{a,b},\set{ S \to XS, X \to aXb \mid \eps},S)$.
    This grammar indeed generates the desired language:
    Intuitively, it produces an infinite number of occurrences of $S$, each of which is then replaced by a finite word from $\Set{a^n b^n}{n \in \N}$.
\end{example}

\begin{remark}%
\label{Remark:OmegaCFGWeakerDefinition}%
    We briefly discuss how to recover the weaker definition of the $\omega$-language of a CFG in which we do not restrict ourselves to right-infinite derivation processes.
    Observe that because a derivation process is linear, at most one branch of the derivation tree can be infinite.
    Since we only consider left-derivations, any node in the tree that is to the right of an infinite branch will not be replaced:
    The infinite branch will always contain a nonterminal that is more to the left and has to be replaced first.
    Similarly, terminal symbols that are to the right of an infinite branch will not contribute to the word generated by the derivation process.
    Since the infinite branch will always contain some nonterminal that occurs earlier, these terminals are not contained in any prefix.

    This observation yields a method to turn a grammar $G$ into a grammar $G'$ so that $\olang{G'}$ is equal to $\Set{ \lim \tprefix{\beta_i}_{i \in \N} \in \Sigma^\omega }{ {(\beta_i)}_{i \in \N} \text{ is a right-infinite left-derivation process of } G}$.
    The idea is to allow any branch of the derivation tree to be infinite by making it the rightmost branch.
    To this end, we simply drop the part of the derivation tree that is to the right of this infinite branch.
    As argued before, dropping this part will not influence the words that can be~produced.

    To implement this idea, we let $G'$ consist of the nonterminals of $G$ as well as a fresh version $X'$ for every nonterminal $X$.
    Intuitively, we will make sure that the infinite branch of the derivation tree will consist of nonterminals of the shape $X'$.
    The initial symbol is $S'$, the copy of $S$.
    The old nonterminals retain their rules.
    For each rule $X \to \eta$ of $G$ and each prefix $\beta.Y$ of $\eta$ that ends in a nonterminal, we add a rule $X' \to \beta.Y'$ for $X'$.
    Note that we use the old versions of the nonterminals in~$\beta$, but we replace $Y$ by its copy $Y'$.

    All sentential forms that can be reached from $S'$ contain exactly one nonterminal of the \nb{shape $X'$}, which is the rightmost symbol.
    Whenever we replace such a terminal, we intuitively pick a \nb{rule $X \to \eta$} of $G$, designate a nonterminal $Y$ in $\eta$ to be on the infinite branch of the derivation tree and drop the rest of $\eta$.

    We apply this construction to the grammar from \cref{Example:OmegaCFGAnBnOmega}.
    We obtain the grammar $G' = (\set{S',X',X},\set{a,b},\set{ S' \to XS' \mid X' , X \to aXb \mid \eps, X' \to aX'},S')$.
    The rules $S' \to XS'$ and $S \to X'$ result from the rule $S \to XS$ of $G$ by choosing $X$ \resp $S$ as the nonterminal on the infinite branch.
    Similarly, $X' \to aX'$ results from $X \to aXb$.
    Note that we have omitted the \nb{nonterminal $S$} because it does not occur in any sentential form that is reachable from~$S'$.

    A right-infinite left-derivation process of this grammar either uses the rule $S' \to XS'$ infinite often.
    In this case, it behaves as the derivation processes of $G$ from \cref{Example:OmegaCFGAnBnOmega} and generates a word of the shape ${(a^{n_i}b^{n_i})}^\omega$.
    Alternatively, the derivation process uses the \nb{rule $S' \to X'$} at some point, followed by an infinite sequence of applications of the rule $X' \to aX'$.
    Hence, the word we obtain is a member of the language $\Set{a^n b^n}{n \in \N}^*.a^\omega$.
    Altogether, $\olang{G} = \Set{a^n b^n}{n \in \N}^\omega \cup \Set{a^n b^n}{n \in \N}^*.a^\omega$.
\end{remark}


The other direction of the proof requires more work.
The first step is to present a different view of the language $\olang{G}$ that separates the infinite rightmost branch of the derivation tree from the other branches.
We start by noting that if $w.X$ is a sentential form in a right-infinite left-derivation process, then the next step will not apply a production rule whose rightmost symbol is a terminal:
Otherwise, we would end up in a sentential form with a terminal as the rightmost symbol, and it becomes impossible to reach another sentential form that ends in \nb{a nonterminal}.

We use this observation to define the \emph{spinal graph}, a finite LTS with nonterminals as configurations that is labeled by sentential forms.
Formally we define $\SpinalGraph = (N,T,S)$ where $X \tow{\beta} Y$ if grammar $G$ contains a production rule $X \to \beta.Y$.
Since the number of nonterminals and productions in the grammar is finite, so is $\SpinalGraph$.
Note that we have not equipped $\SpinalGraph$ with an acceptance condition.
We define $\olang{\SpinalGraph} \subseteq \sententials^\omega$ to be all sequences of labels that occur along infinite paths in $\SpinalGraph$ starting in $S$.
We may flatten $\sententials^\omega = {({(\Sigma \cup N)}^*)}^\omega$ to see such a sequence as an element of~${(\Sigma \cup N)}^\omega$, and hence $\olang{\SpinalGraph}$ as an $\omega$-language over $\Sigma \cup N$.
A problem arises in the case that in an infinite path $\eps$ is the only label occurring infinitely often.
In this case, the resulting sequence over $\Sigma \cup N$ is finite.
We will take care of this problem later by excluding such paths.

It remains to relate the language of $\SpinalGraph$ to $\olang{G}$.
To this end, we define $\calL_G\paren{\eps} = \set{ \eps }$, $\calL_G\paren{a} = a$ for terminals $a \in \Sigma$ and $\calL_G\paren{X} = \lang{N,P,X}$, \ie the context-free language of finite words obtained by seeing $X$ as the initial symbol of grammar $G$.
For finite and infinite sequences \nb{over $N \cup T$}, we define their language to be $\calL_G\paren{\beta_1\beta_2 \ldots} = \calL_G\paren{\beta_1}.\calL_G\paren{\beta_2}\ldots$, the (finite or infinite) concatenation of the respective languages.
For sets of such sequences, we define the language by applying the operator $\calL_G$ element-wise and then taking the union.

\begin{lemma}%
\label{Lemma:CFGOmegaLangSpinalGraph}%
    $\olang{G} = \calL_G\paren{\olang{\SpinalGraph}} \cap \Sigma^\omega$.
\end{lemma}

Note that the intersection with $\Sigma^\omega$ removes paths from $\olang{\SpinalGraph}$ in which $\eps$ is the only label occurring infinitely often.

\begin{proof}
    Consider a word $w$ in $\olang{G}$ and a right-infinite left-derivation process for that word.
    Extracting the infinite sequence of derivation steps that replace nonterminal yields a path $\pi$ in $\SpinalGraph$.
    The infinite sequence of labels over ${(N \cup \Sigma)}^*$ along $\pi$ can be flattened to obtain an infinite \nb{word $\beta \in \olang{\SpinalGraph}$} over $N \cup \Sigma$.
    The rest of the derivation steps in the right-infinite left-derivation process for $w$ can be used to replace each nonterminal $X$ in $\beta$ by a finite word in $\calL_G\paren{X}$.
    The concatenation of these finite words with the nonterminals that may occur in $\beta$ yields the infinite word $w$.
    Altogether, we obtain that $w$ is an infinite word in $\calL_G\paren{\beta} \subseteq \calL_G\paren{\olang{\SpinalGraph}}$ as desired.

    For the other direction, consider an infinite word $w$ in $\calL_G\paren{\olang{\SpinalGraph}}$.
    By definition, it is an element of $\calL_G\paren{\beta}$ for some $\beta \in \olang{\SpinalGraph}$, where $\beta$ corresponds to an infinite path $\pi$ in $\SpinalGraph$.
    We construct a right-infinite left-derivation process for $w$ as follows.
    We start from the initial symbol of the grammar.
    Whenever we have to replace the rightmost nonterminal in the current sentential form, say $X$, we consider the earliest transition in the path $\pi$ that has not been processed yet.
    It is of the shape $X \tow{\itr{\beta}{i}} Y$, corresponding to a rule $X \to \itr{\beta}{i}Y$ of the grammar.
    We apply this rule, producing a finite infix $\itr{\beta}{i}$ of the infinite sequence $\beta$ and a new rightmost nonterminal.
    In order to replace the nonterminals in $\itr{\beta}{i}$, we use $w \in \calL_G\paren{\beta}$ to obtain a finite sequence of left-derivations that replace such a nonterminal by an infix of $w$.
    For each such nonterminal, we use the corresponding sequence of left-derivations until $Y$ is the only nonterminal remaining and we proceed by processing the next transition from $\pi$.
    The infinite word produced by this derivation process is $w$, as it is the concatenation of the terminals in $\beta$ and the finite infixes of~$w$ derived from the nonterminals.
    Hence, $w \in \olang{G}$.
\end{proof}

Finally, we can finish the proof of \cref{Theorem:CFGOmegaLangCharacterization}
by showing the missing implication.

\begin{proposition}
    Each $\omega$-context-free language $\calL \subseteq \Sigma^\omega$ can be written as
    \[
        \calL = \bigcup_{i \in \oneto{n}} U_i . V_i^\omega
    \]
    for some $n \in \N$ with each $U_i, V_i \subseteq \Sigma^*$ context-free.
\end{proposition}

\begin{proof}
    Consider an $\omega$-context-free language $\olang{G}$ for some CFG $G$.
    We may construct the associated spinal graph $\SpinalGraph$.
    For nonterminals $X,Y$, we define $\calL_{X,Y}$ to be set of sequences over $N \cup \Sigma$ that occur in $\SpinalGraph$ as the labels along finite paths from node $X$ to $Y$.

    We claim that
    \[
        \olang{G} = \bigcup_{X \in N} \calL_G\paren{\calL_{S,X}} {(\calL_G\paren{\calL_{X,X}})}^\omega
        \ .
    \]

    Firstly, we argue that the expression on the right-hand side is of the required shape.
    Because there are only finitely many nonterminals, the union is finite.
    Each language $\calL_{X,Y}$ is a regular language over $N \cup \Sigma$.
    To this end, observe that we may see the spinal graph as a finite automaton (after inserting suitable intermediary states to make sure that each transition generates at most one letter) with $X$ as the initial and $Y$ as the unique final state.
    Finally, observe that $\calL_G\paren{\calL_{X,Y}}$ is hence context-free.
    The regular languages are a subclass of the context-free ones, so there is a CFG over $N \cup \Sigma$ for $\calL_{X,Y}$.
    By adding the rules for the nonterminals from $G$ to this grammar, we obtain a CFG over $\Sigma$ that generates
    $\calL_G\paren{\calL_{X,Y}}$.
    In particular, $\calL_G\paren{\calL_{S,X}}$ and $\calL_G\paren{\calL_{X,X}}$ are context-free languages and the expression is as required.

    Secondly, we argue that $\olang{G} \subseteq \bigcup_{X \in N} \calL_G\paren{\calL_{S,X}} {(\calL_G\paren{\calL_{X,X}})}^\omega$.
    By \cref{Lemma:CFGOmegaLangSpinalGraph}, $\olang{G} = \calL_G\paren{\olang{\SpinalGraph}} \cap \Sigma^\omega$.
    Consider a word $w \in \olang{G}$.
    It is a member of $\calL_G\paren{\beta}$ for some $\beta \in \olang{\SpinalGraph}$, \ie $\beta$ is the sequence of labels along an infinite path $\pi$ in $\SpinalGraph$ that starts \nb{in $S$}.
    Because the spinal graph has only finite many nodes, there is some nonterminal $X$ that is visited infinitely often by $\pi$.
    By summarizing transitions in the graph, we may write $\pi$ as
    \[
        S \tow{\eta} X \tow{\gamma_1} X \tow{\gamma_2} X \tow{\gamma_3} \ldots \ ,
    \]
    where $\eta$ and the $\gamma_i$ are words over $N \cup \Sigma$ so that $\beta = \eta.\gamma_1.\gamma_2\ldots$
    By definition, $\eta \in \calL_{S,X}$ and \nb{each $\gamma_i \in \calL_{X,X}$} for all $i$.
    Hence, $\beta \in  \calL_{S,X}.\paren{\calL_{X,X}}^\omega$ and $w \in \calL_G\paren{\beta}$ is an element of~$\calL_G\paren{\calL_{S,X}}.\paren{\calL_G\paren{\calL_{X,X}}}^\omega$ as desired.

    Finally, we consider the other direction.
    Let $w \in \calL_G\paren{\calL_{S,X}}.\paren{\calL_G\paren{\calL_{X,X}}}^\omega$ for some $X \in N$.
    Note that $w$ is an infinite word because the omega-iteration of a language is defined to only produce words in $\Sigma^\omega$.
    By definition, we may write $w = \itr{w}{0}.\itr{w}{1}.\itr{w}{2}\ldots$ with $\itr{w}{0} \in \calL_G\paren{\calL_{S,X}}$ and $\itr{w}{i} \in \calL_G\paren{\calL_{X,X}}$ for all $i > 0$.
    This in turns means there is $\eta \in \calL_{S,X}$ with $\itr{w}{0} \in \calL_G\paren{\eta}$ and $\gamma_i \in \calL_{X,X}$ with $\itr{w}{i} \in \calL_G\paren{\gamma_i}$ for all $i > 0$.
    By the definition of the languages $\calL_{S,X}$ and $\calL_{X,X}$, $\beta = \eta.\gamma_1.\gamma_2\ldots$ is a sequence of labels along an infinite path in the spinal graph starting in $S$, hence $\beta \in \olang{\SpinalGraph}$.
    We have that $w \in \Sigma^\omega$ is an infinite word in $\calL_G(\beta)$ with $\beta \in \olang{\SpinalGraph}$.
    By \cref{Lemma:CFGOmegaLangSpinalGraph}, this is sufficient so show $w \in \olang{G}$, which completes the proof.
\end{proof}


\end{document}
