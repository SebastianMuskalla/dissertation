\documentclass[../../diss.tex]{subfiles}
\begin{document}

\chapter*{Part IV.\newline Separability}

In \cref{Section:IntroSeparability} of the introduction, we have explained that the (regular) separability problem is related to the compositional verification of concurrent systems.
This part of the thesis is concerned with studying regular separability for well-structured transition \nb{systems (WSTSes)}.

\paragraph{Outline}

We start by providing the basic definitions regarding separability and regular separability in \cref{Chapter:Separability}.
We also discuss related work from the literature and summarize the results that we will show in this part of the thesis.

Before actually considering the separability problem for the class of languages of WSTSes, we study the relations among various subclasses thereof in \cref{Chapter:WSTSExpressiveness}.
These results widen the applicability of our main result, which we will present in \cref{Chapter:WSTSSeparability}.
We will prove that under mild preconditions, any two disjoint WSTS languages are regularly separable.
We first prove a technical core result that relates a certain type of invariant of the state space of a WSTS to the existence of a regular separator.
Then, we use the ideal completion of a WSTS to prove that such an invariant always has to exist.

In \cref{Chapter:WSTSSeparatorSize}, we demonstrate our result on regular separability by applying it to Petri net coverability languages, a subclass of the class of WSTS languages.
We explicitly provide the construction of a regular separator, which yields a triply exponential state complexity.
Furthermore, we prove a doubly exponential lower bound for that space complexity.

\paragraph{Publication}

The content we present in this section is mostly taken from the publication~\cite{CzerwinskiLMMKS18} (\resp its full version~\cite{CzerwinskiLMMKS18a}).
We will discuss the author's contributions to this paper in \cref{Chapter:Contributions}.

\end{document}
