\documentclass[../../diss.tex]{subfiles}
\begin{document}

\section{The results and their consequences}%
\label{Section:SeparabilityAndConsequences}%

The main result regarding upward-compatible WSTSes that we will prove in this chapter is the following.

\begin{theorem}%
\label{Theorem:WSTSSeparability}%
    Two disjoint WSTS languages, at least one of them the language of a deterministic WSTS, are regularly separable.
\end{theorem}

In \cref{Separability:Results}, we have promised proving that under mild assumptions, any two disjoint WSTS languages are regularly separable.
Requiring one of the two WSTSes to be deterministic seems to be an assumption that is quite restricting.
However, we may use the expressiveness result from \cref{Chapter:WSTSExpressiveness} to see that it is in fact enough to require one of the two WSTS to be finitely branching or $\omega^2$.
In particular, recall that virtually all WSTSes that are of practical interest satisfy the $\omega^2$-condition.
Hence, the following generalization of the result applies to almost all WSTSes.

\begin{corollary}
    Two disjoint WSTS languages, at least one of them the language of a finitely branching WSTS or of an $\omega^2$-WSTS, are regularly separable.
\end{corollary}

Once we have proven \cref{Theorem:WSTSSeparability}, the corollary follows immediately using \cref{Theorem:WSTSExpressiveness}.

One should emphasize that we simply require both languages to be WSTS languages.
We do not require them to come from the same class of WSTSes.
For example, one could be the language of a Petri net, the other could be the language of a lossy channel system.

It is also noteworthy that our result means that regular separability as a decision problem is essentially equivalent to intersection-emptiness.
Luckily, intersection-emptiness can be decided under mild-assumption.
To be precise, it has been shown that the coverability problem for unlabeled WSTSes, the problem of checking whether a final configuration is reachable from an initial one, is decidable under mild assumptions~\cite{AbdullaJ1993,GeeraertsRV06}.
In the world of labeled WSTSes, this corresponds to checking language-emptiness.
Checking intersection-emptiness amounts to checking language-emptiness for the product system.
Luckily, if both given WSTSes satisfy the requirements for the decidability of coverability, then so does the product system, meaning we can decide intersection-emptiness.

Our result is constructive for most WSTSes.
We will construct a finite automaton whose language is the required regular separator from the so-called ideal decomposition of an invariant.
Such a set is effectively computable in many cases, as demonstrated \eg in~\cite{Finkel16,LazicS15}.

Furthermore, our result has some interesting language-theoretic consequences.
In \cref{Sectiopn:SeparabilityRelatedWork}, we have mentioned that separability is related to the regularity problem.
The following corollary to our result makes this relationship precise.

\begin{corollary}%
\label{Corollary:WSTSNecessarilyRegular}%
    If a language is a deterministic WSTS language and its complement is a WSTS language (or vice versa), then both are necessarily regular.
\end{corollary}

\begin{proof}
    Assume that $\lang{\wsts}$ is the language of a finitely branching WSTS~$W$, and its complement $\overline{\lang{\wsts}} = \lang{\wstsprime}$ is the language of a WSTS.\@
    These languages are disjoint, so by \cref{Theorem:WSTSSeparability}, there is a regular separator $\calR$.
    From $\lang{\wsts} \subseteq \calR$ and $\overline{\lang{\wsts}} \cap \calR = \emptyset$, we conclude $\calR = \lang{\wsts}$.
    Hence, $\lang{\wsts}$ is regular and so is its complement.
\end{proof}

Again, we may apply \cref{Theorem:WSTSExpressiveness} to obtain the following generalization.

\begin{corollary}
    \label{Corollary:WSTSNecessarilyRegular2}%
    If a language is the language of a WSTS that is finitely branching or $\omega^2$, and its complement is a WSTS language, then both are necessarily regular.
\end{corollary}

It has been shown before that Petri net coverability languages have this property~\cite{MukundKRS98,MukundKRS98b}.
A more pointed phrasing of our result is the following weaker statement.

\begin{corollary}%
\label{Corollary:WSTSSeparabilityClosedness}%
    No subclass of the finitely branching or $\omega^2$-WSTS languages beyond $\REG$ is closed under \nb{complement}.
\end{corollary}

\begin{proof}
    Towards a contradiction, assume $\calF$ is such a class.
    Take any non-regular language $\calL$ from $\calF$.
    By assumption, $\overline{\calL}$ is also in $\calF$.
    Both are finitely branching (or $\omega^2$) WSTS languages, so \cref{Corollary:WSTSNecessarilyRegular2} yields that $\calL$ is regular, a contradiction.
\end{proof}

\paragraph{Results for DWSTS}

We will show similar results for downward-compatible WSTSes.

\begin{theorem}%
\label{Theorem:DWSTSSeparability}%
    Two disjoint DWSTS languages, at least one of them the language of a deterministic DWSTS, are regularly separable.
\end{theorem}

With \cref{Theorem:WSTSExpressiveness}, we can generalize the result and require one of the DWSTSes to be $\omega^2$ instead of requiring it to be deterministic.
Unlike in the case of upward-compatible WSTSes, it does not seem to be sufficient to require one of the two DWSTSes to be finitely branching.

\begin{corollary}
    Two disjoint DWSTS languages, at least one of them the language of a deterministic DWSTS or an $\omega^2$-DWSTS, are regularly separable.
\end{corollary}

We also obtain corollaries that are similar to the \cref{Corollary:WSTSSeparabilityClosedness,,Corollary:WSTSNecessarilyRegular,,Corollary:WSTSNecessarilyRegular2}.
The proofs of these corollaries are as in the case of (upward-compatible) WSTSes.

\begin{corollary}
    \begin{thmenumerate}[a)]
        \item If a language is the language of a DWSTS that is deterministic or $\omega^2$ and its complement is a DWSTS language (or vice versa), then both are necessarily regular.
        \item
        No subclass of the deterministic or $\omega^2$-DWSTS languages beyond $\REG$ is closed under \nb{complement}.
    \end{thmenumerate}
\end{corollary}

We leave the question of whether the assumptions that we make are necessary unanswered.
It would be desirable to prove that any two disjoint WSTS languages are regularly separable, without requiring one of the two generating WSTSes to be $\omega^2$ or finitely branching.
This problem is closely related to the unknown strictness of the inclusions in our result on WSTS expressiveness, \cref{Theorem:WSTSExpressiveness}, on which we have commented at the end of \cref{Section:WSTSExpressiveness}.
If one could extend \cref{Theorem:WSTSExpressiveness}, \eg by showing that any WSTS language is the language of a finitely branching WSTS, this also immediately extends our result on regular separability.
If it turns out that the inclusions among WSTS language classes are strict, whether the results on regular separability can be extended remains an open problem.
Although, one would hope that the proof of strictness allows us to gain new insights.
Let us emphasize that virtually all WSTSes of practical interest are known to be $\omega^2$, meaning that our results can be applied.

\end{document}
