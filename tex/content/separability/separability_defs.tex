\documentclass[../../diss.tex]{subfiles}
\begin{document}

\section{Basic definitions}

The separability problem can be seen as an extended version of the intersection-emptiness problem.
The intersection-emptiness problem asks whether two given languages are disjoint.
Usually, we fix the classes from which the languages should come beforehand and expect that the languages are given in the form of an effective representation, \eg we expect context-free languages to be given as context-free grammars.

\begin{problem}
    \problemtitle{Intersection-emptiness for $\calF,\calF'$}
    \probleminput{Language $\calL \subseteq \Sigma^*$ from $\calF$, Language $\calK \subseteq \Sigma^*$ from $\calF'$.}
    \problemquestion{Are $\calL$ and $\calK$ disjoint, \ie $\calL \cap \calK = \emptyset$?}
\end{problem}

For their usage in verification, two variants of this problem are considered to be particularly interesting.
The first is the case in which one of the classes is the class of regular languages and the other class is a class beyond the regular languages.
This models the verification problem in which we have a system, represented by the potentially non-regular language consisting of its possible executions, and a specification, represented by the language of executions that violate it.
The latter can be assumed to be regular because the expressive power of specification languages (\eg the logics LTL~\cite{Pnueli77} and S1S~\cite{Buechi62} on finite words) is typically at most equal to the regular languages.
The system violates the specification if it has an execution violating the specification, \ie the intersection of the languages is non-empty.
This variant of the intersection-emptiness problem is decidable for many classes of languages.
Assuming that the class that we are considering is effectively closed under regular intersection, we can modify the given system.
We obtain a system whose language is the intersection of the original language with the regular language.
Then, we invoke a decider for the emptiness problem for that system.
The assumptions that we have used here are met for example in the case of context-free languages and in the case of Petri net languages with both coverability and reachability as acceptance conditions.

In the second interesting variant of the intersection-emptiness problem, we consider two languages coming from the same class of languages $\calF$; we speak of the intersection-emptiness problem for $\calF$.
This problem can be used to model a setting in which we consider a concurrent system.
Let us for the sake of simplicity assume that the system only has two components.
We model each of the components by a language that contains all executions that (1) respect the behavior of the corresponding component, (2) are arbitrary with respect to the other component, and (3) violate the specification.
See \cref{Section:IntroSeparability} for a more in-depth explanation of the latter two aspects.
We obtain two languages whose intersection is non-empty if and only if there is an execution that respects the behavior of both components of the concurrent system but violates the specification.

This version of the intersection-emptiness problem is typically much more complicated than the intersection-emptiness problem where one of the classes is the class of regular languages.
As in the rest of this thesis, we are interested in an algorithm that does not only decide the problem, but also provides a certificate.
In the case of the two languages not being disjoint, a natural candidate for this certificate is a word in the intersection of the languages.
Consider for example the intersection-emptiness problem for context-free languages, which is well-known to be undecidable (see \eg \cite{HopcroftU79} for a proof).
If we manage to identify a word that is in the intersection, it is easy to verify the output by one membership query for each of the languages.

The problem of providing a certificate for the emptiness of the intersection of two languages is much more involved.
To this end, the notion of a separator can be used.
A \emph{separator} for two languages $\calL, \calK \subseteq \Sigma^*$ is a third language $\calR \subseteq \Sigma^*$ that fully contains $\calL$ and is disjoint from~$\calK$, \ie $\calL \subseteq \calR$, $\calK \cap \calR = \emptyset$.
Obviously, the existence of a separator is equivalent to intersection-emptiness:
If the languages are not disjoint, a separator cannot exist.
Otherwise, $\calL$ is a separator.
Hence, it only makes sense to study separators if we restrict them to come from a simpler class -- one with less expressive power -- than $\calL$ and $\calK$ do.
We call a separator that comes from language class $\calS$ an \emph{$\calS$-separator}, and we say that $\calL$ and $\calK$ are \emph{$\calS$-separable} if an $\calS$-separator exists.
The decision problem of checking the existence of such a separator is the following.

\begin{problem}
    \problemtitle{$\calS$-separability for $\calF, \calF'$}
    \probleminput{Languages $\calL \subseteq \Sigma^*$ from $\calF$ and $\calK \subseteq \Sigma^*$ from $\calF'$.}
    \problemquestion{Are $\calL, \calK$ $\calS$-separable, \ie is there $ \calR \subseteq \Sigma^*$ from $\calS$ with $\calL \subseteq \calR$ and $\calK \cap \calR = \emptyset$?}
\end{problem}

We are mostly interested in the case that both given languages come from the same class $\calF$, which we call the $\calS$-separability problem for $\calF$.
A positive answer to intersection emptiness is necessary for a positive answer to $\calS$-separability, independent of the choice of~$\calS$.
If we chose $\calS$ to be $\calF$, the two problems are equivalent as mentioned before.
If we pick $\calS$ to be a simpler class, the problem becomes interesting.
%
Assume that we have an algorithm solving the $\calS$-separability problem for some class $\calF$ that is constructive in the sense that it outputs a representation of the separator if one exists.
This algorithm can be seen as an algorithm for intersection-emptiness that produces a certificate in the case that the intersection is indeed empty.
Here, we use that checking that a given language $\calR$ is a separator is typically~--~depending on the choice of class~$\calS$~--~much simpler than checking intersection-emptiness.
%
However, one has to be careful since intersection-emptiness does not imply the existence of a separator if the class~$\calS$ is too restrictive.
Consider for example the context-free languages $\calL = \Set{a^n b^n}{n \in \N}$ and $\calK = \overline{\calL}$ that are complements of each other.
The languages are obviously disjoint, but not $\calS$-separable if we choose $\calS$ to be a subclass of the context-free languages that does not contain $\calL$; the regular languages for example.

From the perspective of verification, the most important variant of the separability problem is \emph{regular separability}, \ie checking the existence of a separator from the class of regular languages.
In this case, we call the separators \emph{regular separators} and we call two languages that have such a separator \emph{regularly separable}.
We formally state regular separability as a decision problem.

\begin{problem}
    \problemtitle{Regular separability of $\calF$}
    \probleminput{Languages $\calL, \calK \subseteq \Sigma^*$ from $\calF$.}
    \problemquestion{Are $\calL, \calK$ regularly separable,\newline\ie is there $\calR \subseteq \Sigma^*$ regular with $\calL \subseteq \calR$ and $\calK \cap \calR = \emptyset$?}
\end{problem}

Its importance stems from the fact that regular languages admit a plethora of representations and algorithms that work on these.
On the theoretical side, this means that given a regular candidate language $\calR$, the problem of checking whether $\calR$ is indeed a regular separator is decidable for many language classes.
We have argued above that the intersection-emptiness problem is decidable in many cases when one of the classes is the class of regular languages.
This allows us to check $\calK \cap \calR = \emptyset$.
We can then use that the class of regular languages is closed under complement and check $\calL \subseteq \calR$ by checking $\calL \cap \overline{\calR} = \emptyset$.
Another nice property of the regular separability problem is that it is symmetric in the input: If $\calR$ is a regular separator for $\calL$ and $\calK$, then its complement $\overline{\calR}$ is a regular separator for $\calK$ and $\calL$.
Most language classes beyond the regular languages do not satisfy these properties, a statement that we will make formal in the form of \cref{Corollary:WSTSSeparabilityClosedness}.

\end{document}
