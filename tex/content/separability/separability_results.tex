\documentclass[../../diss.tex]{subfiles}
\begin{document}

\section{Results}%
\label{Separability:Results}

The main result that we will show in this part of the thesis is the following:
\emph{Under mild assumptions, any two disjoint WSTS languages are regularly separable}.
The mild assumptions are in the case of upward-compatible WSTSes that one of the languages is the language of a WSTS that is finitely branching, deterministic or $\omega^2$.
The result also applies in the case of downward-compatible WSTSes (DWSTSes).
Here, we need that one of the languages is the language of a deterministic DWSTS or the language of an $\omega^2$-DWSTS.\@
We have not yet formally defined the restriction of being $\omega^2$; we will do so in \cref{Section:OmegaSquareWSTSes}.
For now, it suffices to know that virtually all WSTSes that occur in practice are $\omega^2$-WSTSes.
In particular, our result applies to Petri net coverability languages.

In \cref{Chapter:WSTSExpressiveness}, we will study several restricted versions of WSTSes and prove inclusions among their languages classes.
We obtain that the aforementioned restriction have in common that they allow us to determinize the corresponding WSTS.\@

We will use this fact in \cref{Chapter:WSTSSeparability} when proving our results.
Firstly, we show a technical core result that relates certain invariants on the level of configurations to the existence of a regular separator on the level of languages.
This result is stated on the level of upward-compatible LTSes, which then allows us to apply it to both DWSTSes and to WSTSes.

Our result on regular separability implies that checking regular separability amounts to checking disjointness, which can be done under mild assumptions.
Furthermore, it is constructive in the sense that from a certain type of proof of disjointness, one can construct a regular separator.
Hence, it provides the desired certificate for intersection-emptiness, which was our initial motivation for studying regular separability.

We will demonstrate the construction in \cref{Chapter:WSTSSeparatorSize} in the case of Petri nets with coverability as the acceptance condition.
We obtain a triply exponential upper bound for the state complexity of a regular separator.
Furthermore, we complement this by providing a doubly exponential lower bound.
We will comment on the fact that the bounds do not match and on several other open questions regarding our results throughout this part of the thesis.

\end{document}
