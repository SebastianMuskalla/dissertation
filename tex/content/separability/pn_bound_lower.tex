\documentclass[../../diss.tex]{subfiles}
\begin{document}

\section{Lower bound}%
\label{Section:PNSeparabilityLowerBound}%

We complement our findings on the regular separability of Petri net coverability languages by a lower bound.

\begin{theorem}%
\label{Theorem:PNSeparabilityLowerBound}%
    Regular separators for Petri net coverability languages may have doubly exponential state complexity and triply exponential index.
\end{theorem}

The lower bound does not match the upper bound: \cref{Theorem:PNSeparabilityUpperBound} provides a separator with triply exponential state complexity (and hence quadruply exponential index).
We have already commented on the fact that this additional exponent is caused by the construction that we have introduced for determinizing one of the nets.
The author conjectures that using a suitable technique, this additional blowup can be avoided.
This would imply that the lower bound that we are going to prove in this section is optimal.

To prove the theorem, we show a proposition that provides for each $n \in \N$ two language-disjoint Petri nets of size polynomial in $n$.
The regular separators of their coverability languages can be shown to have minimal state complexities and indices as required by the lower bound.

\begin{proposition}%
\label{Proposition:PNSeparabilityLowerBound}%
    For all $n \in \N$, there are Petri nets $N_0(n), N_1(n)$ of size polynomial in $n$ with disjoint languages such that any regular separator for the coverability languages $\lang{N_0(n)}$ and $\lang{N_1(n)}$ has state complexity at least $2^{2^n}$ and index at least $2^{2^{2^n}}$.
\end{proposition}

The statement on the index is a strictly stronger statement than the one on the state complexity.
If we show that any DFA separating the languages has at least $2^k$ states, this implies that any NFA with this property has at least $k$ states.
Otherwise, we could take an NFA with less than $k$ states and determinize it to obtain a DFA with less than $2^k$ states.
Hence, it will be sufficient to prove the statement on the index of a separator in the following.

The proof consists of two ingredients.
The first ingredient are two special Petri net instances $(\Ninc,\Minitinc,\vec{0})$ and $(\Ndec,\vec{0},\Mfinaldec)$ that come from Lipton's proof of the $\EXPSPACE$-hardness of coverability~\cite{Lipton76}.
We have stated the properties of these nets in \cref{Proposition:Lipton}.

With these two nets, it would be possible to show that the languages $\Set{a^m}{m < k}$ and $\Set{a^m}{m \geq k}$ for $k = 2^{2^n}$ are the coverability languages of Petri nets of size polynomial in $n$.
These languages are regular languages that are the complement of each other and have state complexity $k$.
Hence, they could be used to show the statement on the state \nb{complexity of separators.}

To also show the stronger statement on the index of separators requires a second ingredient.
It is the classical result that a language that can be represented by a \emph{small} NFA might need a \emph{large} DFA that we have briefly mentioned in \cref{Example:StateComplexityVsIndex}.
We consider the binary alphabet $\set{0,1}$.
For $x \in \set{0,1}$ and a number $k \in \N, k > 0$, the language $\calL_{x @ k}$ is the set of all words whose $k$-last letter is $x$,
\[
    \calL_{x @ k} = \Set{ w \in \set{0,1}^{\geq k}}{ w_{\card{w} - {(k-1)}} = x}
    \ .
\]
It is well-known that this language has state complexity $k+1$, but index $2^k$.
An NFA for the language can guess the occurrence of the $k$-last letter, check that it is indeed $x$, and then verify that it is followed by exactly $k-1$ letters.
A DFA for the language, however, cannot the occurrence of the $k$-last letter.
It needs to store at each point in time the $k$ last letters, which requires $2^k$ states.
\citea{Kozen06} mentions this example without proof.
We will formally prove it in the context of the proof of \cref{Proposition:PNSeparabilityLowerBound}.

Combining the two ingredients yields two Petri nets $N_0(n), N_1(n)$ such that the language of $N_x(n)$ is essentially $\calL_{x @ k}$ for the doubly exponential number $k = 2^{2^n}$.
This means that any DFA whose language separates the two coverability languages is a DFA for $\calL_{x @ k}$, which means that it needs at least $2^k$ states.

By using $\eps$-transitions, we could enforce that the language of $N_x(n)$ equals $\calL_{x @ k}$.
To show that the lower bound holds even if we do not allow $\eps$-transitions, we consider the extended alphabet $\set{0,1,a,b}$.
The language of net $N_x(n)$ will be a subset of $a^*.b.\calL_{x @ k}.b.a^*$, where the prefix $a^*$ and the suffix $a^*$ correspond to a computation of $\Ninc$ and $\Ndec$, respectively.
Since the prefix and the suffix are the same for $N_0(n)$ and $N_1(n)$, they do not influence the size of the separator.
In the following, we will first explain the construction of the nets $N_x(n)$ and then prove that they satisfy the required properties.

\paragraph{The construction of $N_x(n)$}

We state the construction of the Petri net instance $(N_x(n),\Minit,\Mfinal)$.
It is parametric in the size $n$ and in $x \in \set{0,1}$.
Let $(\Ninc,\Minitinc,\vec{0})$ and $(\Ndec,\vec{0},\Mfinaldec)$ be the Petri net instances for the chosen $n$ whose properties are specified in \cref{Proposition:Lipton}.
These nets are independent of $x$; the rest of the construction will be independent of $n$.
The net $N_x(n)$ consists of the disjoint union of $\Ninc$, $\Ndec$, and a constant number of additional places and transitions.
It is depicted schematically, \ie with the internal behavior of $\Ninc$ and $\Ndec$ hidden, in \cref{Figure:SeparabilityLowerBound}.

The places of $N_x(n)$ are the places of $\Ninc$, the places of $\Ndec$, and four  places $p_1, \ldots, p_4$.
These additional places act as control states:
The computations of the net will proceed in four phases such that in phase $i$, place $p_i$ carries one token and the places $p_{j}$ for $j \neq i$ do not carry \nb{any tokens}.

The initial marking assigns $\Minitinc$ to the places of $\Ninc$, one token to $p_1$ and no token elsewhere.
The final marking that should be covered requires a token on $p_4$ and requires tokens as specified by $\Mfinaldec$ on the places of $\Ndec$.

Instead of formally specifying the transitions of $N_x(n)$, we describe the phases that form its computations.
The transitions that can be used in each phase are disjoint.
Each transition $t$ that is used in phase $i$ checks that the control place $p_i$ carries a token, \ie $\i(t,p_i) = \o(t,p_i) = 1$.

\begin{figure}[b]
    \centering%
    \begin{tikzpicture}[->,>=latex,node distance=10em,semithick]

\node (p1) [place,minimum size=3em,tokens=1] {};
\node (p2) [place,minimum size=3em,right of=p1] {};
\node (p3) [place,minimum size=3em,right of=p2] {};
\node (p4) [place,minimum size=3em,right of=p3] {};

\node [below of=p1, node distance=2em] {$p_1$};
\node [below of=p2, node distance=2em] {$p_2$};
\node [below of=p3, node distance=2em] {$p_3$};
\node [below of=p4, node distance=2em] {$p_4$};


\node (Ninc) [draw, minimum width=8em, minimum height=6.5em, inner sep=0em, outer sep=0em, above of=p1, node distance=5em, anchor=south]
    {$\Ninc$};

\node (Ndec) [draw, minimum width=8em, minimum height=6.5em, inner sep=0em, outer sep=0em, above of=p4, node distance=5em, anchor=south]
    {$\Ndec$};

\node [above of=Ninc,node distance=4em] {$a$};
\node [above of=Ndec,node distance=4em] {$a$};

\path
    (Ninc.south west)
    --
    coordinate [pos=0.15] (Ninc1)
    coordinate [pos=0.4] (Ninc2)
    coordinate [pos=0.6] (Ninc3)
    coordinate [pos=0.85] (Ninc4)
    (Ninc.south east);

\path
    (Ndec.south west)
    --
    coordinate [pos=0.15] (Ndec1)
    coordinate [pos=0.4] (Ndec2)
    coordinate [pos=0.6] (Ndec3)
    coordinate [pos=0.85] (Ndec4)
    (Ndec.south east);

\path [ <-> ] (Ninc1) edge (p1);
\path [ <-> ] (Ninc2) edge (p1);
\path [ <-> ] (Ninc3) edge (p1);
\path [ <-> ] (Ninc4) edge (p1);

\path [ <-> ] (Ndec1) edge (p4);
\path [ <-> ] (Ndec2) edge (p4);
\path [ <-> ] (Ndec3) edge (p4);
\path [ <-> ] (Ndec4) edge (p4);

\node (poutinc) at (Ninc.east) [place,fill=white,yshift=+1.5em] {};
\node [above of=poutinc, node distance=2em, xshift=+1.5em] {$p_\outinc$};

\node (phaltinc) at (Ninc.east) [place,fill=white,yshift=-1.5em] {};
\node [above of=phaltinc, node distance=1.5em, xshift=+1.5em] {$p_\haltinc$};

\node (pindec) at (Ndec.west) [place,fill=white,yshift=+1.5em] {};
\node [above of=pindec, node distance=2em, xshift=-2em] {$p_\indec$};

\node (t12) [transition,right of=p1, node distance=5em,inner sep=0.25em] {$\mathstrut$}
    edge [pre] (p1)
    edge [post] (p2)
;

\path [->]
    (phaltinc) edge [in=90,out=-60] (t12)
;

\node (t23) [transition,right of=p2, node distance=5em,inner sep=0.25em] {$\mathstrut$}
    edge [pre] (p2)
    edge [post] (p3)
;

\node (t34) [transition,right of=p3, node distance=5em,inner sep=0.25em] {$\mathstrut$}
    edge [pre] (p3)
    edge [post] (p4)
;

\node [below of=t12,node distance=1.5em] {$b$};
\node [below of=t23,node distance=1.5em] {$x$};
\node [below of=t34,node distance=1.5em] {$b$};

\node (t20) [transition,above of=p2,node distance=4em,xshift=-1em,inner sep=0.25em] {$\mathstrut$}
    edge [<->] (p2)
;
\node (t21) [transition,above of=p2,node distance=4em,xshift=+1em,inner sep=0.25em] {$\mathstrut$}
    edge [<->] (p2)
;

\node [above of=t20,node distance=1.5em] {$0$};
\node [above of=t21,node distance=1.5em] {$1$};

\node (t30) [transition,above of=p3,node distance=4em,xshift=-1em,inner sep=0.25em] {$\mathstrut$}
    edge [<->] (p3)
    % edge [pre, bend right,in=210] (poutinc)
    % edge [post, bend left,in=125] (pindec)
;
\node (t31) [transition,above of=p3,node distance=4em,xshift=+1em,inner sep=0.25em] {$\mathstrut$}
    edge [<->] (p3)
    % edge [pre, bend right,in=220] (poutinc)
    % edge [post, bend left,in=140] (pindec)
;

\path [->]
    (t23.north east) edge [in=180,out=75] (pindec)
    (t30.north east) edge [in=200,out=75] (pindec)
    (t31.north east) edge [in=220,out=75] (pindec)
;

\path [<-]
    (t23.north west) edge [in=-28,out=105,looseness=1.4] (poutinc)
    (t30.north west) edge [in=-15,out=105] (poutinc)
    (t31.north west) edge [in=0,out=105] (poutinc)
;

\node [above of=t30,node distance=1.5em] {$0$};
\node [above of=t31,node distance=1.5em] {$1$};

\end{tikzpicture}
%
    \caption{A schematic representation of the Petri net $N_x(n)$.}%
    \label{Figure:SeparabilityLowerBound}%
\end{figure}

\begin{enumerate}
    \item[1.]
        The first phase is a computation of $\Ninc$.
        We assume that all transitions of $\Ninc$ are labeled by $a$ and that they check the existence of a token on $p_1$.

        Recall that the copy of $\Ninc(n)$ is initialized with the correct initial marking.
        From \cref{Proposition:Lipton} we get that $\Ninc$ has two special places $p_\haltinc, p_\outinc$ such that if the computation reaches a marking that assigns a token to $p_\haltinc$, $p_\outinc$ carries exactly $2^{2^n}$ tokens.
    \item[]
        A $b$-labeled transition signals the beginning of the second phase.
        It moves a token from $p_1$ to $p_2$ and consumes a token from $p_\haltinc$.
        This means that $p_\outinc$ carries $2^{2^n}$ tokens.
    \item[2.]
        The second phase uses two transitions that are labeled by $0$ and $1$, respectively.
        Besides checking for the control token on $p_2$, they have no effect.

        Intuitively, the second phase generates a prefix of a word from $\calL_{x @ k} \subseteq \set{0,1}^*$, namely all but the last $k$ letters of a word from this language.
    \item[$x$.]
        The second phase ends with an $x$-labeled transition that moves the control from $p_2$ to $p_3$.
        It also moves one token from the place $p_\outinc$ of $\Ninc$ to the place $p_\indec$ of $\Ndec$.
%
        Note that this is the only transition that depends on $x \in \set{0,1}$, \ie it is the only part of the construction in which $N_0(n)$ and $N_1(n)$ differ.

        Intuitively, firing this transition corresponds to the $k$-last letter of a word from $\calL_{x @ k}$.
        % The transition being labeled $x$ ensures that the word will indeed by contained in the language.
        % The rest of the net will ensure that the word ends with an arbitrary suffix of length $k-1$.
    \item[3.]
        The third phase consist of two transitions labeled by $0$ and $1$, respectively.
        These transitions check for the control token on $p_3$ and move one token from $p_\outinc$ to $p_\indec$.

        Intuitively, the third phase generates the suffix of length $k-1$ of a word from $\calL_{x @ k}$.
    \item[]
        The last phase starts with a $b$-labeled transition that moves the control from $p_3$ to $p_4$.
    \item[4.]
        The last phase is essentially a computation of $\Ndec$.
        We assume that all transitions of $\Ndec$ are labeled by $a$ and that they check the existence of a token on $p_4$.

        Recall that a computation of $\Ndec$ can only cover the final marking $\Mfinaldec$ if initially, we had at least $2^{2^n}$ tokens on $p_\indec$.
\end{enumerate}

We claim that the language of $N_x(n)$ is
\[
    \lang{N_x(n),\Minit,\Mfinal} = \calL_\textinc.b.\calL_{x @ 2^{2^n}}.b.\calL_\textdec
\]
where $\calL_\textinc, \calL_\textdec \subseteq a^*$ correspond to computations of $\Ninc,\Ndec$.

By the description of the phases of $N_x(n)$ above, any word in the language of $N_x(n)$ has the shape $w_\textinc.b.w.b.w_\textdec$ where $w_\textinc,w_\textdec \in a^*$.
To see that we also have $w \in \calL_{x @ k}$, we argue that the structure of the net ensures that the $k$-last letter of $w$ is $x$.
We may write $w'.x.w''$, where $w',w'' \in \set{0,1}^*$ correspond to the letters generates by Phase~2 and Phase~3, respectively.
To complete the argument, we show that $w''$ has length $k-1$.

At the beginning of Phase~2, $p_\outinc$ carries $k = 2^{2^n}$ tokens and $p_\indec$ carries no token.
At the beginning of Phase~3, \ie after generating $w'.x$, place $p_\outinc$ carries $k-1$ tokens and $p_\indec$ carries one token.
The final marking of $\Ndec$, and hence the final marking of $N_x(n)$, can only be covered if $p_\indec$ carries $k$ token at the beginning of Phase~4.
Thus, $w''$ needs to have length at least length $k-1$ to move at least $k-1$ tokens from $p_\outinc$ to $p_\indec$.
Since there are only $k-1$ tokens on available on $p_\outinc$, this also limits the length of $w''$ by $k_1$.

We have shown that any word from the language of $N_x(n)$ is of the shape $a^*.b.\calL_{x @ 2^{2^n}}.b.a^*$.
For the other direction, one would need to argue that for any word $w \in \calL_{x @ 2^{2^n}}$, the word $w_\textinc.b.w.b.w_\textdec$ obtained by pre- and appending suitable pre- and suffixes is in the language of $N_x(n)$.
This can be shown easily by picking suitable transitions in the Phases~2 and~3.

To finish the proof of \cref{Proposition:PNSeparabilityLowerBound}, we have to show that the languages of $N_0(n)$ and $N_1(n)$ cannot be separated by the regular language of index less than triply exponentially.
Since the languages share the prefix $a^*.b$ and the suffix $b.a^*$, this boils down to proving that $\calL_{1 @ {2^{2^n}}}$ and $\calL_{0 @ {2^{2^n}}}$ cannot be separated by a DFA with less than triply exponentially many states.
We proceed to show this using the well-known fooling-set technique~\cite{Birget92,Birget93}.

\begin{proof}[Proof of \cref{Proposition:PNSeparabilityLowerBound}]
    Let $k = 2^{2^n}$ and assume that $\lang{A}$ is the language of a DFA with strictly less than~$2^k$ states that separates the given languages, \ie $\lang{N_0(n),\Minit,\Mfinal} \subseteq \lang{A}$ and $\lang{N_1(n),\Minit,\Mfinal} \cap \lang{A} = \emptyset$.
    Towards a contradiction, we consider the set of words $\set{0,1}^k$.
    Since its size is larger than the number of states of the DFA $A$, there are distinct words that lead to the same state, a fact that we will exploit later.

    Let $w_\textinc \in a^*$ be a word generated by a computation of $\Ninc$ that creates a token on $p_\haltinc$.
    Similarly, let $w_\textdec \in a^*$ be a word generated by a covering computation of $\Ndec$, assuming that it starts from a marking that places $k$ tokens on $p_\indec$.

    We use the set of words $\set{ w_\textinc.b.w }{ w \in \set{0,1}^k}$ as the fooling set.
    For each word $w_\textinc.b.w$, let $q_w$ be the unique state in which $A$ is after reading it.
    Since $\card{\set{0,1}^k} = 2^k$ is strictly larger than the number of states of $A$, there are two distinct words $w,w'$ with $w \neq w'$ such that $q_{w} = q_{w'}$.
    %
    Since $w \neq w'$, there is some position $i$ with $w_i \neq w_i'$.
    \Wolog, we assume $w_i = 0, w_i' = 1$.
    We define $w_{\text{fill}} = 0^{i-1}$ as a suitable suffix such that the $\nth{i}$ position  of $w$ and $w'$ becomes the $k$-last position in $w.w_{\text{fill}}$ and $w'.w_{\text{fill}}$.
    In particular, we have $w.w_{\text{fill}} \in \calL_{0 @ k}$, $w'.w_{\text{fill}} \in \calL_{1 @ k}$.
%
    This implies
    \[
        w_\textinc.b.w.w_{\text{fill}}.b.w_\textdec \in \lang{N_0(n),\Minit,\Mfinal} \subseteq \lang{A}
        \ ,
    \]
    meaning that the unique run of $A$ on that word is accepting.
    Since the state $q_w$ after reading $w_\textinc.b.w$ respectively $w_\text.b.w'$ is the same and the suffixes $w_{\text{fill}}.b.w_\textdec$ are equal,
    \nb{we also get}
    \[
        w_\textinc.b.w'.w_{\text{fill}}.b.w_\textdec \in \lang{A}
        \ .
    \]
    However, this word contained in $\lang{N_1(n),\Minit,\Mfinal}$, a contradiction to the assumption that $A$ is a separating automaton with $\lang{A} \cap \lang{N_1(n),\Minit,\Mfinal} = \emptyset$.
\end{proof}

With the proof of the lower bound completed, all results outlined in \cref{Separability:Results} have been shown.
Our study of the regular separability of WSTS languages has been completed.

\end{document}
