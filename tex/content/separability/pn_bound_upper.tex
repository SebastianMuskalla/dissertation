\documentclass[../../diss.tex]{subfiles}
\begin{document}

\section{Upper bound}%
\label{Section:PNSeparabilityUpperBound}%

Our upper bound is the following theorem.

\begin{theorem}%
\label{Theorem:PNSeparabilityUpperBound}%
    Two labeled Petri net instances with disjoint languages can be separated by a regular language with triply exponential state complexity.
\end{theorem}

Our proof of the theorem will be constructive in the sense that it provides an invariant in the ideal completion from which one can construct a separating automaton as specified by \cref{Definition:WSTSSeparabilityAutomaton}.
Before we present the construction, we need to deal with two problems.
The first one is the minor technicality that we have allowed $\eps$-transitions in Petri nets, but not in WSTSes.
The second one is the requirement of one of the systems being deterministic.
We will discuss the latter in more detail after solving the first problem.

Throughout the rest of this section, let $(N_1,M_{\init1},M_{\final1})$ and $(N_2,M_{\init2},M_{\final2})$ be two labeled Petri net instances over some alphabet $\Sigma$ of size $n_1$ and $n_2$, respectively.
We assume that the coverability languages of the two nets are disjoint.

\paragraph{Eliminating $\eps$-transitions}

Unlike in Petri nets, we have not allowed $\eps$-transitions in WSTSes for reasons that we explained in \cref{Remark:WSTSEpsTransitions}.
The construction that we will apply to determinize one of the nets will also eliminate $\eps$-transitions in that net as a by-product.
To handle $\eps$-transitions in the other net, we apply a preprocessing step.
It is designed so that it does not change the fact that the nets are language-disjoint, and it has no substantial influence on the size of the separating automaton.

Assume that net $N_1$ contains transitions labeled by $\eps$.
To eliminate them, we consider a fresh letter $a \not\in \Sigma$ and define $\Sigma_a = \Sigma \dotcup \set{a}$.
We relabel all $\eps$-transitions in $N_1$ by letter $a$, resulting in the net $N_1'$.
The initial and final marking remain unchanged.

To account for the changes to $N_1$ in the other net, we add a fresh transition $t_a$ to $N_2$ that is labeled by $a$ and that does neither consume nor produce any tokens, $\i(t_a) = \o(t_a) = \vec{0}$.
Let the resulting net be $N_2'$.
Again, we do not change the initial or the final marking.

Firstly, observe that the size of $N_1$ and $N_2$ is polynomial in the size of $N_1$ and $N_2$, respectively.
Secondly, we will prove that a separator for the original nets can be turned into a separator for the modified nets and vice versa, where both transformations impose a blow up of the separating automaton that is at most polynomial.

\begin{lemma}%
\label{Lemma:PNSeparabilityEliminatingEpsilon}%
    The languages of $N_1$ and $N_2$ are regularly separable iff the languages of $N_1'$ and $N_2'$ are.
\end{lemma}

\begin{proof}
    Assume that $A$ is an NFA over $\Sigma$ whose language $\lang{A}$ is a regular separator for $\lang{N_1,M_{\init1},M_{\final1}}$ and $\lang{N_2,M_{\init2},M_{\final2}}$.
    We define $A'$ over $\Sigma_a$ by adding an $a$-labeled self-loop $q \tow{a} q$ to every state $q$ of $A$.
    Clearly, the size of $A'$ is polynomial in the size of $A$.
    It remains to show that its language separates $\lang{N_1',M_{\init1},M_{\final1}}$ and $\lang{N_2',M_{\init2},M_{\final2}}$.

    We first show $\lang{N_1',M_{\init1},M_{\final1}} \subseteq \lang{A'}$.
    Consider a covering computation $M_{\init1} \fire{\sigma} M \geq M_{\final1}$ of $N_1'$.
    It is also a covering computation of $N_1$, because the two nets differ only in their transition labels.
    Hence, $\lambda(\sigma) \in \lang{A}$ since $\lang{A}$ is assumed to be a separator.
    By using the $a$-labeled self-loops that are present in $A'$, we turn an accepting run of~$A$  on $\lambda(\sigma)$ in into an accepting run of $A'$ on $\lambda'(\sigma)$, which completes the argument.

    To see that $\lang{A'} \cap \lang{N_2',M_{\init2},M_{\final2}} = \emptyset$, assume towards a contradiction that some word $w' \in \Sigma_a^*$ is contained in the intersection.
    Define word $w \in \Sigma^*$ by projecting $w'$ to $\Sigma$, \ie by removing all occurrences of letter $a$.
    We claim that $w \in \lang{A} \cap \lang{N_2,M_{\init2},M_{\final2}}$.
    The only way of generating letter $a$ in automaton $A'$ is using the $a$-labeled self-loops.
    By removing all of them, we turn an accepting run of $A'$ on $w'$ into an accepting run of $A$ on $w$.
    Similarly, a covering computation of $N_2'$ for $w'$ can be turned into a covering computation of $N_2$ for $w$ by removing all occurrences of transition $t_a$.
    Hence, $w \in \lang{A} \cap \lang{N_2,M_{\init2},M_{\final2}}$ which is a contradiction to the assumption that $\lang{A}$ is a separator.

    For the other direction, assume that the language of some NFA $A'$ separates $\lang{N_1',M_{\init1},M_{\final1}}$ and $\lang{N_2',M_{\init2},M_{\final2}}$.
    We define $A$ by relabeling all $a$-transitions of $A'$ to $\eps$.
    The resulting automaton has the same size as $A'$, but may contain $\eps$-transitions.
    However, it is well-known that these transitions can be eliminated without introducing additional states.

    We show that $\lang{N_1,M_{\init1},M_{\final1}} \subseteq \lang{A}$.
    Assume $w \in \lang{N_1,M_{\init1},M_{\final1}}$, and consider a covering computation $M_{\init1} \fire{\sigma} M \geq M_{\final1}$ of $N_1$ with $\lambda(\sigma) = w$.
    We may see this computation as a computation of $N_1'$ that produces the word $w' = \lambda'(\sigma)$, and use the assumption that $\lang{A'}$ is a separator to conclude $w' \in \lang{A'}$.
    Because $w$ is obtained from $w'$ by removing all occurrences of letter $a$, we obtain $w \in \lang{A}$ as desired.

    Finally, we show $\lang{N_2,M_{\init2},M_{\final2}} \cap \lang{A} = \emptyset$.
    Assume that word $w$ is contained in $\lang{A}$.
    We define $w'$ to be a word that is obtained from $w$ by inserting occurrences of letter $a$ so that $w' \in \lang{A'}$.
    It has to exist by the definition of the automaton $A$.
    Using that $\lang{A'}$ is a separator, we obtain that $w'$ is not contained in $\lang{N_2',M_{\init2},M_{\final2}}$.
    Hence, the word $w$ that is re-obtained by removing all occurrences of $a$ cannot be contained in the language of $N_2$:
    A covering computation of $N_2$ producing $w$ would imply the existence of a covering computation of $N_2'$ producing~$w'$ by inserting occurrences of $t_a$.
    This finishes the proof.
\end{proof}

Altogether, we have shown how to eliminate $\eps$-transitions from one of the nets while only introducing a polynomial blowup to the size of the nets and the size of the separator.
In the rest of this section, we can assume without loss of generality that both Petri nets do not contain $\eps$-labeled transitions.

\paragraph{Enforcing determinism}

\cref{Theorem:WSTSSeparability}, our result that shows that the languages of disjoint WSTSes are regularly separable requires one of the two WSTSes to be deterministic.
In general, the WSTSes induced by a labeled Petri net will not satisfy this requirement.
Because the underlying order $(\N^k,\leq)$ of such a WSTS is $\omega^2$, we could use our results on expressiveness, \cref{Lemma:Expressiveness1} in particular, to obtain a deterministic language-equivalent WSTS.\@
The construction from the proof of \cref{Lemma:Expressiveness1} would provide a WSTS whose underlying order is $(\pwrsetdown{\N^k},\subseteq)$.
This poses a problem.
The ideals of $(\pwrsetdown{\N^k},\subseteq)$ are well-understood: They are of the shape $\dcwrt{D}{\subseteq} = \Set{ D' \subseteq D }{ D' \text{ is downward closed in }  \N^k}$, where $D$ ranges over the non-empty downward-closed subsets of $\N^k$~\cite{LazicS15}.
However, we then would need to obtain a bound on the size of the ideal decomposition of an invariant in this order to obtain a bound on the size of the separating automaton.
The author did not succeed in proving such a result.
It seems that the methods that have been developed by \citea{LazicS15} in the context of \emph{alternating} and \emph{branching vector addition systems} do not apply here.

To circumvent the outlined problems, we do not use the construction from the proof of \cref{Lemma:Expressiveness1}.
Instead, we apply another preprocessing step to one of the Petri net which leads to the induced WSTS becoming deterministic.
This leads to a separator for the modified nets, but it is not straightforward to obtain the desired separator for the original ones.
We will discuss this problem in detail after giving the construction.

\paragraph{Enforcing uniqueness}

Assume again that $(N_1,M_{\init1},M_{\final1})$ and $(N_2,M_{\init2},M_{\final2})$ are the given Petri net instances.
The main challenge in enforcing the WSTS induced by $N_1$ to be deterministic is the requirement that the transition relation has to be unique.
This requirement is violated as soon as $N_1$ contains two transitions that have the same label, because there is necessarily a marking that is large enough so that both transitions are enabled.
Hence, we have to equip $N_1$ with the \emph{free labeling}, a labeling that labels each transition by its own name.
In the freely labeled variant $N_1'$ of $N_1$, the set of transitions $T_1$ is the alphabet and the labeling function is the identity $\id \colon T_1 \to T_1$.

It remains to create a modification $N_2'$ of $N_2$ that also uses $T_1$ as the alphabet.
In the product of $N_1$ and $N_2$, a transition $t_1$ of $N_1$ can synchronize with all transitions of $t_2$ that have the same label, $\lambda_1(t_1) = \lambda_2(t_2)$.
The idea behind the construction of $N_2'$ is to preserve this property.
To this end, we replace each transition $t_2$, say with label $\lambda_2(t_2) = a$, by a bunch of copies $t_2(t_1)$ of $t_2$ in $N_2'$, one copy for each transition $t_1$ of $N_1$ with label $\lambda_1(t_1) = a$.
The incoming and outgoing multiplicities of each copy $t_2(t_1)$ are equal to those of $t_2$.
The places, the initial marking, and the final marking of $N_2'$ coincide with $N_2$.
Net $N_2'$ uses $T_1$ as its alphabet, and each transition $t_2(t_1)$ is labeled by $t_1$.

We argue that this construction accomplishes its intended purpose.
Firstly, observe that if two transitions $t_1$ in $N_1$ and $t_2$ in $N_2$ can synchronize in the product of the nets, meaning that they have the same label, then the transition $t_1$ in $N_1'$ can synchronize with transition $t_2(t_1)$ in $N_2'$, the $t_1$-labeled copy of $t_2$.

Secondly, the transition relation of $N_1'$ (and the WSTS induced by $N_1'$) is unique since there is only one transition with a specific label.
However, it is not yet deterministic, since we are missing completeness:
There may be markings in which the unique transition with a specific label is not enabled.
This is a minor issue that we will handle later.

Thirdly, there is a strong connection between the languages of $N_1$ and $N_2$ and the languages of $N_1'$ and $N_2'$.
Consider the labeling function  of $N_1'$, extended into a homomorphism $\lambda_1 \colon T_1^* \to \Sigma^*$.
We have that $\lang{N_1,M_{\init1},M_{\final1}} = \lambda_1 ( \lang{N_1',M_{\init1},M_{\final1}}   )$
and $\lang{N_2,M_{\init2},M_{\final2}} = \lambda_1 ( \lang{N_2',M_{\init2},M_{\final2}}   )$.
This in particular means that the languages of $N_1$ and $N_2$ are disjoint if and only if the languages of $N_1'$ and $N_2'$ are.
It is easy to see that a word $\sigma \in T_1^*$ that is in the languages of $N_1'$ and $N_2'$ leads to $\lambda_1(\sigma)$ being in the language of $N_1$ and~$N_2$.
For the other direction, consider a word $w \in \Sigma^*$ in the intersection of the languages of $N_1$ and $N_2$, and consider firing sequences $\sigma \in T_1^*$, $\tau \in T_2^*$ that induce the covering computation.
We immediately get that $\sigma$ is in the language of $N_1'$.
To see that $\sigma$ is also in the language of $N_2'$, one can use the covering computation induced by the firing sequence $\tau_\sigma$ whose $\nth{i}$ transition is $\tau_i (\sigma_i)$, the $\sigma_i$-labeled copy of $\tau_i$ in $N_2'$.

The most important property is that a regular separator for the languages $N_1'$ and $N_2'$ can be turned in to a regular separator for the languages of the original nets.
One might expect that if $\calR \subseteq T_1^*$ is a regular language separating the languages of the modified nets, then $\lambda_1(\calR)$ is the desired separator over $\Sigma^*$.
However, this turns out to be wrong.
Applying a homomorphism to two disjoint languages can result in two languages with a non-empty intersection.
Assume that $\calR$ is disjoint from the language of $N_2'$, $\calR \cap \lang{N_2',M_{\init2},M_{\final2}} = \emptyset$.
Nevertheless, there may be a word $w \in \lambda_1(\calR) \cap \lambda_1(\lang{N_2',M_{\init2},M_{\final2}})$, which means that $\lambda_1(\calR)$ is not disjoint from $\lambda_1(\lang{N_2',M_{\init2},M_{\final2}}) = \lang{N_2,M_{\init2},M_{\final2}}$.
Since we assume that $\calR$ and the language of $N_2'$ are disjoint, we know that $w$ is not the image of a word over $T_1$ that is both in $\calR$ and in the language of $N_2$.
However, there may be two words that both have $w$ as their image under $\lambda_1$ such that one of them is in $\calR$ and the other is in the language of $N_2'$.
Phrased differently, we need that any word in $\lambda_1(\calR)$ has no preimage in the language of $N_2'$, which is a stronger property than disjointness.

This problem can be overcome by using complementation.
More precisely, let $\calR \subseteq T_1^*$ be a regular language that separates $N_2'$ and $N_1'$ in the sense that $\lang{N_2',M_{\init2},M_{\final2}} \subseteq \calR$ and $\lang{N_1',M_{\init1},M_{\final1}} \cap \calR = \emptyset$.
Then $\lambda_1(\overline{\calR})$ is a regular separator for the languages of $N_1$ and $N_2$.
Note that the order of the two input languages is swapped to account for the complementation.

\begin{proposition}%
\label{Proposition:PNSeparabilityDeterminization}%
    If $\calR$ regularly separates the languages of $N_2'$ and $N_1'$, then $\lambda_1(\overline{\calR})$ regularly separates the languages of $N_1$ and $N_2$.
\end{proposition}

\begin{proof}
    Firstly, observe that our candidate separator is indeed a regular language because the class of regular languages is closed under complementation and homomorphisms.
    Secondly, we prove $\lang{N_1,M_{\init1},M_{\final1}} \subseteq \lambda_1(\overline{\calR})$.
    Since $\calR$ separates the languages of $N_2'$ and $N_1'$, its complement $\overline{\calR}$ separates the languages of $N_1'$ and $N_2'$.
    This in particular means that $\lang{N_1',M_{\init1},M_{\final1}} \subseteq  \overline{\calR}$.
    Inclusions are preserved under homomorphisms, so we may use $\lang{N_1,M_{\init1},M_{\final1}} = \lambda_1(\lang{N_1',M_{\init1},M_{\final1}})$ to deduce the desired statement.

    Finally, we show that $\lambda_1(\overline{\calR})$ is disjoint from $\lang{N_2,M_{\init2},M_{\final2}}$.
    Towards a contradiction, assume that there is a word $w \in \Sigma^*$ in the intersection of the languages.
    On the one hand, this means that there is some sequence $\tau \in \overline{\calR}$ with $\lambda_1(\tau) = w$.
    Note that this sequence consists of (the names of) transitions of $N_1$.
    On the other hand, there is a covering computation $M_{\init2} \fire{\sigma} M \geq M_{\final2}$ of net $N_2$ with $\lambda_2(\sigma) = w$.
    We turn this computation into a computation $\sigma_\tau$ of $N_2'$ as follows:
    We replace the $\nth{i}$ transition $\sigma_i \in T_2$ of $N_2$ by $\sigma_i(\tau_i)$, the copy of transition $\sigma_i$ in $N_2'$ that is labeled by $\tau_i \in T_1$.
    This copy indeed exists because after applying the respective labeling functions, we have $\lambda_2(\sigma_i) = \lambda_1(\tau_i) = w_i$.
%
    The result is a sequence $\sigma_\tau$ of transitions of $N_2'$ that still induces a covering computation, hence $\lambda_1(\sigma_\tau) \in \lang{N_2',M_{\init2},M_{\final2}}$.
    Furthermore, the construction ensures that the labeling of $\sigma_\tau$ under the labeling function of $N_2'$ is $\tau \in \overline{\calR}$.
    This is a contradiction to the assumption that $\overline{\calR}$ and the language of $N_2$ are disjoint because $\calR$ is a regular separator.
\end{proof}

Unfortunately, applying complementation and a homomorphism has a severe impact on the complexity.
We will prove that for two Petri nets, one of them freely labeled, one can construct a separator with a doubly exponential state complexity.
This means that we can represent its complement by a DFA with at most triply exponentially many states.
Applying a homomorphism to this DFA introduces a polynomial blowup to its size, but we lose determinism.
The final result is an NFA of triply exponential size, proving \cref{Theorem:PNSeparabilityUpperBound}.
A priori, the index of the separator may be quadruply exponential.

We will later prove a doubly exponential lower bound for the state complexity of regular separators for coverability languages of non-freely labeled Petri nets.
This matches the upper bound in the case that one of the nets is freely labeled, but not in the general case.
The author conjectures that the determinization construction in this section can be improved to avoid the exponential blowup introduced by the complementation.

\paragraph{Enforcing completeness}

The definition of deterministic WSTSes imposes two restrictions on the transition relation, uniqueness and completeness.
The above construction ensures that each marking has at most one $a$-labeled successor in the WSTS associated to a Petri net.
To get determinism, we need also need to guarantee that there is at least one $a$-labeled successor.
Luckily, enforcing completeness is typically much easier than enforcing uniqueness.
For example, an NFA can be completed by introducing an additional \emph{error state}.
A similar trick works in the case of WSTSes.

Let $\wsts = (\configs,\leq,T,\configs_\init, \configs_\final)$ be a labeled WSTS over some alphabet $\Sigma$.
We define the WSTS $\wstsprime = (\configs \dotcup \set{\bot},\leq,T',\configs_\init,\configs_\final)$ that is obtained by adding a special bottom configuration $\bot$.
The order $\leq$ on $\configs$ is extended to $\configs \dotcup \set{\bot}$ by defining $\bot \leq c$ for all $c \in \configs \dotcup \set{\bot}$.
This order is a WQO assuming that $(\configs,\leq)$ is a WQO.\@
The new configuration is neither initial nor final.

The transition relation $T'$ is a superset of $T$ defined as follows.
If $c \in \configs$ is a configuration that has no $a$-successor in $T$, then we have $c \tow{a} \bot$ in $T'$.
Additionally, we have $\bot \tow{a} \bot$ in $T'$ for all letters $a$.
Upward compatibility is satisfied since $\bot \leq d$ for all $d \in \configs$.
The definition of $T'$ ensures completeness, \ie each configuration has at least one $a$-successor.
If $T$ is unique, then $T'$ is deterministic.

Since the new configuration $\bot$ is absorbing, meaning a computation that enters it cannot leave, but it is not final, we have $\lang{\wsts} = \lang{\wstsprime}$.
If $\wsts$ is the WSTS associated to a freely labeled Petri net, then $\wstsprime$ is a deterministic WSTS with the same language.
With respect to regular separability, the two can be used interchangeably.

When considering the size of the separator, however, it is important to understand the implications that the construction of $\wstsprime$ has on the ideal completion.
If $\calJ$ is an ideal in $(\configs \dotcup \set{\bot},\leq)$, then it is of the shape $\calI \dotcup \set{\bot}$, where $\calI \subseteq \configs$ is an ideal of $(\configs,\leq)$ or the empty set.
Similarly, if $\calI$ is an ideal of $(\configs,\leq)$, then $\calI \dotcup \set{\bot}$ is an ideal of $(\configs \dotcup \set{\bot},\leq)$.
To see that both statements are true, we use the fact that $\bot$ is a bottom element, which implies that it is contained in every non-empty downward-closed subset.

Consequently, if $X$ is an invariant for $\wsts$, and its ideal decomposition $\idealdecomp{\configs}{X}$ has size $k$, then $X \cup \set{\bot}$ is an invariant for $\wstsprime$ and its ideal decomposition has size $k+1$.
The latter consists if the ideals $\ideal \cup \set{\bot}$ where $\ideal \in \idealdecomp{\configs}{X}$ and the additional ideal $\set{\bot}$.
This means that applying the preprocessing step that makes a WSTS complete only adds a single ideal to the ideal decomposition of an invariant.

\paragraph{An upper bound assuming determinism}

The final step in proving \cref{Theorem:PNSeparabilityUpperBound} is showing that if $N_1$ and $N_2$ are language-disjoint Petri nets where $N_1$ is freely labeled, then their languages can be separated by the language of an NFA with doubly exponentially many states.
To this end, we will consider the size of the ideal decomposition of an invariant for the products of the nets.

The main ingredient to showing a doubly exponential bound for the size of this invariant is the following result by \citea{BozzelliG11}.
It provides a doubly exponential bound on the size of a representation of $\reachrev{\wstsN}$, where $\wstsN$ is the WSTS associated to a Petri net $N$.
The result has been obtained by inspecting Abdulla's backward search~\cite{AbdullaCJT96}.

\begin{theorem}[(\citea{BozzelliG11})]%
\label{Theorem:SeparabilityBozzelliGanty}%
    Consider a Petri net instance $(N,M_\init,M_\final)$ and the associated WSTS $\wstsN$.
    One can construct a representation
    $\reachrev{\wstsN}=\uc{\set{M_1, \ldots, M_k}}$, where $k$ as well as all $\norminf{M_i}$ are bounded by
    \[
        g =
        \paren
        {
            \card{T} \cdot
            \paren
            {
                \norminf{N}
                +
                \norminf{M_\init}
                +
                \norminf{M_\final}
                +
                2
            }
        }^{2^{\bigO{\card{P} \cdot \log \card{P}}}}
        \ .
    \]
\end{theorem}

Here, $\norminf{N} = \max \set{ \norminf{\i}, \norminf{\o}}$ is the maximum multiplicity of any transition.
Our challenge in the following will be to convert this bound on a representation of the upward closed set $\reachrev{\wstsN}$ into a bound on the size of the ideal decomposition of its complement $\N^P \setminus \reachrev{\wstsN}$.
This set is guaranteed to be an invariant since $\wstsN$ has empty language in our case.

We have already briefly mentioned in \cref{Section:SeparabilityIdeals} that the ideals in $\N^d$ are of the shape $\dc{M}$, where $M \in \N_\omega^d$ is a generalized marking.
In this section, we will additionally need to consider (1)~how to construct a representation of the intersection of two ideals, and (2)~how to construct the ideal decomposition of the set $\N^d \setminus \uc{M}$, where $M$ is a marking.
The following lemma states the well-known solution to both problems.

%
\cheatpagebreak
%

\begin{lemma}[(see \eg \citea{LazicS15})]%
\label{Lemma:SeparabilityIdealOperations}%
    \begin{thmenumerate}[a)]
        \item
            For ideals $\dc{M_1}, \dc{M_2}$ of $\N^d$, their intersection $\dc{M_1} \cap \ \dc{M_2}$ is the ideal $\dc{M}$ where $M$ is specified by $M(i) = \min \set{ M_1(i), M_2(i)}$ for all $i \in \oneto{d}$.
        \item
            For a marking $M \in \N^d$, the ideal decomposition of $\N^d \setminus \uc{M}$ is $\Set{ \dc{M_i}}{ i \in \oneto{d}}$ where $M_i(i) = M(i) - 1$ and $M_i(j) = \omega$ for $j \neq i$.
    \end{thmenumerate}
\end{lemma}

The proof is straightforward in both cases.
The representation of the intersection simply formalizes the fact that an element of the intersection needs to be smaller in every component than the representations of the intersected ideals.
For b), each ideal in the ideal decomposition of $\N^d \setminus \uc{M}$ describes all markings that are smaller than $M$ in some component, and hence contained in the complement of $\uc{M}$.

With the preliminaries at hand, we can state and prove the main result.

\begin{proposition}%
\label{Proposition:PNSeparabilityUpperBound}%
    Let $N_1,N_2$ be languages-disjoint Petri nets, and let $\wsts_{\times}$ be the WSTS associated to their product.
    The ideal decomposition of the invariant $\N^P \setminus \reachrev{\wsts_{\times}}$ has size at most doubly exponential.
\end{proposition}


\begin{proof}
    Let $(N_1,M_{\init1},M_{\final1})$ and $(N_2,M_{\init2},M_{\final2})$ be the given Petri net instances.
    Let $\wsts_{\times}$ be the WSTS associated to the product of the nets, and let $d$ be the total number of places.
    Since the two Petri nets are language disjoint, $X = \N^d \setminus \reachrev{\wsts_{\times}}$ is a safe inductive invariant, see \cref{Section:WSTSSeparabilityCore}.
    Hence, the downward closure $\dc{Y}$ of its ideal decomposition $Y = \idealdecomp{X}$ is a finitely represented invariant in the ideal completion by \cref{Proposition:WSTSIdealInductiveInvariant}.

    With \cref{Theorem:SeparabilityBozzelliGanty} applied to $\wsts_{\times}$, we have
    $\reachrev{\wsts_{\times}} = \uc{\set{M_1, \ldots, M_k}}$ for suitable markings $M_1, \ldots, M_k$.
    We may write
    \[
        X
        \ = \ \N^d \setminus \reachrev{\wsts_{\times}}
        \ = \ \N^d \setminus \uc{\set{M_1, \ldots, M_k}}
        \ = \ \N^d \setminus \bigcup_{i \in \oneto{k}} \uc{M_i}
        \ = \ \bigcap_{i \in \oneto{k}} \N^d \setminus \uc{M_i}
        \ .
    \]
    With Part~b) of \cref{Lemma:SeparabilityIdealOperations}, we can rewrite each $\N^d \setminus \uc{M_i}$ to $\bigcup_{j \in \oneto{d}} \dc{M_{i,j}}$, where $M_{i,j}(j) = M_i(j) - 1$ and $M_{i,j}(\ell) = \omega$ for $\ell \neq j$.
    Altogether, we have
    \[
        X = \bigcap_{i \in \oneto{k}}  \bigcup_{j_i \in \oneto{d}} \dc{M_{i,j_i}}
        \ .
    \]
%
    We use distributivity to swap the intersection and the union, resulting in
    \[
        X =  \bigcup_{\vec{j} \in \oneto{d}^k} \bigcap_{i \in \oneto{k}} \dc{M_{i,\vec{j}(i)}}
        \ .
    \]
    We can use Part~a) of \cref{Lemma:SeparabilityIdealOperations} to compute for each $\vec{j}$ a single ideal $M_{\vec{j}}$ with $\dc{M_{\vec{j}}} = \bigcap_{i \in \oneto{k}} \dc{M_{i,\vec{j}(i)}}$.
    Altogether, we obtain $X = \bigcup_{\vec{j} \in \oneto{d}^k} \dc{M_{\vec{j}}}$.
    Some ideals in this union might be redundant, but in any case, we obtain that the ideal decomposition of $X$ consists of at most~$d^k$ ideals.
    Note that we have $k \leq g$, where $g$ is the doubly exponential bound specified in \cref{Theorem:SeparabilityBozzelliGanty}.
    Unfortunately, this only provides a triply exponential bound $d^{g}$.

    To get the desired bound, we use that \cref{Theorem:SeparabilityBozzelliGanty} also specifies $\norminf{M_i} \leq g$ for all $i$.
    By the construction of the generalized markings $M_{i,j}$, we have $\norminf{M_{i,j}} \leq \norminf{M_i} \leq g$, where we extend the infinity norm to generalized markings by treating $\omega$-components as zero.
    Finally, we have that each $M_{\vec{j}}$ which represents the intersection of some of the $\dc{M_{i,j}}$ also satisfies $\norminf{M_{\vec{j}}} \leq g$ by definition.
    Hence, any non-$\omega$ component of $M_{\vec{j}}$ is bounded by $g$ from above.

    There are only ${(g+2)}^d$ different generalized $d$-dimensional marking in which all non-$\omega$ components are bounded by $g$.
    Hence, after removing redundant ideals, the ideal decomposition of $X$ has size at most ${(g+2)}^d$.
    Inserting the definition of $g$ from \cref{Theorem:SeparabilityBozzelliGanty} yields the bound
    \[
        \card{Y} \leq {(g+2)}^d
        =
        \paren
        {
        \paren
        {
            \card{T_\times} \cdot
            \paren
            {
                \norminf{N_\times}
                +
                \norminf{M_{\init\times}}
                +
                \norminf{M_{\final\times}}
                +
                2
            }
        }^{2^{\bigO{\card{P_\times} \cdot \log \card{P}}}}
        +2
        }^{d}
        \ .
    \]
    Here $T_\times, N_\times$ and so on specify the components of the product net of $N_1$ and $N_2$ that induces the WSTS $\wsts_{\times}$.
    Using the power laws, the number
    \(
        \paren
        {
            \card{T_\times} \cdot
            \paren
            {
                \norminf{N_\times}
                +
                \norminf{M_{\init\times}}
                +
                \norminf{M_{\final\times}}
                +
                2
            }
        }^{2^{\bigO{\card{P_\times} \cdot \log \card{P}}}}
    \)
    is at most exponential in $\card{N_1} + \card{N_2}$, even if $\norminf{N_\times}$, $\norminf{M_{\init\times}}$, and $\norminf{M_{\final\times}}$ may already be exponential in $\card{N_1} + \card{N_2}$.
    Hence, we get the desired doubly exponential bound.
\end{proof}

\begin{remark}
    The above proof of \cref{Proposition:PNSeparabilityUpperBound} using the result by \citea{BozzelliG11} is a modified version of the proof in the original publication~\cite{CzerwinskiLMMKS18,CzerwinskiLMMKS18a}.
    With the results by \citea{LazicS15}, Corollary~4.6 in particular, a simpler proof would be possible.
    The authors of that paper show that one can compute the invariant $\N^d \setminus \reachrev{\wsts_N}$ as the fixed point of the descending chain
    \[
        \N^d \setminus \uc{\Mfinal}
        \ \supseteq \
        \N^d \setminus \pre{\Sigma^{\leq 1}}{\uc{\Mfinal}}
        \ \supseteq \
        \N^d \setminus \pre{\Sigma^{\leq 2}}{\uc{\Mfinal}}
        \ \supseteq \
        \ldots
        \ ,
    \]
    based on a technique that uses ideals.
    They establish a doubly exponential upper bound for the smallest number $i$ such that $\N^d \setminus \reachrev{\wsts_N} = \N^d \setminus \pre{\Sigma^{\leq i},\uc{\Mfinal}}$ and mention that this implies that the ideal decomposition of the invariant is also of doubly exponential size.
\end{remark}

Recall that our upper bound, \cref{Theorem:PNSeparabilityUpperBound}, states that two labeled Petri net instances with disjoint languages can be separated by a regular language with triply exponential state complexity.
We can now assemble its proof by using the various techniques that we have presented in this section.

\begin{proof}[Proof of \cref{Theorem:PNSeparabilityUpperBound}]
    Let $N_1, N_2$ be two given language-disjoint Petri nets.
    We use the above construction to eliminate all $\eps$-transitions in $N_2$, resulting in the nets $N_1',N_2'$.
    We then consider $N_1''$, the freely labeled variant of $N_1'$, and $N_2''$, the corresponding modification of $N_2'$.
    Note that being freely labeled also implies that $N_1'$ has no $\eps$-transitions.
    %
    The techniques that we have applied preserve language-disjointness, so the intersection of the languages of $N_1''$ and $N_2''$ are disjoint.
    This means that the language of the WSTS $\wsts_\times$ associated to their product is empty, and the set $\N^d \setminus \reachrev{\wsts_\times}{\uc{M_{\final\times}}}$ is an inductive invariant.
    With \cref{Proposition:PNSeparabilityUpperBound}, the size of the ideal decomposition of this invariant is at most doubly exponential.

    The transition relation of the WSTS associated to $N_1''$ is unique, but not complete.
    However, we may apply a preprocessing step as described above and obtain a deterministic WSTS $\wsts_{N_1''\text{complete}}$ with the same language.
    Compared to the original WSTS, this WSTS has one additional ideal.
    This means that the ideal decomposition of the invariant for $\wsts_\times$ translates into an ideal decomposition of an invariant for the product of $\wsts_{N_1''\text{complete}}$ and $\wsts_{N_2''}$ that is larger only by a polynomial factor.
    This product satisfies the assumption of \cref{Theorem:WSTSSeparabilityCore}, proving that the languages of $\wsts_{N_1''\text{complete}}$ and $\wsts_{N_2''}$ can be separated by the language of an NFA with doubly exponentially many states.

    Applying \cref{Proposition:PNSeparabilityDeterminization} yields that the languages of $N_1'$ and $N_2'$ can be separated by a regular separator with triply exponential state complexity.
    Finally, we apply the construction from the proof of \cref{Lemma:PNSeparabilityEliminatingEpsilon} to obtain separator for the languages of the original nets $N_1$ and $N_2$ with the desired triply exponential state complexity.
    This completes the proof.
\end{proof}

\end{document}
