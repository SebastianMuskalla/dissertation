\documentclass[../../diss.tex]{subfiles}
\begin{document}

% \chapter{Outline}

We give a brief overview of the structure of this thesis.
The thesis is partitioned into six parts, the first of which is the introduction that this chapter concludes.

\cref{Part:Prelims}.~consists of preliminaries.
It introduces the models of computation on which the results in the rest of the thesis are based.
It mostly presents results that have been established in the literature without contributions from the author of this thesis.

\cref{Part:Closures}.~presents our results on the closures of Petri net coverability languages.
These correspond to \cref{Section:IntroClosures} of the introduction.
We study  the computation of both the downward and the upward closure of Petri net languages as well as restricted versions of these problems.
We establish upper bounds on the size of the closures by providing algorithms, typically matching the lower bounds that we also present and prove.
The content of this part is based on the publication~\cite{AtigMMS17}.

\cref{Part:Separability}.~is concerned with results related to the regular separability of WSTS languages.
We have given a brief introduction to this topic and highlighted its relevance in \cref{Section:IntroSeparability} of the introduction.
We first establish some basic results on various classes of WSTS languages.
Then, we prove the main result, showing that disjoint WSTS languages are always separable under certain mild conditions.
Finally, we give an upper bound and lower bound for the size of the construction in the case of Petri net coverability languages.
The content of this part is based on the publication~\cite{CzerwinskiLMMKS18}.

\cref{Part:Games}.~presents our results on solving games.
We have explained how games are related to problems in verification and synthesis in \cref{Section:IntroGames} of the introduction.
After providing basic definitions for games, we describe a fixed-point based approach to solving verification problems.
This approach will be our vehicle for solving games defined by context-free grammars and higher-order recursion schemes afterwards.
We conclude the part by exploring the boundaries of the decidability of games.
To this end, we use valence systems over graph monoids as a model that provides a unified theory for various types of automata.
The content of this part is based on the publications~\cite{HolikMM16,HagueMM17,MeyerMZ18,MeyerMN17a}.

\cref{Part:Conclusion}.~forms the conclusion.
Firstly, we summarize the contributions of this thesis.
We make clear for each of the results whether it has been published before.
We also detail who, in addition to the author of this thesis, has contributed to establishing the result.
Secondly, we give an outlook on potential future work.
This includes extensions of the theoretical results as well as a brief discussion on how the results in this thesis could be applied in practice.

We will give a more detailed overview at the beginning of each part of the thesis.

\end{document}
