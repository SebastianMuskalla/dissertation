\documentclass[../../diss.tex]{subfiles}
\begin{document}

\section{Compositional verification and separability}%
\label{Section:IntroSeparability}%

In this section, we continue to consider the compositional verification of concurrent systems.
We have already explained why the verification of concurrent system is essential, but also rather difficult.
We came to the conclusion that in order to tame the complexity of the problem, a compositional approach is necessary.
In the following, we make our explanation more formal.
We will then explain how separability problems are related to the theoretic foundations of compositional verification.

\paragraph{A language-theoretic approach to compositional verification}

We start by explaining how the language of a concurrent system can be constructed, given the languages of its components.
Let $A = A_1 \parcomp A_2 \parcomp \ldots \parcomp A_k$ be a concurrent system that is obtained as the result of the parallel composition of the components $A_1$ to $A_k$.
We assume that each component $A_i$ has an associated alphabet $\Sigma_i$, a set of letters or atomic commands that this component may execute.
We consider a simple model in which two components $A_i, A_j$ synchronize on the actions in the shared alphabet $\Sigma_i \cap \Sigma_j$, while actions that are not shared can be executed independently.
We mimic this model by defining the parallel composition $\calL_i \parcomp \calL_j$ of two languages $\calL_i$ and $\calL_j$ over $\Sigma_i$ and $\Sigma_j$ as the set of words over $\Sigma_i \cup \Sigma_j$ such that if we project a word to one of the alphabets (by removing all letters not contained in that alphabet), we obtain a word in the corresponding language.
% We write this formally as
% \[
%     \calL_i \parcomp \calL_j = \Set{ w \in \paren{\Sigma_i \cup \Sigma_j}^*}{ \proj{\Sigma_i}{w} \in \calL_i, \proj{\Sigma_j}{w} \in \calL_j }
%     \ ,
% \]
% where the function $\proj{\Sigma}{w}$ removes from a word $w$ all letters not in alphabet $\Sigma$.
With this definition, the language of the parallel composition of components is the parallel composition of the languages of the components, $\lang{A_1 \parcomp A_2} = \lang{A_1} \parcomp \lang{A_2}$.
Given that the parallel composition operator is both associative and commutative, extending this to systems with more than two components is straightforward.

There are two special cases of the parallel composition of languages that are noteworthy.
The first is the case that the alphabets are equal, $\Sigma_1 = \Sigma_2$.
In this case, the projections leave the words unchanged, and the parallel composition is equal to the intersection $\calL_1 \cap \calL_2$.
The second interesting case is that the two alphabets are disjoint, $\Sigma_1 \cap \Sigma_2 = \emptyset$.
Then, the parallel composition is equal to the so-called shuffle of the languages, the set of all interleavings of words from $\calL_1$ and $\calL_2$, where two words from each of the languages are interleaved in an arbitrary order, but the order of letters within each of the two words is preserved.

One popular approach to composition verification is assume-guarantee reasoning~\cite{Lamport77,Jones83}.
Our presentation follows~\cite{GiannakopoulouNP18}, but we have adapted it to take a more language-theoretic approach.
Assume-guarantee reasoning is based on triples of the form $\agtriple{\psi}{A}{\varphi}$, where $A$ is a system and $\psi$ and $\varphi$ are specifications.
Such a triple is valid if any concurrent system containing $A$ as a component that satisfies $\psi$ also satisfies $\varphi$.
The name comes from the fact that under the assumption of $\psi$, one can guarantee $\varphi$.

To express an assume-guarantee triple on the level of languages, we make use of the fact that the language of a system containing $A$ as a component can be written as $\calL \parcomp \lang{A}$ for some suitable language $\calL$.
As explained above, satisfying a specification can be expressed as an inclusion.
We have to deal with the problem that the specification and the system may not use the same alphabet.
Under the assumption that $\Sigma_\varphi \subseteq \Sigma_A$, \ie the alphabet associated to the system is an extension of the alphabet used by the specification, we write $\lang{A} \sqsubseteq \lang{\varphi}$ to denote the inclusion $\proj{\Sigma_\varphi}{\lang{A}} \subseteq \lang{\varphi}$, where $\proj{\Sigma_\varphi}{-}$ is the projection that removes from a word all letters not contained in $\Sigma_\varphi$
With this notation at hand, we obtain that the triple $\agtriple{\psi}{A}{\varphi}$ is valid if and only if for every language $\calL$, $\calL \parcomp \lang{A} \ \sqsubseteq  \ \lang{\psi}$ implies $\calL \parcomp \lang{A} \ \sqsubseteq \ \lang{\varphi}$.

One can check whether a triple $\agtriple{\psi}{A}{\varphi}$ is valid by checking if the language $\lang{\psi} \parcomp \lang{A} \parcomp \overline{\lang{\varphi}}$ is empty~\cite{PasareanuGBCB08}.
Here, $\overline{\lang{\varphi}}$ is the complement of $\lang{\varphi}$, \ie the set of all words over $\Sigma_\varphi$ not contained in $\lang{\varphi}$.
As a start, we will assume that both the specifications and the systems under consideration are given as finite-state automata \resp regular languages.
For specifications, this makes sense: Regular languages are not only the class of languages associated to finite automata, but also expressive enough to model common specification logics like MSO and LTL~\cite{Pnueli77}.
For systems, limiting ourselves to finite automata is a heavy restriction.
We will come back to this aspect later.
%
Under the assumption that all components of a triple $\agtriple{\psi}{A}{\varphi}$ are given as finite-state automata, checking the validity of a triple can be automated.
To this end, one can compute an automaton for $\lang{\psi} \parcomp \lang{A} \parcomp \overline{\lang{\varphi}}$ and verify that its language is empty by conducting a reachability check.
However, computing this automaton may be very expensive.
Recall that the system $A$ is typically a concurrent system that is not given explicitly, but as a collection of its components.
If we construct a finite automaton representing~$A$, the whole procedure will suffer from the state explosion problem detailed in the \nb{previous section}.

\paragraph{A proof rule for compositional verification}

This is where compositional verification comes to our aid.
The compositional approach to checking the validity of assume-guarantee triples is based on proof rules.
The goal is to establish the validity of an assume-guarantee triple for a complex system under the premise that the validity of assume-guarantee triples for simpler systems have already been shown.
Here, we focus on the following simple but powerful proof rule:
\[
    \begin{array}{l}
        \agtriple{\true}{A}{\psi}
        \\
        \agtriple{\psi}{B}{\varphi}
        \\
        \hline
        \agtriple{\true}{A \parcomp B}{\varphi}
        \ .
    \end{array}
\]
This rule states that if the two premises, namely the validity of the triples $\agtriple{\true}{A}{\psi}$ and $\agtriple{\psi}{B}{\varphi}$ have been proven, then we obtain the validity of the triple $\agtriple{\true}{A \parcomp B}{\varphi}$.
Phrased differently, in order to establish that $A \parcomp B$ satisfies $\varphi$ in any environment and without precondition\footnote{%
        We see $\true$ as the specification with respect to which every execution is legal, \ie as the language of all words over a suitable alphabet \resp a finite automaton generating this language.
    }%
    , it is sufficient to show that $A$ satisfies $\psi$ in any environment and that $B$ satisfies $\varphi$ in any environment in which it satisfies $\psi$.
The rule enables compositional verification: It decomposes checking the validity of a triple for a parallel composition into two triples, one for each of the two components.
It is not hard to see that the rule is sound, meaning that the premises indeed imply the conclusion.
Assume that the two premises hold, and let $\calL$ be any language. We have to show that $\calL \parcomp \lang{A \parcomp B} \ \sqsubseteq \ \lang{\varphi}$.
We use the first premise, instantiating the environment for $A$ with $\calL \parcomp \lang{B}$, and obtain that the composition satisfies $\psi$, $\paren { \calL \parcomp \lang{B}} \parcomp  \lang{A}  \ \sqsubseteq \ \lang{\psi}$.
Using the associativity and commutativity of parallel composition, the language $\paren { \calL \parcomp \lang{B}} \parcomp  \lang{A}$ can be rewritten as $\paren { \calL \parcomp \lang{A}} \parcomp  \lang{B}$.
We may see $\calL \parcomp \lang{A}$ as an environment for $B$ with which together it satisfies $\psi$.
This allows us to apply the second premise, yielding that the composition also satisfies $\varphi$.
Finally, we rewrite $\paren { \calL \parcomp \lang{A}} \parcomp  \lang{B}$ as $\calL \parcomp \lang{A \parcomp B}$, using commutativity and the correspondence between the parallel composition of languages and the parallel composition of automata.
We obtain $\calL \parcomp \lang{A \parcomp B} \ \sqsubseteq  \ \lang{\varphi}$ as desired.

Surprisingly, the rule is not just sound, but also complete.
To make this notion precise, note that the specification $\psi$ occurring in the premises does not occur in the conclusion.
Hence, if we start with the goal of proving $\agtriple{\true}{A \parcomp B}{\varphi}$, we are free to choose the specification $\psi$.
As long as our choice of $\psi$ leads to the triples in the premises being valid, it is suitable for instantiating the proof rule.
This means that $\psi$ plays the role of a certificate.
Given an appropriate $\psi$, it is easier to check $\agtriple{\true}{A \parcomp B}{\varphi}$ by using the proof rule and checking the triples $\agtriple{\true}{A}{\psi}$ and $\agtriple{\psi}{B}{\varphi}$ than it would be to check the triple for the composition itself.
Also note that conceptually, $\psi$ is an abstraction of the behavior of $A$ in an arbitrary environment.
On the one hand, this abstraction should be coarse in the sense that the finite automaton representing it is smaller than $A$.
The difficulty of automatically checking the validity of a triple as explained above does not only scale with the size of the system, but also with the size of the two specifications.
Hence, the smaller the representation of $\psi$ is, the easier it will be to check the triples $\agtriple{\true}{A}{\psi}$ and $\agtriple{\psi}{B}{\varphi}$.
On the other hand, the abstraction has to be precise enough so that the two triples are actually valid.
In a sense, $\psi$ is the abstract description of $A$ that we use to \nb{verify $B$}.
If it is too imprecise, we will not be able to complete our goal of establishing the property~$\varphi$.

When we say that a proof rule is complete, we mean that whenever the conclusion is true, then there is an instantiation of the premises that allows us to apply the rule.
In our example, this means that if $\agtriple{\true}{A \parcomp B}{\varphi}$ is true, then there is a specification $\psi$ so that the triples $\agtriple{\true}{A}{\psi}$ and $\agtriple{\psi}{B}{\varphi}$ from the premises are valid.
In fact, this specification can be chosen to be simply~$A$.\footnote{%
    Assume that $\agtriple{\true}{A \parcomp B}{\varphi}$ is valid.
    We claim that then $\agtriple{\true}{A}{A}$ and $\agtriple{A}{B}{\varphi}$ are valid.
    For the validity of $\agtriple{\true}{A}{A}$, note that the composition $\calL \parcomp \lang{A}$ of $A$ with any language $\calL$ will satisfy $\calL \parcomp \lang{A} \ \sqsubseteq \ \lang{A}$ by the definition of the parallel composition.
    For the validity of $\agtriple{A}{B}{\varphi}$, let $\calL$ be any language so that $\calL \parcomp \lang{B} \ \sqsubseteq \ \lang{A}$.
    Take any word $w \in \calL \parcomp \lang{B}$ and note that we have both $\proj{\Sigma_B}{w} \in \lang{B}$ and $\proj{\Sigma_A}{w} \in \lang{A}$.
    Hence, $\proj{\Sigma_B \cup \Sigma_A}{w}$ is an element of $\lang{A \parcomp B}$.
    Since we assumed $\agtriple{\true}{A \parcomp B}{\varphi}$ to be valid, this implies $\proj{\Sigma_\varphi}{\proj{\Sigma_B \cup \Sigma_A}{w}} = \proj{\Sigma_\varphi}{w} \in \lang{\varphi}$.%
}
Note that this completeness only applies under the assumption that both the components and the specifications are given in the form of finite automata.
The fact that our simple proof rule is complete means that conceptually, compositional verification is always possible.
However, the existence of an appropriate specification $\psi$ does not mean that it is easy to find one that is not unsuitably large.
Recall that the validity of a triple $\agtriple{\psi}{B}{\varphi}$ is equivalent to the emptiness of $\lang{\psi} \parcomp \lang{B} \parcomp \overline{\lang{\varphi}}$.
Instantiating this result for $\agtriple{\psi}{B}{\varphi}$ and for $\agtriple{\true}{A \parcomp B}{\varphi}$ yields that if $\psi$ is similar in size to $A$, then checking $\agtriple{\psi}{B}{\varphi}$ is not easier than checking $\agtriple{\true}{A \parcomp B}{\varphi}$.
Hence, the proof rule only leads to a substantial reduction in running time if we are able to find a specification $\psi$ that is substantially smaller than $A$.
While the existence of a specification $\psi$ that leads to the premises becoming true is guaranteed, the existence of a small representation for a suitable $\psi$ is not guaranteed.
In fact, complexity-theoretic reasons have led us to believe that it cannot exist in some cases.

The commonly held belief is that the more loosely two components are coupled in a concurrent system, the easier it is to find a specification for their interface that is sufficient for verification.
In the literature, numerous approaches to compositional verification based on assume-guarantee reasoning have been developed.
Notably, there is a learning algorithm that tries to compute the certificate $\psi$ on the fly~\cite{CobleighGP03,PasareanuGBCB08}.
If the triple that should ultimately be proved is indeed valid, this algorithm is guaranteed to terminate with the smallest deterministic automaton representing $\psi$ so that the above proof rule can be applied.
Additionally, there is research on proof rules that work \eg for more than just two components~\cite{PasareanuGBCB08}.

\paragraph{Compositional verification and separability}

All the aforementioned results focus on the simple case that the components and specifications are given as finite automata.
Our goal in the following will be to go beyond finite automata and regular languages.
Fundamentally, the question we want to answer is whether compositional verification is possible for more general types of systems.
To this end, we apply a sequence of simplifications to the problem at hand.
The result is a theory that allows us to clearly formulate the question of whether compositional verification is possible.
While the resulting theory will not immediately lead to an approach to compositional verification that is usable in practice, it can hopefully be extended towards one in the future.

We focus on a triple $\agtriple{\true}{A \parcomp B}{\varphi}$.
The first simplification that we apply is that we discard the idea of having an arbitrary environment.
We are interested in the system $A \parcomp B$, and whenever we focus on one of the two components, the other one will be the environment.
Consequently, we can discard the universal quantification over a language in the definition of the validity of a triple.
Instead of asking whether $\calL \parcomp \lang{A} \parcomp \lang{B} \ \sqsubseteq \ \lang{\varphi}$ for all languages $\calL$, we simply want to check $\lang{A} \parcomp \lang{B} \ \sqsubseteq \ \lang{\varphi}$.

The second simplification is that we assume that $A$, $B$, and $\varphi$ all use the same alphabet $\Sigma$.
This means that $A$ and $B$ synchronize on every action, and not just on the actions represented by letters in a subset of their alphabets.
Luckily, this property can usually be enforced rather easily without changing the behavior of the system.
For each letter of $B$ that was not in the shared alphabet, we introduce appropriate transitions in $A$ that correspond to this letter without actually changing the internal state of $A$.
The resulting system behaves as $A$ does, even when composed with $B$, but uses an extended alphabet.
We apply the same procedure to $B$ and $\varphi$ to make sure that all three objects use the same alphabet.
As mentioned before, the parallel composition of the languages of systems that use the same alphabet is simply the intersection of these languages.
Hence, $\lang{A} \parcomp \lang{B}$ equals $\lang{A} \cap \lang{B}$.
Additionally, we can replace our variant $\sqsubseteq$ of the subset relation that involves a projection by the normal inclusion relation $\subseteq$.
In total, our goal is now checking $\lang{A} \cap \lang{B} \ \subseteq \ \lang{\varphi}$.

Finally, we transform this expression into a form that is even simpler.
The inclusion $\lang{A} \cap \lang{B} \subseteq \lang{\varphi}$ holds if and only if $\paren{ \lang{A} \cap \lang{B} } \cap \overline{\lang{\varphi}}$ is empty, where $\overline{\lang{\varphi}}$ is the complement language of $\lang{\varphi}$.
We rewrite this to $\paren{ \lang{A} \cap \overline{\lang{\varphi}}} \cap \lang{B}$.
Recall that the \nb{language $\lang{\varphi}$} is \nb{regular -- while} we are interested in systems that are more expressive than finite automata, regular languages are usually sufficient to represent the specifications we are interested in.
By the closure properties of regular languages, also the complement language $\overline{\lang{\varphi}}$ is regular.
Additionally, most other language classes are well-behaved when it comes to intersections with regular languages:
Intersecting a language from such a class with a regular language leads to a language that is again in that class.\footnote{%
    Using the nomenclature from the theory of abstract families of languages (AFLs)~\cite{Berstel79}, many classes of languages are \emph{trios}, which in particular means that they are closed under intersection with regular languages.
    This applies to both the context-free~\cite{Berstel79} and the Petri net languages~\cite{Jantzen87}.%
}
For example, given a pushdown automaton and a finite automaton, it is easy to construct a pushdown automaton whose language is the intersection of the two associated languages.
The same is true for Petri nets.
Ultimately, this means that we can construct a system $A'$ whose language is $\lang{A} \cap \overline{\lang{\varphi}}$.
After applying this construction, checking $\lang{A} \cap \lang{B} \subseteq \lang{\varphi}$ amounts to checking $\lang{A'} \cap \lang{B} = \emptyset$.

We have argued before that checking whether a system satisfies a specification means checking whether the intersection of the possible and the illegal executions is empty.
In that case, we usually consider the intersection of a complicated language, the language of possible executions of a system, with a simple language, the language of illegal executions according to a specification.
In the case of checking $\lang{A'} \cap \lang{B} = \emptyset$, we are dealing with a setting that is more involved.
The language of $A'$ is the set of executions that are possible with respect to component $A$ while being illegal with respect to specification $\varphi$.
Both $\lang{A'}$ and $\lang{B}$ are languages of systems that are more complicated than finite automata.

How can we check $\lang{A} \cap \lang{B} = \emptyset$?
(Here and in the following, we drop the notation $A'$ for the sake of simplicity, since $\lang{A}$ and $\lang{A'}$ come from the same class of languages.)
Explicitly constructing an automaton whose language is $\lang{A} \cap \lang{B}$ is not a good idea.
In some cases, it may not even be possible.
For example, the intersection of two languages of pushdown automata is not the language of a pushdown automaton in general.
Even if it is possible, it will suffer from the same state explosion problem that we have explained in the case of finite automata.
To solve this problem, we need a compositional approach.

We propose the following \emph{separability proof rule} as a possible solution:
\[
    \begin{array}{l}
        \lang{A} \subseteq \calR
        \\
        \calR \cap \lang{B} = \emptyset
        \\
        \hline
        \lang{A} \cap \lang{B} = \emptyset
        \ .
    \end{array}
\]
Let us first argue that this rule is analogous to the compositional proof rule for assume-guarantee triples.
In the conclusion, the validity of the triple $\agtriple{\true}{A \parcomp B}{\varphi}$ has been replaced by $\lang{A} \cap \lang{B} = \emptyset$ as explained.
Similarly, in the second premise the property of satisfying a specification $\varphi$ has been replaced with the emptiness of a language.
The most interesting part is the language $\calR$, which replaces the specification $\psi$ in both the first and the second premise.

In order to understand the role of $\calR$, we prove the soundness of the proof rule.
Assume there is a language $\calR$ so that both premises are true.
This means that by the first premise, $\calR$ is bigger than $\lang{A}$.
Additionally, $\calR$ and $\lang{B}$ are disjoint by the second premises.
But if $\lang{B}$ has an empty intersection with a language that is bigger than $\lang{A}$, then also $\lang{A}$ and $\lang{B}$ need to be disjoint and the conclusion is true.
A language $\calR$ that makes the premises become true is called a \emph{separator}.
It is a certificate for the disjointness of $\lang{A}$ and $\lang{B}$.

The question that remains is whether the separability proof rule is complete.
Is it true that whenever two languages are disjoint, a separator has to exist?
The answer to this question depends on the language classes that we consider.
If $\lang{A}$, $\lang{B}$, and $\calR$ come from the same class of languages, completeness obviously holds.
If $\lang{A}$ and $\lang{B}$ are disjoint, then $\lang{A}$ itself is a separator.
However, in that case, we have gained nothing.
While the first premise trivially holds, checking the second premise simply amounts to checking the desired conclusion.
Even in the case that all languages come from the same class, however, it would be interesting to ask whether there is a separator that has a smaller description than $A$.
This is closely related to the observations we made in the case of finite automata.

\paragraph{Regular separability}

We will focus on the case that the separator is required to come from a class that is less expressive than the languages $\lang{A}$ and $\lang{B}$.
To be precise, we will exclusively consider regular separators in this thesis.
This is motivated by two facts.
The first fact is the aforementioned correspondence of regular languages to specification logics.
Secondly, regular languages are very well-behaved when it comes to closure properties and algorithmics.
This means that for a candidate separator that is a regular language, it is highly likely that the two premises of the separability proof rule can indeed be automatically checked.
This goes along with our theme of being interested in certificates that are easier to verify than the initial problem is to solve.

In the following, we consider \emph{regular separability}.
Is it true for a class of languages that whenever two languages are disjoint, a regular separator exists?
If that is not true, can one decide whether a separator exists?
If a separator exists, can it be computed?
For context-free languages, the class of languages defined by pushdown automata, it turns out that separability is difficult.
Given two languages and a regular candidate separator, we can automatically check whether the two premises of the separability proof rule hold.
In that case, the soundness of the rules implies that the two given context-free languages are disjoint.
However, it is easy to construct two context-free languages so that no regular separator can exist.
Additionally, the problem of deciding whether a separator exists has turned out to be undecidable~\cite{SzymanskiW76}.
In the literature, it has been established that this behavior does not only apply to context-free languages, but also to restricted versions of context-free languages that are only minor extensions of the regular languages, \eg the languages of one-counter automata~\cite{CzerwinskiL17} and visibly pushdown automata~\cite{Kopczynski16}.

\paragraph{Outlook}

In Part~\ref{Part:Separability}.~of this thesis, we focus on languages of well-structured transition systems (WSTSes).
This class of languages is a superclass of the class of Petri net languages.
We will show that under mild conditions, the separability proof rule is complete for WSTS languages.
Whenever two WSTS languages are disjoint, there is a regular separator as a certificate for this disjointness.

To be more precise, our main result is that any two disjoint WSTS languages, one of them the language of a deterministic WSTS, are regularly separable.
Requiring one of the WSTSes to be deterministic seems like a substantial restriction.
However, we will establish a hierarchy of subclasses of languages within the class of WSTS languages and show that any WSTS whose transition relation is finitely branching or whose underlying order is $\omega^2$ can be determinized.
Almost all WSTSes that are of practical relevance satisfy at least one of these two requirements.
Hence, our result on the separability of WSTS languages can be applied to almost all WSTSes.

Additionally, one can extract from the proof of the main result an algorithm for computing a finite automaton representing the regular separator.
We will demonstrate this in the case of Petri net languages, obtaining a construction for the separator and an upper bound on its size, which we will accompany by a lower bound.

\end{document}
