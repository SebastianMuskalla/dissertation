\documentclass[../../diss.tex]{subfiles}
\begin{document}

\section{Deterministic target languages}%
\label{Section:CFGamesDeterministic}%

We consider the special case of context-free regular inclusion games where the regular target language is given by a deterministic automaton.
It can be show that solving such games is $\EXPTIME$-complete, instead of $2\EXPTIME$-complete as in the case of an NFA representing the target language.
We modify our algorithm to achieve a matching \nb{running time}.

To show $\EXPTIME$-completeness, on can for example use that the context-free games on the configuration graphs of pushdown automata that were considered by \citea{Walukiewicz01} and \citea{Cachat02} can be translated into a grammar-based context-free game with a DFA target language within polynomial time.
Hence, \Walus proof of $\EXPTIME$-completeness for this type of game applies.

One also can immediately see that our proof of $2\EXPTIME$-hardness, \cref{Theorem:CFGamesComplexityHardness}, does not work.
Dealing with configurations of an alternating Turing machine with exponential space consumption requires dealing with binary strings of polynomial length.
An automaton of polynomial size for this task is necessarily nondeterministic.
One could adapt the proof to the case of a deterministic target language as follows:
We assume that the given alternating Turing machine has polynomial space consumption, using that $\APSPACE = \EXPTIME$.
Hence, the binary strings indexing cells on the tape are only of logarithmic size.
The logarithmic nondeterministic automaton dealing with these strings can be transformed into a polynomially sized deterministic one to complete the proof.

Unfortunately, the algorithm that we have outlined in this chapter cannot exploit the fact that the target language is deterministic to achieve a better running time.
The crucial factor in the running time of the algorithm is the size of transition monoid $\nfatransmonoid{A}$, the number of boxes.
If automaton $A$ is a DFA, then there are fewer potential boxes:
For each of states, there is only a single successor state along each word.
Hence, there are at most $\card{Q}^{\card{Q}} = 2^{\log \card{Q} \cdot \card{Q}}$ different boxes in  $\nfatransmonoid{A}$.
This number is substantially smaller than $2^{\card{Q}^2}$, the number of boxes in the general case, but it still is exponential.

To overcome this problem, we propose a variation of the algorithm that uses $\pBF(Q)$ as domain.
Instead of formulas over the transition monoid, where the atoms track the full behavior of words in the automaton, we have formulas over $Q$, where each atom corresponds to a target state.
To make this approach work, we also need to fix the source state.
Formally, we have a system of equations with variables of the shape $qX$, where $q$ is a (source) state and $X$ is a nonterminal.
An atom $p$ in the formula $\sol{qX}$, taken from the least solution to the system, corresponds to a word $w$ derivable from $X$ with $q \tow{w} p$ in $A$.

However, this introduces a new challenge.
Recall that the composition of two formulas $\sol{Y} \comp \sol{Z}$ is defined so that it captures the structure of words derivable from $Y.Z$:
Each such word is a word derivable from $Y$ followed by a word derivable from $Z$.
When using formulas over the transition monoid, we can simply compose the atoms corresponding to these words using relational composition.
With the modified system of equations, $\sol{qY} \comp \sol{pZ}$ only makes sense if we derive from $Y$ a word that indeed leads to target state $p$.
However, the words derivable from $Y$ can lead to many different target states, and a priori it is not clear to which ones.

The solution to this problem is to compose $\sol{qY}$ not with a single $\sol{pZ}$, but with the family of formulas ${(\sol{pZ})}_{p \in Q}$ for all $p \in Q$ simultaneously.
The composition will be defined so that it first resolves the operators in $\sol{qX}$ until arriving at an atom $p$.
It then selects the appropriate formula $\sol{pZ}$ from the family to complete the composition.

Formally, we define a new composition operator $\supercomp$\ .
It is $(\card{Q} + 1)$-ary, but for the sake of readability we write the last $\card{Q}$ arguments as a family of formulas.
This means we write $F \supercomp {(H_q)}_{q \in Q}$ where $F$ and $H_q$ for every $q \in Q$ are formulas.
The definition is as follows:
%
\begin{align*}
    (F \wedgevee F') \supercomp {(H_q)}_{q \in Q} &= F \supercomp {(H_q)}_{q \in Q} \wedgevee F' \supercomp {(H_q)}_{q \in Q}
    \\
    p \supercomp {(H_q)}_{q \in Q} &= H_p
    \ .
\end{align*}
%
The operation essentially replaces each atom $p$ in the first formula by $H_p$.

With this definition at hand, we can formally define the system of equations.
As variables, we use expressions of the shape $qX$ where $q \in Q$ and $X \in N$.
Assume that $X \to \itr{\beta}{1} \mid \ldots \mid \itr{\beta}{k}$ are all production rules for nonterminal $X$.
For each state $q$, the defining equality for $qX$ is
\[
    qX = \text{preprocess}_q(\itr{\beta}{1}) \wedgevee \ldots \wedgevee \text{preprocess}_q(\itr{\beta}{k})
    \ .
\]
As usual, $\wedgevee$ is disjunction if the owner of $X$ is the existential player and conjunction otherwise.
The operation $\text{preprocess}_q$ is defined on sentential forms $\beta$ as follows:
If $\beta$ consists of a single symbol, $\beta \in \N \cup \Sigma \cup \set{\eps}$, we have
\[
    \text{preprocess}_q(\beta) = q\,\beta
    \ .
\]
If $\beta = \beta_1.\beta_2 \ldots \beta_m$, we prefix the first symbol with $q$, replace the rest of the symbols with the corresponding families, and connect the results using the new composition operator,
\[
    \text{preprocess}_q(\beta_1.\beta_2 \ldots \beta_m) = q\,\beta_1 \supercomp {(p \, \beta_2)}_{p \in Q} \supercomp \ldots \supercomp {(p \, \beta_m)}_{p \in Q}
    \ .
\]

The system of equations uses disjunction, conjunction, the new composition operator and constants of the shape $q \, a$ for $q \in Q$ and $a \in \Sigma \cup \set{\eps}$ as function symbols.
We provide a model over which the system can be interpreted.
The domain is $\pBF(Q)$ factored by logical equivalence, \ie equivalence classes of positive Boolean formulas over states of the DFA, ordered by implication.
The interpretation of conjunction, disjunction, and the new composition operator is as expected.
For each state $q$, the interpretation of $q \eps$ is $q \in \pBF(Q)$.
For state $q$ and terminal $a \in \Sigma$, $q \,  a$ is $p \in \pBF(Q)$, where $p$ is the unique state of the DFA with $q \tow{a} p$.

Given an instance $(G,A)$ of a context-free regular inclusion game with a DFA $A$ representing the target language, one solves the game as follows.
We first construct the system of equations, and then compute its least solution using the above interpretation.
The least solution provides us with a formula $\sol{ \qinit S} \in \pBF(Q)$ for the initial state of $A$ and the initial symbol of $G$.
We consider the variable assignment $Q_{\text{rej}} = Q \setminus \QF \subseteq Q$ that sets to true the non-final states.
The existential player has a winning strategy for the game iff $\sol{ \qinit S}(Q_{\text{rej}})$ \nb{evaluates to true}.

To show this, one can prove a correspondence between variable assignments $Q' \subseteq Q$ that satisfy a formula $\sol{qX}$ and strategies for the existential player that enforce the derivation of a word $w$ from $X$ such that $q \tow{w} q'$ with $q' \in Q'$.
To make this formal, the theory that we have developed in \cref{Section:CFGamesSoundness} can be adapted easily.

The resulting algorithm solves the game in exponential time, which is the optimal complexity.
The new system of equations is substantially larger as it features one equation per state and nonterminal, but its size remains polynomial.
For solving the interpreted system of equations, using $Q$ as the set of atoms is the determining factor for the complexity.
This means that the height of the domain as well as the maximum size of a single formula in conjunctive normal form are singly exponential.
This can be made formal by adapting the proof of \cref{Proposition:CFGamesComplexityMembership}.

Altogether, we have shown the following.

\begin{theorem}
    Context-free regular inclusion games with the target language represented by a DFA are $\EXPTIME$-complete,
    and an algorithm based on effective denotational semantics solves them in exponential time.
\end{theorem}

When we are given a context-free regular inclusion game $(G,A)$ where $A$ is an NFA, we now have two procedures with doubly exponential running time for solving it.
The first is the algorithm that we have discussed earlier in this the chapter.
The second procedure  starts by determinizing the automaton, obtaining a new instance $(G,A^{\text{det}})$ that is potentially exponentially larger.
However, we can now apply the algorithm described in this section to solve this larger instance in singly exponential time (in its size), resulting in doubly exponential time overall.

While both algorithms solve the problem, we expect the first approach to be much more efficient for the same reasons that we have discussed in \cref{Section:CFGamesAlgorithmics}.
An upfront determinization of the automaton introduces a guaranteed blowup, even if some parts of the automaton may actually never be used by words that can be derived in the grammar.
The first approach uses an on-the-fly determinization that may avoid this blowup.

\end{document}
