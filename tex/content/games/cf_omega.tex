\documentclass[../../diss.tex]{subfiles}
\begin{document}

\section{\texorpdfstring{$\omega$}{Omega}-context-free inclusion games}%
\label{Section:CFGamesOmega}%

We complete our study of context-free games by studying $\omega$-context-free $\omega$-regular inclusion games.
In the case of context-free inclusions games with a regular language of finite words as the target language, all infinite plays are won by the universal player.
In order to be able to win, the existential player had to be able to enforce the termination of any play.
Here in this section, we will take the opposite approach.
We define the winning condition so that the universal player wins all plays that do terminate.
For plays that are infinite, the winner is determined by whether the play corresponds to an infinite word in the given $\omega$-regular target language.
To be precise, the existential player wins if the play corresponds to a right-infinite left-derivation process that produces an infinite word that is not contained in the target language.

\paragraph{Formal definition}

Formally, an $\omega$-context-free ($\omega$-regular) inclusion games is given by a context-free game grammar $G = (N_\allplayer \dotcup N_\explayer,P,S)$ and a nondeterministic Büchi automaton $A = (Q, \delta, \qinit, \QF)$.
The game arena is defined as for context-free regular inclusion games in \cref{Section:CFGamesDefinition}:
Starting from the initial symbol, a play is a left-derivation process in which each player chooses the production rules applied to the nonterminals owner by her.
They key difference is that the winning condition for the existential player is now non-inclusion in an $\omega$-regular language.
To this end, she has to enforce that the play of the game corresponds to a right-infinite (left-)derivation process as defined in \cref{Section:CFGOmega}.
Recall that this means that the play is a sequence of sentential forms that have a nonterminal as their rightmost symbol which is replaced infinitely often.
Consequently, the derivation of all other nonterminals is a finite process.
The existential player wins the play if and only if it constitutes such a right-infinite derivation process, deriving an infinite word $w$ not contained in $\olang{A}$.
In all other cases, the universal player wins:
This applies when the play is a right-infinite derivation process deriving a word from $\olang{A}$, but it also applies if the play is finite, or if it is infinite, but it is not a right-infinite derivation process, or if it is a right-infinite derivation process that does not derive an infinite word.
The latter case could occur if starting at some point, the terminal prefix of the sentential form stops growing.
Our hope is to extend the techniques for context-free games in the same way that we extended our algorithm for context-free regular inclusion to $\omega$-context-free $\omega$-regular inclusion in \cref{Section:EDSOmegaRegIncl}.
To do so, there are two major challenges.
The first lies in the fact that the automaton representing the $\omega$-regular target language is nondeterministic.
The second one is that least fixed-points as used in effective denotational semantics seem to be insufficient to capture the semantics of this type of game.
We discuss each of the problems in detail and present our solutions.

\paragraph{Determinizing the automaton}

We have explained in \cref{Chapter:Games} that a nondeterministic automaton representing the target language poses a problem, as it essentially introduces a third player that can be merged with neither of the other two players without changing the semantics of the game.
In the case of context-free inclusion games, we have overcome this problem by using the transition monoid.
We have relied on the fact that using the transition monoid is a way of implicitly determinizing the automaton.
The same is not true for nondeterministic Büchi automata:
Firstly, we cannot determinize an NBA into a deterministic Büchi automaton in general.
Secondly, even by considering a more involved acceptance condition for which deterministic automata are sufficiently expressive, like the parity or Muller acceptance conditions, considering the transition monoid is not enough.
Unlike in the case of automata on finite words, in general there is no deterministic automaton that uses transition monoid elements (boxes) as states that is equivalent to a given NBA.\@
We have used in \cref{Section:EDSOmegaRegIncl} that infinite words can be represented by pairs of boxes $\tau\rho^\omega$.
However, during the derivation of a word, it is impossible to decide on-the-fly which pair of boxes $\tau\rho^\omega$ is the correct one and where the split between the $\lang{\tau}$ and the $\lang{\rho}^\omega$ part happens.
In the case of a verification problem, we could overcome this by guessing nondeterministically.
In the case of a game, this mechanism is invalid:
We cannot let one of the players guess for the same reasons that cannot let one of players pick the moves of the automaton (as explained in \cref{Chapter:Games}).
-
Our solution to this problem is an upfront determinization of the automaton, the very thing that we successfully avoided in the case of context-free inclusion games.
In the rest of this section, we assume that the automaton representing the target language is a deterministic parity automaton $A = (Q, \Sigma, \delta, \qinit, \Omega)$.
Using a famous result by \citea{Safra88}, a given NBA with $n$ states can be transformed into a DPA with $2^{\bigO{n \cdot \log n}}$ states.
These states are so-called Safra-trees, and the construction is a variant of the powerset construction that is rather involved.
We also assume that the priority function $\Omega \colon Q \to \N$ does not assign the priority $0$ to any state, as we will need this priority later for the empty word.
This condition can be easily enforced by incrementing all priorities by $2$.
Let $d$ be the largest priority assigned to any state of the automaton in the following.

\paragraph{Solving finite subgames}

Before actually considering the $\omega$-context-free game, we set up a system of equations whose least solutions characterizes the effect of finite words on the parity automaton.
We re-use the system that we have developed in the previous section for context-free games with the target language being represented by a DFA.\@
The only modification that is needed is tracking the priorities in the automaton.
Formally, we consider formulas in $\pBF(Q \times \zeroto{d})$ whose atoms are tuples of a target state and the largest priority that has been seen.

We quickly recall the construction.
For each state $q$ and each nonterminal $X$, there is a variable $qX$.
Its defining equation is
\[
    qX = \text{preprocess}_q(\itr{\beta}{1}) \wedgevee \ldots \wedgevee \text{preprocess}_q(\itr{\beta}{k})
    \ .
\]
Here, $X \to \itr{\beta}{1} \mid \ldots \mid \itr{\beta}{k}$ are all production rules for nonterminal $X$, and $\wedgevee$ is disjunction if and only if the owner of $X$ is the existential player.
The operation $\text{preprocess}_q$ applied to a sentential form $\beta_1.\beta_2 \ldots \beta_m$ prefixes the first symbol by $q$, replaces all other symbols $\beta_i$ by the family ${(p\,\beta_i)}_{p \in Q}$, and connects them using the composition operator $\supercomp$\ .
Formally, we define
\[
    \text{preprocess}_q(\beta_1.\beta_2 \ldots \beta_m) = q\,\beta_1 \supercomp {(p \, \beta_2)}_{p \in Q} \supercomp \ldots \supercomp {(p \, \beta_m)}_{p \in Q}
    \ .
\]

This system is interpreted over (equivalence classes of) positive Boolean formulas from $\pBF(Q \times \zeroto{d})$, ordered by implication.
Conjunction and disjunction are interpreted in the expected way.
For each terminal $a \in \Sigma$ and each state $q \in Q$, the interpretation of the constant $q \, a$ is the formula just consisting of the atom $(p,j)$, where $p$ is the unique state with $q \tow{a} p$ in the automaton, and $j = \max \set{ \Omega(q), \Omega(p)}$.
For any state $q$, the interpretation of the constant $q \, \eps$ is the atom $(q,0)$.
Our rationale behind using priority $0$ here (which by assumption is not assigned to any state) is that in the $\omega$-context-free game, the existential player has to guarantee the derivation of an infinite word.
If the game derives an infinite sequence of $\eps$ starting at some point, the result of the play is not an infinite word, and the play is won by the universal player.
Therefore, we should assign an even priority to $\eps$, since even priorities correspond to words that are accepted by the automaton (which is good for the universal player).
However, the derivation of infinitely many $\eps$ is only problematic if no non-$\eps$ word is derived infinitely \nb{often -- therefore}, we assign the lowest even priority to $\eps$.

The composition operator $\supercomp$ is a $(\card{Q}+1)$-ary operator that composes a formula $F$ with a family of formulas ${(H_q)}_{p \in Q}$.
Compared to the previous section, the definition is altered so that it keeps track of the maximal priority.
We inductively define $F \supercomp {(H_q)}_{p \in Q}$ to be
\begin{align*}
    (F \wedgevee F') \supercomp {(H_q)}_{q \in Q} &= F \supercomp {(H_q)}_{q \in Q} \wedgevee F' \supercomp {(H_q)}_{q \in Q}
    \\
    (p,j) \supercomp {(H_q)}_{q \in Q} &= \text{max}_j \, H_p
    \ ,
\end{align*}
where $\text{max}_j\, H_p$ is defined by
\begin{align*}
    \text{max}_j\, (H \wedgevee H') &= \text{max}_j H \wedgevee \text{max}_j H'
    \\
    \text{max}_j\, (s,j') &= (s,\max \set{j,j'})
    \ .
\end{align*}
The intuition behind the definition is as follows:
Assume the automaton is in state $q$, and we derive a finite word from the sentential form $Y.Z$.
This means we first derive a finite word from~$Y$, tracking the target state $p$ and the maximum priority $j$ that has occurred during the run of the automaton from $q$ to $p$.
We then derive a word from $Z$ while tracking the target state $s$ of this word from the source state $p$ and the maximum priority $j'$ along the way.
The information of interest for the concatenation of these words is the target state $s$ and the maximum priority $\max \set{j,j'}$ seen along the run from $q$ to $s$.

With the interpretation fixed, we can solve the system of equations.
We obtain for each pair $qX$ of state and nonterminal a formula $\sol{qX} \in \pBF(Q \times \zeroto{d})$.
The atoms $(p,j)$ in that formula correspond to finite words derivable from $X$ that introduce a run from $q$ to $p$ with maximum priority $j$ in the automaton.


\paragraph{Solving the $\omega$-game}

It remains to use the information about the finite subgames to solve the $\omega$-context-free game.
Recall that a derivation of an infinite word in the language of an $\omega$-context-free grammar can be seen as a nested process consisting of an infinite path in the spinal graph of the grammar (as defined in \cref{LTS:OmegaAutomata}) and a sequence of finite derivations from the symbols occurring as transition labels along that path.
Similarly, a play of an $\omega$-context-free game consists of an infinite play in the spinal graph constructing an infinite sentential form, and a sequence of plays from the symbols of that sentential form.
In order to win, the existential player has to enforce that the latter are finite, \ie they are plays of a context-free game.

We use this insight to construct a parity game on a finite arena that is induced by the spinal graph and the parity automaton representing the target language.
For each nonterminal $X$ and state $q$ of the DPA, $(X,q)$ is a position of the game.
The initial position is $(S,\qinit)$, consisting of the initial symbol and the initial state.
When the parity game that we construct is in some position $(X,q)$, it proceeds as follows.
Firstly, the owner of $X$ picks some production of the grammar for $X$.
If this production is not of the shape $X \to \beta.Y$, \ie its rightmost symbol is not a nonterminal, the play does not represent a right-infinite left-derivation process and the universal player wins immediately.
Else, we consider the formula $F = \sol{ \text{preprocess}_q{\beta} }$ that describes the behavior of the finite play from $\beta$ with respect to the initial state $q$.
Here, we have used the $\text{preprocess}_q$ operator to insert the states of the automaton into the sentential form.
Secondly, we let the players resolve the operators in $F$ in the expected way: The existential player resolves disjunctions, the universal player resolves conjunctions.
Finally, the play arrives at an atom of the shape $(p,j)$.
From this position, the play continues at $(Y,p)$, the position formed by the rightmost nonterminal of the production and the state that was reached by the finite word that was derived from $\beta$.
The priority of each position in the parity game is $0$, unless the position is of the shape $(p,j)$, in which case it is $j$.

A play of the parity game in which only transitions of the shape $X \to \beta.Y$ are picked corresponds to a right-infinite left-derivation process.
The finite play that unfolds from $\beta$ is represented by the formula structure.
This means we use the formulas to represent a play of finite, but unbounded length by a play on the formula of bounded length.
The literal $(p,j)$ that is the result of such a bounded play corresponds to the behavior of the finite word that is derived from $\beta$ on the DPA.\@
In particular, the priority $j$ corresponds to the maximum priority seen in the run on the word.
If the existential player wins the parity game, \ie she can enforce a play in which the dominating priority is odd, she can enforce the derivation of an infinite word that is not accepted by the DPA.\@
This means she wins the $\omega$-context-free game by proving non-inclusion.

Before giving the proof, it is necessary to make the construction of the parity game formal.
For the sake of simplicity, we proceed as in the proof of \cref{Proposition:CFGamesComplexityMembership} and assume that all formulas are in conjunctive normal form.
This means that $\sol{ \text{preprocess}_q{\beta} }$ is a set of clauses, and each such clause is a set of literals.
The parity game has as positions a sink $\bot$, positions of the shape $(X,q)$ and positions for the formulas, clauses, and literals,
%
\begin{align*}
    \configs = &\phantom{\cup} \phantom{\cup}  \paren{N \times Q} \cup \set{\bot}
    \\
    & \cup \Set{ \paren{ \sol{\text{preprocess}_q{\beta}},Y }  }{ X \to \beta.Y \text{ is a production} }
    \\
    & \cup \Set{ (K,Y)  }{ K \in \sol{\text{preprocess}_q{\beta}}, X \to \beta.Y \text{ is a production} }
    \\
    & \cup \Set{ \paren{ (p,j) ,Y }  }{ (p,j) \in K \in \sol{\text{preprocess}_q{\beta}}, X \to \beta.Y \text{ is a production} }
    \ .
\end{align*}
%
The initial position is $(S,\qinit)$.
The existential player owns all positions of the shape $(X,q)$ for $X \in N_\explayer$ and all positions $(K,Y)$ corresponding to clauses.
All other positions are owned by the universal player.
%
The priority of a position $(p,j)$ is $j$.
The priority of all other positions is $0$.

The transitions connect positions of the shape $(X,q)$ to formulas according to the productions for $X$, or to the sink state.
The sink state has a self-loop as its unique outgoing transition.
They also connect formulas to their clauses and clauses to their literals.
Empty clauses are connected to the sink state.
Finally, literals are connected to the positions of the shape $(X,q)$,
\begin{align*}
    T = &\phantom{\cup} \phantom{\cup}
    \Set{ (X,q) \to \bot}{ X \to \eta \text{ is a production with } \eta \not\in {\paren{N \cup T}}^*.N}
    \\
    & \cup \set{ \bot \to \bot}
    \\
    & \cup \Set{ (X,q) \to (\sol{\text{preprocess}_q{\beta}},Y)  }{ X \to \beta.Y \text{ is a production} }
    \\
    & \cup \Set{  (\sol{\text{preprocess}_q{\beta}},Y) \to (K,Y)  }{ K \in \sol{\text{preprocess}_q{\beta}},  X \to \beta.Y \text{ is a production} }
    \\
    & \cup \Set{  (K,Y) \to ((p,j),Y)  }{ (p,j) \in K  \in \sol{\text{preprocess}_q{\beta}},  X \to \beta.Y \text{ is a production}  }
    \\
    & \cup \Set{  (K,Y) \to \bot }{ K = \emptyset, K  \in \sol{\text{preprocess}_q{\beta}},  X \to \beta.Y \text{ is a production}  }
    \\
    & \cup \Set{  \paren{ \paren{p,j},Y } \to (Y,p) }{ Y \in N, p \in Q}
    \ .
\end{align*}

With the formal construction at hand, we can prove the main result.

\begin{theorem}
    The winner of the finite parity game equals the winner of the $\omega$-context-free $\omega$-regular game.
\end{theorem}

\begin{proof}
    It is well-known that parity games are determined: Exactly one of the players has a winning strategy.
    We prove that this winning strategy induces a winning strategy for the $\omega$-context-free game.
    We give the formal proof in the case that the existential player wins the parity game.
    The proof for the universal player is easier as one has to care less about various corner cases.

    Assume that some winning strategy for the existential player is fixed.
    We show how to turn a conforming play of the parity game into a play of the context-free game.
    We do this in a way that implicitly defines a winning strategy.
    In particular, whenever the universal player can make a choice, we consider all of these choices.

    Assume that the play of the parity game is in some node $(X,q)$ and the play of the $\omega$-context-free game is in some sentential form $w.X$.
    Initially, we have $X = S$, $w = \eps$, and $q = \qinit$.
    The owner of~$X$ selects a transition $X \to \beta.Y$ in the parity game, and we pick the corresponding production in the $\omega$-context-free game.
    If $X \in N_\explayer$, this transition is chosen according to the winning strategy.
    Note that any production that is not of the shape $X \to \beta.Y$ will never be chosen, since it would lead to the sink state of the parity game in which the \nb{universal player wins}.

    The parity game is now in the state $(\sol{\text{preprocess}_q{\beta}},Y)$, in which the universal player can choose one of the clauses of the formula.
    Since we are following a winning strategy, we know that the existential player can react to any choice that the universal player can make.
    More formally, for each clause $K \in \sol{\text{preprocess}_q{\beta}}$ that the universal player could select, there is one atom $(p,j) \in K$ contained in that clause so that the winning strategy would select the transition $(K,Y) \to ((p,j),Y)$ in the parity game.
    Note that this in particular implies that the formula cannot be equivalent to $\false$:
    A positive Boolean formula in conjunctive formal is unsatisfiable if and only if it contains the empty clause.
    The empty clause in the game is connected to the sink state, which would lead to the universal player winning the play.

    We define a variable assignment that sets the atoms to true that are picked by the winning strategy.
    More precisely, $(p,j)$ is evaluated to true $\true$ if the winning strategy picks the transition $(K,Y) \to ((p,j),Y)$ for some clause $K$.
    Since every clause contains at least one such atom, the formula $\sol{\text{preprocess}_q{\beta}}$ evaluates to true under that assignment.
    By an analogue of \cref{Proposition:CFGamesSoundnessExistential}, adapted to the setting under consideration, one can show that the existential player has a strategy from $\beta$ that enforces the derivation of a finite word $v$ whose effect is described by $(p,j)$.
    This means $q \tow{v} p$ in the DPA, and the maximum priority seen in the run is $j$.
    We invoke this strategy in the $\omega$-context-free game to go from sentential form $w.\beta.Y$ (which we obtained from $w.X$ by applying $X \to \beta.Y$) to $w.v.Y$.
    Now, the play of the $\omega$-context-free game continues from that sentential form, while the play of the parity game continues from $(Y,p)$.
    We repeat this process ad infinitum.

    When following this strategy in the $\omega$-context-free game, we encounter an infinite sequence of sentential forms of the shape
    \[
        S \to^* \itr{w}{1}X_1  \to^* \itr{w}{1}\itr{w}{2} X_2 \to^* \ldots \to^* \itr{w}{1}\itr{w}{2}\ldots\itr{w}{k} X_K \to^* \ldots
    \]
    We argue that this constitutes a right-infinite left-derivation process deriving an infinite word not in the language of the DPA.\@
    Firstly, we note that the strategies that we invoke for the subgames enforce deriving a finite word.
    Thus, every subgame terminates after finitely many steps, and the process is indeed right-infinite.
    Secondly, each infix $\itr{w}{i}$ that is derived in such a subgame corresponds to an atom $(p_i,j_i)$ of some formula that is chosen by the winning strategy in the parity game.
    The fact that the strategy wins the parity game means that the largest priority occurring infinitely often is odd.
    In particular, it is not zero, the priority that we have assigned to $\eps$ above.
    Winning the parity game implies that infinitely many of the $\itr{w}{i}$ are not equal to $\eps$, and we indeed derive an infinite word.
    Finally, the atoms are witnesses for the unique run
    \[
        \qinit \tow{ \itr{w}{1}} p_1 \tow{ \itr{w}{2}} p_2 \tow{ \itr{w}{3}} \ldots
    \]
    of the DPA on the infinite word $\itr{w}{1}\itr{w}{2}\itr{w}{3} \ldots $ that is derived by the process.
    Each priority $j_i$ is the maximum priority seen during the run on the finite infix $\itr{w}{i}$.
    If the largest priority that occurs infinitely often among the $j_i$ is odd, then so is the largest priority that occurs infinitely often in the run.
    (Summarizing finite infixes in the sequence of priorities by taking their maximum does not change the dominating priority.)
    Hence, the run of the DPA on the word is not accepting.
    The existential player has enforced the derivation of an infinite word that is not in the language of the DPA.\@
\end{proof}

With the soundness of the construction proven, we turn to considering its complexity.
Assume that we have fixed a grammar $G$ and a DPA with $Q$ as its sets of states.
We claim that the running time our algorithm is polynomial in $G$ and exponential in $Q$.
Note that we can assume without loss of generality that the maximum priority $d$ is in $\bigO{\card{Q}}$, $d \leq \card{Q} + 4$ to be precise.

Firstly, we can argue similarly to the previous section that computing $\sol{\text{preprocess}_q{\beta}}$ for each transition $X \to \beta.Y$ can be done in that time.
Also, the number of such transitions is bounded by $\card{G}$.
The formulas themselves can be shown to obey a similar bound:
The number of atoms is $\card{ Q \times \zeroto{d}} = \card{Q} \cdot (d+1)$.
This also bounds the size of any clause, and there are at most $2^{\card{Q} \cdot (d+1)}$ clauses, which bounds the size of any formula.
Combining at most $\card{G}$ of these objects into a parity game leads to a parity game whose size is polynomial in $\card{G}$, exponential in $\card{Q}$, and whose maximum priority is in $\bigO{\card{Q}}$.
We can now use the recent breakthrough result~\cite{CaludeJKLS17} that parity games are fixed-parameter tractable:
They can be solved in time polynomial in the size of the arena and exponential only in the maximum priority.
Since the maximum priority is linear in the size of the original input, solving the parity game overall is exponential in the size of the input.
Altogether, we obtain the desired running time.

If we imagine starting with a nondeterministic Büchi automaton instead of a deterministic parity automaton, we first have to apply the Safra construction for determinization.
It transforms an NBA with set of states $Q$ into a parity automaton with a set of states of size $2^{\bigO{ \card{Q} \cdot \log \card{Q}}}$ and maximum priority in $\bigO{\card{Q}}$.
Applying the determinization and then the rest of the algorithm leads to a running time that is polynomial in the size of the grammar and doubly exponential in the size of the NBA.\@
This is the optimal time complexity.
Obviously, solving $\omega$-context-free games with an NBA representing the target language is $2\EXPTIME$-hard:
A context-free game with an NFA representing the target language can be seen as a special case of such a game, and for this case, we have proven $2\EXPTIME$-completeness in \cref{Theorem:CFGamesComplexityHardness}.
To transform the instance, it is sufficient to modify the grammar and the automaton so that they first produce a finite word as usual, and then proceed to generate an infinite sequence of occurrences of a filler symbol.

\paragraph{Can the parity game construction can be avoided?}

We have provided an algorithm for $\omega$-context-free games with the optimal time complexity.
The fact that this algorithm relies on solving a parity game might seem undesirable.
An algorithm that simply solves a system of equations to compute the winner of the game with no additional steps would be a cleaner solution and potentially more efficient.
However, solving parity games is a well-studied topic, so it would be easy enough to actually implement the algorithm by combining a modified version of the solver we have presented in \cref{Section:CFGamesAlgorithmics} with a solver for parity game like Oink~\cite{Dijk18}.

In the following, we want to argue that it might be impossible to devise an algorithm for computing the winning of an $\omega$-context-free inclusion game by simply computing a least fixed point.
In their papers~\cite{SalvatiW15,SalvatiW15b}, Salvati and \Walu~have considered the $\lambda Y$ calculus as a tree-generating mechanism which is closely related to higher-order recursion schemes.
They have shown that a so-called Scott model (which is similar to the model template for HORSes that we will introduce in \cref{Section:HORSTemplate}) that is based on either least or greatest fixed points is not very expressive when it comes to its capabilities of accepting trees generated by $\lambda Y$-calculus terms.
Its expressiveness is equal to boolean combination of $\Omega$-blind automata.
The precise definition of $\Omega$-blind automata is beyond the scope of this thesis; it shall suffice to say that an $\Omega$-blind automata will accept all infinite branches of a tree (and its negation will reject all infinite branches of a tree).
Hence, a boolean combination of $\Omega$-blind automata is insufficient to check liveness properties (like membership in an $\omega$-regular language) along an infinite branch of the tree generated by a $\lambda Y$-calculus term.

In personal communication with the author of this thesis, Roland Meyer has conjectured that the result by Salvati and \Walu~can be used to show that $\omega$-context-free inclusion games cannot be solved with an algorithm based on effective denotational semantics that simply translates the problem of computing the winner into computing the least solution to a system of equations.
Recall that a context-free grammar can be seen as a higher-order recursion scheme.
It should be possible to transform a game grammar into a term in the $\lambda Y$-calculus that generates a tree so that each possible play of the game corresponds to a branch of that tree.
We leave making this correspondence formal for future work.

\Walus~work also provides us with an explanation why this problem can be circumvented by constructing and solving a parity game instead of trying to read off the winner from the solution to the system of equations directly.
He has shown in~\cite{Walukiewicz02} that the winning regions of parity games can be described by formulas in $\mu$-calculus that contain an alternation of least and greatest fixed-point operators.
Hence, solving a parity game based on the solution to the system of equations means computing a so-called \emph{non-extremal} fixed point that is obtained by nesting least and greatest fixed-point operators.
The expressiveness results by Salvati and \Walu~do not apply to semantics featuring non-extremal fixed point operators.

We think that if this part of the computation potentially cannot be avoided, then representing it via parity games, for which there is a plethora of research and numerous solvers like the ones implemented in the aforementioned tool Oink, is the most elegant way to deal with the issue.

\end{document}
