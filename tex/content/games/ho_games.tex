\documentclass[../../diss.tex]{subfiles}
\begin{document}

\section{Higher-order games}

We define inclusion games on arenas induced by word-generating higher-order recursion schemes.
A \emph{higher-order (regular) inclusion game} is of the shape $(G,A)$, where $G$ is a \emph{game HORS}%
\footnote%
{%
    Game HORSes should not be confused with horse games, \eg polo.%
}%
, a word-generating HORS whose nonterminals $N = N_\allplayer \dotcup N_\explayer$ are partitioned into the nonterminals owned by each of the players.
Recall that being word generating means that the terminals are of the shape $T = \Sigma \dotcup \set{ \wordend \colon \ground}$, consisting of a set $\Sigma$ of terminals of kind $\ground \to \ground$ and the word-end marker.
The second component $A$ of the inclusion game is an NFA over $\Sigma$, specifying the regular target language.

A game HORS induces a game arena.
The positions in the arena are the terms of the HORS of kind ground, \ie well-kinded expressions of kind $\ground$ that can be built using the nonterminals, terminals, and variables.
We fix $S$, the initial nonterminal of kind~$\ground$, to be the initial position of interest.
The moves in the arena are defined by outermost-to-innermost (OI) derivation steps.
Recall that this means replacing an outermost redex (a subexpression that starts with a nonterminal and has all parameters present so that it is of kind~$\ground$) using one of the rules of the HORS.\@

The restriction to OI derivations in conjunction with the assumption that the HORS is word generating means that each term of kind $\ground$ has a unique outermost redex.
To be precise, such a term will either not contain a nonterminal, being of the shape $a_1 ( \ldots a_n (\wordend))$, which we identify with the finite word $a_1 \ldots a_n \in \Sigma^*$, or it will be of the shape
\[
    a_1 ( a_2 ( \ldots a_n (F\ t_1 \ldots t_k) \ldots ))
    \ ,
\]
where $F$ is the outermost nonterminal, uniquely determining the outermost redex.
To see that this is true, note that the initial term $S$ satisfies this property and that applying a replacement to the outermost redex in a term of this shape will result in another term of this shape.
Here, it is crucial that we cannot obtain terms of the shape $a\ (F \ldots)\ (G \ldots)$, \ie a terminal with two redexes as parameters, in a word-generating scheme.

The above observation allows us to assign to each term an owner based on its outermost nonterminal:
Player~$\player$ owns the term $a_1 ( a_2 ( \ldots a_n (F\ t_1 \ldots t_k) \ldots ))$ if $F \in N_\player$.
For terms not containing nonterminals, the ownership does not matter.

With the definition of the game arena completed, we observe that a play from $S$ is an OI-derivation process of the HORS in which each player selects the replacement steps for the nonterminals owned by her.
This is similar to the definition of context-free games in the \nb{previous chapter}.

The winning condition is (non-)membership in the regular target language defined by automaton~$A$.
A play that ends in a terminal word $a_1 \ldots a_n$ not contained in $\lang{A}$ is won by the existential player.
All other plays, including all infinite ones, are won by the universal player.

Note that one could also give a definition that fits our framework for games from \cref{Chapter:Games} by basing it on the growth of the terminal prefix $a_1 \ldots a_n$ of a term $a_1 ( a_2 ( \ldots a_n (F\ t_1 \ldots t_k) \ldots ))$.

The rest of this chapter is dedicated to solving higher-order inclusion games by computing the player that has a winning strategy from the initial position.
The difficulty lies in having to compute information about a game on an infinite arena based on the finite syntactical representation of that arena~--~the HORS and the automaton for the target language.

\paragraph{Related work}

Similar to context-free inclusion games, HORS games can be approached from various angles.
Instead of considering games on arenas induces by HORSes, one can consider games induced by higher-order pushdown automata~\cite{Cachat03,BouajjaniM04,KapikNUW05,HagueMOS08,HagueO07,HagueO09,BroadbentCHS12}, which are equivalent to (a subclass of) HORSes~\cite{Damm82,DammG86,KnapikNU02}.
As in Walukiewicz's work on the $\mu$-calculus and context-free games~\cite{Walukiewicz01}, the game aspect can also be present in the form of the specification that is used in a verification problem that takes a HORS as input.
For example, the decidability of monadic second-order logic (MSO) \resp the modal $\mu$-calculus over trees generated by recursion schemes can be seen as such a result~\cite{KnapikNU02,Caucal02,Ong06}, especially since the latter result by Ong explicitly uses game semantics.
Kobayashi~\cite{Kobayashi09,KobayashiO09} has pioneered an approach to $\mu$-calculus model checking on trees generated by HORSes that is based on so-called intersection types.
The typing algorithm that solves the problem then amounts to computing a least-fixed point, which is similar to our approach using effective denotational semantics.

While our definition of HORS inclusion games with a regular target language seems to be novel, the decidability result that we strive to obtain in the rest of this chapter is not particularly surprising.
It would have been possible to obtain the decidability of HORS inclusion games by reducing them to the setting considered in one of the aforementioned papers.
Rather than the result itself, the interesting aspect of our study is the way in which we obtain that result.
It demonstrates the versatility of our approach based on effective denotational semantics.
Furthermore, the technical development will yield results of independent interest, like the framework for exact fixed-point transfer that we are going to present in \cref{Section:HOGamesFramework}, \nb{as a byproduct}.

\end{document}
