\documentclass[../../diss.tex]{subfiles}
\begin{document}

In the previous chapters, we have studied games on infinite game arenas that are defined by a finite syntax.
We have seen that the finite description provided by the syntax is sufficient to ensure decidability in the case of games defined by context-free grammars and HORSes.
In this chapter, our goal is to explore the frontier of decidability.
We want to obtain a classification result that specifies exactly for which automata models we can hope to solve games on the induced infinite game arenas.

As the basis for this classification result, we use valence systems over graph monoids~\cite{Zetzsche15d}.
Valence systems are a general algebraic automaton model.
Some of the models that we have considered in this thesis, including finite-state automata, pushdown automata, and Petri nets, can be seen as restricted valence systems.
Our classification results will show that among all models that can be seen as valence systems, the only ones for which games on infinite arenas are decidable are essentially context-free models and strongly related ones.

\paragraph{Sources}

The first section of this chapter presents material that is standard in the literature.
We will give references in the main text.
The second section is based on work by Roland Meyer, Georg Zetzsche, and the author of this thesis that has not been published.
The final section gives a brief summary of our publication~\cite{MeyerMZ18} (\resp its full version~\cite{MeyerMZ18a}) without presenting all details.
We will discuss the authors author's contributions to the material presented in this chapter in \cref{Chapter:Contributions}.

\end{document}
