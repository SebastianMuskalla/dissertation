\documentclass[../../diss.tex]{subfiles}
\begin{document}

\section{Context-free regular inclusion games}%
\label{Section:CFGamesDefinition}%

We define a context-free inclusion game as a game whose plays represent derivation processes of a context-free grammar.
The winning condition is membership of the derived word in a regular or $\omega$-regular language.
For now, we limit ourselves to the case of a regular target language of finite words.

For the formal development, we use the definition of inclusion games that we have given in \cref{Chapter:Games}.
We apply it to the automata-theoretic view on context-free grammars that was based on prefix-growth, see \cref{Section:CFG}.
Formally, this means that we associate to a context-free grammar $G = (N,P,S)$ an LTS whose configurations are either terminal words or sentential forms of the shape $w.X.\beta$, where $w \in \Sigma^*$ is a terminal prefix, $X \in N$ is the leftmost nonterminal and $\beta \in {(N \cup \Sigma)}^*$ is the rest of the sentential form.
There is a transition $w.X.\beta \to w.v.\eta.\beta$ in that LTS if $X \to v.\eta$ is a production rule of the grammar, with $v$ being the terminal prefix of the right-hand side, the largest prefix that exclusively consists of symbols from $\Sigma$.
The label $\lambda(w.X.\gamma \to w.v.\eta.\gamma) = v$ of this transition is $v$, the growth of the terminal prefix.
Recall that $\lang{G}$ is the set of labels that occur along paths from $S$, the configuration associated to the initial symbol, to terminal words, configurations of the shape $w \in \Sigma^*$.
The labels along a path from $S$ to $w$ form exactly the word~$w$.

To turn this LTS in a game arena, we assume that we start with a \emph{game grammar}, a context-free grammar where the nonterminals $N = N_\allplayer \dotcup N_\explayer$ are partitioned into the nonterminals $N_\allplayer$ owned by the universal player and the nonterminals $N_\explayer$ owned by the existential player.
We usually write such a game grammar as $G = (N_\allplayer \dotcup N_\explayer,P,S)$.

This partition of the nonterminals induces a partition of the sentential forms -- the configurations of the LTS -- based on the leftmost nonterminal.
The terminal words have no outgoing transitions in the LTS;\@ we may arbitrarily assign them to one of players, say the existential one.
Formally, we have $\configs_\explayer = \Sigma^*.N_\explayer.{(N \cup \Sigma)}^* \cup \Sigma^*$, $\configs_\allplayer = \Sigma^*.N_\allplayer.{(N \cup \Sigma)}^*$.
The sentential forms being owned by the owner of the leftmost nonterminal corresponds to the fact that the transitions of the LTS are based on left-derivations.
This means that we can think of a play in the game as a left-derivation process in which each player selects the production rules for the nonterminals that she owns.

To equip the game arena with a winning condition, we assume that we are also given a regular target language $\lang{A} \subseteq \Sigma^*$ over the same alphabet, where $A$ is a finite automaton.
A play is won by the existential player if and only if it produces a finite word that is not contained in $\lang{A}$.
All other plays are won by the universal player, including all infinite plays.

%
\cheatpagebreak
%

\begin{definition}
    A \emph{context-free (regular) inclusion game} is of the shape $(G,A)$, where $G = (N_\allplayer \dotcup N_\explayer,P,S)$ is a game grammar and $A$ is an NFA over the same alphabet $\Sigma$.
    The plays are left-derivation processes in which each player picks the replacement rules for her nonterminals.
    A maximal play is won by the existential player if and only if it is a finite play of the shape $S \to^* w$ with $w \not\in \lang{A}$.
\end{definition}

Most of this chapter is concerned with developing an algorithm that computes the winner of context-free (regular) inclusion games.
This is a non-trivial task:
While the game can be converted into a reachability game as described in \cref{Chapter:Games}, it is played on an infinite arena that we cannot explicitly store in memory and compute on.
Hence, we face the challenge of having to design an algorithm that works with the finite syntax (the game grammar $G$ and the automaton $A$) describing the infinite game arena in order to solve the game.

Other works in the literature have solved similar types of games, \eg games played on the transition systems of pushdown automata.
We summarize all games in which the game arena is based on a type of system that corresponds to the class of context-free languages, \eg context-free grammar and pushdown automata, under the term \emph{context-free games}.
We defer the discussion of the related work to \cref{Section:CFGamesRelWork}  in this chapter, after we have presented our algorithm.
This will allow us to conduct an in-depth comparison of our solution to other approaches.

\end{document}
