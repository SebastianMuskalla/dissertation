\documentclass[../../diss.tex]{subfiles}
\begin{document}

\section{A model template for deterministic schemes}%
\label{Section:HORSTemplate}%

Our goal is to solve HORS games using effective denotational semantics.
In order to do so, we will design a system of equations, find a suitable model to interpret this system, and then argue that the least solution to the interpreted system allows us to determine the winner of the game.
For now, however, we will not focus on HORS games.
Instead, we develop a general framework for interpreting the systems of equations associated to HORSes.
This framework is not only of independent interest, we will also instantiate it multiple times when we solve HORS games later.

\paragraph{Associating equations to a deterministic HORS}

We start with discussing on how to associate a system of equations to a HORS.\@
We will provide this construction for an arbitrary deterministic HORS.\@
% Later, we will determinize a game HORS so that we can apply the construction.

Let $G = (V,N,T,R,S)$ be a deterministic HORS, \ie each nonterminal $F$ has a unique rule $F \to t$.
We design a system of equations representing $G$.
As in the case of context-free grammars, it has one variable for each nonterminal.
In particular, the variables of the HORS do not become variables of the system of equations.
Each rule $F \to t$ of the HORS yield one equation $F = t$ of the system.
To this end, we need to introduce function symbols that allow us to see the HORS term $t$ as a term of the system of equations, where the syntactic fragments that build up the HORS term $t$ yield the syntactic fragments that build the term $t$ of the system of equations.

Following the definition of HORS terms in \cref{Section:HORS}, each HORS term is obtained by composing three types of constructs:
(1) HORS variables, terminals, and nonterminals,
(2) function application, and
(3) lambda abstraction.
As already mentioned, the HORS nonterminals are simply the variables of the system of equations.
For everything else, we introduce appropriate function symbols:
We see terminals and HORS variables as constants, function symbols with arity zero.
Function application is a binary function symbol: If $f$ and $t$ are terms in the system of equation, then $f\appl t$ is also a term.
As usual for function application in the context of HORSes, we use infix notation and omit brackets.
For each variable $x$, the lambda abstraction  $\lambda x$ is a unary function symbol, so if $t$ is a valid term in the system of equations, then so is $\lambda x. t$

Choosing HORS terminals to have arity zero instead of assigning them their arity as specify by the kind might seem peculiar.
The reason here is that a terminal in a HORS term does not always have all its parameter present.
For example, a terminal $a \colon \ground \to \ground$ may occur in a term like $F\appl a$ without any parameter.
Therefore, it makes sense to let terminals and variables be constants and encode function application as a separate function symbol.

Introducing all of these function symbols allows us to see the set of production rules of the HORS as a system of equations.
Note that an (interpreted) system of equations according to our definition from \cref{Section:EDSEqSys} is essentially a first-order construction, as it just deals with values and functions that transform these values.
By associating a system of equations to a HORS, we have essentially turned a higher-order object into a first-order one.
In order to recover the higher-order aspect, we will have to interpret the system of equations accordingly.

\paragraph{The model template}

Later, we will consider several interpretations of the same system of equations associated to the determinization of a game HORS.\@
Each interpretation is of a similar shape.
In order to make this formal, we define a \emph{model template} for the system of equations associated to a deterministic HORS.\@
This model template can be instantiated using a very restricted amount of information, yielding a full model, \ie a domain and interpretations for all function symbols.

Formally, the model template is $\modelt = (\domaint,\interpretationt)$ consisting of a domain $\domaint$ and the interpretation $\interpretationt$ of the function symbols.
Instantiating the template requires a domain $\domain{\ground}$ for elements of kind ground and an interpretation for all terminals.
The template then provides the rest of the model:
The domain for all other kinds and the interpretation of the variables, function application, and lambda abstraction.

\paragraph{The domain}

We start by explaining the domain.
Assume that a CPPO $\domain{\ground}$ has been chosen.
For any other kind $\kappa_1 \to \kappa_2$, the domain $\domain{\kappa_1 \to \kappa_2}$ is then defined to be the set of join-continuous functions from $\domain{\kappa_1}$ to $\domain{\kappa_2}$.
We will denote this domain by $\domain{\kappa_1} \cto \domain{\kappa_2}$ in the following.
Note that any such function is monotonic since join-continuity implies monotonicity.
(Here, we generalize the definitions of join-continuity and monotonicity from functions with signature $D \to D$ to arbitrary functions where source and target set do not have to coincide.)

It is well-known that if $D,D'$ are CPPOs, then the set $D \cto D'$ of join-continuous functions from $D$ to $D'$ is also a CPPO.\@
The ordering is component-wise, \ie $f \leq g$ iff $f(x) \leq g(x)$ for all $x$.
The function $f \colon D \to D'$ that maps all elements of $D$ to the bottom-element of $D'$ is the bottom element of the function domain.
The join of an ascending chain of functions is the function that takes a value and returns the join of the chain of function values,
\[
    \big(\bigsqcup_{i \in \N} f_i \big) (d) = \bigsqcup_{i \in \N} \paren{ f_i(d) }
    \ .
\]
It is easy to verify that this is indeed the join of the chain.
Let us briefly argue that $\bigsqcup_{i \in \N} f_i$ is join-continuous.
Consider an ascending chain of values ${(d_j)}_{j \in \N}$.
We need to prove that
\(
    \paren{ \bigsqcup_{i \in \N} f_i } \paren{\bigsqcup_{j \in \N} d_j}
    =
    \bigsqcup_{j \in \N} \paren{ \paren{ \bigsqcup_{i \in \N} f_i } \paren{d_j}}
    .
\)

By the definition of the join, the left-hand side of the equality is
\(
    \bigsqcup_{i \in \N} \paren{ f_i \paren{ \bigsqcup_{j \in \N} d_j} }
    =
    \bigsqcup_{i \in \N} \bigsqcup_{j \in \N} f_i \paren{d_j}
    ,
\)
using that each $f_i$ is join-continuous.
We now observe that the two joins commute and use the definition of $\bigsqcup_{i \in \N} f_i$ to get the right-hand side of the desired equality.

We have obtained a CPPO $\domain{\kappa}$ for every kind $\kappa \in \kinds$, and we define \( \domain = \bigcup_{\kappa \in \kinds } \domain{\kappa} \) to be their union.
However, the domains of the shape $\domain{\kappa}$ are insufficient to evaluate terms of the kind $\kappa$ that contain free variables.
The right-hand sides of all rules of the HORS are guaranteed to be closed in that each occurrence of a variable is bound by a preceding lambda abstraction, but when applying structural induction, we will encounter subterms that contain free variables.
In order to handle free variables, the elements of our domain are functions that expect a \emph{valuation}, a partial function that assigns a value to each HORS variable that is free in the term of interest.
Hence, the domain $\domaint$ that is used by our template is actually
\[
    \domaint = (V \pto \domain) \cto \domain
    \ ,
\]
the set of join-continuous functions from valuations to $\domain$.
Here $V \pto \domain$, denotes the set of partial functions from the variables to $\domain$, functions that may be undefined for some variables.

In order for the notion of join-continuity to make sense here, we need to see $V \pto \domain$ itself as a CPPO.\@
In order for this to succeed, we will need to restrict $(V \pto \domain)$ in the following way.
For each variable $x$ of kind $\kappa$, we assume that a valuation either is undefined or it assigns a value from $\domain{\kappa}$.
Whenever we write $V \pto \domain$ in the following, we implicitly assume that we only consider valuations that respect the kinds of the variables.
With this restriction in place, we can define an order on valuations.
We define $\val \leq \val'$ by requiring that for each HORS variable~$x$, either $\val(x)$ is undefined or both $\val(x)$ and $\val'(x)$ are defined and $\val(x) \leq \val'(x)$ holds (using the order $\leq$ on $\domain{\kappa}$, where $\kappa$ is the kind of $x$).
The least element of this CPPO is the function that is undefined everywhere.
The join of an ascending chain of valuations ${\paren{\val_i}}_{i \in \N}$ is the valuation $\bigsqcup_{i \in \N} \val_i$ with $\paren{ \bigsqcup_{i \in \N} \val_i} (x) = \bigsqcup_{i \in \N} \paren{ \val_i(x)}$.
For the latter expression to make sense, we extend the order on $\domain{\kappa}$ to an order on $\domain{\kappa}$ together with the undefined value by seeing undefined as the new least element.
In words, $\bigsqcup_{i \in \N} \val_i$ is the valuation that is undefined for a variable $x$ if all $\val_i$ are undefined for $x$ and that returns the join of the defined values $\val_i(x)$ otherwise.

% We implicitly assume that the domain matches the kinds of the term:
% If $x$ is a HORS variable of kind $\kappa$, then any valuation should assign a value from $\domain{\kappa}$ to $x$.
% Similarly, our interpretation should interpret a term of kind $\kappa$ as a value from
% \(
%     (V \pto \domain) \to \domain{\kappa}
%     \ ,
% \)
% which is a subset of $\domaint$.

We summarize our definition of the domain.

\begin{definition}
    Let $\domain{\ground}$ be a CPPO.\@
    We recursively define $\domain{\kappa_1 \to \kappa_2} = \domain{\kappa_1} \cto \domain{\kappa_2}$ for all kinds $\kappa_1, \kappa_2$ to be the CPPO of join-continuous functions from $\domain{\kappa_1}$ to $\domain{\kappa_2}$.
    Let \( \domain = \bigcup_{\kappa \in \kinds } \domain{\kappa} \).

    We define $\domaint = (V \pto \domain) \cto \domain$ to be the set of join-continuous functions from valuations $\val \colon V \pto \domain$ to domain elements $d \in \domain$.
\end{definition}

In order to enable Kleene iteration later, we need a domain that is a CPPO.\@
Formally, we would need to argue that $\domaint = (V \pto \domain) \cto \domain$ is a CPPO.\@
However, we will make sure that we interpret each term $t$ of kind $\kappa$ as a value from $(V \pto \domain) \cto \domain{\kappa}$.
In particular, we will never need to compare a value from $(V \pto \domain) \cto \domain{\kappa}$ to a value from $(V \pto \domain) \cto \domain{\kappa'}$ for distinct kinds $\kappa \neq \kappa'$.
Hence, it will suffice to make sure that $(V \pto \domain) \cto \domain{\kappa}$ is a CPPO for each kind $\kappa$.

\begin{lemma}
    For each kind $\kappa$, $(V \pto \domain) \cto \domain{\kappa}$ is a CPPO.\@
\end{lemma}

\begin{proof}
    The set $V \pto \domain$ of valuations is a CPPO under the assumption that we only consider valuations that respect the kinds of the variables.
    The set $\domain{\kappa}$ is either $\domain{\ground}$ which is a CPPO by assumption or it is a CPPO as the set of join-continuous functions from one CPPO to another.
    Hence, $(V \pto \domain) \cto \domain{\kappa}$ is the CPPO of join-continuous functions from one CPPO to another.

    Note that the least element of $(V \pto \domain) \cto \domain{\kappa}$ is the function that ignores the given valuation and returns the least element of $\domain{\kappa}$.
    The join of a chain of functions is the function that returns the join of the chain of function values.
\end{proof}

\paragraph{The interpretations}

After fixing the domain $\domaint$, it remains to provide interpretations for all function symbols.
We assume that for each terminal $a$ of kind $\kappa$, an interpretation, $a^\interpretationt \in \domain{\kappa}$ has been provided.
We can see $a^\interpretationt$ as a function with signature $(V \pto \domain) \cto \domain$ that ignores the provided valuation and always returns the same element.
Indeed, in the term $a$ itself, no variable occurs freely.

The template should provide the interpretations for all other types of function symbols: variables, function application, and lambda abstraction.
A HORS variable $x$ is interpreted as $x^\interpretationt \colon (V \pto \domain) \cto \domain$, the function that takes a valuation $\val$ and returns $\val(x)$.
This function is indeed join-continuous by the definition of the join on $V \pto \domain$.
Note that if the valuation respects the kinds of the variables, $\val(x)$ will be an element of $\domain{\kappa}$, where $\kappa$ is the kind of variable~$x$.

The interpretation of function application is a function with signature
\[
    {\domaint}^2 \to \domaint
    \enspace
    =
    \enspace
    {\paren{(V \pto \domain) \cto \domain}}^2 \to (V \pto \domain) \cto \domain
    \ .
\]
It expects three parameters: two values $f^\interpretation,  t^\interpretation \in \domaint$ and a valuation $\val$.
We define the interpretation ${(f^\interpretation \appl t^\interpretation)}^\interpretationt \val$ of function application to be $\paren{f^\interpretation} \val \appl (t^\interpretation \val)$.
Intuitively, $f^\interpretation$ and $t^\interpretation$ are interpretations of terms and $\val$ specifies the values of variables that are free in $f$ or $t$.

\begin{lemma}
    The interpretation of function application has signature ${\domaint}^2 \to \domaint$:
    For any $f^\interpretation, t^\interpretation \in \domaint$, the result of function application is the value ${(f^\interpretation \appl t^\interpretation)}^\interpretationt \in \domaint$.

    If $\val \colon V \pto \domain$ respects the kinds of the variables, $f^\interpretation \val \in \domain{\kappa_1 \to \kappa_2}$, and $t^\interpretation \val \in \domain{\kappa_1}$, then ${(f^\interpretation \appl t^\interpretation)}^\interpretationt \val = (f^\interpretation \val) \appl (t^\interpretation \val) \in \domain{\kappa_2}$.
\end{lemma}

\begin{proof}
    We prove the second part of the statement first.
    We have that $f^\interpretation \val$ is a value from $\domain{\kappa_1 \to \kappa_2}$, which is a join-continuous function from $\domain{\kappa_1}$ to $\domain{\kappa_2}$.
    Similarly, $t^\interpretation \val$ is in $\domain{\kappa_1}$.
    Hence, $(f^\interpretationt \val) \appl (t^\interpretationt \val)$ is in $\domain{\kappa_2}$, and $\kappa_2$ is indeed the kind of $f \appl t$.

    To show that the interpretation of function application has the required signature, we need to argue that ${(f^\interpretation \appl t^\interpretation)}^\interpretationt$ is join-continuous on the level of valuations.
    Consider an ascending chain of valuations ${(\val_i)}_{i \in \N}$.
    We need to prove
    \[
        {(f^\interpretation \appl t^\interpretation)}^\interpretationt (\bigsqcup_{i \in \N} \val_i)
        =
        \bigsqcup_{i \in \N}
        {(f^\interpretation \appl t^\interpretation)}^\interpretationt (\val_i)
        \ .
    \]
    By definition, we have
    \(
        {(f^\interpretation \appl t^\interpretation)}^\interpretationt (\bigsqcup_{i \in \N} \val_i)
        = \paren{ f^\interpretation  \paren{\bigsqcup_{i \in \N} \val_i} } \paren{ t^\interpretation  \paren{\bigsqcup_{i \in \N} \val_i} }
        \ .
    \)
    By assumption, $f^\interpretation$ and $t^\interpretation$ are elements of $\domaint$, so they are join-continuous.
    We can rewrite the right-hand side of the equality as
    \(
        \paren{ \bigsqcup_{i \in \N} f^\interpretation \val_i }
        \paren{ \bigsqcup_{i \in \N}  t^\interpretation \val_i }
        \ .
    \)
    As mentioned before, each $f^\interpretation \val_i$ is a join-continuous function from $\domain{\kappa_1 \to \kappa_2}$.
    The join on $\domain{\kappa_1 \to \kappa_2}$ is defined to be simply the function that returns the join of the function values, so we can rewrite this expression as
    \(
        \bigsqcup_{i \in \N} \ \paren{
            \paren{ \ f^\interpretation \val_i }
            \paren{ \bigsqcup_{i \in \N}  t^\interpretation \val_i }
        }
        \ .
    \)
    Because each $f^\interpretation \val_i$ is join-continuous, we finally obtain
    \(
        \bigsqcup_{i \in \N}
        \bigsqcup_{i \in \N}
        \paren{ \ f^\interpretation \val_i }
        \paren{ \ t^\interpretation \val_i }
        \ .
    \)
    Omitting one of the joins does not change the result and applying the definition of the interpretation of function application proves the desired equality.
\end{proof}

To complete the specification of the model, we have to define the interpretation of lambda abstraction.
For each variable $x$, the interpretation ${(\lambda x)}^\interpretationt$ is a function with signature
\[
    \domaint \to \domaint
    \enspace = \enspace
    \paren{(V \pto \domain) \cto \domain} \to (V \pto \domain) \cto \domain
    \ .
\]
It takes as parameters a value $t^\interpretation \colon (V \pto \domain) \cto \domain$, which intuitively is the interpretation of a term, and a valuation $\val$.
% The latter does not have to assign a value to $x$, since $x$ is not free in $\lambda x . t$.
It returns the function $d \mapsto t^\interpretation (\val[x \mapsto d])$ that expects a value $d$ for $x$ and evaluates $t^\interpretation$ at $\val$ with the value for $x$ replaced by $d$.

\begin{lemma}
    The interpretation of lambda abstraction has signature $\domaint \to \domaint$:
    For any variable $x$ and any $t^\interpretation \in \domaint$, the result of lambda abstraction is ${(\lambda x)}^\interpretationt t^\interpretation \in \domaint$.

    If $\val \colon V \pto \domain$ respects the kinds of the variables, $x$ has kind $\kappa_1$ and $t^\interpretation \val \in \domain{\kappa_2}$, then ${(\lambda x)}^\interpretationt t^\interpretation \val \in \domain{\kappa_1 \to \kappa_2}$.
\end{lemma}

\begin{proof}
    We prove the second part of the statement first.
    Assume that variable $x$ has kind $\kappa_1$, and that term $t$ has kind $\kappa_2$.
    We have that $\paren{{(\lambda x)}^\interpretationt t^\interpretation} \val$ is a function that takes $d \in \domain{\kappa_1}$ and returns $t^\interpretation (\val[x \mapsto d]) \in \domain{\kappa_2}$.
    Hence, it is indeed a function from $\domain{\kappa_1}$ to $\domain{\kappa_2}$.
    This matches the kind $\kappa_1 \to \kappa_2$ of the term $\lambda x. t$.
    To conclude that the function  $d \mapsto t^\interpretation (\val[x \mapsto d])$ is an element of $\domain{\kappa_1 \to \kappa_2}$, we need to argue that it is join-continuous.
    Let ${\paren{d_i}}_{i \in \N}$ be an ascending chain in $\domain{\kappa_1}$.
    We first observe that the valuation $\val [ x \mapsto \bigsqcup_{i \in \N} d_i]$ equals the join $\bigsqcup_{i \in \N} \val [ x \mapsto d_i]$ by the definition of the join on $V \pto \domain$.
    We then note that $t^\interpretation \colon (V \pto \domain) \cto \domain$ is join-continuous, and obtain $t^\interpretation (\val[x \mapsto \bigsqcup_{i \in \N} d_i]) = \bigsqcup_{ i \in \N} t^\interpretation (\val[x \mapsto d_i])$ as desired.

    To prove the first part of the statement, we need to argue that ${(\lambda x)}^\interpretationt t^\interpretation$ is join-continuous on the level of valuations.
    Let ${\paren{\val_i}}_{i \in \N}$ be an ascending chain of valuations.
    We need to show
    \[
        {(\lambda x)}^\interpretationt (t^\interpretation) ( \bigsqcup_{i \in \N} \val_i)
        =
        \bigsqcup_{i \in \N} {(\lambda x)}^\interpretationt (t^\interpretation) \val_i
        \ .
    \]
    The left-hand side of the equality is the function
    \(
        d \mapsto t^\interpretation ( ( \bigsqcup_{i \in \N} \val_i ) [ x \mapsto d] )
        \ .
    \)
    The valuation \( (\bigsqcup_{i \in \N} \val_i ) [ x \mapsto d] \) equals \( \bigsqcup_{i \in \N} \paren{ \val_i [x \mapsto d]} \) using the definition of the join on $V \pto \domain$.
    Since $t^\interpretation \colon (V \pto \domain) \cto \domain$ is join-continuous, the function equals
    \(
        d \mapsto \bigsqcup_{i \in \N} t^\interpretation ( \val_i [ x \mapsto d] )
        \ .
    \)
    By the definition of the join on the level of functions, this function is the join of the functions $d \mapsto t^\interpretation ( \val_i [ x \mapsto d] )$, which by the definition of the interpretation of lambda abstraction is the right-hand side of the equality we wanted to show.
\end{proof}

\paragraph{Join-continuity}

We have fully specified our $\modelt = (\domaint,\interpretationt)$ by providing a domain and interpretations for all function symbols.
In order to prove that the model satisfies the requirements for Kleene iteration as presented in \cref{Section:EDSEqSys}, we need to argue that the interpretation of all function symbols are join-continuous.

When we have considered join-continuity in the proofs of the previous lemmas, we were discussion join-continuity on the level of valuations.
We were merely arguing that for each term $t$, its interpretation is a value of $(V \pto \domain) \to \domain$, \ie a join-continuous function that expects a valuation.
To satisfy the requirements of Kleene's theorem, we need to argue that the functions are also join-continuous with respect to their arguments from $\domaint$.

\begin{remark*}
In the previous chapter, it was sufficient to prove that the interpretations are monotonic by relying on finiteness of the domain and \cref{Remark:EDSMonotonocityContinuity}.
In this chapter, we are interested in instantiating the model template for domains that are not necessarily finite and that do not necessarily satisfy the ACC.\@
\end{remark*}

For constants, function symbols without arguments, there is nothing to do.
This applies to the interpretations of terminals and HORS variables.
It remains to consider the unary interpretation of lambda abstraction and the binary interpretation of function \nb{application}.

\begin{lemma}%
\label{Lemma:HORSTemplateLambdaJoinCont}%
    The interpretation of lambda abstraction is a join-continuous function with signature $\domaint \to \domaint$.
\end{lemma}

\begin{proof}
    Let us consider a lambda abstraction $\lambda x$ where $x$ is of kind $\kappa_1$.
    Let ${(t_i)}_{i \in \N}$ be an ascending chain of values in $(V \pto \domain) \to \domain{\kappa_2}$.
    We have to show ${(\lambda x)}^\interpretationt (\bigsqcup_{i \in \N} t_i) = \bigsqcup_{i \in \N} {(\lambda x)}^\interpretationt (t_i)$.
    The expression on the left-hand side is equal to
    \[
        (\nu, d) \mapsto \big( \bigsqcup_{i \in \N } t_i \big) (\nu [x \mapsto d])
        \ ,
    \]
    a function that takes a valuation $\nu$ and a value $d$ for $x$ and evaluates $\bigsqcup_{i \in \N } t_i$ at $\nu [ x \mapsto d]$.
    Similarly, the expression on the right-hand side equals,
    \[
        \bigsqcup_{i \in \N } \big( (\nu, d) \mapsto t_i (\nu [x \mapsto d]) \big)
        \ .
    \]
    Using the definitions of the join on $(V \pto \domain) \cto \domain$ and the join on $\domain{\kappa_1 \to \kappa_2}$, this equals
    \[
        (\nu, d) \mapsto \bigsqcup_{i \in \N } \left( t_i (\nu [x \mapsto d]) \right)
        \ .
    \]
    We use the definition of the join on $(V \pto \domain) \cto \domain$ again to establish
    \[
        \big( \bigsqcup_{i \in \N } t_i \big) (\nu [x \mapsto d]) = \bigsqcup_{i \in \N } \left( t_i (\nu [x \mapsto d]) \right)
        \ ,
    \]
    which yields the desired equality.
\end{proof}

We proceed similarly for function application, which is a binary function.
We need to prove join-continuity in both parameters, which we will do simultaneously.

\begin{lemma}%
\label{Lemma:HORSTemplateApplicationJoinCont}%
    The interpretation of function application is a join-continuous function with signature ${\domaint}^2 \to \domaint$.
\end{lemma}

\begin{proof}
    Let ${(f_i)}_{i \in \N}$ be an ascending chain in $(V \pto \domain) \cto \domain{\kappa_1 \to \kappa_2}$ and let ${(t_j)}_{j \in \N}$ be an ascending chain in $(V \pto \domain) \cto \domain{\kappa_1}$.
    We need to prove the following equality
    \[
        {\Big(
            \big( \bigsqcup_{i \in \N} f_i \big)
            \appl
            \big( \bigsqcup_{j \in \N} t_j \big)
        \Big)}^\interpretationt
        =
        \bigsqcup_{i \in \N}
        \bigsqcup_{j \in \N}
        {\paren{ f_i \appl e_j}}^\interpretationt
        \ .
    \]
    Both expressions are functions that expect a valuation $\val \colon V \pto \domain$ and return a value from $\domain{\kappa_2}$.
    Evaluating the left-hand side at some valuation $\val$ yields
    \[
        \begin{array}{>{\displaystyle}r>{\displaystyle\enspace}l>{\displaystyle\enspace}l}
        {\Big(
            \big( \bigsqcup_{i \in \N} f_i \big)
            \appl
            \big( \bigsqcup_{j \in \N} t_j \big)
        \Big)}^\interpretationt
        \appl \val
        &=
        \Big( \big( \bigsqcup_{i \in \N} f_i \big) \appl \val \Big)
        \appl
        \Big(  \big( \bigsqcup_{j \in \N} t_j \big) \appl\val \Big)
        &=
        \bigsqcup_{i \in \N} \big( f_i \appl \val \big)
        \appl
        \bigsqcup_{j \in \N} \big( t_j \appl \val \big)
        \\
        &=
        \bigsqcup_{i \in \N}
        \Big(
            \big( f_i \appl \val \big)
            \appl
            \bigsqcup_{j \in \N} \big( t_j \appl \val \big)
        \Big)
        &=
        \bigsqcup_{i \in \N}
        \bigsqcup_{j \in \N}
        \Big(
            \big( f_i \appl \val \big)
            \appl
            \big( t_j \appl \val \big)
        \Big)
        \ ,
        \end{array}
    \]
    using the definition of the interpretation of function application,
    the definition of the join on $(V \pto \domain) \to \domain{\kappa}$ for all kinds $\kappa$,
    the definition of the join on $\domain{\kappa_1 \to \kappa_2}$
    and the fact that each $f_i \appl \val$ is a join-continuous function.
    %
    Evaluating the right-hand side of the above expression at $\val$ gives us
    \[
        \big(
            \bigsqcup_{i \in \N}
            \bigsqcup_{j \in \N}
            {\paren{ f_i \appl e_j}}^\interpretationt
        \big)
        \appl
        \val
        \enspace = \enspace
        \bigsqcup_{i \in \N}
        \bigsqcup_{j \in \N}
        \paren{
            {\paren{ f_i \appl e_j}}^\interpretationt
            \appl
            \val
        }
        \enspace = \enspace
        \bigsqcup_{i \in \N}
        \bigsqcup_{j \in \N}
        \Big(
            \big( f_i \appl \val \big)
            \appl
            \big( t_j \appl \val \big)
        \Big)
        \ ,
    \]
    using the definition of the join on $(V \pto \domain) \cto \domain{\kappa_2}$ and the definition of the interpretation of function application.
    The desired equality holds.
\end{proof}

\paragraph{The semantics of terms}

We have completed the definition of the model template $\modelt = \paren{ \domaint, \interpretationt}$.
Each term of the HORS that does not contain nonterminals can be interpreted as a value from $\domain$.
Furthermore, if the kind of term $t$ is $\kappa$ and $\val$ is a valuation that respects the kinds of the variables, then we get that the interpretation of $t$ evaluated at $\val$ is a value from $\domain{\kappa}$.

It remains to consider terms that contain variables of the system of equations, \ie nonterminals of the HORS.\@
The theory from \cref{Section:EDSEqSys} fixes a semantics
\(
    \sem{\modelt}{\term}
\)
for each term $t$.
In particular, it does so for the terms that occur as the right-hand sides of the rules of the deterministic HORS, which are the right-hand sides of the equations in our system.
Let us consider what the signature of $\sem{\model}{\term}$ is.
Formally, it is a function that expects an assignment of the variables of the system of equations and then evaluates $\term$ at this assignment.
The variables of the system of equations are the nonterminals of the HORS.\@
We get the signature
\[
    \sem{\modelt}{\term} \colon (N \to \domaint) \cto \domaint
    \ .
\]
Here, we have used the fact that the interpretation of each function symbol is join-continuous (\cref{Lemma:HORSTemplateLambdaJoinCont,Lemma:HORSTemplateApplicationJoinCont}) and join-continuity is preserved under composition.
Hence, the interpretation of each term is also join-continuous.

By substituting $\domaint$ using its definition, this signature equals
\[
    \paren{N \to \paren{\paren{V \pto \domain} \cto \domain}} \cto (V \pto \domain) \cto \domain
    \ .
\]
Our first observation is that to evaluate the variables (nonterminals), a valuation of the HORS variables is not needed.
A term that only consists of a nonterminal does not contain a free variable.
Hence, we can simplify the expression to
\[
    \paren{N \to \domain} \cto (V \pto \domain) \cto \domain
    \ .
\]
Intuitively, to evaluate a term in order to obtain an element of $\domain$, we need an assignment of values to the nonterminals and a valuation of the HORS variables.
We merge the assignment and the valuation into a single valuation, a partial function with signature $N \dotcup V \pto \domain$.
When evaluating a term $\term$, this valuation should be defined for all nonterminals and all HORS variables that are free in $t$.
Furthermore, we will implicitly assume that we only consider valuations $\val$ that respect the kinds (of both HORS variables and nonterminals), \ie if $F$ \resp $x$ is of kind $\kappa$, then $\val(F)$ \resp $\val(x)$ is a value from $\domain{\kappa}$.
Under this assumption, a term of kind $\kappa$ should evaluate to a value from $\domain{\kappa}$.

Altogether, we have proven the following.

\begin{lemma}%
\label{Lemma:HORSTemplateSemantics}%
    The semantics of a HORS term $t$ is a join-continuous function
    \[
        \sem{\modelt}{t} \colon (N \dotcup V \pto \domain) \cto \domain
        \ .
    \]
    If $\val$ is a valuation that respects the kinds and $t$ is of kind $\kappa$, then $\sem{\modelt}{t}{\val} \in \domain{\kappa}$.
\end{lemma}

\paragraph{Applying Kleene's theorem}

Consider the interpretation of the system of equations under a model that results from instantiating the model template.
The least solution to the interpreted system is the least fixed point of the function that is obtained by combining the interpretations of the right-hand sides of the equations, \ie the rules of the deterministic HORS.\@
Because the model satisfies the requirement of Kleene's theorem, the existence of this least fixed point is guaranteed.

\begin{proposition}%
\label{Proposition:HORSTemplateExistenceLeastSolution}%
    The system of equations associated to a deterministic HORS interpreted under a model that results from instantiating the model template has a least solution.
\end{proposition}

%
\clearpage
%


\begin{figure}
    \onehalfspacing%
    {%
    \setlength{\fboxsep}{1em}
    \fbox%
    {%
        \begin{minipage}{\textwidth-2.5em}
            \vspace*{0.5em}
            %
            \paragraph{\large Overview: Model template}
            %
            \vspace{1em}

            \paragraph{Domain}

            \emph{Input:} CPPO $\domain{\ground}$ for kind ground (partially ordered; least element exists; ascending chains have a join).

            Recursively define $\domain{\kappa_1 \to \kappa_2} = \domain{\kappa_1} \cto \domain{\kappa_2}$, the join-continuous functions from $\domain{\kappa_1}$ to $\domain{\kappa_2}$.
            This is a CPPO with the component-wise order; the least element is the function that always returns the least element; the join of a chain is the function that, given an input, returns the join of the chain of function values.

            We define the domain $\domaint = (V \pto \domain) \to \domain$ with $\domain = \bigcup_{\kappa} \domain{\kappa}$.
            For each kind $\kappa$, $(V \pto \domain) \to \domain{\kappa}$ is a CPPO.\@
            Elements of this domain are functions that take a valuation, an assignment of a value from $\domain{\kappa}$ to each variable of kind $\kappa$.

            \vspace*{1em}

            \paragraph{Interpretations}

            \emph{Input:} For each terminal $s \colon \kappa$ of a deterministic HORS $G$ an interpretation $s^\interpretationt \in \domain{\kappa}$.

            Specify $\interpretationt$ by defining the interpretation of variables ($\sem{\modelt}{x}{\val} = \val(x)$), function application ($\sem{\modelt}{t_1 \appl t_2}{\val} = \sem{\modelt}{t_1}{\val} \appl \sem{\modelt}{t_2}{\val}$), and lambda abstraction ($\sem{\modelt}{\lambda x . t}{\val} \ \colon \ d \mapsto \sem{\modelt}{t}{\paren{\val[x \mapsto d]}}$).
            Altogether, we get a model $\modelt = \paren{\domaint,\interpretationt}$.

            \vspace*{1em}

            \paragraph{Semantics}

            We obtain for each term $\term$ of kind $\kappa$ the semantics
            \[
                \sem{\modelt}{t} \colon (N \dotcup V \pto \domain) \cto \domain{\kappa}
                \ ,
            \]
            a join-continuous function that expects a valuation of the nonterminals and the HORS variables that are free in $\term$.

            \vspace*{1em}

            \paragraph{Kleene iteration}

            For each $i \in \N$, inductively define the valuation $\soli{i}$.\\
            For $i = 0$ and each nonterminal $F$ of kind $\kappa$,\\
            \hspace*{2em}$\sol{i}{F} = \bot_{\kappa}$, the least element of $\domain{\kappa}$.\\
            For $i+1$ and a nonterminal $F$ whose unique rule in the HORS is $F \to t$,\\
            \hspace*{2em}$\sol{i+1}{F} = \sem{\modelt}{t}{\soli{i}}$, the right-hand side for $F$ evaluated at $\soli{i}$.

            The system of equations associated to $G$ interpreted with respect to $\modelt = \paren{\domaint,\interpretationt}$ has a least solution, namely
            \[
                \sol = \bigsqcup_{i \in \N} \soli{i}
                \ .
            \]
            %
            \vspace*{0.1em}
        \end{minipage}
    }
    }
    \caption{Overview: Model template.}%
    \label{Figure:HORSTemplateCheatSheet}%
\end{figure}

\clearpage

\begin{proof}
    We may design a function
    \[
        \rhs \colon (N \to \domain) \cto (N \to \domain)
    \]
    with $\rhs(\val,F) = \sem{\modelt}{t}{\val}$ for each nonterminal $F$, where $F \to t$ is the unique rule of the HORS for $F$.
    Note that since each such $t$ does not contain any free variables, we can omit the valuation for the HORS variables here.
    We have argued that for each term $t$, $\sem{\modelt}{t}$ is a join-continuous function on a CPPO.\@
    Hence, also $\rhs$ is a join-continuous function on a CPPO.\@
    By Kleene's theorem, \cref{Theorem:EDSKleene}, the least fixed point of $\rhs$ is the join
    \[
        \sol = \bigsqcup_{i \in \N} \soli{i}
        \ ,
    \]
    where the $\nth{i}$ approximant $\soli{i} = \rhs^{i} (\bot)$ is obtained by the $i$-fold application of $\rhs$ to the least element of $N \to \domain$.
    This least element is the function that assigns to each nonterminal $F$ of kind $\kappa$ the least element of $\domain{\kappa}$.
    Note that the definition of $\soli{i}$ implies $\sol{i+1}{F} = \sem{\modelt}{t}{\soli{i}}$ for each nonterminal $F$ whose unique rule is $F \to t$.

    Using the theory from \cref{Section:EDSEqSys}, $\sol$ is the least solution to the interpreted system of equations.
\end{proof}


Note that the proposition guarantees the existence of the least solution as the join of an infinite ascending chain.
Unlike in the previous chapter, our domain will not satisfy the ascending chain condition in general.
Hence, we do not necessarily get that the solution equals $\soli{i}$ for some fixed $i$.
Kleene iteration does not allow us to compute the fixed point in finite time.
When we want to apply our theory to solve HORS games, this is a challenge that we will have to overcome.

We conclude the section with an \emph{overview} in the form of \cref{Figure:HORSTemplateCheatSheet}.
It specifies what a user of the template has to provide to instantiate it and what the result of the instantiation is.

\begin{remark*}
    The models that result from instantiating our model template are similar to what \citea{SalvatiW15b} call \emph{Scott models} for $\lambda Y$-calculus.
\end{remark*}

\end{document}
