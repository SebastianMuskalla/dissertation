\documentclass[../../diss.tex]{subfiles}
\begin{document}

\section{Computing the winner}%
\label{Section:CFGamesSoundness}%

We discuss how to compute the winner of a context-free inclusion game.
In the previous section, we have set up a system of equations that represents a game grammar.
We then have provided a model, consisting of the domain of equivalence classes of positive Boolean formulas over the transition monoid.
We have argued that this model satisfies all requirements that are needed to be able to apply Kleene's theorem and the theory from \cref{Section:EDSEqSys}.

Kleene's theorem proves that the least solution to the system of equations exists, and the fact that our domain is finite means that we can explicitly compute it.
The solution is the least fixed point of the function $\rhs$ obtained by interpreting the right-hand sides of the system of equations.
Recall that we have defined $\soli{i} = \rhs^{i}\paren{\bot}$ to be the $i$-fold application of this function to the least element.
In our case, the least element is the vector that assigns to each variable the equivalence class of unsatisfiable formulas, \ie the equivalence class of $\false$.
Because the domain satisfies the ACC, the least fixed point $\sol$ occurs as $\soli{i_0} = \sol$ for some $i_0 \in \N$ so that $\soli{i_0} = \soli{i_0 + 1}$.
% Recall that $\sol$ and the $\soli{i}$ do not only provide us with a class of formula for each nonterminal, but they can be also extended to arbitrary terms.
The goal of this chapter is proving that the winner of a context-free inclusion game can be read off from $\sol$.

We consider $\Mrej \subseteq \nfatransmonoid{A}$, the set of \emph{rejecting boxes}
\[
    \Mrej = \Set{ \tbox{w} }{ w \not\in \lang{A}}
    \ .
\]
Recall that if one word in the language $\lang{\tbox}$ of some box is not contained in $\lang{A}$, then no such word is.
Furthermore, whether a box is rejecting or not can be determined by checking for the existence of a transition from the initial to a final state.

We call a formula $F \in \pBF\paren{\nfatransmonoid{A}}$ \emph{rejecting} if it evaluates to true under $\Mrej$ seen as variable assignment, \ie the assignment that sets exactly the atoms to true that are contained in $\Mrej$.
Since being satisfied under a certain assignment is preserved by logical equivalence, this notion can be extended to equivalence classes of formulas by picking an arbitrary representative.

We claim that for each sentential form $\beta$, the existential player has a winning strategy for the context-free inclusion game from $\beta$ if and only $\sol{\beta}$ is rejecting.
(Recall that we have extended the definition of $\sol$ from variables to arbitrary terms)
In particular, the winner of the context-free game when starting from $S$ can be read off from $\sol{S}$.

More formally, we define a partition of the set of sentential forms, the positions of the game,
 \[
     W_\explayer = \Set{ \beta \in \sententials }{ \sol{\beta} \text{ is rejecting } }
 \]
and
 \[
     W_\allplayer = \sententials \setminus W_\explayer = \Set{ \beta \in \sententials }{ \sol{\beta} \text{ is not rejecting } }
     \ .
 \]
We prove that for each player $\player \in \players$, $W_\player$ is indeed her winning region.

\begin{theorem}%
\label{Theorem:CFGamesSoundness}%
    The set $W_\explayer$ is the winning region of the existential, $W_\explayer$ is the winning region of the existential player.
\end{theorem}

\begin{example}%
\label{Example:CFGamesThree}%
    We continue \cref{Example:CFGamesTwo}.
    The box $\tbox{b}$ is rejecting while $\tbox{ab}$ and $\tbox{\eps}$ are not.
    Hence, the formula $\sol{X} = \tbox{ab} \vee \tbox{\eps}$ is not rejecting and the universal player wins the game $(G,A)$ starting from $X$.
    The formula $\sol{Y} = \tbox{b}$ is rejecting and the existential player wins from $Y$.
\end{example}

To prove the theorem, we prove a more general statement:
For each sentential form $\beta$, the behavior of $\sol{\beta}$ under evaluation describes exactly the behavior of the context-free game from $\beta$.
If $M \subseteq \nfatransmonoid{A}$ is a set of boxes such that $\sol{\beta}(M) = \false$, then the universal player can prevent the derivation of any word $w$ with $\tbox{w} \in M$.
If $\sol{\beta}(M) = \true$, then the existential player can enforce the derivation of a word $w$ with $\tbox{w} \in M$.
Instantiating these results for the set of rejecting boxes $\Mrej$ then yields the desired result.

We start by considering the case of the universal player.

\begin{proposition}%
\label{Proposition:CFGamesSoundnessUniversal}%
    Let $M \subseteq \nfatransmonoid{A}$ be a set of boxes and let $\beta \in \sententials$ be a sentential form.
    If $\sol{\beta}(M) = \false$, then the universal player has a strategy such that every maximal play conforming to it is either infinite or it ends in a terminal word $w$ with $\tbox{w} \not\in M$.
\end{proposition}

\begin{proof}
    We show that the universal player has a strategy that maintains $\sol{\sentform}(M) = \false$ as an invariant for all positions $\sentform \in \sententials$ that occur in the play.
    This invariant guarantees that if a play conforming to it ends after finitely many steps in a terminal word $w$, we must have $\sol{w}(M) = \tbox{w}(M) = \false$.
    This means $\tbox{w} \not\in M$ as desired.

    Let us construct a strategy that maintains the invariant.
    For the initial position, the invariant holds by the assumption.
    % We show that if $\beta$ is an arbitrary sentential form with $\sol{\beta}(M) = \false$ that is not a deadlock, then there is a successor $\gamma$ with  $\sol{\gamma}(M) = \false$.
    If a sentential form does not contain a nonterminal, it is a deadlock in the game.
    Hence, we consider $\sentform = w.X.\gamma$, where $X$ is the leftmost nonterminal.

    Firstly, we look at the case that the owner of $X$ (and hence the owner of $\sentform$) is the universal player.
    We have $\sol{\sentform} = \sol{w . X  . \gamma} = \tbox{w} \comp \sol{X} \comp \sol{\gamma}$ using the associativity of formula composition.
    Let $\itr{\eta}{1}, \ldots, \itr{\eta}{k}$ be the right-hand sides for nonterminal $X$, \ie $X \to \itr{\eta}{1} \mid \ldots \mid \itr{\eta}{k}$ are all productions for $X$.
    Then we have $\sol{X} = \sol{\itr{\eta}{1}} \wedge \ldots \wedge \sol{\itr{\eta}{k}}$ by the definition of the system of equations and the fact that the least solution $\sol$ satisfies all equations.
    Substituting $\sol{X}$ and applying the definition of formula composition yields
    \[
        \sol{\sentform} = \tbox{w} \comp \sol{\itr{\eta}{1}} \comp \sol{\gamma} \wedge \ldots \wedge \tbox{w} \comp \sol{\itr{\eta}{k}} \comp \sol{\gamma}
        \ .
    \]
    Since we had $\sol{\sentform}(M) = \false$ and $\sol{\sentform}$ can be written as the above conjunction, there is at least one $j$ such that
    $
        \paren{ \tbox{w} \comp \sol{\itr{\eta}{j}} \comp \sol{\gamma} } (M) = \false
        \ .
    $
    We define the strategy so that it picks the move $w.X.\gamma \to w.\itr{\eta}{j}.\gamma$ induced by the production $X \to \itr{\eta}{j}$.
    The formula for the new position is $\sol{w} \comp \sol{\itr{\eta}{j}} \comp \sol{\gamma}$, and the construction guarantees that the invariant is maintained.

    If the leftmost nonterminal is owned by the existential player, we argue similarly.
    We have
    \[
        \sol{\sentform} = \tbox{w} \comp \sol{\itr{\eta}{1}} \comp \sol{\gamma} \vee \ldots \vee \tbox{w} \comp \sol{\itr{\eta}{k}} \comp \sol{\gamma}
        \ .
    \]
    This disjunction only evaluates to $\false$ if $\tbox{w} \comp \sol{\itr{\eta}{j}} \comp \sol{\gamma}$ evaluates to $\false$ for all $j$.
    Hence, the invariant is maintained, no matter which move is picked by the existential player.
\end{proof}

The proof in the case of the existential player is more involved.
This is because she has to enforce reaching a word that violates inclusion within finite time.

\begin{proposition}%
\label{Proposition:CFGamesSoundnessExistential}%
    Let $M \subseteq \nfatransmonoid{A}$ be a set of boxes and let $\beta \in \sententials$ be a sentential form.
    If $\sol{\beta}(M) = \true$, then the existential player has a strategy such that every maximal play conforming to it is finite and ends in a terminal word $w$ with $\tbox{w} \in M$.
\end{proposition}

\begin{proof}
    We prove that the desired property already holds if $\sol{i}{\beta}(M) = \true$ for some $i \in \N$.
    Since we have $\sol = \soli{i_0}$ for some $i_0 \in \N$ and that $\sol{i}{\beta} \lleq \sol{\beta}$ for all $i \in \N$, this is a stronger statement.
    Its advantage is that we can now proceed by induction on $i$.

    In the base case, we consider $i = 0$.
    Since $\false$ is the least element in $\pBF\paren{\nfatransmonoid{A}}$ with respect to implication, we have $\sol{0}{X} = \false$ for all nonterminals.
    Hence, $\sol{0}{\beta}(M) = \true$ is only possible if $\beta$ does not contain any nonterminal.
    This means $\beta = w$ is a terminal word and $\sol{0}{\beta} = \tbox{w}$.
    Thus, position $w$ is a deadlock and $\sol{0}{\beta}(M) = \true$ implies  $\tbox{w} \in M$.

    For the inductive step, assume that the statement holds for $i$ and consider $i+1$.
    Let $\beta$ be some sentential form.
    The proof uses an inner induction on the number of nonterminals in $\beta$.
    In the base case, this number is $0$ and $\beta$ is a terminal word.
    We have already considered this case in the base case of the outer induction.

    In the inductive step of the inner induction, we consider $\sol{i+1}{\beta}$ for some sentential form $\beta = w.X.\gamma$, where $X$ is the leftmost nonterminal in $\beta$.
    We have
    \(
        \sol{i+1}{\beta} = \sol{i+1}{w.X.\gamma} = \sol{i+1}{w} \comp \sol{i+1}{X} \comp \sol{i+1}{\gamma} = \tbox{w} \comp \sol{i+1}{X} \comp \sol{i+1}{\gamma}
    \)
    using the associativity of formula composition.

    Let us assume that $X$ is owned by the existential player.
    We may write $\sol{i+1}{X} = \sol{i}{\itr{\eta}{1}} \vee \ldots \vee \sol{i}{\itr{\eta}{k}}$ where $X \to \itr{\eta}{1} \mid \ldots \mid \itr{\eta}{k}$ are all the rules for nonterminal $X$.
    Using the definition of formula composition, we obtain
    \[
        \sol{i+1}{\beta} = \tbox{w} \comp \sol{i}{\itr{\eta}{1}} \comp \sol{i+1}{\gamma} \vee \ldots \vee \tbox{w} \comp \sol{i}{\itr{\eta}{k}} \comp \sol{i+1}{\gamma}
        \ .
    \]
    Using $\sol{i+1}{\beta}(M) = \true$, we conclude that there is at least one $j \in \oneto{k}$ such that $\paren{\tbox{w} \comp \sol{i}{\itr{\eta}{j}} \comp \sol{i+1}{\gamma}}(M) = \true$.

    We would like to apply the induction hypothesis at this point.
    However, the formula $\tbox{w} \comp \sol{i}{\itr{\eta}{j}} \comp \sol{i+1}{\gamma}$ is not of the required shape as we use different approximants for different parts of the formula.
    To solve this problem, we apply induction only to $\sol{i}{\itr{\eta}{j}}$.
    To take the rest of the sentential form $w.\itr{\eta}{j}.\gamma$ into account, we consider a different variable assignment.

    We define the assignment $M'$ by $\tbox(M') = \paren{\tbox{w} \comp \tbox \comp \sol{i+1}{\gamma}}(M)$.
    Intuitively, it evaluates the given box under $M$ in the context of $w$ on the left-hand side and $\gamma$ on the right-hand side.
    By the definition of formula composition, we have that $F(M') = \paren{\tbox{w} \comp F \comp \sol{i+1}{\gamma}}(M)$ for any formula $F$.
    Indeed, formula $F$ is the leftmost operand of $\tbox{w} \comp F \comp \sol{i+1}{\gamma}$ that may contain conjunctions or disjunctions, which will be resolved first when evaluating the composition.
    In particular, we have $\sol{i}{\itr{\eta}{j}}(M') = \paren{\tbox{w} \comp \sol{i}{\eta} \comp \sol{i+1}{\gamma}}(M) = \true$.
    We may use the induction hypothesis of the outer induction to obtain that the existential player has a strategy $\strat'$ from $\itr{\eta}{j}$ to enforce the derivation of a terminal word $v$ with $\tbox{v}(M') = \true$.

    We define the strategy $\strat$ from $\beta$ to first pick the move $\beta \to w.\itr{\eta}{j}.\gamma$ that is induced by the production $X \to \itr{\eta}{j}$.
    Then, it imitates $\strat'$ (by ignoring the prefix~$w$ and the suffix~$\gamma$) until $\itr{\eta}{j}$ has been derived to a terminal word.
    After this process has been completed, we end up with $w.v.\gamma$ such that $\tbox{v}(M') = \true$.
    By the definition of $M'$, we have $\tbox{v}(M') = \paren{\tbox{w} \comp \tbox{v} \comp \sol{i+1}{\gamma}}(M) = \paren{\sol{i+1}{w.v.\gamma}}(M) = \true$.

    The sentential form $w.v.\gamma$ contains strictly fewer terminals than $\beta$, since we have replaced $X$ by a terminal word.
    We may now use the induction hypothesis of the inner induction to obtain that there is a strategy $\strat''$ from $w.v.\gamma$ that is guaranteed to derive a terminal word $w'$ with $\tbox{w'}(M) = \true$, \ie $\tbox{w'} \in M$, within finitely many steps.

    We complete the definition of $\strat$ by letting $\strat$ behave as $\strat''$ does from $w.v.\gamma$ on.

    The case that $X$ is owned by the universal player is similar.
    The formula $\sol{i+1}{X}$ is a conjunction with one conjunct $\tbox{w} \comp \sol{i}{\itr{\eta}{j}} \comp \sol{i+1}{\gamma}$ for each rule $X \to \itr{\eta}{j}$.
    We have that $\sol{i+1}{X}(M) = \true$ implies that each conjunct evaluates to true under $M$.
    For any conjunct $j$, we proceed as above:
    After defining a new assignment $M'_j$ with $\tbox(M'_j) = \paren{ \tbox{w} \comp \tbox \comp \sol{i+1}{\gamma}}$, we can apply the induction hypothesis of the outer induction.
    We get a strategy $\strat'_j$ that enforces the derivation of a terminal word $v_j$ with $(\sol{i+1}(w.v_j.\gamma))(M) = \true$.
    Applying the inner induction to $w.v_j.\gamma$ gives us a strategy $\strat_j''$ that derives a word $w'$ with $\tbox{w'} \in M$.
    We define the strategy $\strat$ for the existential player as follows:
    In position $w.X.\gamma$, the strategy reads the production rule $X \to \itr{\eta}{j}$ that is picked by the universal player.
    Afterwards, it imitates $\strat'_j$ until the sentential form $w.v_j.\gamma$ has been derived.
    Finally, it behaves as $\strat''_j$ until the end of the play.
    The result is again that we guarantee the derivation of a terminal word $w'$ with $\tbox{w'} \in M$ within finitely many steps.
\end{proof}

\begin{example}%
\label{Example:CFGamesFour}%
    We continue \cref{Example:CFGamesThree} to complete our running example.
    The formula associated to $\sol{Y} = \tbox{b}$ is rejecting, so the existential player should win the game starting from $Y$.

    In the first move, the universal player has no choice but to derive $b.X$, and indeed $\sol{b.X} = \sol{Y} = \tbox{b}$ is still rejecting.
    It is now the existential players choice to replace $X$ either using the production $X \to \eps$ or $X \to a.Y$.
    The productions lead to the formulas $\sol{b} = \tbox{b}$ and $\sol{b.a.Y} = \tbox{ba} \comp \sol{Y} = \tbox{bab} = \tbox{b}$, respectively.
    Either move maintains the invariant of being rejecting.
    However, this is insufficient to ensure that the existential player wins.
    If she picks the production $X \to a.Y$ whenever she has to replace $X$, the resulting play is infinite.
    Even though it consists entirely of positions whose associated formula is rejecting, the existential player loses.
    To ensure that she wins, she has to pick the production $X \to \eps$ after finitely many steps, \eg the first time she replaces $X$.

    We show that we read off this property using the approximants.
    Recall that the least solution corresponds to the $\nth{3}$ approximant.
    The game starts from $Y$ and $\sol{3}{Y} = \tbox{b}$ is rejecting.
    After one move, the game is in $bX$ and $\sol{2}{bX} = \tbox{b} \comp \sol{2}{X} = \tbox{b} \cdot \tbox{\eps} = \tbox{b}$ is still rejecting.
    Applying $X \to \eps$ yields the position $b$ with $\sol{1}{b} = \tbox{b}$ being rejecting.
    Using $X \to aY$ however yields $baY$ with $\sol{1}{b.a.Y} = \tbox{ba} \comp \sol{1}{Y} = \tbox{ba} \comp \false = \false$.
    The formula $\false$ is unsatisfiable and evaluates to $\false$ under any assignment.
    In particular, it is not rejecting.
    Hence, the production $X \to \eps$ should be used to ensure that the play terminates and that the existential player wins.
\end{example}

With both propositions at hand, it is not difficult to obtain the proof of the theorem.

\begin{proof}[Proof of \cref{Theorem:CFGamesSoundness}]
    We first show that \( W_\allplayer =  \Set{ \beta \in \sententials }{ \sol{\beta} \text{ is not rejecting } } \) is a subset of the winning region of the universal player.
    This means that the universal player wins from all positions whose associated formula is not rejecting.

    Let $\beta \in \sententials$ be so that $\sol{\beta}(\Mrej) = \false$.
    By \cref{Proposition:CFGamesSoundnessUniversal} the universal player has a strategy from $\beta$ that enforces plays that are either infinite or derive a word $w$ with $\tbox{w} \not\in \Mrej$.
    By the definition of $\Mrej$, the latter means $w \in \lang{A}$.
    Hence, such a strategy is a winning strategy for the universal player in the context-free regular inclusion game.

    It remains to show that the existential player wins from all positions in $W_\explayer = \Set{ \beta \in \theta }{ \sol{\beta} \text{ is rejecting}}$.
    Because $W_\explayer$ and $W_\allplayer$ form a partition of the set of positions $\sententials$, this completes the proof.
    If $\sol{\beta}$ is rejecting, \ie $\sol{\beta}(\Mrej) = \true$, then \cref{Proposition:CFGamesSoundnessExistential} guarantees the existence of a strategy from $\beta$ that derives a word $w$ with $\tbox{w} \in \Mrej$ within finitely many steps.
    By the definition of $\Mrej$, $\tbox{w} \in \Mrej$ implies $w \not\in \lang{A}$.
    Hence, such a strategy is a winning strategy for the existential player.
\end{proof}

\begin{remark}
    There is a different approach to proving the soundness of our procedure.
    For some fixed initial position, one can see the tree of all plays from that position as an infinite formula over $\nfatransmonoid{A}$.
    To this end, we see inner nodes of the tree as disjunctions or conjunctions, depending on the owner of the corresponding position.
    We see a leaf as the atom $\tbox{w}$, where $w$ is the unique word that has been derived by the play reaching that leaf.
    The result is a formula that potentially has infinite depth.

    We equip it with an evaluation semantics that gets rid of this problem:
    We evaluate boxes depending on a variable assignment, \ie a subset $M \subseteq \nfatransmonoid{A}$.
    We then propagate the values for the boxes upwards using the usual rules for conjunction and disjunction.
    This evaluation semantics essentially ignores all infinite paths, \ie it evaluates them to $\false$.

    The result is that to each infinite formula $F$, we associate a function from $\powerset{\nfatransmonoid{A}} \to \set{\true,\false}$ that evaluates the formula under the given variable assignment.
    However, it is well known that all functions of this type can be represented by finite formulas.
    Hence, there is a finite formula that is logically equivalent to the infinite one.

    To prove the soundness of our algorithm, one first shows that the evaluation of the infinite formula under the variable assignment $\Mrej$, the rejecting boxes, indeed yields the winner of the game.
    Secondly, one shows that the formula associated to the initial position by the least solution to the system of equations is equivalent to the infinite formula.
    We have made this approach formal in Section~4 of the full version of our paper~\cite{HolikMM16a}.

    The approach that we have presented above is not only easier.
    It also allows us to extract the winning strategies for each of the players, as we will see in the next section.
\end{remark}

\end{document}
