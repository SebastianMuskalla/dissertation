\documentclass[../../diss.tex]{subfiles}
\begin{document}

\section{Framework for exact fixed-point transfer}%
\label{Section:HOGamesFramework}%

Our goal in this section is to develop a framework for the exact transfer of the fixed point with respect to one model to the fixed point with respect to another model.
Assume we are given two models $\modelc$ and $\modela$ and a function $\abs$ that maps values from one domain to the other.
We want to provide a minimal set of criteria that will allow us to conclude that the function applied to the fixed point \wrt $\modelc$ is the fixed point \wrt $\modela$, $\abs{\solc} = \sola$.

The following is a potential application for this framework.
Assume the goal is solving a decision problem involving HORSes using effective denotational semantics.
We have \emph{concrete} model $\modelc = (\domainc,\interpretationc)$, for which it is easy to prove that the associated fixed point $\solc$ captures the answer to the decision problem.
In this model, it is hard or even impossible to compute the least solution (\eg because the domain is infinite and does not satisfy the ascending chain condition).
We also have an \emph{abstract} model $\modela = (\domaina,\interpretationa)$ that is well-behaved with respect to the fixed-point computation, but proving that $\sola$ also captures the answer to the decision problem may be much harder.
Using the framework, it is sufficient to provide an abstraction function $\abs$ from the concrete to the abstract domain that satisfies some properties.
We get exact fixed-point transfer and have $\abs{\solc} = \sola$.
If the property that allows us to read off the answer to the decision problem is preserved under $\abs$, we conclude that reading off the answer to the decision problem is indeed sound.
Hence, we can replace a complicated proof, namely the proof that $\sola$ captures the solution to the decision problem, by several simpler proofs.

In the next section, we will apply this approach to computing the winner of inclusion games with game arenas defined by game HORSes.
Since we think that the framework for exact fixed-point transfer is of independent interest, we develop our theory for a general setting.

Assume that a deterministic HORS $G$ and its associated system of equations are fixed.
We consider two models $\modelc = (\domainc,\interpretationc)$ and $\modela = (\domaina,\interpretationa)$ that are instantiations of the model template described in \cref{Section:HORSTemplate}
We call $\modelc$ and the associated semantics \emph{concrete} and $\modela$ \emph{abstract} to appeal to the intuition of the reader.
However, these words have no formal meaning here; in particular, both models could be finite or infinite.
Each of the models provides an interpretation $\interpretationc$ \resp $\interpretationa$ of all function symbols used in the system of equations, a domain $\domainc{\kappa}$ \resp $\domaina{\kappa}$ for each kind $\kappa$ and a signature $(N \cup V \pto \domainc) \cto \domainc$ \resp $(N \cup V \pto \domaina) \cto \domaina$ for the semantics of HORS terms.

Assume that an \emph{abstraction function} $\abs \colon \domainc{\ground} \to \domaina{\ground}$ is given.
Note that it acts on the domains for kind ground.
The first step in the development of the framework will be to lift $\abs$ to the domains for other kinds.
Intuitively, the abstraction $\abs{\fc}$ of a function $\fc \in \domainc{\kappa_1 \to \kappa_2}$ should be the function that takes an abstract value $\xa \in \domaina{\kappa_1}$, \emph{concretizes} it by taking a preimage $\xc \in \domainc{\kappa_1}$ with $\abs{\xc} = \xa$, applies $f$ to the preimage and abstracts the result, $\abs{\xc} \appl \xa = \abs{\fc \appl \xc}$.

This definition has two problems:
The abstract value $\xa$ might have no preimage under $\abs$, and even if it has one, the preimage may not be unique.
To solve the second problem, we introduce the notion of \emph{compatibility}:
A function is compatible (with $\abs$) if values with the same abstraction are evaluated to function values with the same abstraction.
Hence, the value of the abstraction of such a function will not depend on which preimage is chosen.

Furthermore, we define $\abs{f} \appl \xa$ to be the least element of the domain for the appropriate kind whenever $\xa$ has no $\abs$-preimage.
This solves the first problem, but causes another one:
We now cannot hope that $\abs{\solc} = \sola$ will hold for the components of the fixed-point that are functions.
Instead, we will only get what is essentially equality on the image of $\abs$.
We formalize this using an equivalence~$\approx$ on the abstract domain that is equality for kind ground and equality on the image of $\abs$ for function kinds.
In the end, our exact fixed-point transfer result will prove $\abs{\solc} \approx \sola$ instead of the desired equality.
Since $\approx$-equivalence is equality for kind ground, this will still give us equality, $\abs{\solc}(t) = \sola{t}$, whenever $t$ is a term of kind ground.

\begin{remark*}
    In our publication~\cite{HagueMM17}, we solved the problem of missing preimages differently, namely by simply requiring the abstraction function $\abs$ to be surjective.
    This solves the problem, but it has a cost attached to it:
    Showing surjectivity is an additional proof obligation for the user of the framework.
    Indeed, it is often natural to use an abstraction that is not surjective.
    For example, we will consider an abstraction function that is not surjective in \cref{Section:HORSSolving}.
    Usually, it would be possible to restrict the abstract domain so that $\abs$ becomes surjective (and this is in fact the approach we took in the aforementioned publication), but this again is an additional step for the user of the framework.
    The version of the framework we present in this thesis works without requiring surjectivity.
\end{remark*}

\paragraph{Basic definition}

We formally define the notions that we have introduced before.
For each kind $\kappa$, we provide
(1)~a notion of compatibility of elements $\xc \in \domainc{\kappa}$,
(2)~an extension of the abstraction function to kind $\kappa$, \ie
        $\abs \colon \domainc{\kappa} \to \domaina{\kappa}$, and
(3)~an equivalence relation $\approx$ on $\domaina{\kappa}$.

These definitions are interleaved with each other, so we have to proceed using induction.
We start with kind ground, where the abstraction function $\abs \colon \domainc{\ground} \to \domaina{\ground}$ is already fixed.

\begin{definition}
    We define every element of $\domainc{\ground}$ to be compatible.
    We define the equivalence~$\approx$ on $\domaina{\ground}$ to be equality.
\end{definition}

Now consider kind $\kappa_1 \to \kappa_2$, where we assume that $\abs$, $\approx$ and compatibility have been defined for $\kappa_1$ and $\kappa_2$.
Note that for each kind $\kappa$, we denote by $\botc{\kappa}$ and $\bota{\kappa}$  the least element of $\domainc{\kappa}$ and $\domaina{\kappa}$, respectively.

\begin{definition}%
\label{Definition:HORSEFPTBasics}%
    \begin{thmenumerate}[(1)]%
        \item
            A function $\fc \in \domainc{\kappa_1 \to \kappa_2}$ is \emph{compatible} if for every compatible value $\xc \in \domainc{\kappa_1}$, $\fc \appl \xc$ is compatible, and
                for every compatible $\xc, \xc' \in \domainc{\kappa_1}$ such that $\abs{\xc} \approx \abs{\xc'}$, we have $\abs{\fc \appl \xc} \approx \abs{\fc \appl \xc'}$.
        \item
            If $\fc  \in \domainc{\kappa_1 \to \kappa_2}$ is compatible, we define its \emph{abstraction} $\abs{\fc} \in \domaina{\kappa_1 \to \kappa_2}$ as follows:
            Whenever $\xa \in \domaina{\kappa_1}$ is $\approx$-related to the $\abs$-image of a compatible element, \ie $\xa \approx \abs{\xc}$ for some $\xc \in \domainc{\kappa_1}$, we define the function value $\abs{\fc} \appl \xa = \abs{\fc \appl \xc}$.
            If no such $\xc$ exists, we define $\abs{\fc} = \bota{\kappa_1}$ to be the least element of $\domaina{\kappa_1}$.

            For non-compatible $\fc$, we define $\abs{\fc} = \bota{\kappa_1 \to \kappa_2}$ as the bottom function for kind $\kappa_1 \to \kappa_2$, the function that maps any value from $\domaina{\kappa_1}$ to $\bota{\kappa_2}$.
        \item
            The \emph{equivalence} $\approx$ on $\domaina{\kappa_1 \to \kappa_2}$ is defined as follows: For $\fa, \fa' \in \domaina{\kappa_1 \to \kappa_2}$, $\fa \approx \fa'$ holds if for any $\xa \in \domaina{\kappa_1}$ that is $\approx$-equivalent to the $\abs$-image of a compatible element, $\xa \cong \abs{\xc}$ for some $\xc \in \domainc{\kappa_1}$, we have $\fa \appl \xa \approx \fa' \appl \xa$.
    \end{thmenumerate}
\end{definition}

Before studying the definition in more detail, we consider an example.

\begin{example}%
\label{Example:HORSEFPTBasicDefinitions}%
    Consider kind $\ground \to \ground$.
    A function $\fc \in \domainc{\ground \to \ground}$ is compatible if for any two $\xc,\xc \in \domainc{\ground}$, $\abs{\xc} = \abs{\xc}$ implies $\abs{\fc \appl \xc} = \abs{\fc \appl \xc'}$.
    Note that this is just the second part of the definition of compatibility, since the first one is trivially satisfied.
    Also, the fact that $\approx$ is just equality on $\domainc{\ground}$ and that every element of $\domainc{\ground}$ is compatible play a rule here.

    Such a compatible function $\fc$ can be abstracted, obtaining $\abs{\fc}$, the function that takes a value $\xa \in \domaina{\ground}$ and returns $\abs{\fc \appl \xc}$, where $\xc \in \domainc{\ground}$ satisfies $\abs{\xc} = \xa$.
    If no such $\xc$ exists, the return value is $\bota{\ground}$.
    This in particular means that $\abs{\fc} \appl \abs{\xc} = \abs{\fc \appl \xc}$ holds for all $\xc \in \domainc{\ground}$.

    Two functions $\fa,\fa \in \domaina{\ground \to \ground}$ are $\approx$-equivalent if for any $\abs{\xc}$, their function \nb{values coincide}.

    This characterization can be generalized easily for any kind of order one with arbitrary arity.
    A concrete function is compatible if swapping its arguments for ones that have the same image under $\abs$ does not change the $\abs$-image of the function value.
    The abstraction of the function takes a preimage for each abstract value, applies the function, and abstracts the result.
    Two abstract functions are $\approx$-equivalent if their function values coincide on the image of $\abs$.
\end{example}

We turn back to discussing the intricacies of \cref{Definition:HORSEFPTBasics}.
It is not hard to verify using induction that $\approx$ is indeed an equivalence.
However, it quickly turns our there is an issue with the well-definedness of $\abs$ for compatible functions $\fc$.
The value of $\abs{\fc} \appl \xa$ depends on which preimage $\xc$ we chose.
However, assume $\xc, \xc'$ are both compatible elements with $\abs{\xc} \approx \abs{\xc'} \approx \xa$.
Because $\fc$ is compatible, we get $\abs{\fc \appl \xc} \approx \abs{\fc \appl \xc'}$.
Even if we chose a different preimage for $\xa$, we end up with an $\approx$-equivalent result.
While $\abs{\fc}$ is technically not well-defined, it is at least well-defined modulo applying the equivalence on the function domain.
We will take care of this issue by making sure that we never compare the function values of $\abs{\fc}$ using equality.
Instead, we always use $\approx$-equivalence in the following.

There is another issue.
Recall that $\domaina{\kappa_1 \to \kappa_2}$ is defined to be the set of join-continuous functions from $\domaina{\kappa_1}$ to $\domaina{\kappa_2}$.
Hence, $\abs{\fc}$ should be join-continuous, which in particular implies that it should be monotonic.
From the definition, it is obvious that monotonicity does not hold:
If we feed $\abs{\fc}$ an abstract value $\xa$ that is not $\approx$-equivalent to some $\abs{\xc}$, then the result $\abs{\fc} \appl \xa$ will be the least element of $\domaina{\kappa_1}$.
However, we will later show that $\abs{\fc}$ is indeed join-continuous and hence monotonic if we restrict ourselves to applying it to values of the shape $\abs{\xc}$ where $\xc$ is compatible.
As in the above case, we will circumvent the issue by making sure that we never use the monotonicity or join-continuity of $\abs{\fc}$ for arguments that are not of this shape.

One could solve both problems -- the well-definedness of $\abs{\fc}$ and its join-continuity -- by modifying the abstract domain.
Instead of considering $\domaina{\kappa}$, we should only consider its subset $\abs{\Set{\xc \in \domainc{\kappa}}{\xc \text{ compatible}}}$, the image of $\abs$ on compatible elements, and we should additionally factorize modulo the equivalence relation~$\approx$.
Not taking this approach will require us to be more careful, but it will also improve readability.

\paragraph{Basic properties}

Later, we will put additional requirements on $\abs$ to be able to prove  exact-fixed point transfer.
For now, we will show some properties of the notions that we have introduced that do not require further restrictions.
The first one is that the equivalence $\approx$ is well-behaved with respect to forming limits:
The join of an ascending chain remains the same (modulo $\approx$) if we exchange the elements of the chain for $\approx$-equivalent ones.

\begin{lemma}%
\label{Lemma:HORSEFPTApproxLimits}%
    If ${(\xai)}_{i \in \N}$ and ${(\xai')}_{i \in \N}$ are ascending chains in $\domaina{\kappa}$ for some $\kappa$ such that $\xai \approx \xai'$ for all $i$, then $\bigsqcup_{i \in \N } \xai \approx \bigsqcup_{i \in \N} \xai'$.
\end{lemma}

\begin{proof}
    We proceed by induction on the kind $\kappa$.
    In the base case, the statement is trivial since $\approx$ is equality for kind ground.

    Consider kind $\kappa_1 \to \kappa_2$, and let ${(\fai)}_{i \in \N}$ and ${(\fai')}_{i \in \N}$ be two ascending chains from $\domaina{\kappa_1 \to \kappa_2}$ such that $\fai \approx \fai'$ for all $i$.
    The join of these chains is a function, and the definition of $\approx$ for functions requires us to consider $\xa \in \domaina{\kappa_1}$ such that $\xa \approx \abs{\xc}$ for some compatible $\xc$.
    We have to argue that
    \[
        \big(\bigsqcup_{i \in \N } \fai\big) \appl \xa
        \approx
        \big(\bigsqcup_{i \in \N} \fai'\big) \appl \xa
    \]
    holds.
    Using the definition of the join on the function domain, this simplifies to showing
    \[
        \bigsqcup_{i \in \N } \paren{\fai \appl \xa}
        \approx
        \bigsqcup_{i \in \N} \paren{\fai' \appl \xa}
        \ .
    \]
    Since $\fai \approx \fai'$, we have $\fai \appl \xa \approx \fai' \appl \xa$ for all $i$.
    Hence, we may apply induction to conclude the desired statement.
\end{proof}

The other result tackles the following problem:
The definition of $\approx$ essentially guarantees that $\fa \approx \fa'$ implies $\fa \appl \xa \approx \fa' \appl \xa$.
However, we will later also need that $\approx$ is a congruence, \ie that $\fa \appl \xa \approx \fa \appl \xa'$ if $\xa \approx \xa'$.
We then get the congruence property that will be required: If $\fa \approx \fa'$ and $\xa \approx \xa'$, then $\fa \appl \xa \approx \fa' \appl \xa'$.

Unfortunately, $\approx$ is not a congruence on $\domaina{\kappa}$ in general.
Instead, we define the notion of \emph{permeable} elements of $\domaina{\kappa}$ such that $\approx$ is a congruence on these elements.
Our definition of permeability is again by induction on the kind.

For kind ground, an element $\xa \in \domaina{\ground}$ is permeable if $\xa = \abs{\xc}$ for some compatible $\xc \in \domainc{\ground}$.
(Since $\approx$ is equality for ground kind, this is equivalent to requiring $\xa \approx \abs{\xc}$.)

A function $\fa \in \domaina{\kappa_1 \to \kappa_2}$ is permeable if
(1) $\fa \approx \abs{\fc}$ for some $\fc \in \domainc{\kappa_1 \to \kappa_2}$,
(2) for any permeable $\xa \in \domain{\kappa_1}$, $\fa \appl \xa \in \domaina{\kappa_2}$ is permeable,
and
(3) for permeable $\xa, \xa' \in \domain{\kappa_1}$ with $\xa \approx \xa'$, we have $\fa \appl \xa \approx \fa \appl \xa'$.

Part~(3) of the definition of permeability is exactly the congruence with respect to $\approx$ that we will need later.
We show that permeability is well-behaved with respect to $\approx$.

\begin{lemma}%
\label{Lemma:HORSEFPTApproxCompatibleWellDef}%
    If $\fa$ is permeable and $\fa \approx \fa'$, then so is $\fa'$.
\end{lemma}

\begin{proof}
    We proceed by induction on the kind, where the base case is trivial since $\approx$ is equality for ground kind.
    Consider functions $\fa, \fa' \in \domaina{\kappa_1 \to \kappa_2}$ such that $\fa$ is permeable and $\fa \approx \fa'$.
    We argue that $\fa'$ is permeable.
    For~(1), we have $\fa' \approx \fa \approx \abs{\fc}$ for some compatible $\fc$ using the fact that $\fa$ is permeable and the transitivity of $\approx$.
    For~(2), consider a permeable $\xa \in \domain{\kappa_1}$.
    We have $\fa \appl \xa \approx \fa' \appl \xa$, where the former value is permeable by the assumption that $\fa$ is permeable.
    We apply induction to get that also the latter value is permeable.
    For~(3), consider permeable $\xa, \xa' \in \domain{\kappa_1}$ with $\xa \approx \xa'$.
    We have $\fa' \appl \xa \approx \fa \appl \xa \approx \fa \appl \xa' \approx \fa' \appl \xa'$, where we use the definition of the equivalence~$\approx$ twice and the fact that $\fa$ is assumed to be permeable.
\end{proof}

The notion of permeability may give the impression that it introduces a new proof obligation on the user of the framework.
However, this is not the case: We will show that any $\abs$-image $\abs{\xc}$ of a compatible $\xc$ is permeable.
By \cref{Lemma:HORSEFPTApproxCompatibleWellDef}, this also means that any element that is $\approx$-equivalent to an $\abs$-image is permeable.
Before we can formally prove this, we need to state some additional requirements on $\abs$ that we will need to make in the rest of this section.

\paragraph{Precision}

The exact fixed-point transfer will require $\abs$ to have some special properties.

\begin{definition}%
\label{Definition:HORSEFPTPrecise}%
    We call an abstraction function $\abs \colon \domainc{\ground} \to \domaina{\ground}$ \emph{precise} if
    \customlabel{Property:HORSPrecisionBottom}{(P1)} $\abs{\botc{\ground}} = \bota{\ground}$,
    \customlabel{Property:HORSPrecisionCont}{(P2)} $\abs$ is join-continuous on $\domainc{\ground}$:
            If ${(\xci)}_{i \in \N}$ is an ascending chain in $\domainc{\ground}$, then $\abs{\bigsqcup_{i \in \N} \xci} = \bigsqcup_{i \in \N} \abs{\xci}$,
    \customlabel{Property:HORSPrecisionCompatibility}{(P3)} for each terminal $s$, $\interpretationc{s}$ is compatible,
    \customlabel{Property:HORSPrecisionInterpretations}{(P4)} for each terminal $s$, $\abs{s^{\interpretationc}} \approx s^{\interpretationa}$.
\end{definition}

The first requirement is that $\abs$ maps the least element of the concrete domain to the least element of the abstract domain.
The second is that $\abs$ is join-continuous.
Since the least fixed point is, by Kleene's theorem, the join of an ascending chain that starts with the bottom element, requiring these properties is not unexpected.
Note that we only require these properties for kind ground.
We will show that this is sufficient to be able to lift them to arbitrary the domains for arbitrary kinds.

The latter two requirements are that the concrete interpretation of the terminals are compatible and that under $\abs$, they are $\approx$-equivalent to the abstract interpretations.
Recall that we have required terminals to be first order.
This means that the definitions if $\abs$ and $\approx$ are simple, as we have explained in \cref{Example:HORSEFPTBasicDefinitions}.
We lift the latter two properties from the terminals to arbitrary HORS terms later.
This lifting result be the key ingredient for proving exact fixed-point transfer.
At the end of the section, we will give an overview of what the user of the framework has to show to be able to use it.

We start by considering the properties of $\botc{\kappa}$ and $\bota{\kappa}$, lifting Property~\ref{Property:HORSPrecisionBottom} in the process.

\begin{lemma}%
\label{Lemma:HORSEFPTLiftBottom}%
    Assume that $\abs$ satisfies Property~\ref{Property:HORSPrecisionBottom}.
    For each kind $\kappa$, $\botc{\kappa}$ is compatible and $\abs{\botc{\kappa}} = \bota{\kappa}$.
\end{lemma}

\begin{proof}
    We proceed by induction on the kind $\kappa$.
    In the base case, compatibility of $\botc{\ground}$ is trivial.
    Property~\ref{Property:HORSPrecisionBottom} guarantees $\abs{\botc{\ground}} = \bota{\ground}$.

    Consider kind $\kappa_1 \to \kappa_2$.
    For compatibility, consider any elements $\xc, \xc' \in \domainc{\kappa_1}$.
    We have that $\botc{\kappa_1 \to \kappa_2} \appl \xc = \botc{\kappa_2}$ is compatible by induction.
    Similarly, we have $\abs{\botc{\kappa_1 \to \kappa_2} \appl \xc} = \abs{\botc{\kappa_2}} = \abs{\botc{\kappa_1 \to \kappa_2} \appl \xc'}$ as required.

    To see that $\abs{\botc{\kappa_1 \to \kappa_2}}$ is $\bota{\kappa_1 \to \kappa_2}$, the function that sends any element $\xa \in \domaina{\kappa_1}$ to $\bota{\kappa_2}$, consider such an $\xa$.
    If $\xa$ is not $\approx$-equivalent to the $\abs$-image of some compatible $\xc \in \domainc{\kappa_1}$, $\abs{\botc{\kappa_1 \to \kappa_2}} \appl \xa = \bota{\kappa_2}$ holds by definition.
    Otherwise, we have
    \(
        \abs{\botc{\kappa_1 \to \kappa_2}} \appl \xa
        =
        \abs{\botc{\kappa_1 \to \kappa_2} \appl \xc}
        =
        \abs{\botc{\kappa_2}}
        =
        \bota{\kappa_2}
    \)
    by induction.
\end{proof}

With this lemma, we can also tie up a loose end from earlier.
It enables us to show that $\abs{\fc}$ is always permeable.
Hence, permeability is not an additional proof obligation; we get it for free for any abstract element that is in the image of $\abs$.

\begin{lemma}%
\label{Lemma:HORSEFPTAbstractionApproxCompatible}%
    Assume that $\abs$ satisfies Property~\ref{Property:HORSPrecisionBottom}.
    For any $\fc \in \domainc{\kappa}$, $\abs{\fc} \in \domaina{\kappa}$ is permeable.
\end{lemma}

\begin{proof}
    We proceed by induction on the kind $\kappa$.
    We additionally prove that $\bota{\kappa}$ is permeable.
    For kind~$\ground$, permeability just means being in the $\abs$-image, which is satisfied by assumption.
    The least element $\bota{\kappa}$ is in the image of $\abs$ by Property~\ref{Property:HORSPrecisionBottom}.

    Consider kind $\kappa_1 \to \kappa_2$.
    We first argue that $\bota{\kappa_1 \to \kappa_2}$ is permeable.
    By \cref{Lemma:HORSEFPTLiftBottom}, it is in the image of $\abs$.
    For Part~(2) of the definition of permeability, consider a permeable $\xa \in \domaina{\kappa_1}$ and note that $\bota{\kappa_1 \to \kappa_2} \appl \xa = \bota{\kappa_2}$, which is permeable by induction.
    Similarly, for Part~(3), consider permeable $\xa, \xa' \in \domaina{\kappa_1}$ and we get $\bota{\kappa_1 \to \kappa_2} \appl \xa = \bota{\kappa_2} = \bota{\kappa_1 \to \kappa_2} \appl \xa'$.

    Let us now consider $\fc \in \domainc{\kappa_1 \to \kappa_2}$.
    We need to prove that $\abs{\fc}$ is permeable, where the first part of the definition is trivially satisfied.
    If $\fc$ is not compatible, $\abs{\fc} = \bota{\kappa_1 \to \kappa_2}$, and we have just argued that this value is permeable.
    Let us assume that $\fc$ is compatible, and let $\xa \approx \xa'$ be permeable values.
    This in particular means $\xa \approx \xa' \approx \abs{\xc}$ for some compatible $\xc$.
    Consider $\abs{\fc} \appl \xa \approx \abs{\fc \appl \xc}$, using the definition of $\abs{\fc}$.
    The latter value is permeable by induction, and so is the former by \cref{Lemma:HORSEFPTApproxCompatibleWellDef}.
    For Part~(3) of the definition, note that $\abs{\fc} \appl \xa \approx \abs{\fc \appl \xc} \approx \abs{\fc} \appl \xa'$ as required.
\end{proof}

As a next step, we lift Property~\ref{Property:HORSPrecisionCont}, the join-continuity of $\abs$, beyond kind ground.
To be precise, we show a restricted version of join-continuity.
We only consider ascending chains of compatible elements, and instead of showing that the join of the function value equals the function value of the join, we show that they are $\approx$-equivalent.
For the proof by induction to work, we need to strengthen the induction hypothesis and also show that the limit of a chain of compatible elements is compatible.

\begin{lemma}%
\label{Lemma:HORSEFPTLiftJoinContinuity}%
    Assume that $\abs$ satisfies the Properties~\ref{Property:HORSPrecisionBottom}~and~\ref{Property:HORSPrecisionCont}
    For all $\kappa$:
    The join $\bigsqcup_{i \in \N} \fci$ of an ascending chain of compatible elements ${(\fci)}_{i \in \N}$ from $\domainc{\kappa}$ is compatible, and $\abs$ is join-continuous on such chains:
    $\abs{ \bigsqcup_{i \in \N} \fci } \approx \bigsqcup_{i \in \N} \abs{\fci}$.
\end{lemma}

\begin{proof}
    We proceed by induction on $\kappa$.
    Any element of $\domainc{\ground}$ is compatible by definition, which in particular applies to the join of a chain.
    The join continuity of $\abs$ is simply Property~{\ref{Property:HORSPrecisionCont}}.

    Consider kind $\kappa_1 \to \kappa_2$, and let $\bigsqcup_{i \in \N} \fci$ be an ascending chain of compatible elements ${(\fci)}_{i \in \N}$ from $\domainc{\kappa_1 \to \kappa_2}$.

    For the first part of the definition of compatibility, consider a compatible $\xc \in \domainc{\kappa_1}$
    We have
    \(
        \paren{ \bigsqcup_{i \in \N} \fci } \appl \xc
        =
        \bigsqcup_{i \in \N} \paren{\fci \appl \xc}
    \)
    by the definition of the join on the function domain $\domainc{\kappa_1 \to \kappa_2}$.
    Since $\xc$ is compatible and each $\fci$ is compatible, then so is each $\fci \appl \xc$.
    Additionally, the $\fci$ forming an ascending chain implies their function values $\fci \appl \xc$ forming an ascending chain.
    Induction proves that $\bigsqcup_{i \in \N} \paren{\fci \appl \xc}$ is compatible as desired.

    For the second part, let $\xc, \xc' \in \domainc{\kappa_1}$ be compatible so that $\abs{\xc} \approx \abs{\xc'}$.
    We need to prove
    \[
        \abs \Big( \big( \bigsqcup_{i \in \N} \fci \big) \appl \xc \Big)
        \enspace
        \approx
        \enspace
        \abs \Big( \big( \bigsqcup_{i \in \N} \fci \big) \appl \xc' \Big)
        \ .
    \]
    Using the definition of the join on both sides, it would be sufficient to show
    \[
        \abs \Big( \bigsqcup_{i \in \N} \big(  \fci \appl \xc  \big) \Big)
        \enspace
        \approx
        \enspace
        \abs \Big( \bigsqcup_{i \in \N} \big(  \fci  \appl \xc' \big) \Big)
        \ .
    \]
    As in the first part of the proof, both ${(\fci \appl \xc)}_{i \in \N}$ and ${(\fci \appl \xc')}_{i \in \N}$ are ascending chains of compatible elements in $\domainc{\kappa_2}$.
    We apply induction and use the join-continuity of $\abs$ on $\domainc{\kappa_2}$ to get the desired statement.

    For join-continuity, we need to show
    \(
        \abs { \bigsqcup_{i \in \N} \fci }
        \approx
        \bigsqcup_{i \in \N} \abs {\fci }
    \).
    By the definition of $\approx$, this means we need to consider some $\xa \in \domain{\kappa_1}$ such that $\xa \approx \abs{\xc}$ for some compatible $\xc$.
    We prove
    \(
        \abs { \bigsqcup_{i \in \N} \fci } \appl \xa
        \approx
        \paren{\bigsqcup_{i \in \N} \abs {\fci }} \appl \xa
        \ .
    \)
    By the definition of $\abs$ for functions, we have
    $\abs { \bigsqcup_{i \in \N} \fci } \appl \xa \approx \abs{ ( \bigsqcup_{i \in \N} \fci ) \appl \xc  }$.
    By the definition of the join, this value equals
    $\abs{ \bigsqcup_{i \in \N} (\fci \appl \xc)  }$.
    As before, ${(\fci \appl \xc)}_{i \in \N}$ is an ascending chain of compatible elements, so we apply induction to get
    \(
        \abs \big( \bigsqcup_{i \in \N} (\fci \appl \xc)  \big)
        \approx
        \bigsqcup_{i \in \N} \abs{ \fci \appl \xc }
        \ .
    \)
    Starting from the other side of the desired equality we get
    \(
        \paren{\bigsqcup_{i \in \N} \abs {\fci }} \appl \xa
        =
        \bigsqcup_{i \in \N} \paren{ \abs {\fci } \appl \xa}
    \) using the definition of the join.
    By the definition of $\abs$, we have $\abs {\fci } \appl \xa \approx \abs{ \fci \appl \xc }$, which completes the proof.
\end{proof}

The previous lemma allows us to tie up another loose end.
At the beginning of the section, we have claimed that $\abs{\fc}$ is join-continuous on the $\abs$-images of compatible elements.
We can now prove this statement.

Let $\fc \in \domainc{\kappa_1 \to \kappa_2}$ be a function.
Consider an ascending chain ${(\xci)}_{i \in \N}$ in $\domainc{\kappa_1}$ of compatible elements.
By \cref{Lemma:HORSEFPTLiftJoinContinuity}, $\abs$ is join-continuous and hence monotonic.
We define $\xai = \abs{\xci}$ for each $i$ and obtain that ${(\xai)}_{i \in \N}$ is an ascending chain in $\domaina{\kappa_1}$.
We prove that $\abs{\fc} \appl \paren{\bigsqcup_{i \in \N} \xai} \approx \bigsqcup_{i \in \N} \paren{ \abs{\fc} \appl \xai}$.
By the join-continuity of $\abs$, we get that the join $\bigsqcup_{i \in \N} \xai$ of the $\xai$ is $\approx$-equivalent to the join of the $\xai$, $\bigsqcup_{i \in \N} \xai \approx \abs{\bigsqcup_{i \in \N} \xci}$.
Hence, the definition of $\abs{\fc}$ gives us
\(
    \abs{\fc} \appl \paren{\bigsqcup_{i \in \N} \xai}
    \approx
    \abs{\fc \appl \paren{\bigsqcup_{i \in \N} \xci}}
    \ .
\)
The function $\fc$ is join-continuous by our definition of $\domainc{\kappa_1 \to \kappa_2}$, so we can transform the latter expression to
\(
    \abs{\bigsqcup_{i \in \N} \fc \appl \xci}
    \ .
\)
We use the join-continuity of $\abs$ and the definition of $\abs{\fc}$ to get
\(
    \bigsqcup_{i \in \N} \abs{\fc \appl \xci}
    \approx
    \bigsqcup_{i \in \N} \paren{\abs{\fc} \appl \xai}
    \ ,
\)
which proves the desired statement.

\paragraph{Exact fixed-point transfer}

We hone in on proving exact fixed-point transfer.
The key step is lifting the Properties~\ref{Property:HORSPrecisionCompatibility}~and~\ref{Property:HORSPrecisionInterpretations} from the definition of precision from the nonterminals to arbitrary HORS terms.

Recall that the semantics $\sem{\model}{t}$ of a HORS term $t$ is a function with signature $(N \dotcup V \pto \domain) \to \domain$ that expects a valuation, an assignment of values to the nonterminals and the HORS variables that are free in $t$.
We extend some of the notions that we have defined in this section to valuations in the expected way.
A concrete valuation $\valc \colon N \dotcup V \pto \domainc$ is compatible if $\valc(x)$ is compatible for all $x$ on which $\valc$ is defined.
For such a concrete valuation $\valc$, $\abs{\valc} \colon N \dotcup V \pto \domaina$ is the abstract valuation with $\abs{\valc}(x) = \abs{\valc{(x)}}$ for all $x$ on which $\valc$ is defined.

We can now prove the proposition that shows that $\semc{t}{\val}$ is compatible and that its $\abs$-image is $\approx$-equivalent to $\sema{t}{\abs{\val}}$.
We also prove that if we modify the valuation $\abs{\val}$ by exchanging its entries for $\approx$-equivalent ones, we obtain an $\approx$-equivalent result.


\begin{proposition}%
\label{Proposition:HORSEFPTLiftToSemantics}%
    Assume $\abs$ to be precise.
    Let $t$ be a term and $\valc$ be a compatible valuation.
    \begin{enumerate}[a)]
        \item $\semc{t}{\valc}$ is compatible.
        \item $\abs{\semc{t}{\valc}} \approx \sema{t}{\abs{\valc}}$.
        \item If an abstract valuation $\vala \colon N \dotcup V \pto \domaina$ satisfies $\vala \approx \abs{\valc}$, meaning $\vala(x) \approx \abs{\valc}(x)$ for all~$x$, then $\sema{t}{\abs{\valc}} \approx \sema{t}{\vala}$.
    \end{enumerate}
\end{proposition}

\begin{proof}
    We proceed by induction on the structure of the term $t$ and prove all statements \nb{simultaneously}.

    \subparagraph{Base case $t = s$ for a terminal $s$}
    %
    We have $\semc{t}{\valc} = s^{\interpretationa}$.
    For Part~a), note that compatibility is simply Property~\ref{Property:HORSPrecisionCompatibility}.
    %
    For Part~b), we have $\abs{\semc{t}{\valc}} = \abs{s^{\interpretationa}} \approx s^{\interpretationa} = \sema{t}{\abs{\valc}}$ using Property~\ref{Property:HORSPrecisionInterpretations} and the fact that the valuation does not play a role when evaluating a terminal.
    %
    The latter fact also proves $\sema{s}{\abs{\valc}} \approx \sema{s}{\vala}$.

    \subparagraph{Base case $t = F$ for a nonterminal $F$}
    %
    The value $\semc{F}{\valc} = \valc(F)$ is compatible since $\valc$ is compatible.
    %
    We have $\abs{\semc{F}{\valc}} = \abs{\valc(F)} = \abs{\valc}(F) = \sema{F}{\abs{\valc}}$ using the definition of $\abs{\valc}$.
    %
    Finally, $\sema{F}{\abs{\valc}} = \abs{\valc}(F) \approx \vala(F) = \sema{F}{\vala}$ since $\abs{\valc} \approx \vala$ by assumption.


    \subparagraph{Base case $t = x$ for a HORS variable $x$}
    %
    The reasoning is exactly the same as for $t = F$.

    \subparagraph{Inductive step, $t = \lambda x . t_1$}
    %
    Observe that $\semc{\lambda x . t_1}{\valc}$ is a function that takes $\xc$ and returns $\semc{t_1}{\paren{\valc[x \mapsto \xc]}}$.

    For Part~a), we prove compatibility using its definition.
    Let $\xc$ be compatible, then the value $\semc{\lambda x . t_1}{\valc} \appl \xc$ equals $\semc{t_1}{\paren{\valc[x \mapsto \xc]}}$.
    Since both $\valc$ and $\xc$ are compatible, $\valc[x \mapsto \xc]$ is a compatible valuation.
    We apply Part~a) of the proposition using induction and obtain that $\semc{t_1}{\paren{\valc[x \mapsto \xc]}}$ is compatible as required.
    Consider compatible $\xc,\xc'$ with $\abs{\xc} \approx \abs{\xc'}$.
    We have $\abs{\semc{\lambda x . t_1}{\valc} \appl \xc} = \abs{\semc{t_1}{\paren{\valc[x \mapsto \xc]}}} \approx \sema{t_1}{\abs{\valc[x \mapsto \xc]}}$ by using Part~b) of the induction hypothesis.
    Similarly, $\abs{\semc{\lambda x . t_1}{\valc} \appl \xc'} \approx \sema{t_1}{\abs{\valc[x \mapsto \xc']}}$.
    We have $\abs{\valc[x \mapsto \xc]} \approx \abs{\valc[x \mapsto \xc']}$ since $\abs{\xc} \approx \abs{\xc'}$ so we can use Part~c) of the induction hypothesis to conclude $\sema{t_1}{\abs{\valc[x \mapsto \xc]}} \approx \sema{t_1}{\abs{\valc[x \mapsto \xc']}}$ as desired.

    For Part~b), consider $\abs{\semc{\lambda x . t_1}{\valc}}$.
    We need to show that it is $\approx$-equivalent to $\sema{\lambda x . t_1}{\abs{\valc}}$.
    The latter is a function that takes $\xa$ and returns $\sema{t_1}{\abs{\valc}[x \mapsto \xa]}$.
    We check the definition of $\approx$ for functions by considering $\xa$ with $\xa \approx \abs{\xc}$ for some compatible $\xc$.
    We have
    \begin{align*}
        \abs{\semc{\lambda x . t_1}{\valc}} \appl \xa
        &\approx \abs{\semc{\lambda x . t_1}{\valc} \appl \xc}
        \\
        &= \abs{\semc{t_1}{\paren{\valc [x \mapsto \xc]}}}
        \\
        &\approx \sema{t_1}{\abs{\valc [x \mapsto \xc]}}
        \\
        &= \sema{t_1}{\paren{\abs{\valc} [x \mapsto \abs{\xc}]}}
    \end{align*}
    using the definition of $\abs$, the definition of the interpretation of lambda abstraction, and the induction hypothesis.
    The last step is simply evaluating $\abs{\valc [x \mapsto \xc]}$.
    We now observe that $\abs{\valc}[x \mapsto \abs{\xc}] \approx \abs{\valc}[x \mapsto \xa]$ since $\xa \approx \abs{\xc}$.
    We use Part~c) of the induction hypothesis and get
    \begin{align*}
       \sema{t_1}{\paren{\abs{\valc} [x \mapsto \abs{\xc}]}}
       &\approx  \sema{t_1}{\paren{\abs{\valc} [x \mapsto \xa]}}
       = \sema{\lambda x. t_1}{\abs{\valc}} \appl \xa
    \end{align*}
    as desired.

    For Part~c), we compare $\sema{\lambda x . t_1}{\abs{\valc}}$ and $\sema{\lambda x . t_1}{\vala}$.
    Consider $\xa$ such that $\xa \approx \abs{\xc}$ for some compatible $\xc$.
    We have $\sema{\lambda x .  t_1}{\vala} \appl \xa = \sema{t_1}{\paren{\vala[x \mapsto \xa]}}$
    and similarly $\sema{\lambda x .  t_1}{\abs{\valc}} \appl \xa = \sema{t_1}{\paren{\abs{\valc}[x \mapsto \xa]}}$.
    We have $\abs{\valc}[x \mapsto \xa] \approx \abs{\valc[x \mapsto \xc]}$, so we apply the induction hypothesis to get
    $\sema{t_1}{\paren{\abs{\valc}[x \mapsto \xa]}} \approx \sema{t_1}{\abs{\valc[x \mapsto \xc]}}$.
    We now use $\abs{\valc[x \mapsto \xc]} \approx \vala[x \mapsto \xa]$ and apply the induction hypothesis again to get
    $\sema{t_1}{\abs{\valc[x \mapsto \xc]}} \approx \sema{t_1}{\paren{\vala[x \mapsto \xa]}}$ as desired.

    \subparagraph{Inductive step, $t = t_1 \appl t_2$}
    %
    For Part~a), we have
    \[
        \semc{t_1 \appl t_2}{\valc} = \semc{t_1}{\valc} \appl \semc{t_2}{\valc}
    \]
    using the definition of the interpretation of function application.
    Both $\semc{t_1}{\valc}$ and $\semc{t_2}{\valc}$ are compatible by induction.
    The value $\semc{t_1}{\valc}$ is a function, and using the definition of compatibility for functions, the function value of a compatible function at a compatible argument is again compatible.
    We obtain that $\semc{t_1}{\valc} \appl \semc{t_2}{\valc}$ is compatible.

    For Part~b), we get
    \begin{align*}
        \abs{ \semc{t_1 \appl t_2}{\valc} }
        &= \abs{ \semc{t_1}{\valc} \appl \semc{t_2}{\valc} }
        \\
        &\approx \abs{ \semc{t_1}{\valc} } \appl \sema{t_2}{\abs{\valc}}
    \end{align*}
    where the last transformation requires some explanation.
    By induction, we have $\abs{\semc{t_2}{\valc}} \approx \sema{t_2}{\abs{\valc}}$.
    Hence, $\semc{t_2}{\valc}$ is an $\abs$-preimage of $\sema{t_2}{\abs{\valc}}$ (modulo~$\approx$) and we may use the definition of $\abs$ for functions.
    We have $\abs{ \semc{t_1}{\valc} } \approx \sema{t_1}{\abs{\valc}}$ by induction, so following the definition of $\approx$ for functions, we obtain
    \begin{align*}
        \abs{ \semc{t_1}{\valc} } \appl \sema{t_2}{\abs{\valc}}
        & \approx \sema{t_1}{\abs{\valc}} \appl \sema{t_2}{\abs{\valc}}
        \\
        &= \sema{t_1 \appl t_2}{\abs{\valc}}
        \ ,
    \end{align*}
    proving the desired statement.

    Finally, we consider Part~c).
    We have
    \[
        \sema{t_1 \appl t_2}{\abs{\valc}}
        \ = \
        \sema{t_1}{\abs{\valc}}
        \appl
        \sema{t_2}{\abs{\valc}}
        \ .
    \]
    By induction $\sema{t_1}{\abs{\valc}} \approx \sema{t_1}{\vala}$ and
    $\sema{t_2}{\abs{\valc}} \approx \sema{t_2}{\vala}$ hold.
    Also by induction, we know that $\sema{t_2}{\abs{\valc}}$ is $\approx$ related to an element in the image of $\abs$ (namely $\semc{t_2}{\valc}$), so we can use the definition of $\approx$ for functions and get
    \[
        \sema{t_1}{\abs{\valc}} \appl \sema{t_2}{\abs{\valc}}
        \ \approx \
        \sema{t_1}{\vala} \appl \sema{t_2}{\abs{\valc}}
        \ .
    \]
    Have you noticed that we have not used the concept of permeability yet?
    The value $\abs{\semc{t_1}{\valc}}$ is permeable by \cref{Lemma:HORSEFPTAbstractionApproxCompatible}.
    Since $\abs{\semc{t_1}{\valc}} \approx  \sema{t_1}{\abs{\valc}} \approx \sema{t_1}{\vala}$, so is $\sema{t_1}{\vala}$ by \cref{Lemma:HORSEFPTApproxCompatibleWellDef}.
    Similarly, $\abs{\semc{t_2}{\valc}}$ is permeable by \cref{Lemma:HORSEFPTAbstractionApproxCompatible}, and both $\sema{t_2}{\abs{\valc}}$ and $\sema{t_2}{\vala}$ are permeable because they are $\approx$-equivalent to this value.
    We use Property~(3) from the definition of permeability to conclude
    \[
        \sema{t_1}{\vala} \appl \sema{t_2}{\abs{\valc}}
        \ \approx \
        \sema{t_1}{\vala} \appl \sema{t_2}{\vala}
        \ = \
        \sema{t_1 \appl t_2}{\vala}
        \ .
    \]
    This completes the proof.
\end{proof}

%
\cheatpagebreak
%

The previous proposition enables us to prove our main result, exact fixed-point transfer.

\begin{theorem}%
\label{Theorem:HORSEFPT}%
    If $\abs$ is precise, then $\sola \approx \abs{\solc}$.
\end{theorem}

Here, we use the notation for valuations, \ie  $\sola \approx \abs{\solc}$ means that for every nonterminal $F$, we have $\sola{F} \approx \abs{\solc}(F) = \abs{\solc{F}}$.
This result also extends to arbitrary terms $t$: $\sola{t} \approx \abs{\solc{t}}$.
If $t$ is variable-closed and of kind $\ground$, we even get $\sola{t} = \abs{\solc{t}}$ because $\approx$ is simply equality on $\domaina{\ground}$.
We state this result as a corollary.

\begin{corollary}%
\label{Corollary:HORSEFPT}%
    If $t$ is a variable-closed term of kind ground, then $\sola{t} = \abs{\solc{t}}$.
\end{corollary}

Note that the corollary in articular applies to the initial nonterminal $S$ of the scheme, which we have required to be of kind $\ground$.

To finish this section, it remains to prove the theorem.


\begin{figure}
    \onehalfspacing%
    {%
    \setlength{\fboxsep}{1em}
    \fbox%
    {%
        \begin{minipage}{\textwidth-2.5em}
            \vspace*{0.5em}
            %
            \paragraph{\large Overview: Exact fixed-point transfer}
            %
            \vspace{1em}
            %
            Consider a deterministic HORS $G$, a concrete model $\modelc = (\domainc,\interpretationc)$, and an abstract model $\modela = (\domaina,\interpretationa)$, both instantiations of the model template described in \cref{Section:HORSTemplate} (see \cref{Figure:HORSTemplateCheatSheet}).
            Note that these models in particular provide domains $\domainc{\ground}$ and $\domaina{\ground}$ for kind~$\ground$.

            We get the existence of the least solution $\solc$ and $\sola$ of the interpreted system of equations associated to $G$ for these models.
            %
            \vspace*{1em}
            %
            \paragraph{Precision}
            Assume we are given an abstraction function $\abs \colon \domainc{\ground} \to \domaina{\ground}$ satisfying the \nb{following properties}.
            \begin{itemize}
                \item
                    $\abs$ maps the least element $\botc{\ground} \in \domainc{\ground}$ to the least element $\bota{\ground} \in \domaina{\ground}$, $\abs{\botc{\ground}} \ = \  \bota{\ground}$.
                \item
                    $\abs \colon \domainc{\ground} \to \domaina{\ground}$ is join-continuous.
                \item
                    For each terminal symbol $s$ of order one, say with arity $k$, consider $x_1, x_1', \ldots, x_k, x_k' \in \domainc{\ground}$ so that $\abs{x_i} = \abs{x_i'}$ for all $i \in \oneto{k}$.
                    \\
                    The concrete interpretation of $s$ should satisfy
                    \[
                        \abs{ s^{\interpretationc} \appl x_1 \ldots x_k }
                        \ = \
                        \abs{ s^{\interpretationc} \appl x_1' \ldots x_k' }
                        \ .
                    \]
                \item
                    For each terminal symbol $s$, say with arity $k$, consider $x_1, \ldots, x_k \in \domainc{\ground}$.
                    \\
                    The interpretations of $s$ should satisfy
                    \[
                        \abs{ s^{\interpretationc} \appl x_1 \ldots x_k }
                        \ = \
                        s^{\interpretationa} \appl \abs{x_1} \ldots \abs{x_k}
                        \ .
                    \]
            \end{itemize}

            \paragraph{Exact fixed-point transfer}
            Then for each variable-closed term $t$ of kind ground, the abstraction of the value assigned to $t$ by the concrete least solution is the value of $t$ in the abstract least solution,
            \[
                \abs{\solc{t}} \ = \ \sola{t}
                \ .
            \]
            \vspace*{0.1em}
        \end{minipage}
    }
    }
    \caption{Overview: Exact fixed-point transfer.}%
    \label{Figure:HORSEFPTCheatSheet}%
\end{figure}

\begin{proof}[Proof of \cref{Theorem:HORSEFPT}.]
    Firstly, we show
    \[
        \abs{\solci{i}} \approx \solai{i}
    \]
    for every $i \in \N$ using induction.
    We also argue that each $\solci{i}$ is compatible.

    In the base case, consider $i = 0$.
    For a nonterminal $F$ of kind $\kappa$, $\sola{0}{F} = \bota{\kappa}$, as the $\nth{0}$ approximant is the least element in every component.
    Similarly, $\abs{\solc{0}{F}} = \abs{\botc{\kappa}} = \bota{\kappa}$ using \cref{Lemma:HORSEFPTLiftBottom}.
    Also by \cref{Lemma:HORSEFPTLiftBottom}, we get that $\solci{0}$ is compatible.

    For the induction step, assume we have proven $\abs{\solci{i}} \approx \solai{i}$ and consider $i+1$.
    Consider a nonterminal $F$ of kind $\kappa$ whose unique rule in the HORS is $F \to t$.
    By the definition of the $\nth{(i + 1)}{st}$ approximant, we have
    \( \solc{i+1}{F} = \semc{t}{\solci{i}} \), similar for $\sola$.
    By induction, $\solci{i}$ is compatible, then so is $\semc{t}{\solci{i}}$ using Part~a) of \cref{Proposition:HORSEFPTLiftToSemantics}.
    We get
    \begin{align*}
        \abs{\solc{i+1}{F}}
        &=
        \abs{\semc{t}{\solci{i}}}
        \\
        &\approx
        \sema{t}{\abs{\solci{i}}}
        \\
        &\approx
        \sema{t}{\solai{i}}
        \\
        &= \sola{i+1}{F}
        \ .
    \end{align*}
    The first and the last equality are the above observation.
    The second transformation is Part~b) of \cref{Proposition:HORSEFPTLiftToSemantics}, which we can apply because $\solci{i}$ is compatible by induction.
    The penultimate transformation is $\abs{\solci{i}} \approx \solai{i}$ together with Part~c) of \cref{Proposition:HORSEFPTLiftToSemantics}.

    Equipped with the knowledge that $\abs{\solci{i}} \approx \solai{i}$ for all $i$, we turn to proving $\abs{\solc} \approx \sola$.
    Kleene's theorem, \cref{Theorem:EDSKleene}, gives us
    \[
        \solc \ = \ \bigsqcup_{i \in \N} \solci{i}
        \qquad \text{and}
        \qquad
        \sola \ = \ \bigsqcup_{i \in \N} \solai{i}
        \ ,
    \]
    see also \cref{Proposition:HORSTemplateExistenceLeastSolution}.
    We get
    \begin{align*}
        \abs{\solc}
        \ = \
        \abs \big( \bigsqcup_{i \in \N} \solci{i} \big)
        \ \approx \
        \bigsqcup_{i \in \N} \abs{\solci{i}}
        \ \approx \
        \bigsqcup_{i \in \N} \solai{i}
        =
        \sola
        \ .
    \end{align*}
    The first and the last equality are Kleene's theorem.
    The second transformation is the join-continuity of $\abs$ on chains of compatible elements, \cref{Lemma:HORSEFPTLiftJoinContinuity}, together with our earlier observation that each $\solci{i}$ is compatible.
    The penultimate transformation is the very first lemma in this section, \cref{Lemma:HORSEFPTApproxLimits}, together with $\abs{\solci{i}} \approx \solai{i}$ for all $i$.
\end{proof}

We conclude the section with an \emph{overview} in the form of \cref{Figure:HORSEFPTCheatSheet}.
It lists what a potential user of the framework has to show to guarantee that \cref{Corollary:HORSEFPT} holds.
The contents of the overview are based on the basic definition from \cref{Definition:HORSEFPTBasics}, the observation of what these definitions mean for order one in \cref{Example:HORSEFPTBasicDefinitions}, and the definition of precision in \cref{Definition:HORSEFPTPrecise}.


\end{document}
