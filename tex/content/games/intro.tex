\documentclass[../../diss.tex]{subfiles}
\begin{document}

\chapter*{Part V.\newline Games}

This part of the thesis is concerned with solving games.
We have argued in \cref{Section:IntroGames} of the introduction that games whose game arenas are defined by automata are related to the area of program verification.
In particular, the task of a solving synthesis problem can be approached by translating it into an equivalent game.

\paragraph{Outline}

We start by formally define board games for two players with perfect information in \cref{Chapter:Games}.
We discuss the intricacies that arise from considering inclusion games where the winning condition is membership in a given target language.

Before we actually turn to solving games, we explain effective denotational semantics in \cref{Chapter:EDS}.
Effective denotational semantics is an approach to solving decision problems that is based on computing the least solution to a system of equations, and it will be our preferred course of action throughout the rest of this part.
We demonstrate how to use effective denotational semantics by applying it to the regular inclusion problem for context-free languages and the $\omega$-regular inclusion problem for $\omega$-context-free languages.

In \cref{Chapter:ContextFreeGames}, we consider the problem of solving context-free inclusion games, games whose game arena is defined by a context-free grammar and whose winning condition is membership in a regular target language.
We present an algorithm solving such games with the optimal time complexity that is based on effective denotational semantics.

In \cref{Chapter:HOGames}, we extend this approach to games defined by higher-order recursion schemes.
In addition to solving such games, we study effective denotational semantics for verification problems based on higher-order schemes on a more general level, proving an exact fixed-point transfer result that is of independent interest.

Finally, \cref{Chapter:ValenceGames} is concerned with studying the frontier of the decidability of games.
To this end, we use the model of valence systems, a type of automaton that generalizes well-known models like pushdown automata and Petri nets.
We will prove a classification result that shows that context-free games are essentially the only decidable type of reachability games that can be modeled using valence systems over graph monoids.
We propose using bounded context switching as an approach to obtain decidability in a wider range of cases.

\paragraph{Publication}

This part of the thesis is based on the publications~\cite{HolikMM16} (\resp its full version~\cite{HolikMM16a}),~\cite{HagueMM17} (\resp its full version~\cite{HagueMM17a}),~\cite{MeyerMZ18} (\resp its full version~\cite{MeyerMZ18a}), and~\cite{MeyerMN17a}.
At the beginning of each chapter, we will provide more detailed information about which paper the chapter is based on.
In \cref{Chapter:Contributions}, we will discuss the contribution the author made to these publications.

\end{document}
