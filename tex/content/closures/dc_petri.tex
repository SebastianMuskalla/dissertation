\documentclass[../../diss.tex]{subfiles}
\begin{document}

\section{Downward closures for Petri nets}%
\label{Section:PNDC}%

We first consider the problem of computing the downward closure of a given Petri net coverability language, formalized as follows.

\begin{compproblem}
    \problemtitle{Computing the downward closure}
    \problemshort{($\PNDC$)}
    \probleminput{A labeled Petri net instance $(N,\Minit,\Mfinal)$.}
    \problemquestion{An NFA $A$ with $\lang{A} = \dc{\lang{N,\Minit,\Mfinal}}$.}
\end{compproblem}

Our goal is to prove the following result.

\begin{theorem}%
\label{Theorem:PNDCGeneral}%
    Downward closures of Petri net coverability languages have non-primitive-recursive state complexity, and the corresponding automata can be constructed in non-primitive recursive time.
    These bounds are tight.
\end{theorem}

The computability has already been shown by~\citea{HabermehlMW10}.
We will start by recalling the proof.
Later, we give a matching lower bound.

\paragraph{Computability / Upper bound}

It has been shown in~\cite{HabermehlMW10} that the downward closure of Petri net languages is computable.
In addition to the main result, which shows the computability for Petri net reachability languages, the paper also provides an algorithm specifically for coverability languages.
The advantage of the latter is that the construction does not rely on techniques from the proof of the decidability of Petri net reachability~\cite{Lambert92}, but only on the coverability graph~\cite{KarpM69}.

\begin{theorem}[(\citea{HabermehlMW10})]%
\label{Theorem:PNComputeDC}%
    The downward closures of Petri net languages are computable.
\end{theorem}

We briefly recall how, given a Petri net instance $(N,\Minit,\Mfinal)$, an NFA $A$ representing the downward closure, \ie $\lang{A} = \dc{\lang{N,\Minit,\Mfinal}}$, can be constructed.
For the proof of correctness, we refer to~\cite{HabermehlMW10}.
The construction relies on the \emph{Karp-Miller tree}~\cite{KarpM69} associated to the Petri net, a standard construction to represent all markings that can be covered from the initial one.

To explain the construction, we will need the notion of \emph{generalized markings} that we have presented in \cref{Section:PNCovLang}.
We briefly recall the definition.
We define $\N_\omega = \N \cup \set{\omega}$, the natural numbers plus a new top element.
The order and addition are extended from $\N$ to $\N_\omega$ by defining $n < \omega$ and $\omega + n = \omega - n = \omega$ for all numbers $n \in \N$.
(In the following, we will never need to add $\omega$ to or subtract $\omega$ from itself.)
A \emph{generalized marking} is a function $M \colon P \to \N_\omega$ that assigns to each place a finite number of tokens, or the special value $\omega$.
Intuitively, $M(p) = \omega$ means that $p$ carries unboundedly many tokens.
A transition $t$ is enabled in generalized marking $M$ if $M \geq \i(t)$ (which means that places $p$ with $M(p) = \omega$ are essentially ignored), and firing it then yields $M \fire{t} M + e(t)$.

With these preliminaries at hand, we can specify the construction of the Karp-Miller tree.
It is a finite tree in which the nodes are labeled by generalized markings, and in which the edges are labeled by transitions.
During the construction, we store for each node of the graph whether it has been processed or not.
The construction is set up so that all unprocessed nodes are leaves of the tree.

Initially, we consider a tree that solely consists of the root node that is labeled by the initial marking $\Minit$ and not yet processed.
While the tree contains an unprocessed leaf, we proceed as follows:
Pick an unprocessed leaf $v$, say labeled by generalized marking $M$, and mark it as processed.
For each transition $t$ that is enabled in $M$, compute $M'$ with $M \fire{t} M'$.
For each place $p$, check whether on the unique path from the root of the tree to $v$, there is a marking $\hat{M}$ with $\hat{M} \leq M'$ and $\hat{M}(p) < M'(p)$.
If so, we modify $M'$ by setting $M'(p) = \omega$.
After iterating over all places, we insert a new leaf $v'$ into the tree, labeled by $M'$, and add a $t$-labeled edge from $v$ to $v'$.
If the tree already contains a node labeled by $M'$, mark $v'$ as processed.
Else, it is unprocessed.

Intuitively, the construction explores the behavior of the Petri net while tracking \emph{accelerations} or \emph{pumps}, \ie infixes of the computation that lead from a marking $\hat{M}$ to a marking $M'$ that is strictly greater than $\hat{M}$.
If such a pump has been discovered, it could be inserted as often as desired to obtain arbitrarily high token counts on the places $p$ with $\hat{M}(p) < M'(p)$.

The termination of the algorithm relies on WQO arguments:
$(\N_\omega,\leq)$ is a WQO, similar to $(\N,\leq)$, see Part~\ref{Example:WQONaturals}) of \cref{Example:WQO}.
Hence, also the product order on $\N_\omega^P$ is a WQO.\@
The WQO properties guarantee that each path in the tree is finite:
After finitely many steps, for each place we have either discovered a pump that sets this place to $\omega$ in the successor, or we have explored all the finitely many values that $M(p)$ can obtain in reachable markings.
Now observe that the out-degree of each node of the tree is bounded by the number of transitions.
Both facts combined yield the finiteness of the Karp-Miller tree.

One should mention that the Karp-Miller tree is not unique, as its construction depends on the order in which unprocessed leafs and transitions are considered.
However, the interesting properties of the tree are independent of this order.
In the literature, the Karp-Miller tree is often transformed into the so-called \emph{coverability graph} by merging all nodes that are labeled by the same generalized marking.
For our purpose, we will keep the tree structure.

The Karp-Miller tree is commonly used to represent the set \emph{coverability set} of a net $N$ together with an initial marking $\Minit$.
A marking $\Mfinal$ is coverable in $N$ from $\Minit$ (\ie it is contained in the coverability set) if and only if the Karp-Miller tree contains some node labeled by a generalized marking $M$ with $\Mfinal \leq M$.
% Phrased differently, the downward closure (with respect to the product order on $N_\omega^P$) of the set of generalized markings labeling nodes of the Karp-Miller tree intersected with $\N^P$ is exactly the coverability set.
Since the tree is finite, this yields a simple procedure to decide coverability.
Unlike Rackoff's procedure~\cite{Rackoff78}, its worst-case complexity is not optimal, as we will discuss below.

\citea{HabermehlMW10} provide a procedure that constructs the downward closure of a Petri net language based on the Karp-Miller tree.
Consider the NFA $A = (Q, \delta, v_\init, Q_F)$, where $Q$ is the set of nodes of the Karp-Miller tree, $v_\init$ is the root node, and $Q_F$ is the set of nodes labeled by generalized markings $M'$ with $\Mfinal \leq M'$.
The transitions decompose into the transitions induced by the Karp-Miller tree and back edges.
Each edge $v \to v'$ in the Karp-Miller tree, say labeled by transition $t$, induces the transition $v \tow{\lambda(t)} v'$ in $A$.
If node $v$ is labeled by the generalized marking $M$, and $\hat{v}$ is a node on the unique path from the root to $v$, labeled by a marking $\hat{M}$ with $\hat{M} \leq M$, we add the transition $v \tow{\eps} \hat{v}$.

NFA $A$ has the property that its downward closure coincides with the downward closure of the coverability language.

\begin{lemma}[(\citea{HabermehlMW10})]%
    $\dc{\lang{A}} = \dc{\lang{N,\Minit,\Mfinal}}$.
\end{lemma}

A representation for $\dc{\lang{A}}$ is easy to compute by inserting an $\eps$-labeled variant of each transition in $A$ downward \cref{Section:PNRelwork}.
Hence, this proves \cref{Theorem:PNComputeDC}.

The result is not accompanied by a statement on the complexity of the algorithm and the state complexity of the downward closure.
An upper bound for the latter is the size of the Karp-Miller, which is well-known to be non-primitive recursive in the worst case.
This fact can be shown similar to the proof of our lower bound below.

\paragraph{Lower bound}

Our lower bound will be based on the well-known fact that Petri nets can weakly compute fast-growing functions.
We say that a Petri net weakly computes a function $f$ if for each number $m$ of tokens on a special input place in the initial marking, there is a computation that produces $f(m)$ tokens on a special output place, and no computation produces more than $f(m)$ tokens on the output place.
More formally, consider a net $N$ with a designed input place $p_{\text{in}}$ and a designed output place $p_{\text{out}}$ and an initial marking $\Minit$.
% and a final marking $\Mfinal$.
We define by $\Minit(m)$ the marking that coincides with $\Minit$ but for $M(p_{\text{in}}) = m$.
The net $N$ \emph{weakly computes} function $f \colon \N \to \N$ if for any $m \in \N$ and any marking $M$ reachable from $\Minit(m)$, $M(p_{\text{out}}) \leq m$ holds, and there is at least one reachable marking $M$ with $M(p_{\text{out}}) = m$.

It has been observed by \citea{MayrM81} that a function closely related to the non-primitive recursive Ackermann function can be computed by a Petri net.
We adapt their construction to show that for each number $n \in \N$ there is a Petri net of size linear in $n \in \N$ that weakly computes the function $\Acker{n}{-}$, \ie a version of the Ackermann function in which the first parameter is fixed.
With this net at hand, we will be able to prove the following result.

\begin{proposition}%
\label{Proposition:PNAckermann}%
     For all $n, m \in \N$, there is a Petri net instance $( N(n), \Minit(n,m), \Mfinal(n) )$ of size polynomial in $n + m$ such that
     $\lang{ N(n), \Minit(n,m), \Mfinal(n) } = \set{ a^k \mid k \leq \Acker{n}{m} }$.
\end{proposition}

\begin{figure}[hb!]
    \centering%
    \begin{tikzpicture}[->,semithick,>=latex,node distance=4em]

\node[place] (A1) [minimum size=3em] {$\pin^0$};

\node[transition] (B1) [right of=A1, node distance=9em, inner sep = 0.25em] {$\tcopy^0$}
    edge [pre] (A1);

\node[place] (A4) [ right of=B1, minimum size=3em, node distance=9em] {$\pout^0$}
    edge [pre] (B1);

\node[place] (A2) [ below of=A1, minimum size=3em,  node distance = 4em] {$\pstart^0$};

\node[transition] (B2) [right of=A2, xshift=0.5em, inner sep = 0.25em] {$\tstart^0$}
    edge [pre] (A2);

\node[place] (A3) [below of=B1, minimum size=3em, node distance = 4em] {$\pcopy^0$}
    edge [pre] (B2)
    edge [pre, bend left] (B1)
    edge [post,bend right] (B1);

\node[transition] (B3) [right of=A3, xshift=0.5em, inner sep = 0.25em] {$\tstop^0$}
    edge [pre] (A3)
    edge [post] (A4) ;

\node[place] (A5) [ below of=A4, minimum size=3em, node distance = 4em] {$\pstop^0$}
    edge [pre] (B3);

\end{tikzpicture}
%
    \caption{The Petri net $\AckerNet{0}$.}%
    \label{AckerNetBaseCase}%
\end{figure}

This language is already downward closed and any NFA for it needs at least $\Acker{n}{m}$ states, see \cref{Example:StateComplexityEqualsIndex}.
Hence, it proves that the downward closure of a Petri net coverability language can have non-primitive recursive state complexity.
This also implies that the time needed to construct the corresponding automaton may be non-primitive recursive.
\cref{Proposition:PNAckermann} together with \cref{Theorem:PNComputeDC} prove the main result of this section, \cref{Theorem:PNDCGeneral}.

The rest of this section is dedicated to the proof of \cref{Proposition:PNAckermann}.
We first present an unlabeled family of Petri nets ${(\AckerNet{n})}_{n \in \N}$ such that $\AckerNet{n}$ weakly computes $\Acker{n}{-}$.
Later, we show how to modify it to obtain the family of language required by the proposition.

The definition of $\AckerNet{n}$ is inductive in $n$ and imitates the definition of the Ackermann function.

\begin{figure*}[ht!]
    \centering%
    \begin{tikzpicture}[->,semithick,>=latex,node distance=4em]

\node[place] (A1) [minimum size=4em] {$\pin^{n+1}$};

\node[place] (A2)  [below of=A1, node distance = 5em, minimum size=4em] {$\pstart^{n+1}$};

\node[draw, thick, minimum width=7.5em, minimum height=7.5em, inner sep=0em, outer sep=0em, right of=A1, node distance=12em, yshift=-3em, ]{$\AckerNet{n}$};

\node [place, fill=white, right of=A1, node distance=9em, minimum size=3em] (AA1){$\pin^{n}$};

\node [place, fill=white, node distance=9em, minimum size=3em] (AA2) [right of=A2] {$\pstart^{n}$};

\node [place, fill=white] (AB1) [right of=AA1, node distance = 6.5em, minimum size=3em] {$\pout^{n}$};

\node [place, fill=white] (AB2) [right of=AA2, node distance = 6.5em, minimum size=3em] {$\pstop^{n}$};

\node[transition] (B1) [right of=A2, xshift=1em, inner sep = 0.25em] {$\tstart^{n+1}$}
    edge [pre] (A2)
    edge [post] (AA1)
    edge [post] (AA2);

\node[transition] (B2) [above of=AA1, xshift=3em, inner sep = 0.25em] {$\tswap^{n+1}$}
    edge [pre, bend left=40] (AB1)
    edge [post, bend right=40] (AA1);

\node[place] (A3) [above of=B2, node distance = 4em, minimum size=4em] {$\pswap^{n+1}$}
    edge [pre, bend left] (B2)
    edge [post, bend right] (B2);

\node[transition] (B3) [left of=B2, node distance = 6.5em, inner sep = 0.25em] {$\trestart^{n+1}$}
    edge [pre,bend left] (A3)
    edge [post, bend right] (AA2);

\node[transition] (B4) [above of=A3, node distance = 4em, inner sep = 0.25em] {$\tin^{n+1}$}
    edge [pre, bend left=60] (AB2)
    edge [pre, out=180,in=90] (A1)
    edge [post] (A3);

\node[transition] (B4) [right of=AB1, node distance=9em, inner sep = 0.25em] {$\tcopy^{n+1}$}
    edge [pre] (AB1);

\node[transition] (B5) [right of=AB2, xshift=1em, inner sep = 0.25em] {$\ttmp^{n+1}$}
    edge [pre] (AB2);

\node[place] (A4) [right of=AB2, node distance=9em, minimum size=4em] {$\ptmp^{n+1}$ }
    edge [pre, bend left] (B4)
    edge [pre] (B5)
    edge [post, bend right] (B4);

\node[transition] (B6) [right of=A4, xshift=1em, inner sep = 0.25em] {$\tstop^{n+1}$}
    edge [pre] (A4);

\node[place] (A5) [right of=A4, node distance=9em, minimum size=4em] {$\pstop^{n+1}$}
    edge[pre] (B6);

\node[place] (A6) [right of=B4, node distance=9em, minimum size=4em] {$\pout^{n+1}$}
    edge [pre] (B4);

\end{tikzpicture}
%
    \caption{The Petri net $\AckerNet{n+1}$.}%
    \label{AckerNetStep}%
\end{figure*}

\begin{definition}%
\label{Definition:PNAckermann}%
    We define the Petri net $\AckerNet{0}$ to be
    \begin{align*}
        \AckerNet{0} &= (P^0,T^0,\i^0,\o^0) \quad \text{ with }\\
        P^0 &= \set{ \pin^0, \pout^0, \pstart^0, \pstop^0, \pcopy^0 }\text{, and} \\
        T^0 &= \set{ \tstart^0, \tstop^0, \tcopy^0 }.
    \end{align*}
    %
    The transition multiplicities are given by Figure~\ref{AckerNetBaseCase}, where each edge has a weight of $1$.

    For $n \in \N$, we define $\AckerNet{n+1}$ inductively by
    \begin{align*}
        \AckerNet{n+1}
        &= (P^{n+1},T^{n+1},\i^{n+1},\o^{n+1})
        \quad \text{ with }
        \\
        P^{n+1}
        &= P^n \cup \set{ \pin^{n+1}, \pstart^{n+1}, \pcopy^{n+1}, \pout^{n+1}, \pstop^{n+1},\pswap^{n+1}, \ptmp^{n+1}}\text{, and}
        \\
        T^{n+1}
        &= T^n \cup\set{ \tstart^{n+1}, \tcopy^{n+1}, \tstop^{n+1},\trestart^{n+1},\tin^{n+1}, \tswap^{n+1}, \ttmp^{n+1} }.
    \end{align*}
    %
    The transition multiplicities are given by Figure~\ref{AckerNetStep}, where again each edge has a weight of $1$.

    Let us furthermore define for each $m \in \N$ the marking $\Minit(n,m)$ of $\AckerNet{n}$ that places one token on $\pstart^{n}$, $m$ tokens on $\pin^{n}$ and no token elsewhere.
\end{definition}

We show that the family of nets ${(\AckerNet{n})}_{n \in \N}$ has the desired property.

\begin{lemma}%
\label{Lemma:PNWeaklyComputeAckermann}%
    For all $n \in \N$, $\AckerNet{n}$ weakly computes $\Acker{n}{-}$:
    For each $m \in \N$, there is a computation $\Minit(n,m) \fire{\sigma} M$ of $\AckerNet{n}$ such that $M(\pout^n) = \Acker{n}{m}$, $M(\pstop^n) = 1$, and there is no computation $\Minit(n,m) \fire{\sigma} M$ with $M(\pout^n) > \Acker{n}{m}$.
\end{lemma}

Before starting the proof, recall the recursive definition of the Ackermann function that we have already given in \cref{Section:Complexity}:
$\Acker{0}{m} = m + 1 $,
$\Acker{n+1}{0} = \Acker{n}{1}$, and
$\Acker{n+1}{m+1} = \Acker{n}{\Acker{n+1}{m}}$.
We will need the following two properties of the Ackermann function, which are easy to show using induction.
For all $n, m, m' \in \N$, we~have
\begin{align}
    \label{Equation:AckerOne}\tag{A1}%
    \Acker{n+1}{m}
    &=
    \underbrace{ \AckerT(n, \AckerT(n, \ldots \AckerT(n, }_{(m+1) \text{ times}} 1) \ldots ))
\\
    \label{Equation:AckerTwo}\tag{A2}%
    \Acker{n}{m} + m' &\leq \Acker{n}{m + m'}
\\[1em]
    \label{Equation:AckerThree}\tag{A3}%
    \Acker{n}{m} &\leq \Acker{n}{\Acker{n}{m}}
    \ .
\end{align}

\begin{proof}[Proof of \cref{Lemma:PNWeaklyComputeAckermann}]
    We proceed by an induction on $n$ that proves both statements simultaneously.

    \subproof{Base case, $n = 0$}

    All firing sequences that are enabled from $\Minit(n,m)$ and produce a token on $\pstop^0$ are of the shape $\tstart^0.{(\tcopy^0)}^k.\tstop^0$, where $k \in \zeroto{m}$.
    Such a transition sequence creates between $1$ and $m+1 = \Acker{0}{m}$ tokens on $\pout^0$.
    It is impossible to create more than $m+1$ tokens on $\pout^0$.

    \subproof{Inductive step, $n \to n+1$}

    We first demonstrate how to create a token on $\pstop^n$ and $\Acker{n}{1}$ tokens on $\pout^n$.
    The only transition that is enabled in $\Minit(n,m)$ is $\tstart^{n+1}$.
    Firing it creates one token on $\pin^n$ and one token on $\pstart^n$.
    We can now execute the computation of $\AckerNet{n}$ that creates $\Acker{n}{1}$ tokens on $\pout^n$ and one token on $\pstop^n$, which exists by induction.

    Next, we show how to create $\Acker{n}{\Acker{n}{1}}$ tokens on $\pout^n$.
    Using $\tin^{n+1}$, we consume one token from $\pin^{n+1}$ and the token on $\pstop^n$ to create a token on $\pswap^{n+1}$.
    This token allows us to swap all $\Acker{n}{1}$ tokens from $\pout^{n}$ to $\pin^{n}$ using $\tswap^{n+1}$.
    After doing this, we move the token from $\pswap^{n+1}$ to $\tstart^n$ using the restart transition $\trestart^{n+1}$.
    We are now able to execute the computation of $\AckerNet{n}$ that creates $\Acker{n}{\Acker{n}{1}}$ tokens on $\pout^n$ and one token on $\pstop^n$, which exists by induction.

    The process described in the previous paragraph can be repeated for each of the $m$~tokens on place $\pin^{n+1}$.
    After doing so, we end up with one token on $\pstop^n$ and
    $\underbrace{ \AckerT(n, \AckerT(n, \ldots \AckerT(n, }_{(m+1) \text{ times}} 1) \ldots )) = \Acker{n+1}{m}$ tokens on $\pout^n$ using \cref{Equation:AckerOne}.

    It remains to use $\ttmp^{n+1}$ once to get a token on $\ptmp^{n+1}$.
    Then we can transfer all $\Acker{n+1}{m}$ tokens from $\pout^n$ to $\pout^{n+1}$ with $\tcopy^{n+1}$.
    Finally, we use $\tstop^{n+1}$ to create a token on $\tstop^{n+1}$.

    To prove the second part of the statement, we argue that it is not possible to create more than $\Acker{n+1}{m}$ tokens on $\pout^n$.
    Obviously, the number of tokens on $\pout^n$ during the computation also limits the number of tokens on $\pout^{n+1}$ at the end.
    Having $\ell$ tokens on $\pin^n$, we cannot create more than $\Acker{n}{\ell}$ tokens on $\pout^n$ by induction.
    This in particular means that initially, with only one token on $\pin^n$, we cannot create more than $\Acker{n}{1}$ tokens on $\pout^{n}$.

    Afterwards, if we do not execute $\tswap^{n+1}$ as often as possible, say we leave $\ell'$ out of $\ell$ tokens on $\pout^n$, we end up with $\Acker{n}{\ell - \ell'} + \ell' \leq \Acker{n}{\ell}$ tokens, using \cref{Equation:AckerTwo}.
    This means that in each iteration, we maximize the number of tokens on $\pout^n$ by using $\tswap^{n+1}$ as often as possible, as in the previously described computation.

    Finally, we use that $\Acker{n}{\ell} \leq \Acker{n}{\Acker{n}{\ell}}$, \cref{Equation:AckerThree}.
    We obtain the maximum number of tokens on $\pout^n$ by conducting the maximum number of iterations consuming all tokens on $\pin^{n+1}$.

    Altogether, the computation of $\AckerNet{n+1}$ as previously described maximizes the number of tokens on $\pout^{n+1}$; no computation can create more than $\Acker{n+1}{m}$ tokens.
\end{proof}

It remains to modify the unlabeled Petri net $\AckerNet{n}$ to obtain the Petri net instance $\big(N(n), \Minit(n,m), \Mfinal(n)\big)$ with $\lang{ N(n), \Minit(n,m), \Mfinal(n) } = \set{ a^k \mid k \leq \Acker{n}{m} }$.

\begin{proof}[Proof of \cref{Proposition:PNAckermann}]
    Let $n,m \in \N$.
    Consider $\AckerNet{n}$, the net from \cref{Definition:PNAckermann}, as a labeled Petri net with all transitions labeled by $\eps$.
    We define $N(n)$ to be the Petri net that is obtained from it by adding an $a$-labeled transition $t_a$ that consumes one token from $\pout^n$.
    The initial marking is the marking $\Minit(n,m)$ of $\AckerNet{n}$, the final marking is the zero vector.

    By \cref{Lemma:PNWeaklyComputeAckermann}, there is a computation creating $\Acker{n}{m}$ tokens on $\pout^n$, and no computation creates more than $\Acker{n}{m}$ tokens.
    Hence, a computation of $N(n)$ can fire $t_a$ up to $\Acker{n}{m}$ times, producing the same number of $a$s in the process.
    This proves $\lang{ N(n), \Minit(n,m), \Mfinal(n) } = \set{ a^k \mid k \leq \Acker{n}{m} }$.

    Note that the size of $N(n)$ is polynomial in $\AckerNet{n}$, which has a number of places and transitions linear in $n$.
    The binary encoding of the initial marking is linear in $n + \ceil{\log m}$.
\end{proof}

With the proof of \cref{Proposition:PNAckermann} completed, we have provided the lower bound matching the construction from \cref{Theorem:PNComputeDC}, showing \cref{Theorem:PNDCGeneral}.

\paragraph{The unary case}

\cref{Theorem:PNDCGeneral} also holds if we define the size of Petri nets based on their unary encoding.
The upper bound obviously continues to hold.
For the lower bound, observe that $\AckerNet{n}$ only uses transition multiplicities from the set $\set{0,1}$, and that the unary encoding of the initial marking is linear in $n + m$.

\end{document}
