\documentclass[../../diss.tex]{subfiles}
\begin{document}

\section{SRE inclusion in the downward closures for BPP nets}

It should come as no surprise that the final section of this chapter combines the two restricted versions of the downward closure computation into a single problem.
We study the complexity of checking whether the language of a given SRE is contained in the downward closure of the language of a given BPP net.

\begin{problem}
    \problemtitle{SRE inclusion in the downward closure for BPP nets}
    \problemshort{($\SREBPPDC$)}
    \probleminput{A labeled BPP net instance $(N,\Minit,\Mfinal)$, an SRE $\sre$.}
    \problemquestion{$\lang{\sre} \subseteq \dc{\lang{N,\Minit,\Mfinal}}$?}
\end{problem}

We prove that the complexity of the problem is the same as the problem $\SREBPPUC$ in the case of the downward closure:
$\SREBPPDC$ is complete for the class of problems solvable in polynomial time by a nondeterministic algorithm.

\begin{theorem}%
\label{Theorem:SREBPPDC}%
    $\SREBPPDC$ is $\NPTIME$-complete.
\end{theorem}

While the statement is similar, the proof is vastly more complex than the proof of \cref{Theorem:BPPSREUC}.
The same reasoning as in \cref{Section:PNSREDC} applies:
A downward-closed language cannot be represented by finitely many maximal words.
Hence, to solve the problem, it is not sufficient to conduct a finite number of membership queries.
Before we go into detail about the proof approach, let us briefly comment on the lower bound.

\paragraph{Lower bound}

The $\NPTIME$-hardness of $\SREBPPDC$ is a direct consequence of the $\NPTIME$-hardness of coverability in BPP nets, \cref{Lemma:BPPCOVhardness}.
The formal reasoning is similar to the one for the $\EXPSPACE$-hardness of $\SREUC$ (see \cref{Lemma:PNSREUChardness}) and $\SREDC$, and the $\NPTIME$-hardness of $\SREBPPUC$.

\paragraph{Upper bound}

With the lower bound out of the way, it remains to show that $\SREBPPDC$ can be solved by a nondeterministic algorithm in polynomial time.

\begin{theorem}%
\label{Theorem:SREBPPDCMembership}%
    $\SREBPPDC$ is in $\NPTIME$.
\end{theorem}

The proof of this theorem is the most complex proof in this part of the thesis.
It combines various insights that we have gathered in the previous chapters.

Let us start by considering the proof of the $\EXPSPACE$-membership of $\SREDC$, \cref{Theorem:PNSREDCMembership}, and understand which parts of it can be reused.
The proof proceeds in four steps:
(1)~Observe that instead of checking $\lang{\sre} \subseteq \dc{\lang{N,\Minit,\Mfinal}}$, we can check $\lang{\product} \subseteq \dc{\lang{N,\Minit,\Mfinal}}$ for each of the polynomially many products of the SRE.\@
(2)~Instead of checking this inclusion, we can check $\lang{\linof{\product}} \subseteq \dc{\lang{N,\Minit,\Mfinal}}$ where $\linof{\product}$ is the linearization of the product from \cref{Definition:Linearization}.
(3)~Construct the product of $\dc{N}$, the downward closure of the given net, and $N_\product$, a net with language $\lang{\linof{\product}}$.
(4)~Construct an instance of the simultaneous unboundedness problem $\PNSU$ based on this product.

The first two steps carry over.
The fact that the inclusion can be checked for each product as well as \cref{Lemma:Linearization} remain unchanged if we consider an input net that is a BPP net.
Unfortunately, there is a problem with the third step.
While $\dc{N}$ and $N_\product$ both are BPP nets, their product will not be a BPP net.
Even if it would be a BPP net, there is also a problem with the reduction to $\PNSU$. The problem $\PNSU$ is $\EXPSPACE$-complete, so we would not get the desired complexity.

The question if $\PNSU$ becomes $\NPTIME$-complete if we restrict the input net to be a BPP net arises naturally.
One could approach this problem by first using \cref{Theorem:BPPPResburger} which yields that the reachability set of a BPP net is effectively semi-linear, and then trying to encode simultaneous unboundedness as a Presburger formula.
While the latter seems to be easy, one will obtain a formula that contains a universal quantification, so we cannot use \cref{Theorem:SatisfiabilityEPA}, the $\NPTIME$-completeness of existential Presburger arithmetic.
We conjecture that the proof that we will present in the rest of this section could be adapted to show that $\PNSU$ is $\NPTIME$-complete for BPP nets.
However, since this result would not help us settle the main question, we refrain from investigating this further.

We come back to the problem under consideration.
The following proposition is the result that we need.
Together with the first to steps from the proof of \cref{Theorem:PNSREDCMembership} as outlined above, it proves \cref{Theorem:SREBPPDCMembership}.

\begin{proposition}%
\label{Proposition:BPPLinearizationInclusion}%
    Given a product $\product$ and a BPP net $N$, one can decide whether the inclusion
    $\lang{\linof{\product}} \subseteq \dc{\lang{N,\Minit,\Mfinal}}$
    holds in $\NPTIME$.
\end{proposition}

The proof of the proposition combines various techniques:
In addition to the linearization from \cref{Section:PNSREDC}, it also uses the tricks that we have used to show that the word problem for BPP nets is in $\NPTIME$, \cref{Proposition:BPPWordProblem}, and the pumping lemma for BPP nets, \cref{Lemma:BPPPumping}.

%
\cheatpagebreak
%

Let $(N,\Minit,\Mfinal)$ be the given BPP net, and let $\product$ be the given product.
We assume that the linearization of $\product$ is
\[
    \dtr{w}{0} \paren{w_{\Gamma_1}}^* \dtr{w}{1} \ldots a_{m-1} \paren{w_{\Gamma_m}}^* \dtr{w}{m}
\]
with words $\dtr{w}{i} \in \Sigma^*$ and subalphabets $\Gamma_i \subseteq \Sigma$ for all $i$.
The idea is the following:
We check that there is a computation for words from the linearization in which the parts corresponding to $\paren{w_{\Gamma_i}}^*$ can be repeated.
This means that we require that after seeing $\paren{w_{\Gamma_i}}$ once, we are in a larger marking than before.

We want to encode this property in Presburger arithmetic.
To avoid having to express unbounded markings, we use the exponential constant
\[
    k = \paren{\normone{\Minit}+2} \cdot \paren{ \paren{\card{P}+1} \cdot \norminf{N}}^{ \paren{\card{T} + 2} }
\]
from \cref{Lemma:BPPPumping}.
For two markings $M, M'$, we write $M \leqk M'$ if for all places $p\in P$ we have that $M'(p) < k$ implies $M(p) \leq M'(p)$.

Using this notation, we can formalize the above explanation.

\begin{definition}%
\label{Definition:BPPWitness}%
    A \emph{witness} for $\linof{\product}$ is a covering computation
    \[
        \Minit
        =
        M_0 \fire{\sigma_0} M_0'
        \fire{\tau_1}
        M_1 \fire{\sigma_1} M_1'
        \fire{\tau_2}
        \ldots
        M_{m-1}'
        \fire{\tau_{m}}
        M_m \fire{\sigma_m} M_m'
        \geq \Mfinal
        \ ,
    \]
    of $\dc{N}$ where $\sigma_i, \tau_i \in T^*$ are firing sequences so that
    (1)~$\lambda(\sigma_i) = \dtr{w}{i}$ for all $i \in  \zeroto{m}$,
    (2)~$\lambda(\tau_i) = w_{\Gamma_i}$ for all $i \in \oneto{m}$, and
    (3)~$M_i'\leqk M_{i+1}$ for all $i \in \zeroto{m-1}$.
\end{definition}

Note that unlike in the rest of this thesis, $\sigma_i$ and $\tau_i$ denote firing sequences instead of single transitions here.
The first two properties express that the word generated by the computation is $\dtr{w}{0} w_{\Gamma_1} \ldots w_{\Gamma_m} \dtr{w}{m}$.
The last property means that each infix $\tau_i$ that correspond to the occurrence of some $w_{\Gamma_i}$ can be repeated.
Because we only require $M_i'\leqk M_{i+1}$ (instead of $M_i'\leq M_{i+1}$) we might need to insert pumps, as in the proof of \cref{Theorem:BPPUC}.

With the definition of a witness at hand, the proof of \cref{Proposition:BPPLinearizationInclusion} decomposes into two steps.
We first show that checking the inclusion between the linearization of a product and the language of $\dc{N}$ amounts to checking the existence of a witness.
Then, we show how the latter can be checked in $\NPTIME$ by encoding it into an existential Presburger formula.

\begin{lemma}%
\label{Lemma:BPPSREDCWitness}%
    The inclusion $\lang{\linof{\product}} \subseteq \lang{\dc{N}, \Minit,\Mfinal}$ holds if and only if $\linof{\product}$ has a witness.
\end{lemma}

\begin{proof}
    Let us assume that the computation
    \[
        \Minit
        =
        M_0 \fire{\sigma_0} M_0'
        \fire{\tau_1}
        M_1 \fire{\sigma_1} M_1'
        \fire{\tau_2}
        \ldots
        M_{m-1}'
        \fire{\tau_{m}}
        M_m \fire{\sigma_m} M_m'
        \geqk \Mfinal
        \ ,
    \]
    of $\mathmbox{\dc{N}}$ is a witness for $\linof{\product}$.
    Consider the word
    $w = \dtr{w}{0} \paren{w_{\Gamma_1}}^{j_1} \ldots \paren{w_{\Gamma_m}}^{j_m} \dtr{w}{m} \in \lang{\linof{\product}}$.
    Our goal is to show that $w \in \lang{\dc{N},\Minit,\Mfinal}$.
    The firing sequence $\sigma_0 . {\tau_1}^{j_1} . \sigma_1 \ldots {\tau_m}^{j_m} . \sigma_m$ has the correct label and intuitively should induce a covering computation of $\dc{N}$.
    However, we have not required $M_i'\leq M_{i+1}$ in the definition of a witness, but only $M_i'\leqk M_{i+1}$.

    Instead of using \cref{Lemma:BPPPumping} to insert pumps as in the proof of \cref{Theorem:BPPUC}, we simply use \cref{Theorem:BPPUC}:
    We have shown that the downward closures of the language of $N$ and of $\langoak{k}{N,\Minit,\Mfinal}$, the $\zeroto{k}$-$\omega$-overapproximation coincide.
    We first project the transition sequence back to the net $N$, meaning that we use the non-$\eps$-labeled variant of each transition wherever needed.
    This sequence then induces an accepting run of the automaton $A_{>k}$ with $\lang{A_{>k}} = \langoak{k}{N,\Minit,\Mfinal}$ on a superword of $w$.
    Indeed, the automaton is defined to ignore the precise value of markings on places where bound $k$ has been exceeded.
    Hence, requiring $M_i'\leqk M_{i+1}$ is sufficient to show that the transitions can be repeated.
    Altogether, we have $w \in \dc{\langoak{k}{N,\Minit,\Mfinal}} = \lang{\dc{N},\Minit,\Mfinal}$.

    For the other direction, define $\ell = {(k+2)}^{\card{P}}$,
    and consider the word
    \[
        \dtr{w}{0} \paren{w_{\Gamma_1}}^{\ell} \dtr{w}{1} \ldots \dtr{w}{m-1} \paren{w_{\Gamma_m}}^{\ell} \dtr{w}{m}
        \in \lang{\linof{\product}}
        \ .
    \]
    Since we assume that the inclusion holds, $\dc{N}$ has a computation
    \[
        \Minit
        =
        M_0 \fire{\sigma_0} M_0'
        \fire{\tau_1}
        M_1 \fire{\sigma_1} M_1'
        \fire{\tau_2}
        \ldots
        M_{m-1}'
        \fire{\tau_{m}}
        M_m \fire{\sigma_m} M_m' \geq \Mfinal
        \ ,
    \]
    where $\lambda(\sigma_i) = \dtr{w}{i}$ for all $i$ and $\lambda(\tau_i) = \paren{w_{\Gamma_i}}^{\ell}$ for all $i$.

    This computation already covers the final marking and satisfies the first property from \cref{Definition:BPPWitness}.
    For the other two properties, we essentially need to identify for each $\tau_i$ an infix $\tau_i'$ such that the marking before firing $\tau_i''$ is $\leqk$-smaller than the marking after firing $\tau_i'$.
    The number $\ell$ has been chosen so that this is possible.

    Let us focus on one infix $M_{i-1}' \fire{\tau_{i}} M_i$.
    Since $\lambda(\tau_i) = \paren{w_{\Gamma_i}}^{\ell}$, we may split it into
    \[
        M_{i-1}'
        = \itr{M}{0}
        \displayfire{ \itr{\tau}{1} }
        \itr{M}{1}
        \displayfire{ \itr{\tau}{2} }
        \ldots \
        \displayfire{ \itr{\tau}{\ell} }
        \itr{M}{\ell}
        = M_i
    \]
    where $\lambda( \itr{\tau}{j} ) = w_{\Gamma_i}$ for each $j$ and $\tau_i = \itr{\tau}{1} \ldots \itr{\tau}{\ell}$.
    Now observe that the set $\paren{\zeroto{k} \cup \set{\omega}}^{P}$ has cardinality ${(k+2)}^{\card{P}}=\ell$.
    This means that if we identify all numbers strictly greater than $k$ with each other, there are only $\ell$ different markings.
    The sequence $\itr{M}{0}, \itr{M}{1}, \ldots, \itr{M}{\ell}$ of markings from the above computation contains $\ell+1$ markings.
    Hence, there need to be indices $j < j'$ such that $M^{(j)}$ and $M^{(j')}$ assign more than $k$ tokens to the same places, and coincide on the rest of the places.
    This in particular means that $M^{(j)} \leqk M^{(j')}$.

    We may rewrite $M_{i-1}' \fire{\tau_{i}} M_i$ as
    \[
        M_{i-1}'
        \displayfire{ \itr{\tau}{1} \cdots \itr{\tau}{j} }
        M^{(j)}
        \displayfire{ \itr{\tau}{j+1} \cdots \itr{\tau}{j'} }
        M^{(j')}
        \displayfire{ \itr{\tau}{j'+1} \cdots \itr{\tau}{\ell} }
        M_i
        \ .
    \]
    To obtain a witness, it remains to get rid of superfluous letters.
    Recall that in $\dc{N}$, each transition has an $\eps$-labeled variant.
    We thus may define $\tau_i'$ to be obtained from $\itr{\tau}{1} \ldots \itr{\tau}{j}$ by replacing all non-$\eps$-labeled transitions by their $\eps$-labeled variant, $\tau_i'''$ by applying the same to $\itr{\tau}{j'+1} \ldots \itr{\tau}{\ell}$, and finally $\tau_i''$ to be obtained from $\tau_{(j+1)} \ldots \tau_{(j')}$ by keeping the non-$\eps$-labeled transitions in $\tau_{(j')}$, but applying the replacement to all other transitions.
    We obtain \nb{$\lambda(\tau_i') = \lambda(\tau_i''') = \eps$, $\lambda(\tau_i'') = w_{\Gamma_i}$.}

    To finish the construction, we replace $\tau_i$ by $\tau_i'.\tau_i''.\tau_i'''$, then merge $\tau_i'$ with the preceding sequence $\sigma_{i-1}$ and $\tau_i'''$ with $\sigma_{i}$.
    This means we replace
    \[
        M_{i-1} \fire{\sigma_{i-1}} M_{i-1}' \fire{\tau_i} M_i \fire{\sigma_i} M_i'
    \]
    by
    \[
        M_{i-1} \fire{\sigma_{i-1}.\tau_i\lowerprime} \itr{M}{j} \fire{\tau_i\lowerprime\lowerprime} \itr{M}{j'} \fire{\tau_i\lowerprime\lowerprime\lowerprime.\sigma_i} M_i'
        \ ,
    \]
    where we have $\lambda(\sigma_{i-1}.\tau_i') = a_{i-1}$,
    $\lambda(\tau_i'') = w_{\Gamma_i}$,
    $\lambda(\tau_i'''.\sigma_{i}) = a_{i}$,
    and $\itr{M}{j} \leqk \itr{M}{j'}$.

    Applying this replacement to each $\tau_i$ yields a witness.
\end{proof}

With the lemma proven, it remains to check the existence of witnesses in nondeterministic polynomial time.
To solve this problem, we construct a formula in existential Presburger arithmetic $\varphi$ that is satisfiable if and only if a witness exists.
We proceed similar to the proof of $\NPTIME$ membership of the word problem for BPP nets, \cref{Proposition:BPPWordProblem}.
The formula is a conjunction of the formula characterizing reachability in a modified net $N'$ and several formulas that express the conditions that we impose on a witness.

Consider the word
\[
    w = \dtr{w}{0} w_{\Gamma_1} \dtr{w}{1} \ldots \dtr{w}{m-1} w_{\Gamma_1} \dtr{w}{m} \in \lang{\linof{\product}}
\]
in which we take one iteration of each $w_{\Gamma_i}$.
The proof of \cref{Proposition:BPPWordproblem} yields a formula
$\varphi = \varphi_{({N \downarrow},\Minit,\Mfinal),w}$ in existential Presburger arithmetic
of size polynomial in the input size that is satisfiable if and only if $w \in \lang{\dc{N},\Minit,\Mfinal}$.

We recall the construction on a high level of abstraction:
For each of the letters $w_i$ of $w$, the formula guesses the transition $t_i$ of $\dc{N}$ with $\lambda(t_i) = w_i$ that will generate it.
To take care of the sequences of $\eps$-labeled transitions between the letters, we use the characterization of reachability in BPP nets by an existential Presburger formula.
Intuitively, for each letter $w_i$, the formula existentially quantifies over the marking that will be reached before \resp after transition $t_i$ has been fired.
To  implement this, we consider the net $N_\eps$ obtained from $\dc{N}$ by removing all non-$\eps$-labeled transitions.
For each letter $w_i$ of $w$, we create two copies of $\dc{N}$.
In one copy, the sequence of $\eps$-labeled transitions before $t_i$ is executed.
The initial marking for this copy should be the marking reached after firing the transition corresponding to the letter before $w_i$.
Since we do not have this marking at hand, we add transitions that allow a computation to populate the copy with an arbitrary initial marking.
These transitions generate the same markings in both copies.
The other copy does not have any transitions; it is simply used to keep the initial marking available for the formula.

The formula $\varphi$ that we construct then checks that there is a marking that is reachable in the modified net (which in particular implies that in each copy, the transition relation of net $\dc{N}$ has been respected), the marking reached in the copy for letter $w_i$ enables transition $t_i$, the initial marking of the copy for the next letter after $w_i$ is populated with the marking that results from firing transition $t_i$,
and the final marking reached in the last copy covers $\Mfinal$.

To be able to use the construction from the proof of \cref{Proposition:BPPWordproblem}, we need to incorporate a small modification.
The formula $\varphi$ makes available to us the markings that occur before and after each of the letters of the word $w$ in the form of its free variables.
However, the markings $M_i$ and $M_i'$ that we need to consider when checking for the existence of a witness may occur not directly before or after a proper letter, but in the midst of a sequence of $\eps$-labeled transitions.
To overcome this problem, we change the construction so that for the last letter of each infix $w_{\Sigma_i}$, the formula actually makes available to us two markings:
The marking that occurs after the transition corresponding to the letter, and another marking that is seen after a (potentially empty) sequence of $\eps$-transitions.
Similarly, for the first letter of each infix $w_{\Sigma_i}$, we assume that in addition to the marking directly before the transition corresponding to the letter, we also have a marking from which the first can be reached by $\eps$-transitions available.
Formally implementing this is an extension of the proof of \cref{Proposition:BPPWordproblem} that is straightforward.
We simply insert additional copies of the nets at the appropriate locations and extend the formulas to ensure that these copies are populated correctly.

With the modification in place, we can assume that the modified formula $\varphi$ makes available to us the following markings in the form of free variables:
$M_i$, the freshly introduced marking in the corresponding copy of $N_\eps$ before (a sequence of $\eps$-transitions before) the first letter of $w_{\Gamma_i}$,
and $M_{i+1}$, the marking reached in the copy for the last letter of $w_{\Gamma_i}$ (and a sequence of $\eps$-transitions).

Finally, we construct a new formula $\varphi \wedge \psi$ that checks for the existence of a witness, where $\psi$ checks Condition~(3), $M_i'\leqk M_{i+1}$:
The marking before seeing $w_{\Gamma_i}$ should be $\leqk$-smaller than the marking after seeing $w_{\Gamma_i}$.
Formally, we have
\[
    \psi = \bigwedge_{i \in \oneto{m}}
    \bigwedge_{p \in \P}
    (M_i'(p) < k)
    \implies
    M_i(p) \leq M_{i+1}(p)
    \ .
\]

We claim that this formula can be constructed in polynomial time.
The only critical part is the encoding of $k$.
Number $k = \paren{\normone{\Minit}+2} \cdot \paren{ \paren{\card{P}+1} \cdot \norminf{N}}^{ \paren{\card{T} + 2} }$ is exponential, but its binary encoding is of size polynomial in the size of the input.

Altogether, the construction proves the following lemma.

\begin{lemma}%
\label{Lemma:BPPSREDCWitnessFormula}%
    We can construct in polynomial time a formula in existential Presburger arithmetic that is satisfiability if and only if there is a witness for $\linof{\product}$.
\end{lemma}

With that lemma at hand, we simply need to collect all our results to get the proof of \cref{Proposition:BPPLinearizationInclusion}, which implies \cref{Theorem:SREBPPDCMembership}.

\begin{proof}[Proof of \cref{Proposition:BPPLinearizationInclusion}]
    Consider a product $\product$ and a BPP net $(N,\Minit,\Mfinal)$ for which we want to check the inclusion $\lang{\linof{\product}} \subseteq \dc{\lang{N,\Minit,\Mfinal}}$.
    By \cref{Lemma:BPPSREDCWitness}, it holds if and only if there is a witness for $\linof{\product}$.
    With \cref{Lemma:BPPSREDCWitnessFormula}, we can construct a formula in existential Presburger arithmetic of polynomial size that is satisfiable if and only if a witness exists.
    By the $\NPTIME$-completeness of $\EPASAT$, \cref{Theorem:SatisfiabilityEPA}, this can be checked in nondeterministic polynomial time.
\end{proof}

This completes this section and the proof of \cref{Theorem:SREBPPDC}.
Note that the problem remains $\NPTIME$-complete in the unary case since the lower bound continues to hold, as in \cref{Section:BPPSREUC}.

Altogether, this finishes our study of the downward closure of Petri net coverability languages.
We have proven all results regarding upward closures that we mentioned in \cref{Section:PNResults}.

\end{document}
