\documentclass[../../diss.tex]{subfiles}
\begin{document}

\section{Basic definitions}%
\label{Closures:Defs}%

We start by making formal the notions of superwords and subwords and the sets of such.
To be precise, we show that they can be incorporated into the framework of WQOs, see \cref{Section:WSTS}.

Let $(X,\leq)$ be a quasi-order.
The \emph{subsequence ordering} $(X^*,\leq^*)$ on the set of finite-length sequences over $X$ is defined as follows.
A sequence $v$ is smaller than sequence $w$, $v \leq^* w$, if it can be obtained from $v$ by iteratively deleting elements and replacing elements by smaller elements with respect to the underlying order $(X,\leq)$.
If $v = a_1 a_2 \ldots a_n$ and $v \leq^* w$, this means that we can write
$w = \itr{c}{0} b_1 \itr{c}{1} b_2 \ldots \itr{c}{n-1} b_n \itr{c}{n}$ for sequences $\itr{c}{i} \in X^*$ and elements $b_i \in X$ with $a_i \leq b_i$ for all $i$.
Note that here, we use $X^*$ to denote the finite-length sequences over $X$ following the notation for words, even if $X$ is not finite.

Higman's lemma~\cite{Higman52} guarantees that the subsequence ordering induced by a WQO is a WQO again.

\begin{lemma}[(\citea{Higman52})]%
\label{Lemma:Higman}%
    If $(X,\leq)$ is a well-quasi-order, then so is $(X^*,\leq^*)$.
\end{lemma}

Throughout this thesis, we will be exclusively interested in the case where the underlying order is $(\Sigma,=)$, a finite alphabet ordered by equality.
Indeed, such a set is a WQO, see Part~a) of \cref{Example:WQO}, so Higman's Lemma guarantees that $(\Sigma^*,=^*)$ is a WQO.\@
In this special case, we call the order the \emph{subword order} and denote it by $(\Sigma^*,\subword)$.
Since the underlying order is equality, the definition of the subsequence order is restricted: It is impossible to replace an element by a distinct smaller element.
Hence, the definition is as expected: A word $w$ is a subword of another word $v$, $w \subword v$, if it can be obtained by deleting letters.
In that case, we also say that $v$ is a \emph{superword}, which means that it can be obtained from $w$ by inserting letters.
Note that when we say \emph{subword}, we are talking about \emph{scattered subwords}, \ie we do not require subwords to occur as infixes.
It is also noteworthy that the subword ordering inherits the property of being partial order from $(\Sigma,=)$, meaning that unlike WQOs in general, it is antisymmetric.

In \cref{Section:WSTS}, we have introduced the notion of the upward and the downward closure of a set.
Here, we apply these operations to a language with respect to the subword ordering.
Given a language $\calL$, its downward closure
\(
    \dc{\calL} = \Set{ w \in \Sigma^*}{ \exists v \in \calL \colon w \subword v}
\)
is the set of all words that are a subword of a word in the language.
Similarly, the upward closure
\(
    \uc{\calL} = \Set{ w \in \Sigma^*}{ \exists v \in \calL \colon v \subword w}
\)
is the set of all superwords of a word in the language, or equivalently, all words that have a subword in the language.
Unless stated otherwise, all closure operators will refer to the closures with respect to the subword ordering in this part of the thesis.

\paragraph{Regularity and simple regular expressions}

With Higman's Lemma, we also get that the various properties of WQOs stated in \cref{Lemma:WQOProperties} apply to the subword ordering.
A surprising consequence of this is the classical result states that both the upward closures and the downward closures of arbitrary languages are guaranteed to be regular.
We give the proof to show how the properties of well-quasi-orders come into play.

\begin{theorem}[(\citea{Haines69})]%
\label{Theorem:Haines}%
    For any language $\calL \subseteq \Sigma^*$, $\dc{\calL}$ and $\uc{\calL}$ are regular.
\end{theorem}

\begin{proof}
    By Higman's Lemma $(\Sigma^*,\subword)$ is a WQO, so every language over $\Sigma$ has a finite set of minimal elements by Item~(\ref{Property:WQOFiniteBasis}) of \cref{Lemma:WQOProperties}.
    Applying this property to $\uc{\calL}$ yields a finite set of words $\itr{w}{1}, \ldots, \itr{w}{k}$ such that $\uc{\calL} = \uc{\set{ \itr{w}{1}, \ldots, \itr{w}{k}}} = \uc{\itr{w}{1}} \cup \ldots \cup \uc{\itr{w}{k}}$.
%
    To show that $\uc{\calL}$ is regular, we observe that the upward closure of a single word $\itr{w}{i} = a_1 \ldots a_n$ can be represented by the regular expression $\Sigma^* a_1 \Sigma^* \ldots \Sigma^* a_n \Sigma^*$, and that the class of regular languages is closed under finite unions.

    To see that the downward closures are regular as well, we use that the complement of a downward-closed language is upward-closed, hence regular, and complements of regular languages are again regular.
\end{proof}

A more recent result~\cite{AbdullaCBJ04} shows that a certain type of restricted regular expressions is sufficient to describe downward and upward closures.
A \emph{simple regular expression (SRE)} is a choice among products,
\[
     \sre \bnfdef \product \bnf \sre \cup \sre
     \ ,
\]
where a product is a concatenation of letters, optional letters, and iterations over subsets $\Gamma\subseteq \Sigma$ of the alphabet,
\[
    \product \bnfdef a \bnf (a\cup\varepsilon)\bnf\Gamma^*\bnf\product.\product
    \ .
\]
SREs are sufficient to describe all upward and downward-closed languages.

\begin{theorem}[(\citea{AbdullaCBJ04})]
    For any language $\calL \subseteq \Sigma^*$, there are simple regular expressions describing its downward and upward closures.
\end{theorem}

For upward closures, the proof follows directly from the proof of \cref{Theorem:Haines}.
For downward closures, one needs to carefully look at the shape of the language obtained by complementing an upward-closed language.

Actually, the original definition of SREs in~\cite{AbdullaCBJ04} is simpler than ours.
In~\cite{AbdullaCBJ04}, a product is defined to be a concatenation $p.p$ of products, an optional letter $(a\cup\varepsilon)$, or the Kleene iteration of a subalphabet $\Gamma^*$, omitting the case of a single non-optional letter.
With this version, one obtains a tight correspondence with the downward-closed languages:
The language of an SRE is downward closed, and any downward-closed language can be described by an SRE.\@
We have modified the definition to ensure that also upward-closed languages are covered, losing the property that every language of an SRE is downward closed.

\paragraph{Effectivity}

As we have seen, closures of languages are guaranteed to be regular.
This in turn means that they cannot be \emph{effectively regular} in general.
We know that they can be represented by \eg an NFA, but it may be impossible to compute such a representation.
We have given a sketch of the corresponding proof in \cref{Section:IntroClosures}, which we repeat here for the sake of completeness.
By the well-known results of Turing and others, all non-trivial decision problems for languages of Turing machines are undecidable.
This in particular means that the emptiness problem for Turing machine languages is undecidable.
However, the upward and downward closure of a language are empty if and only if the language itself is empty.
Hence, the computability of an effective representation of the closure of a Turing machine language (\eg an NFA or a regular expression) would allow us to decide the emptiness problem, a contradiction.

\end{document}
