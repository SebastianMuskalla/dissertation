\documentclass[../../diss.tex]{subfiles}
\begin{document}

\section{Downward closures for BPP nets}%
\label{Section:BPPDC}%

The non-primitive recursive complexity of the downward closure in the case of general Petri nets is a dire insight.
On the practical side, one should take into account that we are talking about the worst-case complexity.
It has been shown in experiments that the worst-case behavior occurs rarely.
On many instances that are of practical interest (\eg for the purpose of verifying systems), the Karp-Miller tree, which can be used to represent the downward closure, can be computed within acceptable time.
On the theoretic side, this unexpectedly tame behavior of these instances can partially be explained by the fact that they may stem from restricted subclasses of Petri nets.
Hence, we study the problem of computing downward closures for one such subclass.
As in \cref{Chapter:UC}, we focus on the class of BPP nets.
This means we consider the following problem.

\begin{compproblem}
    \problemtitle{Computing the downward closure for BPP net languages}
    \problemshort{($\BPPDC$)}
    \probleminput{A labeled BPP net instance $(N,\Minit,\Mfinal)$.}
    \problemquestion{An NFA $A$ with $\lang{A} = \dc{\lang{N,\Minit,\Mfinal}}$.}
\end{compproblem}

In this section, we will prove that this problem can be solved in exponential time.
This also implies that the state complexity is at most exponential.

\begin{theorem}%
\label{Theorem:BPPDCGeneral}%
    Downward closures of BPP net coverability languages have exponential state complexity, and the corresponding automata can be constructed in exponential time. These bounds are tight.
\end{theorem}

The result is similar to \cref{Theorem:BPPUCGeneral} in the case of upward closures of BPP nets.
However, in the case of the upward closure, the exponential complexity of the problem for BPP nets was only a moderate improvement over the doubly exponential complexity in the general case.
Here, the exponential complexity is a vast improvement over the non-primitive recursive complexity in the case of unrestricted Petri nets.

As usual, we prove upper and lower bound separately, starting with the upper bound.
We comment on the proof approach for each of the bounds after formally stating the result.

\paragraph{Upper bound}

We prove one direction of \cref{Theorem:BPPDCGeneral}:
Within exponential time, we can construct an NFA of exponential size accepting the downward closure of the language of a given BPP \nb{net instance}.

%
\cheatpagebreak
%

\begin{theorem}%
\label{Theorem:BPPDC}%
    Given a labeled BPP net instances $(N,\Minit,\Mfinal)$, we can compute in exponential time an NFA $A$ with $\lang{A} = \dc{\lang{N,\Minit,\Mfinal}}$.
    The state complexity of $\dc{\lang{N,\Minit,\Mfinal}}$ is at most exponential, as witnessed by automaton $A$.
\end{theorem}

To show the theorem, we prove a result that can be seen as some sort of \emph{pumping lemma} for BPP nets:
If it is possible to create more than an exponential number of tokens on a place (where the precise number will be given later), then it is possible to create arbitrarily many tokens on that place.
To be precise, to get the desired number of tokens one can insert a suitable number of occurrences of a pump, a sequence of transitions, into the computation that creates these tokens.
The insertion of these pumps contributes to the word generated by the computation, but the downward closure in which we are interested will also contain a version of the word in which the additional letters are deleted.

Let us formally state this pumping lemma.
We assume that the labeled BPP net instance $(N,\Minit,\Mfinal)$ of size $n$ is fixed.
We use $\norminf{N}$ to denote $\max_{p \in P} \set{ \norminf{\i(p)}, \norminf{\o(p)}}$, the maximum multiplicity of any transition.

\begin{lemma}%
\label{Lemma:BPPPumping}%
     If $\Minit \fire{\sigma} M$ and there is a place $p$ with $M(p) > k$, where
     \[
         k = \paren{\normone{\Minit}+2} \cdot \paren{ \paren{\card{P}+1} \cdot \norminf{N}}^{ \paren{\card{T} + 2} }
         \ ,
     \]
     then for each $m \in \N$, there is $\Minit \fire{\hat{\sigma}} \hat{M}$ such that
     (1)~$\sigma \subword \hat{\sigma}$,
     (2)~$M \leq \hat{M}$, and
     (3)~$\hat{M}(p) > m$.
\end{lemma}

We defer the proof of the lemma, and first show that under its assumption, \cref{Theorem:BPPUC} is easy to prove.
We consider the regular $\zeroto{k}$-$\omega$-overapproximation from \cref{Lemma:PNOverapproximation}.
It provides an automaton $A_{> k}$ that simulates the BPP net, but only tracks the number of tokens up to a bound $k$.
We instantiate it with $k$ as in the above lemma.
It is not hard to see that the downward closures of the language of $(N,\Minit,\Mfinal)$ and the downward close of $\langoak{k}{N,\Minit,\Mfinal}$, the language of the automaton $A_{> k}$, coincide.

\begin{proof}[Proof of \cref{Theorem:BPPDC}]
    Let $k = \paren{\normone{\Minit}+2} \cdot \paren{ \paren{\card{P}+1} \cdot \norminf{N}}^{ \paren{\card{T} + 2} }$ be as in \cref{Lemma:BPPPumping}.
    Consider the automaton $A_{> k}$ for the $\zeroto{k}$-$\omega$-approximation and its language $\lang{A_{>k}} = \langoak{k}{N,\Minit,\Mfinal}$ from \cref{Section:PNCovLang}.
    As argued in \cref{Lemma:PNOverapproximation}, $\lang{N,\Minit,\Mfinal} \subseteq \langoak{k}{N,\Minit,\Mfinal}$ and hence $\dc{\lang{N,\Minit,\Mfinal}} \subseteq \dc{\langoak{k}{N,\Minit,\Mfinal}}$ hold.
    For the converse inclusion, consider $w \in \dc{\langoak{k}{N,\Minit,\Mfinal}}$.
    We consider an accepting run of $A_{> k}$ on a word $v \in \lang{A_{> k}}$ with $w \subword v$ that has to exist by the definition of the downward closure.
    The transition of $A_{> k}$ used in this run induce a firing sequence $\sigma$ that is formed by the corresponding transitions of net $N$.

    If the accepting run of $A_{> k}$ only uses markings from $\N^P$, \ie no component is ever $\omega$, this firing sequence $\sigma$ is valid.
    This means it induces a covering computation of $N$, proving $v \in \lang{N,\Minit,\Mfinal}$ and thus $w \in \dc{\lang{N,\Minit,\Mfinal}}$.

    Otherwise, the firing sequence decomposes into $\sigma = \sigma'.\sigma''$ such that $\Minit \fire{\sigma\lowprime} M'$ is a valid computation of $N$, and the state $M'$ of $A_{> k}$ reached by the transitions of the automaton corresponding to $\sigma'$ is the first state in which some place, say place $p$, is set to $\omega$.

    In the following, we demonstrate how to obtain a new sequence of transitions $\hat{\sigma}'.\sigma''$ such that $\sigma' \subword \hat{\sigma}'$ and all transitions used in the sequence are enabled with respect to place~$p$.
    More formally, we associate to a sequence of transitions $\sigma$ a pseudo-computation $\Minit = M_0 \fire{\sigma_1} M_1 \ldots \fire{\sigma_{\card{\sigma}}} M_{\card{\sigma}}$, where the markings are allowed to have negative entries, as in \cref{Section:PNUC}.
    In all markings $M_i$ occurring in the pseudo-computation associated to $\hat{\sigma}'.\sigma''$, we will have $M_i(p) > 0$, $M_i(p) \geq \i(t,p)$ if $t$ is the $\nth{i}$ transition, and the last marking $M_{\card{\sigma}}$ satisfies $M_{\card{\sigma}}(p) \geq \Mfinal(p)$.

    Let $m = \sum_{j = 1}^{\card{\sigma''}} \i(\sigma''_j,p) + \Mfinal(p)$.
    It is an upper bound for the number of tokens required on place $p$ so that all transitions in $\sigma''$ are enabled and so that the final marking is covered with respect to place $p$.
    Since after firing $\sigma'$, we had obtained value $\omega$ for place $p$ in the run of the automaton, we have $M' (p) > k$.
    We instantiate \cref{Lemma:BPPPumping} for $m$.
    There is a firing sequence $\hat{\sigma}'$ with $\Minit \fire{\hat{\sigma}\lowerprime} \hat{M}'$, $\sigma' \subword \hat{\sigma}'$, $M' \leq \hat{M}'$, and $\hat{M}'(p) > m$.
    The desired modification of $\sigma'$ is $\hat{\sigma}'$:
    % Indeed, $\sigma_1^{(j)}$ is a superword of $\sigma_1$.
    The number $m$ was chosen so that in the pseudo-computation associated to $\hat{\sigma}'.\sigma''$, all markings are enabled with respect to place $p$ and the final marking is larger than $\Mfinal$ for place $p$.

    After repeating this process for each of the places, \ie at most $\card{P}$ times, we obtain a pseudo-computation associated to some transition sequence $\tau$ in which all markings are non-negative, all transitions are enabled whenever they are fired, and the final marking \nb{covers $\Mfinal$}.
    Hence, $\Minit \fire{\tau} M \geq \Mfinal$ is a covering computation of $N$, proving that $\lambda(\tau) \in \lang{N,\Minit,\Mfinal}$, and thus $w \subword \lambda(\sigma'.\sigma'') \subword \lambda(\tau)$ is in the downward-closure $\dc{\lang{N,\Minit,\Mfinal}}$.

    We have established that the downward closures of $\langoak{k}{N,\Minit,\Mfinal}$ and $\lang{N,\Minit,\Mfinal}$ coincide.
    To obtain a representation of $\dc{\langoak{k}{N,\Minit,\Mfinal}}$, we take the automaton $A_{> 0}$ and add for each transition an $\eps$-labeled variant as described in \cref{Section:PNRelwork}.
    This does not change the number of states, so to finish the proof, we argue that the state complexity of $A_{> k}$ is as desired.
    It is as most
    \[
         2^n \cdot k^n
         % \leq
         % 2^n \cdot \paren{\paren{\normone{\Minit}+2} \cdot \paren{ \paren{\card{P}+1} \cdot \norminf{N}}^{ \paren{\card{T} + 2} }}^n
         \in \bigO{
             2^n
             \cdot
             \paren{\normone{\Minit}+2}^n
             \cdot
             \paren{ \paren{\card{P}+1} \cdot \norminf{N}}^{ n^2 }
         }
    \]
    using \cref{Lemma:PNOverapproximation} and the definition of $k$.
    By the power laws, this number is singly exponential (even if $\normone{\Minit}$ and $\norminf{N}$ are exponential).
\end{proof}

It remains to show the lemma on which the proof of the theorem relies.
For the proof, we will again need the unfolding of a BPP net and its properties, see \cref{Section:UCBPP}.
This in particular means that we have to eliminate spontaneous transitions from the net.
To this end, we assume that have applied the preprocessing described in \cref{Section:UCBPP}.
In the following, we will consider the bound $k = \normone{\Minit} \cdot \paren{ \card{P} \cdot \norminf{N}}^{ \paren{\card{T} + 1} }$.
This bound is subtly different from the bound $\paren{\normone{\Minit}+2} \cdot \paren{ \paren{\card{P}+1} \cdot \norminf{N}}^{ \paren{\card{T} + 2} }$ stated in \cref{Lemma:BPPPumping}.
The modifications take into account that the preprocessing adds a new place, a new transition, and a new token in the initial marking.
We also assume that the original net satisfies $\norminf{N} \geq 2$, which means that the preprocessing does not increase $\norminf{N}$.

\begin{proof}[Proof of \cref{Lemma:BPPPumping}]
    Let $(O,h)$ be the unfolding $\text{Unf}(N, \Minit)$ of the given net with respect to the initial marking, as defined in \cref{Section:UCBPP}.
    Assume that $\Minit \fire{\sigma} M$ is a computation of $N$ reaching a marking $M$ such that $M(p) > k = \normone{\Minit} \cdot \paren{ \card{P} \cdot \norminf{N}}^{ \paren{\card{T} + 1} }$ for some place $p$.

    We consider a computation $\min O \fire{\sigma\lowerprime} M'$ of $O$ induced by $\sigma$, \ie we have $h(\sigma') = \sigma$.
    We also have $h(M') = M$, meaning that for each place $p_1$ of $N$, the total number of tokens in places $p_1'$ with $h(p_1') = p_1$ is equal to $M(p)$.
    In particular, we have
    \[
        \sum_{\substack{p' \in P',\\  h(p') = p}} M' (p') = M(p) > k
    \]
    for the place $p$ fixed above.

    Recall that $O$ can be seen as a forest, each of its tree rooted in a place that carries a token in $\min O$.
    Considering only the part of $O$ that is used by the computation associated to $\sigma'$, \ie the transitions that are used in $\sigma'$ and the places that carry a token at some point during the computation, yields a finite prefix of $O$ that is again a forest.
    The places that carry a token in the final marking $M'$ form the leaves of the trees in the forest.
    We will call the places $p'$ with $M'(p') = 1$ and $h(p') = p$ the \emph{$p$-leaves} of the forest.
    The name is justified by the above discussion.

    We identify in the forest associated to $\sigma'$ the tree with the largest number of $p$ leaves.
    Let us denote this tree by $\calT$ and its root node by $r'$.
    The number of $p$-leaves in $\calT$ is at least
    \[
        k_1 = \frac{k}{\normone{M_\init}} = {(\card{P} \cdot \norminf{N})}^{(\card{T} + 1)}
        \ .
    \]
    The forest consists of exactly $\normone{\Minit}$ trees, since $h(\min O) = \Minit$ and each token in the initial marking uniquely identifies the root of a tree.
    If every tree had less than $\frac{k}{\normone{\Minit}}$ many $p$-leaves, then it could not be true that the forest has at least $k$ many $p$-leaves in total.

    In the following, we will show that the minimal subtree of $\calT$ that contains the root and reaches all its $p$-leaves contains a \emph{pump}, \ie a substructure that can be replicated to obtain arbitrarily many tokens in place $p$.
    The idea is similar to the one behind the proof of the pumping lemma for context-free languages executed on parse trees.
    However, the structure of the unfolding causes some technical difficulties that we have to overcome.

    Let us denote by $\calT'$ the subtree of $\calT$ that consists of the $p$-leaves and their ancestors.
    Its leaves are exactly the $p$-leaves, and every other place has out-degree exactly one.
    That the out-degree cannot be larger than one is clear by the definition of $O$.
    A place with no outgoing edge has to be a $p$-leaf, as we have removed all nodes that are not ancestors of $p$-leaves.
    The out-degree of the transitions occurring in $\calT'$ is at most $\norminf{N} \card{P}$, since each transition of $N$ can create at most $\norminf{N}$ tokens in each of the $\card{P}$ places.

    Similar to the proof of \cref{Proposition:BPPBound}, we will call a transition of out-degree at least two a \emph{join transition}, as it leads to at least two different $p$-leaves.
    The latter property is guaranteed by the acyclicity of $O$.
    The definition of join transitions is simpler here (compared to the proof of \cref{Proposition:BPPBound}), because we only consider the part of the computation that leads to $p$-leaves.

    We claim $\calT'$ has a branch with at least $\card{T} + 1$ join transitions on it.
    To this end, we state and prove the following claim.

    \subproof{Claim}

    Let $T$ be a tree with $\ell$ leaves in which all nodes have out-degree at most $c$.
    Then $T$ has a branch with at least $\log_c \ell$ nodes of out-degree at least $2$.

    To prove the claim, assume that in any branch of such a tree $T$, the number of nodes of out-degree greater than two in any branch of the tree is $m < \log_c \ell$.
    Such a tree has at most $c^m < c^{\log_c \ell} = \ell$ nodes, which is a contradiction to the assumption that $T$ has $\ell$ leaves.

    We now use the claim for the tree $\calT'$, $\ell = k_1$, $c = \norminf{N} \cdot \card{P}$, and obtain that $\calT'$ has at least
    \[
        k_2 = \log_{\norminf{N} \cdot \card{P} } k_1 = \card{T} + 1
    \]
    many join transitions.
    This means that $\calT'$ has a branch with transitions $t',t''$ of $O$ such that $h(t') = h(t'') = t$ for some transition $t$ of $N$, since $N$ has only $\card{T}$ different transitions.

    Since $t'$ is a join transition, $t'$ is contained in (at least) two branches:
    The first is a branch that also contains $t''$ and ends in a $p$-leaf; let $\sigma_a'.t'.\sigma_b'.t''.\sigma_c'$ be the transitions along that branch.
    The other branch does not contain $t''$ and ends in a different $p$-leaf; let $\sigma_a'.t'.\sigma_d'$ be the associated transitions.
    Note that $\sigma_a', \sigma_b', \sigma_c', \sigma_d' \in {T'}{}^*$ denote sequences of transitions here.

    We now argue that we can first use $t'.\sigma_b'$ to create an arbitrary number of tokens on the places from which transition $t$ of $N$ consumes a token, and later use $t'.\sigma_d'$ to create an arbitrary number of tokens on place $p$ as desired.
%
    More formally, we have that $\sigma_a'.t'.\sigma_b'.t''.\sigma_c'$ is a subword of the computation $\sigma'$ of $O$, and hence that $h(\sigma_a'.t'.\sigma_b'.t''.\sigma_c')$ is a subword of $\sigma$.
    Hence, we may write $\sigma = \sigma_A.t.\sigma_B$, where $t$ corresponds to the occurrence of $t'$ in $\sigma_a'.t'.\sigma_b'.t''.\sigma_c'$.

    Let $m \in \N$ be an arbitrary number.
    We have to construct $\hat{\sigma}$ such that $\Minit \fire{\hat{\sigma}} \hat{M}$ satisfies
    (1)~$\sigma \subword \hat{\sigma}$,
    (2)~$M\leq \hat{M}$, and
    (3)~$\hat{M}(p)>m$.
    We define $\text{pump}_1 = h(\sigma_b'.t'')$, and $\text{pump}_2 = h(\sigma_d')$.
    We claim that $\hat{\sigma} = \sigma_A.t.{\text{pump}_1}^m.\sigma_B.{\text{pump}_2}^m$ is as required.
    Firstly, note that indeed transition $t$ produces the token required by the first transition of $\text{pump}_1$ and the token required by $\sigma_B$.
    Secondly, the last transition of $\text{pump}_1$ is again $h(t'') = t$.
    Altogether, $\sigma_A.t.{\text{pump}_1}^m.\sigma_B$ is a valid firing sequence.
    Furthermore, the $m$ iterations of $\text{pump}_1$ produce at least $m$ tokens on the place from which $\text{pump}_2$ consumes a token.
    Hence, $\Minit \fire{\hat{\sigma}} \hat{M}$ is a valid firing sequence.
    The properties (1) -- (3) are easy to justify: (1)~$\sigma = \sigma_A.t.\sigma_B$ is a subword of $\hat{\sigma}$ by definition, (2)~the pumps only create additional tokens, and (3)~in particular, each iteration of ${\text{pump}_2}$ creates a token on place $p$, since $\sigma_a'.t'.\sigma_d'$ corresponds to a branch in $O$ leading to a $p$-leaf.
\end{proof}

With the proof of \cref{Lemma:BPPPumping} completed, the upper bound, \cref{Theorem:BPPDC} is proven.

\begin{remark*}
In the case of the upward closure, the proofs of the upper bounds were based on the length-$k$ approximation, both in the general case and in the case of BPP nets.
In the case of the downward closure, the constructions seem to be fundamentally different:
In the general case, the construction was based on the Karp-Miller tree, while in the case of BPP nets, we have used the $\zeroto{k}$-$\omega$-approximation.
We claim that these differences can be overcome:
The proof concept from this section can be generalized so that it also applies to the case of general nets.
We claim that to obtain the downward closure of a Petri net, we can take the downward closure of the $\zeroto{k}$-$\omega$-approximation, where $k$ is the greatest non-$\omega$ number occurring in a general marking labeling a node in the Karp-Miller tree.
Proving this fact is not difficult, but relies on some properties of the Karp-Miller tree that we have not stated here.
Hence, we forgo giving the proof.
\end{remark*}

\paragraph{Lower Bound}

To prove that the above construction is optimal, we present a family of BPP languages for which the state complexity of the downward closure is exponential in the size of the nets.
The proof is similar to the proof of \cref{Proposition:BPPUChardness} and uses that exponential numbers can be stored in polynomial space using their binary encoding.
The difference is that this time, we use an exponential number in the initial marking instead of in the final marking.

\begin{proposition}%
\label{Propositition:BPPDCHardness}%
    For each $n \in \N$, there is a labeled BPP net instance $(N, \Minit(n), \Mfinal)$ of size polynomial in $n$ such that $\dc{\lang{N, \Minit(n), \Mfinal}}$ has state complexity at least $2^n$.
\end{proposition}

\begin{proof}
    We define $N$ to be the net with a single place $p$ and a single $a$-labeled transition $t$ that consumes a token from this place.
    The final marking $\Mfinal$ is the zero vector, the initial marking $\Minit$ places $2^n$ tokens on $p$.
    This net is depicted in \cref{Figure:BPPDCLowerBoundBinary}.

    Any computation of the net is covering, and each sequence $t^k$ for $k \leq \Minit(p) = 2^n$ is a valid firing sequence.
    Hence, $\lang{N, \Minit(n), \Mfinal} = \Set{ a^k }{ k \leq 2^n}$.
    This language is already downward closed and has state-complexity $2^n + 1$, see \cref{Example:StateComplexityEqualsIndex}.
    To conclude the proof, note that the size of $N$ and $\Minit$ is constant, while the size of $\Mfinal$ is linear in $n$.
\end{proof}

The proof of the proposition also completes the proof of \cref{Theorem:BPPDCGeneral}.

\begin{figure}[t]
    {\centering\subcaptionbox{In the binary case.%
    \label{Figure:BPPDCLowerBoundBinary}}[0.25\textwidth][c]{
        \begin{tikzpicture}[->,>=latex,node distance=5em,semithick]

\node[place] (P) [minimum size=3em, tokens=9, yshift=+0.5em] {};

\node at (P) [fill=white,circle,inner sep=0em, scale=0.75] {$2^n$};

\node [transition,square] (A) [right of=P,inner sep=0.25em] {$t$}
    edge [pre] (P)
;

\node [below of=P, node distance=2.5em] {$p$};
\node [below of=A, node distance=1.5em] {$a$};

\end{tikzpicture}
}
    }
    {\centering\subcaptionbox{In the unary case.\label{Figure:BPPDCLowerBoundUnary}}[0.7\textwidth][c]{
        \begin{tikzpicture}[->,>=latex,node distance=8em,semithick]

\node[place] (P0) [minimum size=3em,tokens=1] {};
\node[place] (P1) [minimum size=3em, right of=P0] {};
\node (P2) [right of=P1] {$\cdots$};
\node[place] (PN) [minimum size=3em, right of=P2,node distance=4em] {};

\node [transition,square] (T1) [right of=P0,node distance=4em,inner sep=0.25em]
    {$t_{1}$}
    edge [pre] (P0)
    edge [post] node [above] {2} (P1)
;
\node [transition,square] (T2) [right of=P1,node distance=4em,inner sep=0.25em]
    {$t_{2}$}
    edge [pre] (P1)
    edge [post] node [above] {2} (P2)
;

\node [transition,square] (A) [right of=PN,node distance=4em,inner sep=0.25em]
    {$t_{a}$}
    edge [pre] (PN)
;

\path [->] (P2) edge  node [above] {2} (PN);

\node [below of=P0, node distance=2.5em] {$p_0$};
\node [below of=P1, node distance=2.5em] {$p_1$};
\node [below of=PN, node distance=2.5em] {$p_n$};

\node [below of=T1, node distance=2em] {$\eps$};
\node [below of=T2, node distance=2em] {$\eps$};

\node [below of=A, node distance=2em] {$a$};

\end{tikzpicture}
}
    }
    \caption{BPP nets to prove the exponential lower bound for the downward closure.}%
    \label{Figure:BPPDCLowerBound}%
\end{figure}

\paragraph{The unary case}

We complete our study by considering the case in which the BPP net is encoded in unary.
Consider the upper bound.
The size of the automaton that we construct is based on the number $k = \paren{\normone{\Minit}+2} \cdot \paren{ \paren{\card{P}+1} \cdot \norminf{N}}^{ \paren{\card{T} + 2} }$ from \cref{Lemma:BPPPumping}.
This number is exponential, even if we assume that the net is encoded in unary.
Hence, we do not get an improved \nb{upper bound}.

The proof of the lower bound, \cref{Propositition:BPPDCHardness}, obviously becomes invalid if we consider nets to be encoded in unary.
With a unary encoding, it is not possible to assign $2^n$ tokens to a place in the initial marking of an instance whose size is polynomial in $n$.
However, it turns out that we can construct an instance $(N,\Minit,\vec{0})$ of size polynomial in $n$ whose language coincides with the language of the net from \cref{Propositition:BPPDCHardness}.

The construction is more involved than the one from the proof of \cref{Propositition:BPPDCHardness}.
It uses the well-known fact that a context-free grammar of size polynomial in $n$ can generate the word $a^{2^n}$.
Here, we adapt this construction to BPP nets.

The net that we construct is depicted schematically in \cref{Figure:BPPDCLowerBoundUnary}.
It has $n+1$ places $p_0, p_1, \ldots, p_n$.
The initial marking assigns one token to place $p_0$ and zero tokens elsewhere.
For each $i \in \oneto{n}$, there is a transition $t_i$ that consumes a token from place $p_{i-1}$ and produces two tokens on $p_i$.
All these transitions are labeled by $\eps$.
Additionally, there is an $a$-labeled transition $t_a$ that consumes a token from $p_n$.
The final marking is the zero vector.
Note that this net is indeed a BPP net, every transition consumes exactly one token.

Let us prove that the language of the instance is indeed $\Set{ a^k }{ k \leq 2^n}$.
By firing the transitions $t_i$ exhaustively, one can generate $2^n$ tokens on place $2^n$: Assuming that $2^{i}$ tokens are present on place $p_i$, one can fire transition $t_{i+1}$ exactly $2^{i}$ times to generate $2 \cdot 2^{i} = 2^{i+1}$ tokens on $p_i$.
Now, we may fire transition $t_a$ up to $2^n$ times to generate an arbitrary number of $a$s between zero and $2^n$.

To conclude the proof, it remains to observe that the size of the instance is indeed polynomial in $n$.
Hence, we have re-proven the lower bound, \cref{Propositition:BPPDCHardness}.
We obtain a matching exponential lower and upper bound in the case of BPP nets encoded in unary.

\end{document}
