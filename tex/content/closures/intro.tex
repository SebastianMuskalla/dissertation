\documentclass[../../diss.tex]{subfiles}
\begin{document}

\chapter*{Part III.\newline Closures of Petri net languages}

This part of the thesis provides the theoretical development corresponding to \cref{Section:IntroClosures} of the introduction.
We study the closures of Petri net languages with respect to the subword order.
We do so for both general Petri nets and for restrictions thereof.

\paragraph{Outline}

In \cref{Chapter:Closures}, we give some basic definitions.
In particular, we formally introduce the upward and downward closures of languages and mention some of their properties.
We also discuss related work from the literature and we summarize the results contained in this part of the thesis.

\cref{Chapter:UC} is concerned with the upward closures of Petri nets, while \cref{Chapter:DC} contains our work on downward closures.

Finally, \cref{Chapter:BeingUCDC} contains our study on the problem of deciding whether a given Petri net language is upward \resp downward closed.
In this context, we also explore the related problem of deciding regular containment.

\paragraph{Publication}

Most of the content of this section has been published in the form the paper~\cite{AtigMMS17} (\resp its full version~\cite{AtigMMS17a}).
The author's contributions to that publication are discussed in detail in \cref{Chapter:Contributions}.
In comparison to the publication, this thesis adds some details, \eg a study of the state complexity of the language closures in terms of the size of the Petri net encoded in~unary.

\end{document}
