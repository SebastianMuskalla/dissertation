\documentclass[../../diss.tex]{subfiles}
\begin{document}

\section{SRE inclusion in the downward closures for Petri nets}%
\label{Section:PNSREDC}%

Similar to \cref{Section:PNSREUC}, we study the problem of checking whether a given SRE is contained in the downward closure of a given Petri net coverability language.

\begin{problem}
    \problemtitle{SRE inclusion in the downward closure}
    \problemshort{($\SREDC$)}
    \probleminput{A labeled Petri net instance $(N,\Minit,\Mfinal)$, an SRE $\sre$.}
    \problemquestion{$\lang{\sre} \subseteq \dc{\lang{N,\Minit,\Mfinal}} $?}
\end{problem}

The goal of this section is to show that the problem is complete for $\EXPSPACE$, the class of problems solvable with exponential space.
This matches the $\EXPSPACE$-completeness of $\SREUC$, and is a huge improvement over the non-primitive recursive complexity of computing the downward closure.

\begin{theorem}%
\label{Theorem:PNSREDC}%
    $\SREDC$ is $\EXPSPACE$-complete.
\end{theorem}

We first briefly comment on the lower bound, before we turn to developing some techniques needed for proving $\EXPSPACE$-membership.

\paragraph{Lower bound}

Similar to the $\EXPSPACE$-hardness of the problem $\SREUC$, \cref{Lemma:PNSREUChardness}, the hardness of $\SREDC$ directly follows from the hardness of coverability.
We briefly recall the idea of the proof, but refer to \cref{Lemma:PNSREUChardness} for the technical details.

If we modify an unlabeled Petri net instance $(N,\Minit,\Mfinal)$ by labeling all transitions by $\eps$, we have $\lang{N,\Minit,\Mfinal} = \set{\eps}$ if $\Mfinal$ is coverable from $\Minit$.
If this is not the case, we have $\lang{N,\Minit,\Mfinal} = \emptyset$.
Both these languages are already downward closed.
Hence, we have that $\emptyset^* = \set{\eps} \subseteq \uc{\lang{N,\Minit,\Mfinal}}$ holds if and only if $(N,\Minit,\Mfinal)$ is a yes-instance of coverability.


\paragraph{Upper bound}

It remains to prove $\EXPSPACE$ membership.

\begin{theorem}%
\label{Theorem:PNSREDCMembership}%
    $\SREDC$ can be solved using exponential space.
\end{theorem}

Let $(N,\Minit,\Mfinal)$ be the Petri net instance of interest, and let $\sre = \product_1 \cup \ldots \cup \product_k$ be the given SRE.\@
Our goal is to check whether the inclusion $\lang{\sre} \subseteq \dc{\lang{N,\Minit,\Mfinal}}$ holds.

In \cref{Section:PNSREUC}, we did proceed as follows to solve the problem in the case of the upward closure:
Firstly, we observed that the inclusion holds if and only if $\lang{\product_i} \subseteq \uc{\lang{N,\Minit,\Mfinal}}$ holds for each product $\product_i$.
Secondly, we specified an operation that computes for each product $\product_i$ the unique minimal word of $\uc{\lang{\product_i}}$.
Checking the inclusion then amounts to checking whether this minimal word is contained in the downward closure of the net.
Altogether, we reduced the problem $\SREUC$ to a polynomial number of membership queries.

The first observation still holds true in the case of $\SREDC$:  $\lang{\sre} \subseteq \dc{\lang{N,\Minit,\Mfinal}}$ holds if and only if $\lang{\product_i} \subseteq \dc{\lang{N,\Minit,\Mfinal}}$ for all products $\product_i$.
Since the number of products is polynomial in the size of the SRE $\sre$, this will allow us to focus on the simpler problem of checking whether the language of a single product $\product$ is contained in the downward closure.

Unfortunately, the second observation does not carry over.
Unlike upward-closed languages, downward-closed languages cannot be represented by a single word in general.
Consider for example the product $\Sigma^*$ (whose language is already downward closed).
A priori, there is no reason why there should be a single word such that inclusion of the product and the membership problem for that word are equivalent.
This means that checking whether a product is contained in the downward closure is much more involved.

Nevertheless, we can provide a syntactic transformation of the product that will later simplify solving the problem.
Intuitively, our goal is to replace each occurrence of the iteration of a subalphabet by the iteration of a single word.
To this end, we fix some total order on the alphabet, \eg $a_1 < a_2 < \ldots < a_m$, where $\Sigma = \set{a_1, \ldots, a_m}$.
We obtain a word $w_\Sigma = a_1 a_2 \ldots a_m$ by concatenating the letters in ascending order.
To each subalphabet $\Gamma \subseteq \Sigma$, we assign the word $w_\Gamma = \proj{\Gamma}{w}$ that is obtained from $w_\Sigma$ by removing all letters not present in $\Sigma$.

We can now define a linearization operation that turns a product into a regular expression.
%
\begin{definition}%
\label{Definition:Linearization}%
    The \emph{linearization} of a product is inductively defined as follows.
    \begin{align*}
        \linof{a \cup \varepsilon} &=a
        &\linof{a}&=a\\
        \linof{\Gamma^*}&={w_\Gamma}^*
        &\linof{\product_1.\product_2} &= \linof{\product_1}.\linof{\product_2}\ .
    \end{align*}
\end{definition}

\begin{example}%
\label{Example:PNLinearization}%
    For example, if $\Sigma=\set{a, b, c}$ and we take $w_{\Sigma}=abc$, then $\product={(a\cup c)}^*(a\cup\eps){(b\cup c)}^*$ is turned into $\linof{\product}={(ac)}^*a{(bc)}^*$.
\end{example}

The following lemma states that the linearization indeed allows us to simplify the problem:
The language of a product is contained in the downward closure if and only if the language of its linearization is.

\begin{lemma}%
\label{Lemma:Linearization}%
    For a product $\product$, we have
    $\lang{\product} \subseteq \dc{\lang{N, \Minit, \Mfinal}}$
    if and only if
    $\lang{\linof{\product}}\subseteq \dc{\lang{N, \Minit, \Mfinal}}$.
\end{lemma}

\begin{proof}
    We claim that $\dc{\lang{\product}} = \dc{\lang{\linof{\product}}}$, which implies the desired statement.
    We proceed by induction on the structure of the product.
    We have $\dc{\lang{a \cup \eps}} = \set{a,\eps} = \dc{\lang{a}} = \dc{\lang{\linof{a}}} = \dc{\lang{\linof{a \cup \eps}}}$, which proves two of the base cases.

    Consider $\Gamma^*$.
    We have $\dc{\lang{\Gamma^*}} = \Gamma^*$.
    It remains to show that $\dc{\lang{{w_\Gamma}^*}} = \Gamma^*$.
    One direction, $\dc{\lang{{w_\Gamma}^*}} \subseteq \Gamma^*$ is clear since $w_\Gamma$ only contains letters from $\Gamma$.
    For the other direction, consider $v \in \Gamma^*$.
    We claim that $v$ is a subword of ${(w_\Gamma)}^{\card{v}}$: Each letter of $v$ is from $\Gamma$, so it occurs in $w_\Gamma$.
    Dropping all other letters in each iteration of ${(w_\Gamma)}^{\card{v}}$ proves the statement.
    Hence, $v \in \dc{\lang{{w_\Gamma}^*}}$ as desired.

    Finally, for the induction step, assume that $\dc{\lang{\product_i}} = \dc{\lang{\linof{\product_i}}}$ for $i \in \set{1,2}$.
    We have
    \begin{align*}
    \dc{\lang{\product_1.\product_2}}
    &= \dc{\paren{\lang{\product_1}.\ \lang{\product_2}}}
    = \dc{\lang{\product_1}}.\ \dc{\lang{\product_2}}
    \\
    &= \dc{\lang{\linof{\product_1}}}.\dc{\lang{\linof{\product_2}}}
    = \dc{\paren{\lang{\linof{\product_1}}.\lang{\linof{\product_2}}}}
    \\
    &= \dc{\lang{\linof{\product_1}.\linof{\product_2}}}
    = \dc{\lang{\linof{\product_1.\product_2}}}
    \ .
    \end{align*}
    Here, we have used induction, the fact that the downward closure distributes over concatenation, and the definition of the linearization.
\end{proof}

With the lemma at hand, the remaining task is the following:
For each infix ${w_\Gamma}^*$ occurring in the linearization of the product, we need to check whether arbitrarily long instantiations lead to a word contained in the downward closure.
Intuitively, this means that we need to count the number of instantiations of each ${w_\Gamma}^*$.
Formally, we reduce checking the inclusion to an unboundedness problem.

The specific type of unboundedness that we require is \emph{simultaneous unboundedness}, formally defined in the following.

\begin{problem}
    \problemtitle{Simultaneous unboundedness for Petri nets}
    \problemshort{($\PNSU$)}
    \probleminput{Petri net $N$, initial marking $\Minit$, set of places $X$ of $N$.}
    \problemquestion{Are the places in $X$ \emph{simultaneously unbounded}, \ie is there for each $m \in \N$\newline
    a computation $\Minit \fire{\sigma} M$ with $M(p) \geq m$ for all $p \in X$?}
\end{problem}

Simultaneous unboundedness has been introduced by \citea{Demri13} as a generalization of several classical notions of unboundedness, \ie the questions whether a given net Petri is unbounded in some arbitrary \resp a specified component.
Indeed, we need this simultaneous unboundedness to encode our problem:
It is not sufficient to separately check that for each infix ${w_\Gamma}^*$, arbitrarily long instantiations lead to a word in the downward closure.
Rather, we need that even if we combine arbitrarily long instantiations of each of the infixes, we still obtain a word in the downward closure.

All aforementioned unboundedness problems could be solved using the Karp-Miller tree.
For example, the places in $X$ are simultaneously unbounded if the Karp-Miller tree contains a node labeled by a generalized marking in which all places $p \in X$ attain value $\omega$.
However, this procedure would inherit the non-primitive recursive complexity of constructing the \nb{Karp-Miller tree}.

Luckily, Demri~\cite{Demri13} has shown that the complexity of $\PNSU$ is in fact $\EXPSPACE$.

\begin{theorem}[(\citea{Demri13})]%
\label{Theorem:PNSU}%
    $\PNSU$ is $\EXPSPACE$-complete.
\end{theorem}

The proof of this result is an extension of Rackoff's technique~\cite{Rackoff78}.
Demri shows that if a set of places is simultaneously unbounded, then there are specific computations that witness this fact.
With a Rackoff-like argumentation, one obtains a doubly exponential bound on the length of these computations.
This finally allows for checking all such computations using only exponential space by using counters encoded in binary and Savitch's theorem~\cite{Savitch70}.

With the $\EXPSPACE$ membership of $\PNSU$ at hand, it remains to reduce the inclusion $\lang{\linof{\product}}\subseteq \dc{\lang{N, \Minit, \Mfinal}}$ to an instance of $\PNSU$ in polynomial time.
The first step is to consider the net $\dc{N}$ with $\dc{\lang{N, \Minit, \Mfinal}} = \lang{\dc{N}, \Minit, \Mfinal}$, see \cref{Section:PNRelwork}.

As explained above, we need to check whether for each infix ${w_\Gamma}^*$, arbitrarily long instantiations still lead to words in the language of $\dc{N}$
We encode this by introducing for each infix ${w_\Gamma}^*$ a place that tracks that $w_\Gamma$ has been seen.
The instance of the unboundedness problem that we construct requires all such places to be simultaneously unbounded.
We will make this formal in the following.
Afterwards, the minor problem of also encoding the final marking of the net into the unboundedness problem remains to be solved.

To make the construction formal, we see $\linof{\product}$ as a Petri net.
To be precise, we construct an NFA that is language-equivalent to $\linof{\product}$, \eg using the construction by \citea{McNaughtonY60}.
Then, we see this NFA as a Petri net.
We have given a construction in \cref{Example:RegularPNCOV}, but here, it will be important that we obtain a net with no $\eps$-labeled transitions.
To this end, we assume that the NFA has a unique final state, a condition that is easy to enforce, especially for the NFAs with language $\linof{\product}$.
Then, we construct a Petri net instance $(N_\product,M_{\init,\product},M_{\final,\product})$.
Net $N_\product$ has one place per state of the NFA and one transition per transition of the NFA.\@
The labeling and effect of the transitions is as expected.
The initial marking $M_{\init,\product}$ puts one token on the place corresponding to the initial state of the NFA and no tokens elsewhere.
The final marking $M_{\final,\product}$ just requires one token on the place corresponding to the unique final state of the NFA.
We obtain $\lang{\linof{\product}} = \lang{N_\product,M_{\init,\product},M_{\final,\product}}$ as desired.

It remains to modify the net so that for each infix ${w_\Gamma}^*$ of $\linof{\product}$, it counts how many iterations of ${w_\Gamma}^*$ have occurred.
For such an infix, net $N_\product$ has a transition that correspond to the last letter of $w_\Gamma$.
We add a new place $p_\Gamma$ and modify the transition so that it additionally creates a token on $p_\Gamma$.
The desired result is that if a computation produces $m$ tokens on $p_\Gamma$, then ${w_\Gamma}^m$ was the instantiations of ${w_\Gamma}^*$ that has been read.

Note that the notation is intentionally a bit sloppy here:
The linearization may contain several occurrences of ${w_\Gamma}^*$ for the same subalphabet $\Gamma$.
In this case, each occurrence gets its own \nb{place $p_\Gamma$}.
To improve readability, we forgo introducing additional indices

\begin{example}%
\label{Example:PNLinearizationNet}%
    Consider $\linof{ {(a\cup c)}^*(a\cup\varepsilon){(b\cup c)}^* } = {(ac)}^*a{(bc)}^*$ from \cref{Example:PNLinearization}.
    The net $N_\product$ for this product is depicted in \cref{Figure:PNLinearizationNet}, including the two places $p_{\set{a,c}}$ and $p_{\set{b,c}}$ that count the iterations.
    To improve readability, some names of places and transitions have been omitted.
    % The initial marking puts one token on $p_\init$, the final marking requires one token on $p_\final$.
\end{example}

\begin{figure}[t]
    \centering%
    \begin{tikzpicture}[->,>=latex,node distance=7em,semithick]

\node[place] (Pinit)  [minimum size=3em,tokens=1] {};
\node[place] (Ploop1) [minimum size=3em, above of=P0] {};
\node[place] (Pfinal) [minimum size=3em, right of=Pinit, node distance=21em] {};
\node[place] (Ploop2) [minimum size=3em, above of=Pfinal] {};

\node[place] (Gamma1) [minimum size=3em, right of=Ploop1] {};
\node[place] (Gamma2) [minimum size=3em, left of=Ploop2] {};

\node [transition] (Aloop1) [above left of=Pinit,node distance=5em,inner sep=0.4em] {$a\mathstrut$}
    edge [pre] (Pinit)
    edge [post] (Ploop1)
;

\node [transition] (Cloop1) [above right of=Pinit,node distance=5em,inner sep=0.4em] {$c\mathstrut$}
    edge [pre] (Ploop1)
    edge [post] (Pinit)
    edge [post] (Gamma1)
;

\node [transition] (A) [right of=Pinit,node distance=11em,inner sep=0.4em] {$a\mathstrut$}
    edge [pre] (Pinit)
    edge [post] (Pfinal)
;

\node [transition] (Bloop2) [above right of=Pfinal,node distance=5em,inner sep=0.4em] {$b\mathstrut$}
    edge [pre] (Pfinal)
    edge [post] (Ploop2)
;

\node [transition] (Cloop2) [above left of=Pfinal,node distance=5em,inner sep=0.4em] {$c\mathstrut$}
    edge [pre] (Ploop2)
    edge [post] (Pfinal)
    edge [post] (Gamma2)
;

\node [below of=Pinit, node distance=2em] {$p_{0}$};
\node [below of=Pfinal, node distance=2em] {$p_{f}$};
\node [below of=Gamma1, node distance=2em] {$p_{\set{a,c}}$};
\node [below of=Gamma2, node distance=2em] {$p_{\set{b,c}}$};

\end{tikzpicture}
%
    \caption{A net $N_\product$ with language ${(ac)}^*a{(bc)}^*$ that also counts the number of iterations.}%
    \label{Figure:PNLinearizationNet}%
\end{figure}

We now construct the synchronized product of $\mathmbox{\dc{N}}$ and $N_\product$, which we will denote by $(N',\Minit',\Mfinal')$.
Recall that its set of places is the disjoint union of the places of $\dc{N}$ and $N_\product$.
Transitions labeled by $\eps$ can be fired freely, while transitions labeled by $a \in \Sigma$ are synchronized in that we can only fire an $a$-labeled transition of $\dc{N}$ and an $a$-labeled transition of $N_\product$ together.
For the correctness of the construction, it will be crucial that the synchronized product acts as a one-sided synchronized product in the special case at hand:
(1)~$N_\product$ has no $\eps$-labeled transitions, so all transitions of $N_\product$ have to be synchronized with an appropriate transition of $\dc{N}$, while
(2)~$\dc{N}$ has an $\eps$-labeled version of each transition, so each transition can be fired freely.

We would like to have that the desired inclusion holds if and only if the set of places $\Set{ p_{\Gamma}}{ {w_{\Gamma}}^* \text{ infix of } \linof{\product}}$ is simultaneously unbounded in $N'$ from $\Minit'$
However, we still have to take the final marking into account.
To achieve this, we add a new place to $N'$ that can only become unbounded as soon as the final marking has been covered.
Formally, we add a new place $p_\final$ to $N'$.
We add a transition $t_\final$ with $\i(t_\final) = \Mfinal'$ that consumes tokens from all places as specified by the final marking and that produces one token on $p_\final$ (and no token elsewhere).
Finally, we add a transition $t_\pump$ that consumes one token on $p_\final$, but produces two tokens on $p_\final$.
Let $N''$ denote the resulting net, and let $\Minit''$ be the initial marking for $N''$ that coincides with $\Minit'$ on the places of $N'$ and assigns no token to $p_\final$.
The problem $\PNSU$ works on unlabeled nets, but for the proof of correctness later, it will be helpful to consider both $t_\final$ and $t_\pump$ to be labeled by $\eps$.

As desired, we obtain that the inclusion holds if and only if the set of places $\Set{ p_{\Gamma}}{ {w_{\Gamma}}^* \text{ infix of } \linof{\product}} \cup \set{p_\final}$ is simultaneously unbounded in $N''$.
To formally prove the statement, let us assume that $\linof{\product} = \dtr{w}{0} {w_{\Gamma_1}}^* \dtr{w}{1} \ldots w_{k-1} {w_{\Gamma_k}}^* \dtr{w}{k}$ for some words $\dtr{w}{i} \in \Sigma^*$ and subalphabets $\Gamma_i\subseteq \Sigma$ for all $i$.

\begin{proposition}%
\label{Proposition:PNSREDCReduction}%
    The inclusion
    $\lang{\linof{\product}} \subseteq \lang{\mathmbox{\dc{N}}, \Minit, \Mfinal}$
    holds iff the places in
    $X = \set{ p_\final } \cup \Set{ p_{\Gamma_i}}{i \in \oneto{k}}$
    are simultaneously unbounded in $N''$ from $\Minit''$.
\end{proposition}

\begin{proof}
    Assume that the inclusion holds, and let $m \in \N$ be arbitrary.
    We have to show that there is a marking $M$ with $\Minit \fire{\sigma} M$ in $N''$ and $M(p_{\Gamma_i}) \geq m$ for all $i$ and $M(p_\final) \geq m$.
    Consider the word $w = \dtr{w}{0} {w_{\Gamma_1}}^m \dtr{w}{1} \ldots w_{k-1} {w_{\Gamma_k}}^m \dtr{w}{k}$ obtained by taking $m$ iterations of each $w_\Gamma$.
    We have $w \in \lang{\linof{\product}}$ by construction, and since we assume that the inclusion holds also $w \in \lang{\dc{N}, \Minit, \Mfinal}$.
    Recall that the language of the synchronized product is the intersection of the languages of the original nets.
    Hence, we have $w \in \lang{N'',\Minit'',\Mfinal''}$, where $\Mfinal''$ coincides with $\Mfinal'$ on the places of $N'$ and requires no token on $p_\final$.
    Let us consider a corresponding computation $\Minit'' \fire{\sigma} M \geq \Mfinal''$ of $N''$ with $\lambda(\sigma) = w$.
    Since the non-$\eps$-labeled transitions in $N'$ can only be fired in a synchronized fashion, for each infix ${w_{\Gamma_i}}^m$ the transition that corresponds to the last letter of $w_{\Gamma_i}$ is fired $m$ times.
    Hence, we have $m$ tokens on each place $p_{\Gamma_i}$ that will not be consumed by other transitions, and for each $i$, $M(p_{\Gamma_i}) \geq m$ as desired.
    To also obtain $m$ tokens on $p_\final$, we extend the run and consider the firing sequence $\sigma.t_\final.{t_\pump}^m$.
    Note that $t_\final$ is indeed enabled in $M$ since $M \geq \Mfinal''$, and that firing ${t_\pump}^m$ creates $m$ tokens on $p_\final$.
    The marking reached by firing this sequence has the desired properties.

    For the other direction, assume that the places in $X$ are simultaneously unbounded.
    Consider a word $w \in \lang{\linof{\product}}$, say $w = \dtr{w}{0} {w_{\Gamma_1}}^{m_1} \dtr{w}{1} \ldots {w_{\Gamma_k}}^{m_k} \dtr{w}{k}$.
    To prove that $w \in \lang{\mathmbox{\dc{N}}, \Minit, \Mfinal}$, we first show that $w$ is in the language of the product net $N'$.

    To this end, we use the assumption that the places in $X$ are simultaneously unbounded.
    We define $m = \max \set{1,\max_{i \in \oneto{k}} m_i}$, and obtain a computation $\Minit \fire{\sigma} M$ of $N''$ such that $M(p) \geq m$ for all $p \in X$.
    We may assume \wolog that $\sigma$ is of the shape $\sigma = \sigma'.t_\final.{t_\pump}^{m_\pump}$, where $\sigma'$ does not contain $t_\final$ and $t_\pump$.
    Indeed, any sequence that is not of this shape can be reordered while preserving its validity (since executing $t_\final$ later yields greater intermediary markings) and the word that is generated (since $t_\final$ and $t_\pump$ are labeled by $\eps$).
    Note that $\sigma$ needs to contain occurrences of $t_\final$ and $t_\pump$ because we had $M(p_\final) \geq m$.
    If $t_\final$ occurs multiple times in $\sigma$, we simply drop all occurrences but one.

    Consider the computation $\Minit \fire{\sigma\lowerprime} M'$, which we may see as a computation of the product net $N'$.
    It produces the same word, and we have $M' \geq \Mfinal'$ since $t_\final$ was enabled in $M'$.
    Hence,
    \[
        \lambda(\sigma) = \lambda(\sigma') \in  \lang{N',\Minit,\Mfinal'} = \lang{\linof{\product}} \cap \lang{\dc{N},\Minit,\Mfinal}
        \ ,
    \]
    using that the language of the synchronized product is the intersection of the languages.

    Since we have $M'(p_{\Gamma_i}) \geq m$ for all $i$, each infix $w_{\Gamma_i}$ occurs at least $m$ times in $\lambda(\sigma)$.
    We have that $\lambda(\sigma) = \dtr{w}{0} {w_{\Gamma_1}}^{m_1'} \dtr{w}{1} \ldots {w_{\Gamma_k}}^{m_k'} \dtr{w}{k}$, where $m_i' \geq m \geq m_i$ for each $i$.
    Hence, the word $w \in \lang{\linof{\product}}$ that we started with is a subword of $\lambda(\sigma)$.
    We obtain that $\lambda(\sigma) \in \lang{\dc{N},\Minit,\Mfinal}$ implies $w \in \lang{\dc{N},\Minit,\Mfinal}$, which we wanted to show.
\end{proof}


With the proposition at hand, it only remains to combine all ingredients.

\begin{proof}[Proof of \cref{Theorem:PNSREDCMembership}]
    The inclusion $\lang{\sre} \subseteq \dc{\lang{N,\Minit,\Mfinal}}$ holds if and only if $\lang{\product} \subseteq \dc{\lang{N,\Minit,\Mfinal}}$ holds for each product $\product$ of the SRE $\sre$.
    The number of products is polynomial in the size of the SRE.\@

    For each product $\product$, it is easy to construct $\linof{\product}$ in polynomial time.
    In general, an automaton representing a regular expression can be substantially larger than the expression~\cite{GruberH15}.
    For the expression $\linof{\product}$ which has nesting-depth at most $2$, however, it is easy to construct first an automaton and then a Petri net instance $(N_\product,M_{\init,\product},M_{\final,\product})$.
    This net is of size polynomial in the size of $\product$, and it has the same language as $\linof{\product}$.
    By definition, the net $\dc{N}$ is of size polynomial in the size of $N$.
    Combining the two facts yields that the size of the product net $N'$ and also the size of the net $N''$ are polynomial in the input size.

    Altogether, we obtain that to decide the inclusion, we need to invoke a polynomial number of queries for instances of the problem $\PNSU$ of polynomial size.
    We use that $\PNSU$ can be solved in $\EXPSPACE$, \cref{Theorem:PNSU}, and that the calling polynomially many exponential space algorithms still results in an exponential space algorithm to deduce that the overall procedure runs using exponential space.

    The correctness results directly from \cref{Lemma:Linearization} and \cref{Proposition:PNSREDCReduction}.
\end{proof}

With the proof of \cref{Theorem:PNSREDCMembership} completed and the matching lower bound, we have shown that $\SREDC$ is $\EXPSPACE$-complete as stated in \cref{Theorem:PNSREDC}.
Note that since the lower bound is based on the lower bound for coverability, the result also holds if we consider Petri nets to be encoded in unary.

\end{document}
