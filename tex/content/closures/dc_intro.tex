\documentclass[../../diss.tex]{subfiles}
\begin{document}

In this chapter, we study the downward closures of Petri net coverability languages.
Our goal is to obtain a finite automaton representing it with a minimal number of states.
The structure of the chapter is similar to the structure of \cref{Chapter:UC}.
Firstly, we consider the problem of computing a representation of the downward closure of a Petri net.
This problem has already been solved in~\cite{HabermehlMW10} with a procedure that is based on the Karp-Miller tree.
Hence, the resulting automaton can be of non-primitive recursive size.
Here, we complement the result from~\cite{HabermehlMW10} by a matching lower bound, using the well-known fact that a Petri net can weakly compute a variant of the Ackermann function.

This high complexity motivates studying the same restrictions that we have considered in \cref{Chapter:UC}:
We consider the problems of computing the downward closure of BPP net languages, of checking whether a given SRE is included in the downward closure of a given Petri net, and finally whether an SRE is included in the downward closure of a BPP net.
The results that we obtain are similar to the results in the previous chapter.
The upward closure of a BPP net can be computed in exponential time, which we will show by proving some sort of pumping lemma for BPP nets.
%
The SRE-inclusion problems are $\EXPSPACE$ and $\NPTIME$-complete in the case of Petri nets and BPP nets, respectively.
While the complexities match those for the problems $\SREUC$ and $\SREBPPUC$, the proofs are much more involved.
%
Also note that in contrast to the results from the previous chapter, the complexities of the three restricted versions are drastic improvements over the non-primitive complexity of the problem in the general case.

\end{document}
