\documentclass[../../diss.tex]{subfiles}
\begin{document}

\section{SRE inclusion in the upward closures for Petri nets}%
\label{Section:PNSREUC}%

In the first two sections, we have established that the upward closure can be computed in doubly exponential time for Petri nets and in exponential time for BPP nets.
These intractable complexities motivate studying further restrictions of the problem.
We have argued before that the upward and downward closure of any language can be represented by simple regular expressions (SREs).
This result yields a new approach to the closures of Petri net languages:
Instead of computing the closure, we check for a given SRE whether the closure is equal to the language of that SRE.\@
Let us focus on the case of the upward closure for now.
Formally, we will check $\uc{\lang{N,M_\init,M_\final}} = \lang{\sre}$, where $(N,M_\init,M_\final)$ is the given Petri net and $\sre$ is an SRE.\@
This SRE may come from an oracle that iteratively creates candidate SREs based on membership queries and on whether the checks for the previously generated candidates have been successful.

Checking the equality decomposes into checking both inclusion.
Checking one of the inclusions is conceptually easy, namely checking $\mathmbox{\uc{\lang{N,M_\init,M_\final}}} \subseteq \lang{\sre}$.
We proceed as follows:
We use that the inclusion holds if and only if the intersection $\uc{\lang{N,M_\init,M_\final}} \cap \ \overline{\lang{\sre}}$ is empty.
To represent $\overline{\lang{\sre}}$, we construct an NFA for the language of the SRE, determinize it and construct a DFA for the complement which we see as a Petri net in the following.
To represent $\uc{\lang{N,M_\init,M_\final}}$, we may use the Petri net $\uc{N}$ whose language is $\uc{\lang{N,M_\init,M_\final}}$.
Finally, we construct the synchronized product of the two nets and check language emptiness, which amounts to a coverability query.

The complexity of this procedure depends on the size of the representation of $\overline{\lang{\sre}}$.
In principle, an NFA for the complement of a regular language could be exponentially larger than an NFA for the language itself.
We leave it as future work to see whether this worst-case behavior can actually occur in the case of languages defined by SREs.
Instead, we focus on checking the other inclusion.

The reason for focusing on the inclusion $\lang{\sre} \subseteq \uc{\lang{N,M_\init,M_\final}}$ is the following:
We envision a refinement procedure in which an oracle iteratively outputs candidate SREs $\sre_1, \sre_2, \ldots$%.
For each candidate, we check whether the inclusion $\lang{\sre_i} \subseteq \uc{\lang{N,M_\init,M_\final}}$ holds.
The answer to each check is sent to the oracle, which uses it for refinement.
Once we are sufficiently sure that some $\sre_i$ represents the actual upward closure of the language (\eg because the inclusion $\lang{\sre_i} \subseteq \uc{\lang{N,M_\init,M_\final}}$ holds, but the inclusion for SREs with a larger language  fails), we check $\uc{\lang{N,M_\init,M_\final}} \subseteq \lang{\sre_i}$.
This procedure uses a high number of inclusion checks with the SRE as the left-hand side, compared to a low number of inclusion checks with the SRE as the right-hand side.

%
\cheatpagebreak
%

Formally, our goal in this section is to study the complexity of the following decision problem.

\begin{problem}
    \problemtitle{SRE inclusion in the upward closure}
    \problemshort{($\SREUC$)}
    \probleminput{A labeled Petri net instance $(N,\Minit,\Mfinal)$, an SRE $\sre$.}
    \problemquestion{$\lang{\sre} \subseteq \uc{\lang{N,\Minit,\Mfinal}} $?}
\end{problem}

We will show that this problem is $\EXPSPACE$-complete.
This is an improvement over the doubly exponential time needed to construct a representation of the upward closure, as proven in the first section of this chapter.
Furthermore, it matches the complexity of the coverability problem.
Indeed, the lower bound is a straightforward reduction from coverability, while the upper bound reduces $\SREUC$ to a polynomial number of membership queries.

\begin{theorem}%
\label{Theorem:PNSREUC}%
    $\SREUC$ is $\EXPSPACE$-complete.
\end{theorem}

We proceed by proving the upper bound, \ie $\EXPSPACE$ membership, and the lower bound, $\EXPSPACE$-hardness, separately.

\paragraph{Upper bound}

We show one direction of \cref{Theorem:PNSREUC} by proving $\EXPSPACE$ membership.

\begin{proposition}%
\label{Proposition:SREUCMembership}%
    The problem $\SREUC$ can be solved in exponential space.
\end{proposition}

To prove the proposition, we use the following observation.
A language $\calL$ is contained in an upward-closed language $\calL'$ if every minimal word (\wrt the subword relation) of $\calL$ is a member of $\calL'$, \ie $\min \calL \subseteq \calL'$.
Indeed, $\calL \subseteq \uc{\min \calL} \subseteq \uc{  \calL' } = \calL'$ holds in this case.
We even know that the set of minimal words of a language is finite, by Property~\ref{Property:WQOFiniteBasis} from \cref{Lemma:WQOProperties} and the fact that the subword relation is a WQO.\@
The problem with using this observation to decide inclusions is that it can be difficult or even impossible to compute the set of minimal words.

Here, we use that for an SRE, it is easy to construct the set of minimal words in polynomial time.
To be precise, we can construct for each product $p$ in the SRE of interest a word $\min p$ that is the unique minimal word in $\lang{p}$, \ie $\uc{(\min p)} = \uc{\lang{p}}$, in linear time.
It then only remains to check that for each product $p$ of the given SRE, $\min p$ is contained in the upward closure of $\lang{N,M_\init,M_\final}$.
To conduct this check, we do not need to construct an NFA representation of the upward closure.
It is sufficient to use the net $\uc{N}$ whose size is polynomial in the size of $N$.

We show how to extract from a product its minimal word, then give the formal proof.

\begin{definition}
   Let $p$ be the product of an SRE.\@
   We define $\min p$ by induction as follows:
    \begin{align*}
        \min a &= a
        &
        \min p.p' &= \min p . \min p'
        \\
        \min a \cup \eps &= \eps
        &
        \min \Gamma^* &= \eps\ .
    \end{align*}
\end{definition}

Using the definitions, it is easy to see that for each product $\product$, we have $\uc{\lang{\product}} = \uc{\min \product}$.
To obtain $\min{\lang{\sre}}$, the minimal words of the language of an SRE, we can construct the minimal words for each of the products.
Some of them might be comparable, so we obtain $\min{\lang{\sre}}$ by removing the non-minimal ones.
The proof of \cref{Proposition:SREUCMembership} follows immediately.

\begin{proof}[Proof of \cref{Proposition:SREUCMembership}]
    Let $(N,\Minit,\Mfinal)$ be the given net and let $\sre = p_1 \cup \ldots \cup p_k$ be the given SRE, consisting of the products $p_i$.
    As argued above, we have that $\lang{\sre} \subseteq \uc{\lang{N,\Minit,\Mfinal}}$ if and only if for each $i$, $\min p_i \in \uc{\lang{N,\Minit,\Mfinal}}$.
    We consider the net $\mathmbox{\uc{N}}$ with $\lang{\mathmbox{\uc{N}},\Minit,\Mfinal} = \mathmbox{\uc{\lang{N,\Minit,\Mfinal}}}$, and check $\min p_i \in \lang{\uc{N},\Minit,\Mfinal}$.
    Note that both the number of products and the length of each $\min p_i$ is polynomial in the size of the SRE.\@
    Hence, we have reduced the problem $\SREUC$ to a polynomial amount of membership queries for instances of polynomial size.
    The desired result follows as $\WORDPNCOV$ is in $\EXPSPACE$, see \cref{Section:PNCovLang}.
\end{proof}

\paragraph{Lower bound}

The lower bound for $\SREUC$ follows directly from \cref{Theorem:PNLipton}, the $\EXPSPACE$-hardness of coverability.

\begin{lemma}%
\label{Lemma:PNSREUChardness}%
    $\SREUC$ is $\EXPSPACE$-hard.
\end{lemma}

\begin{proof}
    We reduce from the coverability problem for unlabeled Petri nets, which is $\EXPSPACE$-hard by \cref{Theorem:PNLipton}.
    Let $(N,\Minit,\Mfinal)$ be a Petri net instance.
    We see $N$ as a labeled Petri net in which all transitions are labeled by $\eps$.
    Consequently, we have $\lang{N,\Minit,\Mfinal} = \set{\eps}$ if $\Mfinal$ is coverable from $\Minit$ in $N$, else $\lang{N,\Minit,\Mfinal} = \emptyset$.
    Hence, $\uc{\lang{N,\Minit,\Mfinal}}$ is either $\Sigma^*$ or $\emptyset$.
    We obtain that the instance $\emptyset^* \subseteq \uc{\lang{N,\Minit,\Mfinal}}$ of $\SREUC$ is equivalent to the coverability problem, since the SRE $\emptyset^*$ expresses the language $\set{\eps}$.
\end{proof}

This completes the proof of \cref{Theorem:PNSREUC}.
Note that the same result holds in the case of the unary encoding, since coverability remains $\EXPSPACE$-complete for nets encoded in unary:
Lipton's construction only uses transition multiplicities from the set $\set{0,1}$.

\end{document}
